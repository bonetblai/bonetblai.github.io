\documentclass{gkibeamer}

\usepackage{tikz}[2008/02/13]

\title[AI Planning]{Principles of AI Planning}
\author[B.~Nebel,\\R.~Mattm\"uller]{Bernhard Nebel and Robert Mattm\"uller}
\date[]{Winter Semester 2011/2012}

%% Use \alert instead of \textred where it makes sense.
%% (\textred makes sense when you *really* want the color to be red,
%% independent of the beamer style settings.)
\newcommand{\hilite}[1]{{\color{structure} #1}}
\newcommand{\textred}[1]{{\color{red} #1}}
\newcommand{\textblue}[1]{{\color{blue} #1}}
\newcommand{\textgreen}[1]{{\color{green} #1}}
\newcommand{\textbrown}[1]{{\color{brown} #1}}
\newcommand{\textviolet}[1]{{\color{violet} #1}}
\newcommand{\textbg}[1]{{\color{bg} #1}}

\newcommand{\eg}{e.\,g.}
\newcommand{\ie}{i.\,e.}

%% State variables and values for state variables
\newcommand{\var}[1]{\textit{#1}}
\newcommand{\val}[1]{\text{#1}}

%% Various stuff
\newcommand{\astar}{A${}^*$}
\newcommand{\idastar}{IDA${}^*$}

%% Blocksworld stuff
\newcommand{\AONB}{\var{A-on-B}}
\newcommand{\AONC}{\var{A-on-C}}
\newcommand{\AOND}{\var{A-on-D}}
\newcommand{\BONA}{\var{B-on-A}}
\newcommand{\BONC}{\var{B-on-C}}
\newcommand{\BOND}{\var{B-on-D}}
\newcommand{\CONA}{\var{C-on-A}}
\newcommand{\CONB}{\var{C-on-B}}
\newcommand{\COND}{\var{C-on-D}}
\newcommand{\DONA}{\var{D-on-A}}
\newcommand{\DONB}{\var{D-on-B}}
\newcommand{\DONC}{\var{D-on-C}}
\newcommand{\AONTABLE}{\var{A-on-T}}
\newcommand{\BONTABLE}{\var{B-on-T}}
\newcommand{\CONTABLE}{\var{C-on-T}}
\newcommand{\CLEARA}{\var{A-clear}}
\newcommand{\CLEARB}{\var{B-clear}}
\newcommand{\CLEARC}{\var{C-clear}}

\newcommand{\applyop}[2]{\textit{app}_{#1}(#2)}
\newcommand{\applyops}[2]{\textit{app}_{#1}(#2)}
\newcommand{\applyplan}[2]{\textit{app}_{#1}(#2)}
\newcommand{\regr}[2]{\textit{regr}_{#1}(#2)}
\newcommand{\regrstrips}[2]{\textit{sregr}_{#1}(#2)}
\newcommand{\eprecon}[2]{\textit{EPC}_{#1}(#2)}

\newcommand{\compl}[1]{\overline{#1}}

\newcommand{\changes}[2]{\lbrack #1\rbrack_{#2}}
\newcommand{\CEF}{\vartriangleright}

\newcommand{\relaxation}[1]{{#1}^+}
\newcommand{\onset}[1]{\textit{on}(#1)}
\newcommand{\hplus}{\ensuremath{h^+}}
\newcommand{\hmax}{\ensuremath{h_{\text{max}}}}
\newcommand{\hadd}{\ensuremath{h_{\text{add}}}}
\newcommand{\hlmcut}{\ensuremath{h_{\text{LM-cut}}}}
\newcommand{\hff}{\ensuremath{h_{\text{FF}}}}
\newcommand{\hcs}{\ensuremath{h_{\text{cs}}}}
\newcommand{\hsa}{\ensuremath{h_{\text{sa}}}}
\newcommand{\hhhh}{\ensuremath{h_{\text{HHH}}}}

\newcommand{\sasplus}{SAS${}^+$}

\newcommand{\graphequiv}{\stackrel{\textup{G}}{\sim}}

\newcommand{\cg}{\ensuremath{\textit{CG}}}

%% Turing machines and complexity theory
\newcommand{\accept}{{\textsf{Y}}}

\newcommand{\easier}{\ensuremath{\le_{\text{p}}}}

\newcommand{\decisionclass}[1]{\ensuremath{\textsf{\textup{#1}}}}
\newcommand{\dtime}{\decisionclass{DTIME}}
\newcommand{\ntime}{\decisionclass{NTIME}}
\newcommand{\dspace}{\decisionclass{DSPACE}}
\newcommand{\nspace}{\decisionclass{NSPACE}}

\newcommand{\ptime}{\decisionclass{P}}
\newcommand{\np}{\decisionclass{NP}}
\newcommand{\pspace}{\decisionclass{PSPACE}}
\newcommand{\npspace}{\decisionclass{NPSPACE}}
\newcommand{\exptime}{\decisionclass{EXP}}
\newcommand{\nexptime}{\decisionclass{NEXP}}
\newcommand{\expspace}{\decisionclass{EXPSPACE}}
\newcommand{\dblexptime}{\decisionclass{2EXP}}

\newcommand{\planex}{\textsc{PlanEx}}
\newcommand{\planlen}{\textsc{PlanLen}}

\newcommand{\ms}[1]{\mathchoice{\mbox{\normalsize\it #1}}{\mbox{\normalsize\it #1}}{\mbox{\scriptsize\it #1}}{\mbox{\tiny\it #1}}}
\newcommand{\image}[2]{\ms{img}_{#1}(#2)}
\newcommand{\spreimage}[2]{\ms{spreimg}_{#1}(#2)}
\newcommand{\wpreimage}[2]{\ms{wpreimg}_{#1}(#2)}
\newcommand{\disbwd}[2]{\delta^{\ms{bwd}}_{#1}(#2)}
\newcommand{\ats}[1]{\ms{scope}(#1)}
\newcommand{\ndopcpc}[2]{\tau^{\ms{nd}}_{#1}(#2)}
\newcommand{\choice}{\mathop{|}}
\newcommand{\opcpc}[2]{\tau_{#1}(#2)}

%% Don't use \begin{proof}/\end{proof} for multi-part proofs because
%% this places a QED symbol.
\newenvironment{proofstart}{\begin{block}{Proof.}}{\end{block}}
\newenvironment{proofmid}{\begin{block}{Proof (ctd.)}}{\end{block}}
\newenvironment{proofend}{\begin{proof}[Proof (ctd.)]}{\end{proof}}

\newtheorem{remark}[theorem]{Remark}

%% Macros to align the width of things.
\newlength{\mywidth}
\newcommand{\setmywidth}[1]{\settowidth{\mywidth}{#1}}
\newcommand{\usemywidth}[1]{\makebox[\mywidth][l]{#1}}
\newcommand{\usemywidthmath}[1]{\usemywidth{\ensuremath{#1}}}

%% A tighter "align" environment.
\newenvironment{tightalign}[1][c]{\par\(\begin{array}[#1]{@{}r@{}l}}
  {\end{array}\)\par}
\newenvironment{wrappedmath}[1][t]{\begin{array}[#1]{@{}l}}{\end{array}}

%% Nondeterministic transition systems with connector arcs
\newcommand{\arc}[4]{
  \begin{pgfscope}
    \pgfsetlinewidth{1pt}
    \pgfpathmoveto{\pgfpointanchor{#3}{center}}
    \pgfpathlineto{\pgfpointanchor{#1}{#2}}
    \pgfpathlineto{\pgfpointanchor{#4}{center}}
    \pgfusepath{clip}
    \pgfpathcircle{\pgfpointanchor{#1}{#2}}{1.8mm}
    \pgfusepath{draw}
  \end{pgfscope}
}


\begin{document}

\lectureno{6}
\subtitle{Planning as search: search algorithms}
\date{November 15th, 2011}
\maketitles

\section[Introduction]{Introduction to search algorithms for planning}
\subsection[Nodes and states]{Search nodes \& search states}

\begin{frame}{Our plan for the next lectures}
  Choices to make:
  \begin{enumerate}
  \item search direction: progression/regression/both
    \\ $\leadsto$ previous chapter
  \item search space representation: states/sets of states
    \\ $\leadsto$ previous chapter
  \item search algorithm: uninformed/heuristic; systematic/local
    \\ $\leadsto$ \alert{this chapter}
  \item search control: heuristics, pruning techniques
    \\ $\leadsto$ next chapters
  \end{enumerate}
\end{frame}

\begin{frame}{Search}
  \begin{itemize}
  \item Search algorithms are used to find solutions (plans) for
    \alert{transition systems} in general, not just for planning
    tasks.
  \item Planning is \alert{one application} of search among many.
  \item In this chapter, we describe some popular and/or
    representative search algorithms, and (the basics of) how they
    apply to planning.
  \item Most of this is review of material that should be known \\
    (details: Russell and Norvig's textbook).
  \end{itemize}
\end{frame}

\begin{frame}{Search states vs.\ search nodes}
  In search, one distinguishes:
  \begin{itemize}
  \item \alert{search states} \alert{$s$} $\leadsto$ states
    (vertices) of the transition system
  \item \alert{search nodes} \alert{$\sigma$} $\leadsto$ search
    states plus information on where/when/how they are encountered
    during search
  \end{itemize}

  \begin{block}{What is in a search node?}
    Different search algorithms store different information in a
    search node $\sigma$, but typical information includes:
    \begin{itemize}
    \item \alert{$\textit{state}(\sigma)$:} associated search state
    \item \alert{$\textit{parent}(\sigma)$:} pointer to search node
      from which $\sigma$ is reached
    \item \alert{$\textit{action}(\sigma)$:} an action/operator
      leading from $\textit{state}(\textit{parent}(\sigma))$ to
      $\textit{state}(\sigma)$
    \item \alert{$g(\sigma)$:} cost of $\sigma$ (length of path from
      the root node)
    \end{itemize}
    For the root node, $\textit{parent}(\sigma)$ and
    $\textit{action}(\sigma)$ are undefined.
  \end{block}
\end{frame}

\begin{frame}{Search states vs.\ planning states}
  Search states $\neq$ (planning) states:
  \begin{itemize}
  \item \alert{Search states} don't have to correspond to
    \alert{states} in the planning sense.
    \begin{itemize}
    \item progression: search states $\approx$ \alert{(planning)
      states}
    \item regression: search states $\approx$ \alert{sets of
      states} (formulae)
    \end{itemize}
  \item Search algorithms for planning where search states are
    planning states are called \alert{state-space search}
    algorithms.
  \item Strictly speaking, regression is \alert{not} an example of
    state-space search, although the term is often used loosely.
  \item However, we will put the emphasis on progression, which is
    almost always state-space search.
  \end{itemize}
\end{frame}

\begin{frame}{Required ingredients for search}
  A general search algorithm can be applied to any transition system
  for which we can define the following three operations:
  \begin{itemize}
  \item \hilite{$\text{init}()$}: generate the \alert{initial state}
  \item \hilite{$\text{is-goal}(s)$}: test if a given state is a
    \alert{goal state}
  \item \hilite{$\text{succ}(s)$}: generate the set of
    \alert{successor states} of state $s$, along with the
    \alert{operators} through which they are reached \\
    (represented as pairs $\langle o, s'\rangle$ of operators
    and states)
  \end{itemize}
  Together, these three functions form a \alert{search space}
  (a very similar notion to a transition system).
\end{frame}

\subsection{Search for planning}

\begin{frame}{Search for planning: progression}
  Let $\Pi = \langle A,I,O,\gamma\rangle$ be a planning task.

  \begin{block}{Search space for progression search}
    \hilite{states:} all states of $\Pi$ (assignments to $A$)
  \begin{itemize}
  \item $\hilite{\text{init}()} = I$
  \item $\hilite{\text{is-goal}(s)}
    = \begin{cases}
      \textbf{true} & \text{if~} s \models \gamma \\
      \textbf{false} & \text{otherwise}
    \end{cases}$
    %% We use 1/0 for true/false in propositional logic valuations,
    %% but this stuff here is considered closer to pseudocode, not
    %% mathematical logic notation.
  \item $\hilite{\text{succ}(s)} =
    \{ \langle o, s'\rangle \mid
      o \in O, s' = \applyop{o}{s}\}$
  \end{itemize}
  \end{block}
\end{frame}

\begin{frame}{Search for planning: regression}
  Let $\Pi = \langle A,I,O,\gamma\rangle$ be a planning task.

  \begin{block}{Search space for regression search}
    \hilite{states:} all formulae over $A$ (how many?)
  \begin{itemize}
  \item $\hilite{\text{init}()} = \gamma$
  \item $\hilite{\text{is-goal}(\varphi)}
    = \begin{cases}
      \textbf{true} & \text{if~} I \models \varphi \\
      \textbf{false} & \text{otherwise}
    \end{cases}$
  \item $\hilite{\text{succ}(\varphi)} =
    \{ \langle o, \varphi'\rangle \mid
      o \in O, \varphi' = \regr{o}{\varphi},
      \varphi'~\text{is satisfiable}\}$ \\
    (modified if splitting is used)
  \end{itemize}
  \end{block}
\end{frame}

\begin{frame}{Classification of search algorithms}
  uninformed search vs.\ heuristic search:
  \begin{itemize}
  \item \alert{uninformed search algorithms} only use the basic
    ingredients for general search algorithms
  \item \alert{heuristic search algorithms} additionally use
    \alert{heuristic functions} which estimate how close a node is to
    the goal
  \end{itemize}

  \medskip

  systematic search vs.\ local search:
  \begin{itemize}
  \item \alert{systematic algorithms} consider a large number of
    search nodes simultaneously
  \item \alert{local search algorithms} work with one (or a few)
    candidate solutions (search nodes) at a time
  \item not a black-and-white distinction; there are
    \alert{crossbreeds} (\eg, enforced hill-climbing)
  \end{itemize}
\end{frame}

\begin{frame}{Classification: what works where in planning?}
  uninformed vs.\ heuristic search:
  \begin{itemize}
  \item For \alert{satisficing} planning, heuristic search
    vastly outperforms uninformed algorithms on most domains.
  \item For \alert{optimal} planning, the difference is less
    pronounced.%% \\ An efficiently implemented uninformed algorithm is not
    %% easy to beat in most domains.
  \end{itemize}

  \medskip

  systematic search vs.\ local search:
  \begin{itemize}
  \item For \alert{satisficing} planning, the most successful
    algorithms are somewhere between the two extremes.
  \item For \alert{optimal} planning, systematic algorithms are
    required.
  \end{itemize}
\end{frame}

\subsection[Common procedures]{Common procedures for search
  algorithms}

\begin{frame}{Common procedures for search algorithms}
  Before we describe the different search algorithms, we introduce
  three procedures used by all of them:
  \begin{itemize}
  \item \alert{make-root-node:} Create a search node without parent.
  \item \alert{make-node:} Create a search node for a state generated
    as the successor of another state.
  \item \alert{extract-solution:} Extract a solution from a search
    node representing a goal state.
  \end{itemize}
\end{frame}

\begin{frame}{Procedure \text{make-root-node}}
  \alert{make-root-node:} Create a search node without parent.
  \begin{block}{Procedure \text{make-root-node}}
    \textbf{def} \text{make-root-node}($s$): \\
    {}\qquad $\sigma := \textbf{new}~\text{node}$ \\
    {}\qquad $\textit{state}(\sigma) := s$ \\
    {}\qquad $\textit{parent}(\sigma) := \text{undefined}$ \\
    {}\qquad $\textit{action}(\sigma) := \text{undefined}$ \\
    {}\qquad $g(\sigma) := 0$ \\
    {}\qquad \textbf{return} $\sigma$
  \end{block}
\end{frame}

\begin{frame}{Procedure \text{make-node}}
  \alert{make-node:} Create a search node for a state generated
  as the successor of another state.
  \begin{block}{Procedure \text{make-node}}
    \textbf{def} \text{make-node}($\sigma$, $o$, $s$): \\
    {}\qquad $\sigma' := \textbf{new}~\text{node}$ \\
    {}\qquad $\textit{state}(\sigma') := s$ \\
    {}\qquad $\textit{parent}(\sigma') := \sigma$ \\
    {}\qquad $\textit{action}(\sigma') := o$ \\
    {}\qquad $g(\sigma') := g(\sigma) + 1$ \\
    {}\qquad \textbf{return} $\sigma'$
  \end{block}
\end{frame}

\begin{frame}{Procedure \text{extract-solution}}
  \alert{extract-solution:} Extract a solution from a search
  node representing a goal state.
  \begin{block}{Procedure \text{extract-solution}}
    \textbf{def} \text{extract-solution}($\sigma$): \\
    {}\qquad $\textit{solution} := \textbf{new}~\text{list}$ \\
    {}\qquad \textbf{while} $\textit{parent}(\sigma)$ is defined: \\
    {}\qquad\qquad
    $\textit{solution}.\text{push-front}(\textit{action}(\sigma))$ \\
    {}\qquad\qquad $\sigma := \textit{parent}(\sigma)$ \\
    {}\qquad \textbf{return} $\textit{solution}$
  \end{block}
\end{frame}

\section[Uninformed search]{Uninformed search algorithms}
\subsection[Breadth-first w/o duplicate detection]{Breadth-first
  search without duplicate detection}

\begin{frame}{Uninformed search algorithms}
  \begin{itemize}
  \item Uninformed algorithms are less relevant for planning than
    heuristic ones, so we keep their discussion brief.
  \item Uninformed algorithms are mostly interesting to us because we
    can compare and contrast them to related heuristic search
    algorithms.
  \end{itemize}

  \bigskip

  Popular uninformed systematic search algorithms:
  \begin{itemize}
  \item \alert{breadth-first search}
  \item depth-first search
  \item iterated depth-first search
  \end{itemize}

  \bigskip

  Popular uninformed local search algorithms:
  \begin{itemize}
  \item \alert{random walk}
  \end{itemize}
\end{frame}

\begin{frame}{Breadth-first search without duplicate detection}
  \begin{block}{Breadth-first search}
    \small
    \textit{queue} := \textbf{new} \text{fifo-queue} \\
    $\textit{queue}.\text{push-back}(\text{make-root-node}(\hilite{\text{init}()}))$ \\
    \textbf{while not} $\textit{queue}.\text{empty}()$: \\
    {}\qquad $\sigma = \textit{queue}.\text{pop-front}()$ \\
    {}\qquad \textbf{if}
    $\hilite{\text{is-goal}}(\text{state}(\sigma))$: \\
    {}\qquad\qquad \textbf{return} $\text{extract-solution}(\sigma)$
    \\
    {}\qquad \textbf{for each} $\langle o, s \rangle \in
    \hilite{\text{succ}}(\textit{state}(\sigma))$: \\
    {}\qquad\qquad $\sigma' := \text{make-node}(\sigma, o, s)$ \\
    {}\qquad\qquad $\textit{queue}.\text{push-back}(\sigma')$ \\
    \textbf{return} unsolvable
  \end{block}
  \begin{itemize}
  \item Possible improvement: \alert{duplicate detection} (see next
    slide).
  \item Another possible improvement: test if $\sigma'$ is a goal
    node; if so, terminate immediately. (We don't do this because it
    obscures the similarity to some of the later algorithms.)
  \end{itemize}
\end{frame}

\subsection[Breadth-first with duplicate detection]{Breadth-first
  search with duplicate detection}

\begin{frame}<all:1,2>{Breadth-first search with duplicate detection}
  \begin{block}{Breadth-first search with duplicate detection}
    \small
    \textit{queue} := \textbf{new} \text{fifo-queue} \\
    $\textit{queue}.\text{push-back}(\text{make-root-node}(\hilite{\text{init}()}))$
    \\
    \alert<all:1>{$\textit{closed} := \emptyset$} \\
    \textbf{while not} $\textit{queue}.\text{empty}()$: \\
    {}\qquad $\sigma = \textit{queue}.\text{pop-front}()$ \\
    {}\qquad \alert<all:1>{\textbf{if} $\textit{state}(\sigma) \notin
      \textit{closed}$:} \\
    {}\qquad\qquad \alert<all:1>{$\textit{closed} := \textit{closed} \cup
      \{\textit{state}(\sigma)\}$} \\
    {}\qquad\qquad \textbf{if}
    $\hilite{\text{is-goal}}(\text{state}(\sigma))$: \\
    {}\qquad\qquad\qquad \textbf{return} $\text{extract-solution}(\sigma)$
    \\
    {}\qquad\qquad \textbf{for each} $\langle o, s \rangle \in
    \hilite{\text{succ}}(\textit{state}(\sigma))$: \\
    {}\qquad\qquad\qquad $\sigma' := \text{make-node}(\sigma, o, s)$ \\
    {}\qquad\qquad\qquad $\textit{queue}.\text{push-back}(\sigma')$ \\
    \textbf{return} unsolvable
  \end{block}
\end{frame}

\subsection{Random walk}

\begin{frame}{Random walk}
  \begin{block}{Random walk}
    $\sigma := \text{make-root-node}(\hilite{\text{init}}())$ \\
    \textbf{forever}: \\
    {}\qquad \textbf{if}
    $\hilite{\text{is-goal}}(\text{state}(\sigma))$: \\
    {}\qquad\qquad \textbf{return} $\text{extract-solution}(\sigma)$
    \\
    {}\qquad Choose a random element $\langle o, s\rangle$
    from $\text{\hilite{succ}}(\text{state}(\sigma))$. \\
    {}\qquad $\sigma := \text{make-node}(\sigma, o, s)$
  \end{block}

  \begin{itemize}
  \item The algorithm usually does not find any solutions, unless
    almost every sequence of actions is a plan.
  \item Often, it runs indefinitely without making progress.
  \item It can also fail by reaching a \alert{dead end}, a state with
    no successors. This is a weakness of many local search approaches.
  \end{itemize}
\end{frame}

\section[Heuristic search]{Heuristic search algorithms}
\subsection[Heuristics]{Heuristics: definition and properties}

\begin{frame}{Heuristic search algorithms:
    \only<all:1>{systematic}\only<all:2>{local}}
  \begin{itemize}
  \item Heuristic search algorithms are the most common and overall
    most successful algorithms for classical planning.
  \end{itemize}

  \bigskip

  \begin{overprint}
    \onslide<all:1>
    Popular systematic heuristic search algorithms:
    \begin{itemize}
    \item \alert{greedy best-first search}
    \item \alert{\astar}
    \item \alert{weighted \astar}
    \item {\idastar}
    \item depth-first branch-and-bound search
    % \item breadth-first heuristic search
    \item \dots
    \end{itemize}
    \onslide<all:2>
    Popular heuristic local search algorithms:
    \begin{itemize}
    \item \alert{hill-climbing}
    \item \alert{enforced hill-climbing}
    \item beam search
    \item tabu search
    \item genetic algorithms
    \item simulated annealing
    \item \dots
    \end{itemize}
  \end{overprint}
\end{frame}

\begin{frame}{Heuristic search: idea}
  \begin{center}
    \begin{tikzpicture}[scale=.9]
      \node[circle, draw, fill=gray, minimum height=5mm,
        label=above:$\hilite{\text{goal}}$] (Goal) at (9,0) {}; 
      \node[circle,draw, inner sep=0.5mm,
        label=above:$\hilite{\text{init}}$] (N0) at (0,0) {};
      \node[circle,draw, inner sep=0.5mm] (N00) at (2,-1.5) {};
      \node[circle,draw, inner sep=0.5mm] (N01) at (2,1.5) {};
      \node[circle,draw, inner sep=0.5mm] (N000) at (4,-2.25) {};
      \node[circle,draw, inner sep=0.5mm] (N001) at (4,-0.75) {};
      \node[circle,draw, inner sep=0.5mm] (N010) at (4,0.75) {};
      \node[circle,draw, inner sep=0.5mm] (N011) at (4,2.25) {};

      \draw[->,red, thick] (N0) -- (N00);
      \draw[->,blue, thick] (N0) -- (N01);
      \draw[->,red, thick] (N00) -- (N000);
      \draw[->,blue, thick] (N00) -- (N001);
      \draw[->,blue, thick] (N01) -- (N010);
      \draw[->,blue, thick] (N01) -- (N011);
      
      \draw[dashed, violet] (N000) -- node[above,sloped] {
        \scriptsize distance estimate} (Goal);
      \draw[dashed, violet] (N001) -- node[above,sloped] {
        \scriptsize distance estimate} (Goal);
      \draw[dashed, violet] (N010) -- node[above,sloped] {
        \scriptsize distance estimate} (Goal);
      \draw[dashed, violet] (N011) -- node[above,sloped] {
        \scriptsize distance estimate} (Goal);
    \end{tikzpicture}
  \end{center}
\end{frame}

\begin{frame}{Required ingredients for heuristic search}
  A \alert{heuristic search algorithm} requires one more operation \\
  in addition to the definition of a search space.

  \begin{definition}[heuristic function]
    Let $\Sigma$ be the set of nodes of a given search space.

    A \alert{heuristic function} or \alert{heuristic} (for that search
    space) is a function $h: \Sigma \to \mathbb N_0 \cup \{\infty\}$.
  \end{definition}

  The value $h(\sigma)$ is called the \alert{heuristic estimate} or
  \alert{heuristic value} of heuristic $h$ for node $\sigma$. It is
  supposed to estimate the distance from $\sigma$ to the nearest goal
  node.
\end{frame}

\begin{frame}{What exactly is a heuristic estimate?}
  What does it mean that $h$ ``estimates the goal distance''?
  \begin{itemize}
  \item For most heuristic search algorithms, $h$ does not need to
    have any strong properties for the algorithm to work (= be
    correct and complete).
  \item However, the \alert{efficiency} of the algorithm closely
    relates to how accurately $h$ reflects the actual goal distance.
  \item For some algorithms, like \astar, we can prove strong formal
    relationships between properties of $h$ and properties of the
    algorithm (optimality, dominance, run-time for bounded error,
    \dots)
  \item For other search algorithms, ``it works well in practice'' is
    often as good an analysis as one gets.
  \end{itemize}
\end{frame}

\begin{frame}{Heuristics applied to nodes or states?}
  \begin{itemize}
  \item Most texts apply heuristic functions to \alert{states}, not
    \alert{nodes}.
  \item This is slightly \alert{less general} than our definition:
    \begin{itemize}
    \item Given a state heuristic $h$, we can define an
      equivalent node heuristic as $h'(\sigma) :=
      h(\textit{state}(\sigma))$.
    \item The opposite is not possible. (Why not?)
    \end{itemize}
  \item There is good justification for only allowing state-defined
    heuristics: why should the estimated distance to the goal depend
    on \alert{how} we ended up in a given state $s$?
  \item We call heuristics which don't just depend on
    $\textit{state}(\sigma)$ \alert{pseudo-heuristics}.
  \item In practice there are sometimes good reasons to have the
    heuristic value depend on the generating path of $\sigma$ \\
    (\eg, the \alert{landmark pseudo-heuristic}, Richter et al.\ 2008).
  \end{itemize}
\end{frame}

\begin{frame}{Perfect heuristic}
  Let $\Sigma$ be the set of nodes of a given search space.

  \begin{definition}[optimal/perfect heuristic]
    The \alert{optimal} or \alert{perfect heuristic} of a search space
    is the heuristic $\alert{h^*}$ which maps each search node
    $\sigma$ to the length of a shortest path from
    $\textit{state}(\sigma)$ to any goal state.
  \end{definition}
  \hilite{Note:} $h^*(\sigma) = \infty$ iff no goal state is reachable
  from $\sigma$.
\end{frame}

\begin{frame}{Properties of heuristics}
  A heuristic $h$ is called
  \begin{itemize}
  \item \alert{safe} if
    $h^*(\sigma) = \infty$ for all $\sigma \in \Sigma$
    with $h(\sigma) = \infty$
  \item \alert{goal-aware} if
    $h(\sigma) = 0$ for all goal nodes $\sigma \in \Sigma$
  \item \alert{admissible} if
    $h(\sigma) \le h^*(\sigma)$ for all nodes $\sigma \in \Sigma$
  \item \alert{consistent} if
    $h(\sigma) \le h(\sigma') + 1$ for all nodes
    $\sigma, \sigma' \in \Sigma$ \\
    such that $\sigma'$ is a successor of $\sigma$
  \end{itemize}

  \medskip

  Relationships?
\end{frame}

\subsection[Systematic search]{Systematic heuristic search algorithms}

\begin{frame}{Greedy best-first search}
  \begin{block}{Greedy best-first search (with duplicate detection)}
    \small
    $\textit{open} := \textbf{new}~\text{min-heap}~\text{ordered by}~
    (\sigma \mapsto h(\sigma))$ \\
    $\textit{open}.\text{insert}(\text{make-root-node}(\hilite{\text{init}()}))$
    \\
    $\textit{closed} := \emptyset$ \\
    \textbf{while not} $\textit{open}.\text{empty}()$: \\
    {}\qquad $\sigma = \textit{open}.\text{pop-min}()$ \\
    {}\qquad \textbf{if} $\textit{state}(\sigma) \notin
    \textit{closed}$: \\
    {}\qquad\qquad $\textit{closed} := \textit{closed} \cup
    \{\textit{state}(\sigma)\}$ \\
    {}\qquad\qquad \textbf{if}
    $\hilite{\text{is-goal}}(\text{state}(\sigma))$: \\
    {}\qquad\qquad\qquad \textbf{return} $\text{extract-solution}(\sigma)$
    \\
    {}\qquad\qquad \textbf{for each} $\langle o, s \rangle \in
    \hilite{\text{succ}}(\textit{state}(\sigma))$: \\
    {}\qquad\qquad\qquad $\sigma' := \text{make-node}(\sigma, o, s)$
    \\
    {}\qquad\qquad\qquad \textbf{if} $h(\sigma') < \infty$: \\
    {}\qquad\qquad\qquad\qquad $\textit{open}.\text{insert}(\sigma')$ \\
    \textbf{return} unsolvable
  \end{block}
\end{frame}

\begin{frame}{Properties of greedy best-first search}
  \begin{itemize}
  \item one of the three most commonly used algorithms for satisficing
    planning
  \item \alert{complete} for safe heuristics
    (due to duplicate detection)
  \item \alert{suboptimal} unless $h$ satisfies some very strong
    assumptions (similar to being perfect)
  \item invariant under all strictly monotonic transformations of $h$
    (\eg, scaling with a positive constant or adding a constant)
  \end{itemize}
\end{frame}

\begin{frame}{\astar}
  \begin{block}{{\astar} (with duplicate detection and reopening)}
    \small
    $\textit{open} := \textbf{new}~\text{min-heap}~\text{ordered by}~
    (\sigma \mapsto g(\sigma) + h(\sigma))$ \\
    $\textit{open}.\text{insert}(\text{make-root-node}(\hilite{\text{init}()}))$
    \\
    $\textit{closed} := \emptyset$ \\
    $\textit{distance} := \emptyset$ \\
    \textbf{while not} $\textit{open}.\text{empty}()$: \\
    {}\qquad $\sigma = \textit{open}.\text{pop-min}()$ \\
    {}\qquad \textbf{if} $\textit{state}(\sigma) \notin
    \textit{closed}$ \textbf{or}
    $g(\sigma) < \textit{distance}(\textit{state}(\sigma))$: \\
    {}\qquad\qquad $\textit{closed} := \textit{closed} \cup
    \{\textit{state}(\sigma)\}$ \\
    {}\qquad\qquad $\textit{distance}(\textit{state}(\sigma)) := g(\sigma)$ \\
    {}\qquad\qquad \textbf{if}
    $\hilite{\text{is-goal}}(\text{state}(\sigma))$: \\
    {}\qquad\qquad\qquad \textbf{return} $\text{extract-solution}(\sigma)$
    \\
    {}\qquad\qquad \textbf{for each} $\langle o, s \rangle \in
    \hilite{\text{succ}}(\textit{state}(\sigma))$: \\
    {}\qquad\qquad\qquad $\sigma' := \text{make-node}(\sigma, o, s)$
    \\
    {}\qquad\qquad\qquad \textbf{if} $h(\sigma') < \infty$: \\
    {}\qquad\qquad\qquad\qquad $\textit{open}.\text{insert}(\sigma')$ \\
    \textbf{return} unsolvable
  \end{block}
\end{frame}

\begin{frame}<all:1-5>{{\astar} example}
  \framesubtitle{Example}
  \begin{center}
    \begin{tikzpicture}
      \node[circle, draw, fill=gray, label=above:$\gamma$, minimum height=5mm] (Goal) at (9,0) {};
      
      \visible<1->{\node[circle, fill=black, inner sep=0.5mm, label=below:$I$, label=above:0+3] (N0) at (0,0) {};}
      \only<1>{
	\node[circle, draw, fill=white, inner sep=0.5mm] at (0,0) {};
	\draw[dashed,violet] (N0) -- node[above,sloped] {3} (Goal);
      }
      \visible<2->{
	\node[circle, fill=black, inner sep=0.5mm, label=above:1+3] (N00) at (2,-1.5) {};
	\node[circle, fill=black, inner sep=0.5mm, label=above:1+2] (N01) at (2,1.5) {};
	\draw[->, blue, thick] (N0) -- (N00);
	\draw[->, red, thick] (N0) -- (N01);}
      \visible<3->{
	\node[circle, draw, inner sep=0.5mm, label=above:2+7] (N010) at (4,0.75) {};
	\node[circle, draw, inner sep=0.5mm, label=above:2+6] (N011) at (4,2.25) {};
	\draw[->, blue, thick] (N01) -- (N010);
	\draw[->, red, thick] (N01) -- (N011);
	\draw[dashed,violet] (N010) -- node[above,sloped] {7} (Goal);
	\draw[dashed,violet] (N011) -- node[above,sloped] {6} (Goal);
      }
      \visible<4->{
	\node[circle, draw, inner sep=0.5mm, label=above:2+5] (N000) at (4,-0.75) {};
	\node[circle, draw, fill=black, inner sep=0.5mm, label=above:2+2] (N001) at (4,-2.25) {};
	\draw[->, blue, thick] (N00) -- (N001);
	\draw[->, red, thick] (N00) -- (N000);
	\draw[dashed,violet] (N000) -- node[above,sloped] {5} (Goal);
      }
      \visible<5->{
	\node[circle, draw, inner sep=0.5mm, label=above:3+5] (N0010) at (6,-1.5) {};
	\node[circle, draw, fill=black, inner sep=0.5mm, label=below:3+1] (N0011) at (6,-3) {};
	\draw[->, red, thick] (N001) -- (N0010);
	\draw[->, blue, thick] (N001) -- (N0011);
	\draw[dashed,violet] (N0010) -- node[above,sloped] {5} (Goal);
      }
      \visible<6->{
	\node[circle, draw, fill=black, inner sep=0.5mm, label=below:4+8] (N00111) at (8,-2.25) {};
	\draw[->, red, thick] (N0011) -- (Goal);
	\draw[->, blue, thick] (N0011) -- (N00111);
	\draw[dashed,violet] (N00111) -- node[below] {8} (Goal);
      }
      \only<2>{
	\node[circle, draw, fill=white, inner sep=0.5mm] at (2,1.5) {};
	\draw[dashed,violet] (N01) -- node[above,sloped] {2} (Goal);
      }
      \only<2-3>{
	\node[circle, draw, fill=white, inner sep=0.5mm] at (2,-1.5) {};
	\draw[dashed,violet] (N00) -- node[above,sloped] {3} (Goal);
      }
      \only<4>{
	\node[circle, draw, fill=white, inner sep=0.5mm] at (4,-2.25) {};
	\draw[dashed,violet] (N001) -- node[above,sloped] {2} (Goal);
      }
      \only<5>{
	\node[circle, draw, fill=white, inner sep=0.5mm] at (6,-3) {};
	\draw[dashed,violet] (N0011) -- node[above,sloped] {1} (Goal);
      }
    \end{tikzpicture}
  \end{center}
\end{frame}

\begin{frame}{Terminology for \astar}
  \begin{itemize}
  \item \alert{$f$ value} of a node: defined by $f(\sigma) :=
    g(\sigma) + h(\sigma)$
  \item \alert{generated nodes:} nodes inserted into \textit{open}
    at some point
  \item \alert{expanded nodes:} nodes $\sigma$ popped from
    \textit{open} for which the test against \textit{closed} and
    \textit{distance} succeeds
  \item \alert{reexpanded nodes:} expanded nodes for which
    $\textit{state}(\sigma) \in \textit{closed}$ upon expansion
    (also called \alert{reopened} nodes)
  \end{itemize}
\end{frame}

\begin{frame}{Properties of \astar}
  \begin{itemize}
  \item the most commonly used algorithm for optimal planning
  \item rarely used for satisficing planning
  \item \alert{complete} for safe heuristics (even without duplicate
    detection)
  \item \alert{optimal} if $h$ is admissible (even without duplicate
    detection)
    %% Note: A previous version said ``admissible and/or consistent''
    %% here, but this is wrong. Consistency without admissibility is
    %% not good enough for optimality. This can be seen from the
    %% state space s1 <-- s2 --> s3 --> s4 where s1 and s4 are goals,
    %% h(s1) = h(s2) = 2, h(s3) = 1, h(s4) = 0, and s2 is the initial
    %% state. This is consistent, but not admissible, and it causes
    %% A* to find the suboptimal solution s2 --> s3 --> s4 instead of
    %% the optimal s2 --> s1.
  \item never reopens nodes if $h$ is consistent
  \end{itemize}
  
  \bigskip

  \textblue{Implementation notes:}
  \begin{itemize}
  \item in the heap-ordering procedure, it is considered a good idea
    to break ties in favour of lower $h$ values
  \item can simplify algorithm if we know that we only
    have to deal with consistent heuristics
  \item common, hard to spot bug: test membership in \textit{closed}
    at the wrong time
  \end{itemize}
\end{frame}

\begin{frame}{Weighted \astar}
  \begin{block}{Weighted {\astar} (with duplicate detection and reopening)}
    \small
    $\textit{open} := \textbf{new}~\text{min-heap}~\text{ordered by}~
    (\sigma \mapsto g(\sigma) + \alert{W} \cdot h(\sigma))$ \\
    $\textit{open}.\text{insert}(\text{make-root-node}(\hilite{\text{init}()}))$
    \\
    $\textit{closed} := \emptyset$ \\
    $\textit{distance} := \emptyset$ \\
    \textbf{while not} $\textit{open}.\text{empty}()$: \\
    {}\qquad $\sigma = \textit{open}.\text{pop-min}()$ \\
    {}\qquad \textbf{if} $\textit{state}(\sigma) \notin
    \textit{closed}$ \textbf{or}
    $g(\sigma) < \textit{distance}(\textit{state}(\sigma))$: \\
    {}\qquad\qquad $\textit{closed} := \textit{closed} \cup
    \{\textit{state}(\sigma)\}$ \\
    {}\qquad\qquad $\textit{distance}(\sigma) := g(\sigma)$ \\
    {}\qquad\qquad \textbf{if}
    $\hilite{\text{is-goal}}(\text{state}(\sigma))$: \\
    {}\qquad\qquad\qquad \textbf{return} $\text{extract-solution}(\sigma)$
    \\
    {}\qquad\qquad \textbf{for each} $\langle o, s \rangle \in
    \hilite{\text{succ}}(\textit{state}(\sigma))$: \\
    {}\qquad\qquad\qquad $\sigma' := \text{make-node}(\sigma, o, s)$
    \\
    {}\qquad\qquad\qquad \textbf{if} $h(\sigma') < \infty$: \\
    {}\qquad\qquad\qquad\qquad $\textit{open}.\text{insert}(\sigma')$ \\
    \textbf{return} unsolvable
  \end{block}
\end{frame}

\begin{frame}{Properties of weighted \astar}
  The \alert{weight} $W \in \mathbb R^+_0$ is a parameter of the
  algorithm.
  \begin{itemize}
  \item for $W = 0$, behaves like breadth-first search
  \item for $W = 1$, behaves like \astar
  \item for $W \to \infty$, behaves like greedy best-first search
   \end{itemize}

  \bigskip
  
  Properties:
  \begin{itemize}
  \item one of the most commonly used algorithms for satisficing
    planning
  \item for $W > 1$, can prove similar properties to \astar,
    replacing \alert{optimal} with \alert{bounded suboptimal}: generated
    solutions are at most a factor $W$ as long as optimal ones
  \end{itemize}
\end{frame}

\subsection[Local search]{Heuristic local search algorithms}

\begin{frame}{Hill-climbing}
  \begin{block}{Hill-climbing}
    $\sigma := \text{make-root-node}(\hilite{\text{init}}())$ \\
    \textbf{forever}: \\
    {}\qquad \textbf{if}
    $\hilite{\text{is-goal}}(\text{state}(\sigma))$: \\
    {}\qquad\qquad \textbf{return} $\text{extract-solution}(\sigma)$
    \\
    {}\qquad $\Sigma' := \{\,\text{make-node}(\sigma, o, s) \mid
      \langle o, s\rangle \in
      \text{\hilite{succ}}(\text{state}(\sigma))\,\}$ \\
    {}\qquad $\sigma := \text{an element of}~\Sigma'
    \text{~minimizing~} h$ (random tie breaking)
  \end{block}

  \begin{itemize}
  \item can easily get stuck in \alert{local minima} where
    immediate improvements of $h(\sigma)$ are not possible
  \item many variations: tie-breaking strategies, restarts
  \end{itemize}
\end{frame}

\begin{frame}{Enforced hill-climbing}
  \begin{block}{Enforced hill-climbing: procedure \text{improve}}
    \textbf{def} $\textit{improve}(\sigma_0)$: \\
    \small
    {}\qquad\textit{queue} := \textbf{new} \text{fifo-queue} \\
    {}\qquad$\textit{queue}.\text{push-back}(\sigma_0)$ \\
    {}\qquad$\textit{closed} := \emptyset$ \\
    {}\qquad\textbf{while not} $\textit{queue}.\text{empty}()$: \\
    {}\qquad\qquad $\sigma = \textit{queue}.\text{pop-front}()$ \\
    {}\qquad\qquad \textbf{if} $\textit{state}(\sigma) \notin
      \textit{closed}$: \\
    {}\qquad\qquad\qquad $\textit{closed} := \textit{closed} \cup
      \{\textit{state}(\sigma)\}$ \\
    {}\qquad\qquad\qquad \textbf{if} $h(\sigma) < h(\sigma_0)$: \\
    {}\qquad\qquad\qquad\qquad \textbf{return} $\sigma$
    \\
    {}\qquad\qquad\qquad \textbf{for each} $\langle o, s \rangle \in
    \hilite{\text{succ}}(\textit{state}(\sigma))$: \\
    {}\qquad\qquad\qquad\qquad $\sigma' := \text{make-node}(\sigma, o, s)$ \\
    {}\qquad\qquad\qquad\qquad $\textit{queue}.\text{push-back}(\sigma')$ \\
    {}\qquad\textbf{fail}
  \end{block}

  $\leadsto$ breadth-first search for more promising node than
  $\sigma_0$
\end{frame}

\begin{frame}{Enforced hill-climbing (ctd.)}
  \begin{block}{Enforced hill-climbing}
    $\sigma := \text{make-root-node}(\hilite{\text{init}}())$ \\
    \textbf{while not} $\hilite{\text{is-goal}}(\text{state}(\sigma))$: \\
    {}\qquad $\sigma := \text{improve}(\sigma)$ \\
    {}\textbf{return} $\text{extract-solution}(\sigma)$
  \end{block}

  \begin{itemize}
  \item one of the three most commonly used algorithms for satisficing
    planning
  \item can fail if procedure improve fails (when the goal is
    unreachable from $\sigma_0$)
  \item complete for \alert{undirected} search spaces (where the
    successor relation is symmetric) if $h(\sigma) = 0$ for all goal
    nodes and only for goal nodes
  \end{itemize}
\end{frame}

\section*{Summary}

\begin{frame}{Summary}
  \begin{itemize}
  \item distinguish: \alert{planning states}, \alert{search states},
    \alert{search nodes}
    \begin{itemize}
    \item \alert{planning state}: situation in the world
      modelled by the task
    \item \alert{search state}: subproblem remaining to be solved
      \begin{itemize}
      \item In \alert{state-space search} (usually progression
        search), \\ planning states and search states are identical.
      \item In regression search, search states usually describe
        \\ sets of states (``subgoals'').
      \end{itemize}
    \item \alert{search node:} search state + info on ``how we got
      there''
    \end{itemize}
  \item search algorithms mainly differ in \alert{order of node expansion}
    \begin{itemize}
    \item \alert{uninformed} vs.\ \alert{informed} (\alert{heuristic})
      search
    \item \alert{local} vs.\ \alert{systematic} search
    \end{itemize}
  \end{itemize}
\end{frame}

\begin{frame}{Summary (ctd.)}
  \begin{itemize}
  \item \alert{heuristics:} estimators for ``distance to goal node''
    \begin{itemize}
    \item usually: the more accurate, the better performance
    \item desiderata: \alert{safe}, \alert{goal-aware},
      \alert{admissible}, \alert{consistent}
    \item the ideal: \alert{perfect heuristic $h^*$}
    \end{itemize}
  \item most common algorithms for \alert{satisficing planning}:
    \begin{itemize}
    \item \alert{greedy best-first search}
    \item \alert{weighted \astar}
    \item \alert{enforced hill-climbing}
    \end{itemize}
  \item most common algorithm for \alert{optimal planning}:
    \begin{itemize}
    \item \alert{\astar}
    \end{itemize}
  \end{itemize}
\end{frame}

\end{document}
