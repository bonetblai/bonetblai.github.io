\documentclass{gkibeamer}

\usepackage{tikz}[2008/02/13]

\title[AI Planning]{Principles of AI Planning}
\author[B.~Nebel,\\R.~Mattm\"uller]{Bernhard Nebel and Robert Mattm\"uller}
\date[]{Winter Semester 2011/2012}

%% Use \alert instead of \textred where it makes sense.
%% (\textred makes sense when you *really* want the color to be red,
%% independent of the beamer style settings.)
\newcommand{\hilite}[1]{{\color{structure} #1}}
\newcommand{\textred}[1]{{\color{red} #1}}
\newcommand{\textblue}[1]{{\color{blue} #1}}
\newcommand{\textgreen}[1]{{\color{green} #1}}
\newcommand{\textbrown}[1]{{\color{brown} #1}}
\newcommand{\textviolet}[1]{{\color{violet} #1}}
\newcommand{\textbg}[1]{{\color{bg} #1}}

\newcommand{\eg}{e.\,g.}
\newcommand{\ie}{i.\,e.}

%% State variables and values for state variables
\newcommand{\var}[1]{\textit{#1}}
\newcommand{\val}[1]{\text{#1}}

%% Various stuff
\newcommand{\astar}{A${}^*$}
\newcommand{\idastar}{IDA${}^*$}

%% Blocksworld stuff
\newcommand{\AONB}{\var{A-on-B}}
\newcommand{\AONC}{\var{A-on-C}}
\newcommand{\AOND}{\var{A-on-D}}
\newcommand{\BONA}{\var{B-on-A}}
\newcommand{\BONC}{\var{B-on-C}}
\newcommand{\BOND}{\var{B-on-D}}
\newcommand{\CONA}{\var{C-on-A}}
\newcommand{\CONB}{\var{C-on-B}}
\newcommand{\COND}{\var{C-on-D}}
\newcommand{\DONA}{\var{D-on-A}}
\newcommand{\DONB}{\var{D-on-B}}
\newcommand{\DONC}{\var{D-on-C}}
\newcommand{\AONTABLE}{\var{A-on-T}}
\newcommand{\BONTABLE}{\var{B-on-T}}
\newcommand{\CONTABLE}{\var{C-on-T}}
\newcommand{\CLEARA}{\var{A-clear}}
\newcommand{\CLEARB}{\var{B-clear}}
\newcommand{\CLEARC}{\var{C-clear}}

\newcommand{\applyop}[2]{\textit{app}_{#1}(#2)}
\newcommand{\applyops}[2]{\textit{app}_{#1}(#2)}
\newcommand{\applyplan}[2]{\textit{app}_{#1}(#2)}
\newcommand{\regr}[2]{\textit{regr}_{#1}(#2)}
\newcommand{\regrstrips}[2]{\textit{sregr}_{#1}(#2)}
\newcommand{\eprecon}[2]{\textit{EPC}_{#1}(#2)}

\newcommand{\compl}[1]{\overline{#1}}

\newcommand{\changes}[2]{\lbrack #1\rbrack_{#2}}
\newcommand{\CEF}{\vartriangleright}

\newcommand{\relaxation}[1]{{#1}^+}
\newcommand{\onset}[1]{\textit{on}(#1)}
\newcommand{\hplus}{\ensuremath{h^+}}
\newcommand{\hmax}{\ensuremath{h_{\text{max}}}}
\newcommand{\hadd}{\ensuremath{h_{\text{add}}}}
\newcommand{\hlmcut}{\ensuremath{h_{\text{LM-cut}}}}
\newcommand{\hff}{\ensuremath{h_{\text{FF}}}}
\newcommand{\hcs}{\ensuremath{h_{\text{cs}}}}
\newcommand{\hsa}{\ensuremath{h_{\text{sa}}}}
\newcommand{\hhhh}{\ensuremath{h_{\text{HHH}}}}

\newcommand{\sasplus}{SAS${}^+$}

\newcommand{\graphequiv}{\stackrel{\textup{G}}{\sim}}

\newcommand{\cg}{\ensuremath{\textit{CG}}}

%% Turing machines and complexity theory
\newcommand{\accept}{{\textsf{Y}}}

\newcommand{\easier}{\ensuremath{\le_{\text{p}}}}

\newcommand{\decisionclass}[1]{\ensuremath{\textsf{\textup{#1}}}}
\newcommand{\dtime}{\decisionclass{DTIME}}
\newcommand{\ntime}{\decisionclass{NTIME}}
\newcommand{\dspace}{\decisionclass{DSPACE}}
\newcommand{\nspace}{\decisionclass{NSPACE}}

\newcommand{\ptime}{\decisionclass{P}}
\newcommand{\np}{\decisionclass{NP}}
\newcommand{\pspace}{\decisionclass{PSPACE}}
\newcommand{\npspace}{\decisionclass{NPSPACE}}
\newcommand{\exptime}{\decisionclass{EXP}}
\newcommand{\nexptime}{\decisionclass{NEXP}}
\newcommand{\expspace}{\decisionclass{EXPSPACE}}
\newcommand{\dblexptime}{\decisionclass{2EXP}}

\newcommand{\planex}{\textsc{PlanEx}}
\newcommand{\planlen}{\textsc{PlanLen}}

\newcommand{\ms}[1]{\mathchoice{\mbox{\normalsize\it #1}}{\mbox{\normalsize\it #1}}{\mbox{\scriptsize\it #1}}{\mbox{\tiny\it #1}}}
\newcommand{\image}[2]{\ms{img}_{#1}(#2)}
\newcommand{\spreimage}[2]{\ms{spreimg}_{#1}(#2)}
\newcommand{\wpreimage}[2]{\ms{wpreimg}_{#1}(#2)}
\newcommand{\disbwd}[2]{\delta^{\ms{bwd}}_{#1}(#2)}
\newcommand{\ats}[1]{\ms{scope}(#1)}
\newcommand{\ndopcpc}[2]{\tau^{\ms{nd}}_{#1}(#2)}
\newcommand{\choice}{\mathop{|}}
\newcommand{\opcpc}[2]{\tau_{#1}(#2)}

%% Don't use \begin{proof}/\end{proof} for multi-part proofs because
%% this places a QED symbol.
\newenvironment{proofstart}{\begin{block}{Proof.}}{\end{block}}
\newenvironment{proofmid}{\begin{block}{Proof (ctd.)}}{\end{block}}
\newenvironment{proofend}{\begin{proof}[Proof (ctd.)]}{\end{proof}}

\newtheorem{remark}[theorem]{Remark}

%% Macros to align the width of things.
\newlength{\mywidth}
\newcommand{\setmywidth}[1]{\settowidth{\mywidth}{#1}}
\newcommand{\usemywidth}[1]{\makebox[\mywidth][l]{#1}}
\newcommand{\usemywidthmath}[1]{\usemywidth{\ensuremath{#1}}}

%% A tighter "align" environment.
\newenvironment{tightalign}[1][c]{\par\(\begin{array}[#1]{@{}r@{}l}}
  {\end{array}\)\par}
\newenvironment{wrappedmath}[1][t]{\begin{array}[#1]{@{}l}}{\end{array}}

%% Nondeterministic transition systems with connector arcs
\newcommand{\arc}[4]{
  \begin{pgfscope}
    \pgfsetlinewidth{1pt}
    \pgfpathmoveto{\pgfpointanchor{#3}{center}}
    \pgfpathlineto{\pgfpointanchor{#1}{#2}}
    \pgfpathlineto{\pgfpointanchor{#4}{center}}
    \pgfusepath{clip}
    \pgfpathcircle{\pgfpointanchor{#1}{#2}}{1.8mm}
    \pgfusepath{draw}
  \end{pgfscope}
}

\newcommand{\drawblock}[5]{% PARAMETERS: COLOR, CORNERCOORDS, SIZE
% TOP SIDE
\color{#1!55}
\pgfmoveto{\pgfrelative{\pgfxyz(#2,#3,#4)}{\pgfxyz(0,#5,0)}}
\pgflineto{\pgfrelative{\pgfxyz(#2,#3,#4)}{\pgfxyz(#5,#5,0)}}
\pgflineto{\pgfrelative{\pgfxyz(#2,#3,#4)}{\pgfxyz(#5,#5,#5)}}
\pgflineto{\pgfrelative{\pgfxyz(#2,#3,#4)}{\pgfxyz(0,#5,#5)}}
\pgflineto{\pgfrelative{\pgfxyz(#2,#3,#4)}{\pgfxyz(0,#5,0)}}
\pgffill
\color{black}
\pgfmoveto{\pgfrelative{\pgfxyz(#2,#3,#4)}{\pgfxyz(0,#5,0)}}
\pgflineto{\pgfrelative{\pgfxyz(#2,#3,#4)}{\pgfxyz(#5,#5,0)}}
\pgflineto{\pgfrelative{\pgfxyz(#2,#3,#4)}{\pgfxyz(#5,#5,#5)}}
\pgflineto{\pgfrelative{\pgfxyz(#2,#3,#4)}{\pgfxyz(0,#5,#5)}}
\pgflineto{\pgfrelative{\pgfxyz(#2,#3,#4)}{\pgfxyz(0,#5,0)}}
\pgfstroke
% RIGHT SIDE
\color{#1!65}
\pgfmoveto{\pgfrelative{\pgfxyz(#2,#3,#4)}{\pgfxyz(#5,0,0)}}
\pgflineto{\pgfrelative{\pgfxyz(#2,#3,#4)}{\pgfxyz(#5,#5,0)}}
\pgflineto{\pgfrelative{\pgfxyz(#2,#3,#4)}{\pgfxyz(#5,#5,#5)}}
\pgflineto{\pgfrelative{\pgfxyz(#2,#3,#4)}{\pgfxyz(#5,0,#5)}}
\pgflineto{\pgfrelative{\pgfxyz(#2,#3,#4)}{\pgfxyz(#5,0,0)}}
\pgffill
\color{black}
\pgfmoveto{\pgfrelative{\pgfxyz(#2,#3,#4)}{\pgfxyz(#5,0,0)}}
\pgflineto{\pgfrelative{\pgfxyz(#2,#3,#4)}{\pgfxyz(#5,#5,0)}}
\pgflineto{\pgfrelative{\pgfxyz(#2,#3,#4)}{\pgfxyz(#5,#5,#5)}}
\pgflineto{\pgfrelative{\pgfxyz(#2,#3,#4)}{\pgfxyz(#5,0,#5)}}
\pgflineto{\pgfrelative{\pgfxyz(#2,#3,#4)}{\pgfxyz(#5,0,0)}}
\pgfstroke
%FRONT
\color{#1!99}
\pgfmoveto{\pgfxyz(#2,#3,#4)}
\pgflineto{\pgfrelative{\pgfxyz(#2,#3,#4)}{\pgfxyz(#5,0,0)}}
\pgflineto{\pgfrelative{\pgfxyz(#2,#3,#4)}{\pgfxyz(#5,#5,0)}}
\pgflineto{\pgfrelative{\pgfxyz(#2,#3,#4)}{\pgfxyz(0,#5,0)}}
\pgflineto{\pgfxyz(#2,#3,#4)}
\pgffill
\color{black}
\pgfmoveto{\pgfxyz(#2,#3,#4)}
\pgflineto{\pgfrelative{\pgfxyz(#2,#3,#4)}{\pgfxyz(#5,0,0)}}
\pgflineto{\pgfrelative{\pgfxyz(#2,#3,#4)}{\pgfxyz(#5,#5,0)}}
\pgflineto{\pgfrelative{\pgfxyz(#2,#3,#4)}{\pgfxyz(0,#5,0)}}
\pgflineto{\pgfxyz(#2,#3,#4)}
\pgfstroke
}

\newcommand{\RsGsB}{\begin{pgfpicture}{0.5mm}{0.5mm}{12mm}{5mm}
\pgfsetxvec{\pgfpoint{0.4cm}{0cm}}
\pgfsetyvec{\pgfpoint{0cm}{0.4cm}}
\pgfsetzvec{\pgfpoint{0.15cm}{0.15cm}}
\drawblock{red}{0}{0}{0}{1}
\drawblock{green}{1}{0}{0}{1}
\drawblock{blue}{2}{0}{0}{1}
\end{pgfpicture}
}

\newcommand{\twostacksA}[3]{\begin{pgfpicture}{0.5mm}{0.5mm}{0.8cm}{0.9cm}
\pgfsetxvec{\pgfpoint{0.4cm}{0cm}}
\pgfsetyvec{\pgfpoint{0cm}{0.4cm}}
\pgfsetzvec{\pgfpoint{0.15cm}{0.15cm}}
\drawblock{#2}{0}{0}{0}{1}
\drawblock{#1}{0}{1}{0}{1}
\drawblock{#3}{1}{0}{0}{1}
\end{pgfpicture}
}

\newcommand{\twostacksB}[3]{\begin{pgfpicture}{0.5mm}{0.5mm}{0.8cm}{0.9cm}
\pgfsetxvec{\pgfpoint{0.4cm}{0cm}}
\pgfsetyvec{\pgfpoint{0cm}{0.4cm}}
\pgfsetzvec{\pgfpoint{0.15cm}{0.15cm}}
\drawblock{#1}{0}{0}{0}{1}
\drawblock{#3}{1}{0}{0}{1}
\drawblock{#2}{1}{1}{0}{1}
\end{pgfpicture}
}

\newcommand{\onestack}[3]{\begin{pgfpicture}{0.5mm}{0.5mm}{4mm}{13mm}
\pgfsetxvec{\pgfpoint{0.4cm}{0cm}}
\pgfsetyvec{\pgfpoint{0cm}{0.4cm}}
\pgfsetzvec{\pgfpoint{0.15cm}{0.15cm}}
\drawblock{#3}{0}{0}{0}{1}
\drawblock{#2}{0}{1}{0}{1}
\drawblock{#1}{0}{2}{0}{1}
\end{pgfpicture}
}

\newcommand{\RGsB}{\twostacksA{red}{green}{blue}}

\newcommand{\RBsG}{\twostacksB{green}{red}{blue}}
\newcommand{\RBsGr}{\twostacksA{red}{blue}{green}}

\newcommand{\BRsG}{\twostacksA{blue}{red}{green}}
\newcommand{\BRsGr}{\twostacksB{green}{blue}{red}}

\newcommand{\RsBG}{\twostacksB{red}{blue}{green}}

\newcommand{\GRsB}{\twostacksA{green}{red}{blue}}
\newcommand{\RsGB}{\twostacksB{red}{green}{blue}}

\newcommand{\RGB}{\onestack{red}{green}{blue}}
\newcommand{\RBG}{\onestack{red}{blue}{green}}
\newcommand{\GRB}{\onestack{green}{red}{blue}}
\newcommand{\GBR}{\onestack{green}{blue}{red}}
\newcommand{\BRG}{\onestack{blue}{red}{green}}
\newcommand{\BGR}{\onestack{blue}{green}{red}}


\begin{document}

\lectureno{3}
\subtitle{PDDL}
\date{October 28th, 2011}
\maketitles

%% TODO: In PDDL part, use alltt instead of verbatim and add 
%%       "syntax highlighting".
%% TODO: Prettify, e.g. get rid of [containsverbatim] and \frametitle.

\section{Schematic operators}
\subsection[Schemata]{Schematic operators}

\begin{frame}{Schematic operators}
  \begin{itemize}
  \item Description of state variables and operators in terms of a
    given finite \alert{set of objects}.
  \item Analogy: propositional logic vs.\ predicate logic
  \item Planners take input as schematic operators
    and translate them into (\alert{ground}) operators. This is called
    \alert{grounding}.
  \end{itemize}
\end{frame}

\begin{frame}{Schematic operators: example}
  Schematic operator \texttt{drive\_car\_from\_to($x$,$y_1$,$y_2$)}:
  \[
    \begin{array}{l}
      x \in \{ \val{car1},\val{car2}\}, \\
      y_1 \in \{ \val{Freiburg}, \val{Strasbourg}\}, \\
      y_2 \in \{ \val{Freiburg}, \val{Strasbourg}\}\\[\medskipamount]
      %%y_1 \neq y_2 \\[\medskipamount]
      \langle \var{in}(x, y_1), \var{in}(x, y_2) \land
      \neg \var{in}(x, y_1)\rangle
    \end{array}
  \]

  corresponds to the operators
  \[
    \begin{array}{l}
      \langle \var{in}(\val{car1}, \val{Freiburg}),
      \var{in}(\val{car1},\val{Strasbourg}) \land
      \neg \var{in}(\val{car1}, \val{Freiburg})\rangle, \\
      \langle \var{in}(\val{car1}, \val{Strasbourg}),
      \var{in}(\val{car1}, \val{Freiburg}) \land
      \neg \var{in}(\val{car1}, \val{Strasbourg})\rangle, \\
      \langle \var{in}(\val{car2}, \val{Freiburg}),
      \var{in}(\val{car2}, \val{Strasbourg}) \land
      \neg \var{in}(\val{car2}, \val{Freiburg})\rangle, \\
      \langle \var{in}(\val{car2}, \val{Strasbourg}),
      \var{in}(\val{car2}, \val{Freiburg}) \land
      \neg \var{in}(\val{car2}, \val{Strasbourg})\rangle,
    \end{array}
  \]
  plus four operators that are never applicable (inconsistent change set!)
  and can be ignored, like
  \[
  \langle \var{in}(\val{car1}, \val{Freiburg}),
  \var{in}(\val{car1},\val{Freiburg}) \land
  \neg \var{in}(\val{car1}, \val{Freiburg})\rangle.
  \]
\end{frame}

\begin{frame}{Schematic operators: quantification}

\begin{block}{Existential quantification (for formulae only)}
Finite disjunctions $\varphi(a_1)\lor\dots\lor\varphi(a_n)$
represented as $\exists x\in\{a_1,\dots,a_n\}:\varphi(x)$.
\end{block}

\begin{block}{Universal quantification (for formulae and effects)}
  Finite conjunctions $\varphi(a_1)\land\dots\land\varphi(a_n)$
  represented as $\forall x\in\{a_1,\dots,a_n\}:\varphi(x)$.
\end{block}

\begin{example}
$\exists x\in\{\val{A},\val{B},\val{C}\}: \var{in}(x,\val{Freiburg})$
is a short-hand for

$\var{in}(\val{A},\val{Freiburg})\lor\var{in}(\val{B},\val{Freiburg})\lor\var{in}(\val{C},\val{Freiburg}).$
\end{example}

\end{frame}

\section{PDDL}
\subsection{Overview}

\begin{frame}[containsverbatim]{PDDL: the Planning Domain Definition Language}

\begin{itemize}
\item used by almost all implemented systems for deterministic planning
\item supports a language comparable to what we have defined above
(including schematic operators and quantification)
\item syntax inspired by the Lisp programming language: e.g.\ prefix notation
for formulae
\begin{verbatim}
(and (or (on A B) (on A C))
     (or (on B A) (on B C))
     (or (on C A) (on A B)))
\end{verbatim}
\end{itemize}


\end{frame}

\subsection{Domain files}

\begin{frame}
\frametitle{PDDL: domain files}

A domain file consists of
\begin{itemize}
\item (define (domain DOMAINNAME)
\item a :requirements definition (use :strips :typing by default)
\item definitions of types (each parameter has a type)
\item definitions of predicates
\item definitions of operators
\end{itemize}

\end{frame}

\begin{frame}[containsverbatim]
\frametitle{Example: blocks world (with hand) in PDDL}

\begin{itemize}
\item \alert{Note:} Unlike in the previous chapter, here we use a
  variant of the blocks world domain with an explicitly modeled
  gripper/hand.
\end{itemize}

\begin{verbatim}
(define (domain BLOCKS)
  (:requirements :strips :typing)
  (:types block)
  (:predicates (on ?x - block ?y - block)
               (ontable ?x - block)
               (clear ?x - block)
               (handempty)
               (holding ?x - block)
               )
\end{verbatim}

\end{frame}

\begin{frame}[containsverbatim]
\frametitle{PDDL: operator definition}

\begin{itemize}
\item (:action OPERATORNAME
\item list of parameters: (?x - type1 ?y - type2 ?z - type3)
\item precondition: a formula
\begin{verbatim}
  <schematic-state-var>
  (and <formula> ... <formula>)
  (or <formula> ... <formula>)
  (not <formula>)
  (forall (?x1 - type1 ... ?xn - typen) <formula>)
  (exists (?x1 - type1 ... ?xn - typen) <formula>)
\end{verbatim}

\alert{Note:} Pyperplan only supports atoms and conjunctions of atoms.

\end{itemize}

\end{frame}

\begin{frame}[containsverbatim]

\begin{itemize}
\item effect:
\begin{verbatim}
  <schematic-state-var>
  (not <schematic-state-var>)
  (and <effect> ... <effect>)
  (when <formula> <effect>)
  (forall (?x1 - type1 ... ?xn - typen) <effect>)
\end{verbatim}

\alert{Note:} Pyperplan only supports literals and conjunctions of literals.

\end{itemize}

\end{frame}

\begin{frame}[containsverbatim]

\begin{verbatim}
(:action stack
    :parameters (?x - block ?y - block)
    :precondition (and (holding ?x) (clear ?y))
    :effect (and (not (holding ?x))
                 (not (clear ?y))
                 (clear ?x)
                 (handempty)
                 (on ?x ?y)))
\end{verbatim}

\end{frame}


\subsection{Problem files}

\begin{frame}
\frametitle{PDDL: problem files}

A problem file consists of
\begin{itemize}
\item (define (problem PROBLEMNAME)
\item declaration of which domain is needed for this problem
\item definitions of objects belonging to each type
\item definition of the initial state (list of state variables initially true)
\item definition of goal states (a formula like operator precondition)
\end{itemize}

\end{frame}

\begin{frame}[containsverbatim]

\begin{verbatim}
(define (problem example)
  (:domain BLOCKS)
  (:objects a b c d - block)
  (:init (clear a) (clear b) (clear c) (clear d)
         (ontable a) (ontable b) (ontable c)
         (ontable d) (handempty))
  (:goal (and (on d c) (on c b) (on b a)))
)
\end{verbatim}

\end{frame}

\subsection{Example}

\begin{frame}[containsverbatim]
\frametitle{Example run on the Pyperplan planner}
\small
\begin{verbatim}
# ./pyperplan.py blocks-dom.pddl blocks-prob.pddl
[...]
2011-10-27 22:29:21,326 INFO   Search start: example
2011-10-27 22:29:21,330 INFO   Goal reached. [...]
2011-10-27 22:29:21,330 INFO   114 Nodes expanded
2011-10-27 22:29:21,330 INFO   Search end: example
[...]
2011-10-27 22:29:21,331 INFO   Plan length: 6
[...]
\end{verbatim}

\end{frame}


\begin{frame}[containsverbatim]
\frametitle{Example plan found by the Pyperplan planner}

\begin{verbatim}
# cat blocks-prob.pddl.soln
(pick-up b)
(stack b a)
(pick-up c)
(stack c b)
(pick-up d)
(stack d c)
\end{verbatim}

\end{frame}


\begin{frame}[containsverbatim]
\frametitle{Example: blocks world in PDDL}

\begin{verbatim}
(define (domain BLOCKS)
  (:requirements :strips :typing)
  (:types block)
  (:predicates (on ?x - block ?y - block)
	       (ontable ?x - block)
	       (clear ?x - block)
	       (handempty)
	       (holding ?x - block)
	       )
\end{verbatim}

\end{frame}

\begin{frame}[containsverbatim]

\begin{verbatim}
(:action pick-up
    :parameters (?x - block)
    :precondition (and (clear ?x) (ontable ?x)
                       (handempty))
    :effect (and (not (ontable ?x))
                 (not (clear ?x))
                 (not (handempty))
                 (holding ?x)))
\end{verbatim}
\end{frame}

\begin{frame}[containsverbatim]

\begin{verbatim}
(:action put-down
    :parameters (?x - block)
    :precondition (holding ?x)
    :effect (and (not (holding ?x))
                 (clear ?x)
                 (handempty)
                 (ontable ?x)))
\end{verbatim}

\end{frame}

\begin{frame}[containsverbatim]
\begin{verbatim}
(:action stack
    :parameters (?x - block ?y - block)
    :precondition (and (holding ?x) (clear ?y))
    :effect (and (not (holding ?x))
                 (not (clear ?y))
                 (clear ?x)
                 (handempty)
                 (on ?x ?y)))
\end{verbatim}
\end{frame}

\begin{frame}[containsverbatim]
\begin{verbatim}
(:action unstack
    :parameters (?x - block ?y - block)
    :precondition (and (on ?x ?y) (clear ?x)
                       (handempty))
    :effect (and (holding ?x)
                 (clear ?y)
                 (not (clear ?x))
                 (not (handempty))
                 (not (on ?x ?y))))
\end{verbatim}
\end{frame}

\begin{frame}[containsverbatim]

\begin{verbatim}
(define (problem example)
  (:domain BLOCKS)
  (:objects a b c d - block)
  (:init (clear a) (clear b) (clear c) (clear d)
         (ontable a) (ontable b) (ontable c)
         (ontable d) (handempty))
  (:goal (and (on d c) (on c b) (on b a)))
)
\end{verbatim}
\end{frame}

\end{document}
