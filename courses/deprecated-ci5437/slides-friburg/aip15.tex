\documentclass{gkibeamer}

\usepackage{tikz}[2008/02/13]

\title[AI Planning]{Principles of AI Planning}
\author[B.~Nebel,\\R.~Mattm\"uller]{Bernhard Nebel and Robert Mattm\"uller}
\date[]{Winter Semester 2011/2012}

%% Use \alert instead of \textred where it makes sense.
%% (\textred makes sense when you *really* want the color to be red,
%% independent of the beamer style settings.)
\newcommand{\hilite}[1]{{\color{structure} #1}}
\newcommand{\textred}[1]{{\color{red} #1}}
\newcommand{\textblue}[1]{{\color{blue} #1}}
\newcommand{\textgreen}[1]{{\color{green} #1}}
\newcommand{\textbrown}[1]{{\color{brown} #1}}
\newcommand{\textviolet}[1]{{\color{violet} #1}}
\newcommand{\textbg}[1]{{\color{bg} #1}}

\newcommand{\eg}{e.\,g.}
\newcommand{\ie}{i.\,e.}

%% State variables and values for state variables
\newcommand{\var}[1]{\textit{#1}}
\newcommand{\val}[1]{\text{#1}}

%% Various stuff
\newcommand{\astar}{A${}^*$}
\newcommand{\idastar}{IDA${}^*$}

%% Blocksworld stuff
\newcommand{\AONB}{\var{A-on-B}}
\newcommand{\AONC}{\var{A-on-C}}
\newcommand{\AOND}{\var{A-on-D}}
\newcommand{\BONA}{\var{B-on-A}}
\newcommand{\BONC}{\var{B-on-C}}
\newcommand{\BOND}{\var{B-on-D}}
\newcommand{\CONA}{\var{C-on-A}}
\newcommand{\CONB}{\var{C-on-B}}
\newcommand{\COND}{\var{C-on-D}}
\newcommand{\DONA}{\var{D-on-A}}
\newcommand{\DONB}{\var{D-on-B}}
\newcommand{\DONC}{\var{D-on-C}}
\newcommand{\AONTABLE}{\var{A-on-T}}
\newcommand{\BONTABLE}{\var{B-on-T}}
\newcommand{\CONTABLE}{\var{C-on-T}}
\newcommand{\CLEARA}{\var{A-clear}}
\newcommand{\CLEARB}{\var{B-clear}}
\newcommand{\CLEARC}{\var{C-clear}}

\newcommand{\applyop}[2]{\textit{app}_{#1}(#2)}
\newcommand{\applyops}[2]{\textit{app}_{#1}(#2)}
\newcommand{\applyplan}[2]{\textit{app}_{#1}(#2)}
\newcommand{\regr}[2]{\textit{regr}_{#1}(#2)}
\newcommand{\regrstrips}[2]{\textit{sregr}_{#1}(#2)}
\newcommand{\eprecon}[2]{\textit{EPC}_{#1}(#2)}

\newcommand{\compl}[1]{\overline{#1}}

\newcommand{\changes}[2]{\lbrack #1\rbrack_{#2}}
\newcommand{\CEF}{\vartriangleright}

\newcommand{\relaxation}[1]{{#1}^+}
\newcommand{\onset}[1]{\textit{on}(#1)}
\newcommand{\hplus}{\ensuremath{h^+}}
\newcommand{\hmax}{\ensuremath{h_{\text{max}}}}
\newcommand{\hadd}{\ensuremath{h_{\text{add}}}}
\newcommand{\hlmcut}{\ensuremath{h_{\text{LM-cut}}}}
\newcommand{\hff}{\ensuremath{h_{\text{FF}}}}
\newcommand{\hcs}{\ensuremath{h_{\text{cs}}}}
\newcommand{\hsa}{\ensuremath{h_{\text{sa}}}}
\newcommand{\hhhh}{\ensuremath{h_{\text{HHH}}}}

\newcommand{\sasplus}{SAS${}^+$}

\newcommand{\graphequiv}{\stackrel{\textup{G}}{\sim}}

\newcommand{\cg}{\ensuremath{\textit{CG}}}

%% Turing machines and complexity theory
\newcommand{\accept}{{\textsf{Y}}}

\newcommand{\easier}{\ensuremath{\le_{\text{p}}}}

\newcommand{\decisionclass}[1]{\ensuremath{\textsf{\textup{#1}}}}
\newcommand{\dtime}{\decisionclass{DTIME}}
\newcommand{\ntime}{\decisionclass{NTIME}}
\newcommand{\dspace}{\decisionclass{DSPACE}}
\newcommand{\nspace}{\decisionclass{NSPACE}}

\newcommand{\ptime}{\decisionclass{P}}
\newcommand{\np}{\decisionclass{NP}}
\newcommand{\pspace}{\decisionclass{PSPACE}}
\newcommand{\npspace}{\decisionclass{NPSPACE}}
\newcommand{\exptime}{\decisionclass{EXP}}
\newcommand{\nexptime}{\decisionclass{NEXP}}
\newcommand{\expspace}{\decisionclass{EXPSPACE}}
\newcommand{\dblexptime}{\decisionclass{2EXP}}

\newcommand{\planex}{\textsc{PlanEx}}
\newcommand{\planlen}{\textsc{PlanLen}}

\newcommand{\ms}[1]{\mathchoice{\mbox{\normalsize\it #1}}{\mbox{\normalsize\it #1}}{\mbox{\scriptsize\it #1}}{\mbox{\tiny\it #1}}}
\newcommand{\image}[2]{\ms{img}_{#1}(#2)}
\newcommand{\spreimage}[2]{\ms{spreimg}_{#1}(#2)}
\newcommand{\wpreimage}[2]{\ms{wpreimg}_{#1}(#2)}
\newcommand{\disbwd}[2]{\delta^{\ms{bwd}}_{#1}(#2)}
\newcommand{\ats}[1]{\ms{scope}(#1)}
\newcommand{\ndopcpc}[2]{\tau^{\ms{nd}}_{#1}(#2)}
\newcommand{\choice}{\mathop{|}}
\newcommand{\opcpc}[2]{\tau_{#1}(#2)}

%% Don't use \begin{proof}/\end{proof} for multi-part proofs because
%% this places a QED symbol.
\newenvironment{proofstart}{\begin{block}{Proof.}}{\end{block}}
\newenvironment{proofmid}{\begin{block}{Proof (ctd.)}}{\end{block}}
\newenvironment{proofend}{\begin{proof}[Proof (ctd.)]}{\end{proof}}

\newtheorem{remark}[theorem]{Remark}

%% Macros to align the width of things.
\newlength{\mywidth}
\newcommand{\setmywidth}[1]{\settowidth{\mywidth}{#1}}
\newcommand{\usemywidth}[1]{\makebox[\mywidth][l]{#1}}
\newcommand{\usemywidthmath}[1]{\usemywidth{\ensuremath{#1}}}

%% A tighter "align" environment.
\newenvironment{tightalign}[1][c]{\par\(\begin{array}[#1]{@{}r@{}l}}
  {\end{array}\)\par}
\newenvironment{wrappedmath}[1][t]{\begin{array}[#1]{@{}l}}{\end{array}}

%% Nondeterministic transition systems with connector arcs
\newcommand{\arc}[4]{
  \begin{pgfscope}
    \pgfsetlinewidth{1pt}
    \pgfpathmoveto{\pgfpointanchor{#3}{center}}
    \pgfpathlineto{\pgfpointanchor{#1}{#2}}
    \pgfpathlineto{\pgfpointanchor{#4}{center}}
    \pgfusepath{clip}
    \pgfpathcircle{\pgfpointanchor{#1}{#2}}{1.8mm}
    \pgfusepath{draw}
  \end{pgfscope}
}


\begin{document}
\subtitle{13.~Computational complexity of classical planning}
\date{July 20th, 2010}
\maketitles

\tikzset{LabelStyle/.style = {sloped,above}}

\newcommand{\Navigation}[2]{
  \definecolor{Safe}{rgb}{1,1,1}
  \definecolor{HighestProb}{gray}{0.5}
  \definecolor{HighProb}{gray}{0.6}
  \definecolor{MediumProb}{gray}{0.7}
  \definecolor{LowProb}{gray}{0.8}
  \definecolor{VeryLowProb}{gray}{0.9}

  \node[fill=Safe,minimum size=1cm, inner sep=0, outer sep=0]
  at (0.5,0.5) {}; 
  \node[fill=Safe,minimum size=1cm, inner sep=0, outer sep=0]
  at (1.5,0.5) {}; 
  \node[fill=Safe,minimum size=1cm, inner sep=0, outer sep=0]
  at (2.5,0.5) {}; 
  \node[fill=Safe,minimum size=1cm, inner sep=0, outer sep=0]
  at (3.5,0.5) {}; 
  \node[fill=Safe,minimum size=1cm, inner sep=0, outer sep=0]
  at (4.5,0.5) {}; 
  \node[fill=Safe,minimum size=1cm, inner sep=0, outer sep=0]
  at (5.5,0.5) {};

  \node[fill=Safe,minimum size=1cm, inner sep=0, outer sep=0]
  at (0.5,1.5) {}; 
  \node[fill=VeryLowProb,minimum size=1cm, inner sep=0, outer sep=0]
  at (1.5,1.5) {}; 
  \node[fill=LowProb,minimum size=1cm, inner sep=0, outer sep=0]
  at (2.5,1.5) {}; 
  \node[fill=MediumProb,minimum size=1cm, inner sep=0, outer sep=0]
  at (3.5,1.5) {}; 
  \node[fill=HighProb,minimum size=1cm, inner sep=0, outer sep=0]
  at (4.5,1.5) {}; 
  \node[fill=HighestProb,minimum size=1cm, inner sep=0, outer sep=0]
  at (5.5,1.5) {}; 

  \node[fill=Safe,minimum size=1cm, inner sep=0, outer sep=0]
  at (0.5,2.5) {}; 
  \node[fill=VeryLowProb,minimum size=1cm, inner sep=0, outer sep=0]
  at (1.5,2.5) {}; 
  \node[fill=LowProb,minimum size=1cm, inner sep=0, outer sep=0]
  at (2.5,2.5) {}; 
  \node[fill=MediumProb,minimum size=1cm, inner sep=0, outer sep=0]
  at (3.5,2.5) {}; 
  \node[fill=HighProb,minimum size=1cm, inner sep=0, outer sep=0]
  at (4.5,2.5) {}; 
  \node[fill=HighestProb,minimum size=1cm, inner sep=0, outer sep=0]
  at (5.5,2.5) {};

  \node[fill=Safe,minimum size=1cm, inner sep=0, outer sep=0]
  at (0.5,3.5) {}; 
  \node[fill=VeryLowProb,minimum size=1cm, inner sep=0, outer sep=0]
  at (1.5,3.5) {}; 
  \node[fill=LowProb,minimum size=1cm, inner sep=0, outer sep=0]
  at (2.5,3.5) {}; 
  \node[fill=MediumProb,minimum size=1cm, inner sep=0, outer sep=0]
  at (3.5,3.5) {}; 
  \node[fill=HighProb,minimum size=1cm, inner sep=0, outer sep=0]
  at (4.5,3.5) {}; 
  \node[fill=HighestProb,minimum size=1cm, inner sep=0, outer sep=0]
  at (5.5,3.5) {}; 
  
  \node[fill=Safe,minimum size=1cm, inner sep=0, outer sep=0]
  at (0.5,4.5) {}; 
  \node[fill=Safe,minimum size=1cm, inner sep=0, outer sep=0]
  at (1.5,4.5) {}; 
  \node[fill=Safe,minimum size=1cm, inner sep=0, outer sep=0]
  at (2.5,4.5) {}; 
  \node[fill=Safe,minimum size=1cm, inner sep=0, outer sep=0]
  at (3.5,4.5) {}; 
  \node[fill=Safe,minimum size=1cm, inner sep=0, outer sep=0]
  at (4.5,4.5) {}; 
  \node[fill=Safe,minimum size=1cm, inner sep=0, outer sep=0]
  at (5.5,4.5) {};


  \node[fill=Safe,minimum height=0.5cm, minimum width = 1cm, inner sep=0, outer sep=0]
  at (7,1.25) {\footnotesize$0.0$}; 
  \node[fill=VeryLowProb,minimum height=0.5cm, minimum width = 1cm, inner sep=0, outer sep=0]
  at (7,1.75) {\footnotesize$0.05$};
  \node[fill=LowProb,minimum height=0.5cm, minimum width = 1cm, inner sep=0, outer sep=0]
  at (7,2.25) {\footnotesize$0.1$};
  \node[fill=MediumProb,minimum height=0.5cm, minimum width = 1cm, inner sep=0, outer sep=0]
  at (7,2.75) {\footnotesize$0.15$};
  \node[fill=HighProb,minimum height=0.5cm, minimum width = 1cm, inner sep=0, outer sep=0]
  at (7,3.25) {\footnotesize$0.2$};
  \node[fill=HighestProb,minimum height=0.5cm, minimum width = 1cm, inner sep=0, outer sep=0]
  at (7,3.75) {\footnotesize$0.25$};

  \draw [color=black] (6.5,1) -- (6.5,4);
  \draw [color=black] (7.5,1) -- (7.5,4);

  \draw [color=black] (6.5,1) -- (7.5,1);
  \draw [color=black] (6.5,1.5) -- (7.5,1.5);
  \draw [color=black] (6.5,2) -- (7.5,2);
  \draw [color=black] (6.5,2.5) -- (7.5,2.5);
  \draw [color=black] (6.5,3) -- (7.5,3);
  \draw [color=black] (6.5,3.5) -- (7.5,3.5);
  \draw [color=black] (6.5,4) -- (7.5,4);

  \draw (0,0) grid (6,5);


  \node[draw, shape=circle, minimum size=0.6cm, inner sep=0, outer
  sep=0, color=red, line width=1pt] at (#1+0.5,#2+0.5) {};
  
  \node[shape=circle, minimum size=0.8cm, inner sep=0, outer
  sep=0, color=red]  at (5.5,4.5) {X};
}

\section*{Motivation}
\begin{frame}{Probabilistic Planning}
  \begin{itemize}
    \item The world is not predictable, but sometimes we can estimate
    \alert{how likely} unpredictable events happen.
    \item Example from AI robotics:
    \begin{itemize}
      \item robot navigation in unknown terrain
      \item supported by satellite maps
      \item traversability of terrain can only be estimated
    \end{itemize}
    \item Use \alert{probabilities} to describe nondeterministic
    effects.
    \item Different \alert{objectives} (that lead to different
    algorithms):
    \begin{itemize}
      \item Reach a goal state with highest possible probability. 
      \item Reach a goal state with smallest possible \alert{expected
        cost}.
      \item Gain highest possible \alert{reward} (over a finite
      \alert{horizon} or with a \alert{discount factor} in the infinite
      case).
    \end{itemize}
  \end{itemize}
\end{frame}

\begin{frame}{Example: Navigation}
  \begin{center}
    \begin{tikzpicture}
      \Navigation{5}{0}
      \invisible{
        %Just that the picture won't jump
        \node[fill=Safe,inner sep=0, outer sep=0] at (8.1,0.35)
        {\tiny$r(s,a,s') = \begin{cases} 0 & \textnormal{if } s'
            \textnormal{ is goal}\\
            -1 & \textnormal{else}\end{cases}$};
      }
      \end{tikzpicture}
  \end{center}
\end{frame}

\begin{frame}{Objective: Reach Goal with highest possible probability}
  \begin{center}
    \begin{tikzpicture}
      \Navigation{5}{0}
      \draw[color=red, line width=1pt] (0.5,0.1) to (5.1,0.1);
      \draw[color=red, line width=1pt] (0.5,0.1) to node[LabelStyle]
      {\scriptsize$1.0$} (0.5,4.9);
      \draw[color=red, line width=1pt] (0.5,4.9) to (5.1,4.9);

      \draw[line width=1pt] (1.5,0.25) to (5.1,0.25);
      \draw[line width=1pt] (1.5,0.25) to node[LabelStyle]
      {\scriptsize$\approx 0.86$} (1.5,4.75);
      \draw[line width=1pt] (1.5,4.75) to (5.1,4.75);

      \draw[line width=1pt] (2.5,0.4) to (5.1,0.4);
      \draw[line width=1pt] (2.5,0.4) to node[LabelStyle]
      {\scriptsize$\approx 0.73$} (2.5,4.6);
      \draw[line width=1pt] (2.5,4.6) to (5.1,4.6);

      \draw[line width=1pt] (3.5,0.55) to (5.1,0.55);
      \draw[line width=1pt] (3.5,0.55) to node[LabelStyle]
      {\scriptsize$\approx 0.61$} (3.5,4.45);
      \draw[line width=1pt] (3.5,4.45) to (5.1,4.45);

      \draw[line width=1pt] (4.5,0.7) to (5.1,0.7);
      \draw[line width=1pt] (4.5,0.7) to node[LabelStyle]
      {\scriptsize$\approx 0.51$} (4.5,4.3);
      \draw[line width=1pt] (4.5,4.3) to (5.1,4.3);

      \draw[line width=1pt] (5.5,0.85) to node[LabelStyle]
      {\scriptsize$\approx 0.42$} (5.5,4.15);

      \invisible{
        %Just that the picture won't jump
        \node[fill=Safe,inner sep=0, outer sep=0] at (8.1,0.35)
        {\tiny$r(s,a,s') = \begin{cases} 0 & \textnormal{if } s'
            \textnormal{ is goal}\\
            -1 & \textnormal{else}\end{cases}$};
      }
    \end{tikzpicture}
  \end{center}
\end{frame}

\begin{frame}{Objective: Highest reward with finite horizon}
  \begin{center}
    \begin{tikzpicture}
      \Navigation{5}{0}
      \draw[line width=1pt] (0.5,0.1) to (5.1,0.1);
      \draw[line width=1pt] (0.5,0.1) to node[LabelStyle]
      {\scriptsize$-13$} (0.5,4.9);
      \draw[line width=1pt] (0.5,4.9) to (5.1,4.9);

      \draw[line width=1pt] (1.5,0.25) to (5.1,0.25);
      \draw[line width=1pt] (1.5,0.25) to node[LabelStyle]
      {\scriptsize$\approx -12.28$} (1.5,4.75);
      \draw[line width=1pt] (1.5,4.75) to (5.1,4.75);

      \draw[color=red, line width=1pt] (2.5,0.4) to (5.1,0.4);
      \draw[color=red, line width=1pt] (2.5,0.4) to node[LabelStyle]
      {\scriptsize$\approx -11.98$} (2.5,4.6);
      \draw[color=red, line width=1pt] (2.5,4.6) to (5.1,4.6);

      \draw[line width=1pt] (3.5,0.55) to (5.1,0.55);
      \draw[line width=1pt] (3.5,0.55) to node[LabelStyle]
      {\scriptsize$\approx -12.02$} (3.5,4.45);
      \draw[line width=1pt] (3.5,4.45) to (5.1,4.45);

      \draw[line width=1pt] (4.5,0.7) to (5.1,0.7);
      \draw[line width=1pt] (4.5,0.7) to node[LabelStyle]
      {\scriptsize$-12.32$} (4.5,4.3);
      \draw[line width=1pt] (4.5,4.3) to (5.1,4.3);

      \draw[line width=1pt] (5.5,0.85) to node[LabelStyle]
      {\scriptsize$\approx -12.83$} (5.5,4.15);

      \node[inner sep=0, outer sep=0] at (6.85,0.75)
        {\tiny$h = 20$};
      \node[inner sep=0, outer sep=0] at (8.1,0.35)
        {\tiny$r(s,a,s') = \begin{cases} 0 & \textnormal{if } s'
            \textnormal{ is goal}\\
            -1 & \textnormal{else}\end{cases}$};
    \end{tikzpicture}
  \end{center}
\end{frame}

\begin{frame}{Probabilistic Planning}
  \begin{itemize}
    \item In \alert{deterministic planning} we have assumed that the
    only changes taking place in the world are those caused by us and
    that we can \alert{exactly predict} the results of our actions.
    \item In \alert{nondeterministic planning} we extended this
    framework by formalizing other agents and processes beyond our
    control as \alert{nondeterminism}.
    \item If it is possible to assign \alert{probabilities} to
    nondeterministic effects we talk about \alert{probabilistic
      planning}.
  \end{itemize}
\end{frame}

\section{Transition systems and planning tasks}
\subsection{Transition systems}

\begin{frame}{Markov Decision Process}
  \begin{definition}[Markov Decision Process]
    A \alert{Markov Decision Process} (MDP) is a 4-tuple $\mathcal T =
    \langle S, L, P, R \rangle$ where
    \begin{itemize}
    \item $S$ is a finite set of \alert{states},
    \item $L$ is a finite set of (transition) \alert{labels},
    \item $P: S \times L \times S \mapsto [0;1]$ is the
    \alert{transition probability} % (where $\sum_{s' \in S} P(\langle
    % s,a,s' \rangle) = 1$ or $\sum_{s' \in S} P(\langle
    % s,a,s' \rangle) = 0$ for all $\langle s,a\rangle \in S \times A$)
    , and 
    \item $R: S \times L \times S \mapsto \mathbb{R}$ is the
    \alert{reward function}
    \end{itemize}
  \end{definition}
\end{frame}


\section{Basic Algorithms}
\subsection{Value Iteration}
\subsection{Policy Iteration}

\section{Heuristic Search Algorithms}
\subsection{Real-Time Dynamic Programming}

\end{document}
