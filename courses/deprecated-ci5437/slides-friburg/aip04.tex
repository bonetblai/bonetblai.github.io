\documentclass{gkibeamer}

\usepackage{tikz}[2008/02/13]

\title[AI Planning]{Principles of AI Planning}
\author[B.~Nebel,\\R.~Mattm\"uller]{Bernhard Nebel and Robert Mattm\"uller}
\date[]{Winter Semester 2011/2012}

%% Use \alert instead of \textred where it makes sense.
%% (\textred makes sense when you *really* want the color to be red,
%% independent of the beamer style settings.)
\newcommand{\hilite}[1]{{\color{structure} #1}}
\newcommand{\textred}[1]{{\color{red} #1}}
\newcommand{\textblue}[1]{{\color{blue} #1}}
\newcommand{\textgreen}[1]{{\color{green} #1}}
\newcommand{\textbrown}[1]{{\color{brown} #1}}
\newcommand{\textviolet}[1]{{\color{violet} #1}}
\newcommand{\textbg}[1]{{\color{bg} #1}}

\newcommand{\eg}{e.\,g.}
\newcommand{\ie}{i.\,e.}

%% State variables and values for state variables
\newcommand{\var}[1]{\textit{#1}}
\newcommand{\val}[1]{\text{#1}}

%% Various stuff
\newcommand{\astar}{A${}^*$}
\newcommand{\idastar}{IDA${}^*$}

%% Blocksworld stuff
\newcommand{\AONB}{\var{A-on-B}}
\newcommand{\AONC}{\var{A-on-C}}
\newcommand{\AOND}{\var{A-on-D}}
\newcommand{\BONA}{\var{B-on-A}}
\newcommand{\BONC}{\var{B-on-C}}
\newcommand{\BOND}{\var{B-on-D}}
\newcommand{\CONA}{\var{C-on-A}}
\newcommand{\CONB}{\var{C-on-B}}
\newcommand{\COND}{\var{C-on-D}}
\newcommand{\DONA}{\var{D-on-A}}
\newcommand{\DONB}{\var{D-on-B}}
\newcommand{\DONC}{\var{D-on-C}}
\newcommand{\AONTABLE}{\var{A-on-T}}
\newcommand{\BONTABLE}{\var{B-on-T}}
\newcommand{\CONTABLE}{\var{C-on-T}}
\newcommand{\CLEARA}{\var{A-clear}}
\newcommand{\CLEARB}{\var{B-clear}}
\newcommand{\CLEARC}{\var{C-clear}}

\newcommand{\applyop}[2]{\textit{app}_{#1}(#2)}
\newcommand{\applyops}[2]{\textit{app}_{#1}(#2)}
\newcommand{\applyplan}[2]{\textit{app}_{#1}(#2)}
\newcommand{\regr}[2]{\textit{regr}_{#1}(#2)}
\newcommand{\regrstrips}[2]{\textit{sregr}_{#1}(#2)}
\newcommand{\eprecon}[2]{\textit{EPC}_{#1}(#2)}

\newcommand{\compl}[1]{\overline{#1}}

\newcommand{\changes}[2]{\lbrack #1\rbrack_{#2}}
\newcommand{\CEF}{\vartriangleright}

\newcommand{\relaxation}[1]{{#1}^+}
\newcommand{\onset}[1]{\textit{on}(#1)}
\newcommand{\hplus}{\ensuremath{h^+}}
\newcommand{\hmax}{\ensuremath{h_{\text{max}}}}
\newcommand{\hadd}{\ensuremath{h_{\text{add}}}}
\newcommand{\hlmcut}{\ensuremath{h_{\text{LM-cut}}}}
\newcommand{\hff}{\ensuremath{h_{\text{FF}}}}
\newcommand{\hcs}{\ensuremath{h_{\text{cs}}}}
\newcommand{\hsa}{\ensuremath{h_{\text{sa}}}}
\newcommand{\hhhh}{\ensuremath{h_{\text{HHH}}}}

\newcommand{\sasplus}{SAS${}^+$}

\newcommand{\graphequiv}{\stackrel{\textup{G}}{\sim}}

\newcommand{\cg}{\ensuremath{\textit{CG}}}

%% Turing machines and complexity theory
\newcommand{\accept}{{\textsf{Y}}}

\newcommand{\easier}{\ensuremath{\le_{\text{p}}}}

\newcommand{\decisionclass}[1]{\ensuremath{\textsf{\textup{#1}}}}
\newcommand{\dtime}{\decisionclass{DTIME}}
\newcommand{\ntime}{\decisionclass{NTIME}}
\newcommand{\dspace}{\decisionclass{DSPACE}}
\newcommand{\nspace}{\decisionclass{NSPACE}}

\newcommand{\ptime}{\decisionclass{P}}
\newcommand{\np}{\decisionclass{NP}}
\newcommand{\pspace}{\decisionclass{PSPACE}}
\newcommand{\npspace}{\decisionclass{NPSPACE}}
\newcommand{\exptime}{\decisionclass{EXP}}
\newcommand{\nexptime}{\decisionclass{NEXP}}
\newcommand{\expspace}{\decisionclass{EXPSPACE}}
\newcommand{\dblexptime}{\decisionclass{2EXP}}

\newcommand{\planex}{\textsc{PlanEx}}
\newcommand{\planlen}{\textsc{PlanLen}}

\newcommand{\ms}[1]{\mathchoice{\mbox{\normalsize\it #1}}{\mbox{\normalsize\it #1}}{\mbox{\scriptsize\it #1}}{\mbox{\tiny\it #1}}}
\newcommand{\image}[2]{\ms{img}_{#1}(#2)}
\newcommand{\spreimage}[2]{\ms{spreimg}_{#1}(#2)}
\newcommand{\wpreimage}[2]{\ms{wpreimg}_{#1}(#2)}
\newcommand{\disbwd}[2]{\delta^{\ms{bwd}}_{#1}(#2)}
\newcommand{\ats}[1]{\ms{scope}(#1)}
\newcommand{\ndopcpc}[2]{\tau^{\ms{nd}}_{#1}(#2)}
\newcommand{\choice}{\mathop{|}}
\newcommand{\opcpc}[2]{\tau_{#1}(#2)}

%% Don't use \begin{proof}/\end{proof} for multi-part proofs because
%% this places a QED symbol.
\newenvironment{proofstart}{\begin{block}{Proof.}}{\end{block}}
\newenvironment{proofmid}{\begin{block}{Proof (ctd.)}}{\end{block}}
\newenvironment{proofend}{\begin{proof}[Proof (ctd.)]}{\end{proof}}

\newtheorem{remark}[theorem]{Remark}

%% Macros to align the width of things.
\newlength{\mywidth}
\newcommand{\setmywidth}[1]{\settowidth{\mywidth}{#1}}
\newcommand{\usemywidth}[1]{\makebox[\mywidth][l]{#1}}
\newcommand{\usemywidthmath}[1]{\usemywidth{\ensuremath{#1}}}

%% A tighter "align" environment.
\newenvironment{tightalign}[1][c]{\par\(\begin{array}[#1]{@{}r@{}l}}
  {\end{array}\)\par}
\newenvironment{wrappedmath}[1][t]{\begin{array}[#1]{@{}l}}{\end{array}}

%% Nondeterministic transition systems with connector arcs
\newcommand{\arc}[4]{
  \begin{pgfscope}
    \pgfsetlinewidth{1pt}
    \pgfpathmoveto{\pgfpointanchor{#3}{center}}
    \pgfpathlineto{\pgfpointanchor{#1}{#2}}
    \pgfpathlineto{\pgfpointanchor{#4}{center}}
    \pgfusepath{clip}
    \pgfpathcircle{\pgfpointanchor{#1}{#2}}{1.8mm}
    \pgfusepath{draw}
  \end{pgfscope}
}

\input{blocksworld}

\begin{document}

\lectureno{4}
\subtitle{Normal forms}
\date{November 4th, 2011}
\maketitles


\section*{Motivation}
\begin{frame}{Motivation}
  Similarly to normal forms in propositional logic (DNF, CNF, NNF,
  \dots) we can define \alert{normal forms for effects, operators 
  and planning tasks}.
  \bigskip

  This is useful because algorithms (and proofs) then only need to deal
  with effects (resp. operators or tasks) in normal form.
\end{frame}

\section[Effect normal form]{Effect normal form}
\subsection[Equivalences]{Equivalence of operators and effects}

\begin{frame}{Equivalence of operators and effects}
  \begin{definition}[equivalent effects]
    Two effects $e$ and $e'$ over state variables $A$ are
    \alert{equivalent}, written \alert{$e \equiv e'$}, if
    for all states $s$ over $A$, $\changes{e}{s} = \changes{e'}{s}$.
  \end{definition}

  \begin{definition}[equivalent operators]
    Two operators $o$ and $o'$ over state variables $A$ are
    \alert{equivalent}, written \alert{$o \equiv o'$}, if
    they are applicable in the same states, and for all states
    $s$ where they are applicable, $\applyop{o}{s} = \applyop{o'}{s}$.
  \end{definition}

  \begin{theorem}
    Let $o = \langle \chi, e\rangle$ and $o' = \langle \chi', e'\rangle$ be
    operators with $\chi \equiv \chi'$ and $e \equiv e'$. Then
    $o \equiv o'$.
  \end{theorem}
  \hilite{Note:} The converse is not true. (Why not?)
  %% Equivalent operators must have equivalent preconditions, but not
  %% equivalent effects: for example, [e]_s may differ from [e']_s on
  %% states s where the precondition is unsatisfied. As a trivial
  %% example, consider operators with unsatisfiable preconditions,
  %% where the effect specification doesn't matter at all. (As another
  %% difference, for literals implied by the precondition, it doesn't
  %% matter whether they're listed in the effects or not.)
\end{frame}

\begin{frame}{Equivalence transformations for effects}
  \begin{eqnarray}
    e_1 \land e_2 & \equiv & e_2 \land e_1 \\
    (e_1 \land e_2) \land e_3 & \equiv & e_1 \land (e_2 \land e_3) \\
    \top \land e & \equiv & e \\
    \chi \CEF e & \equiv & \chi' \CEF e \qquad \text{if~} \chi \equiv \chi' \\
    \top \CEF e & \equiv & e \\
    \bot \CEF e & \equiv & \top \\
    \chi_1 \CEF (\chi_2 \CEF e) & \equiv & (\chi_1 \land \chi_2) \CEF e \\
    \chi \CEF (e_1 \land \dots \land e_n) & \equiv &
    (\chi \CEF e_1) \land \dots \land (\chi \CEF e_n) \\
    (\chi_1 \CEF e) \land (\chi_2 \CEF e) & \equiv & (\chi_1 \lor \chi_2) \CEF e
  \end{eqnarray}
\end{frame}

\subsection{Definition}
\begin{frame}{Normal form for effects}
  We can define a \alert{normal form for effects}:

  \begin{itemize}
  \item Nesting of conditionals, as in $a \CEF (b \CEF c)$, can be
    eliminated.
  \item Effects $e$ within a conditional effect $\varphi \CEF e$ can be
    restricted to atomic effects ($a$ or $\neg a$).
  \end{itemize}
  %% TODO: Jan findet die Formulierung oben nicht so klar.
  %% Gegenvorschlag: statt ``can be'' aktiver? "Nesting ... is
  %% forbidden." "Within conditional effects ..., e *must be* an
  %% atomic effect."?

  Transformation to this effect normal form only gives a small polynomial size
  increase. 
  \medskip

  \hilite{Compare:} transformation to CNF or DNF may increase formula
  size exponentially.
\end{frame}

\begin{frame}{Normal form for operators and effects}
  \begin{definition}
    An operator $\langle \chi,e\rangle$ is in \alert{effect normal form (ENF)}
    if for all occurrences of $\chi'\CEF e'$ in $e$ the effect $e'$ is either $a$
    or $\neg a$ for some $a\in A$, and there is at most one occurrence of any
    atomic effect in $e$.
  \end{definition}
  %% TODO: Jan findet diese ``at most one''-Formulierung ungenau.

  \begin{theorem}
    For every operator there is an equivalent one in effect normal form.
  \end{theorem}

  Proof is constructive: we can transform any operator into effect
  normal form using the equivalence transformations for effects.
\end{frame}

\subsection{Example}
\begin{frame}{Effect normal form example}
  \begin{example}
    \[
      \begin{array}{r@{\,}c@{\,}l}
        (a & \CEF & (b\land{} \\
        && (c\CEF(\neg d\land e))))\land{} \\
        (\neg b & \CEF & e)
      \end{array}
      \]
    transformed to effect normal form is
    \[
      \begin{array}{r@{\,}c@{\,}l}
        (a & \CEF & b)\land{} \\
        ((a\land c) & \CEF & \neg d)\land{} \\
        ((\neg b\lor(a\land c)) & \CEF & e)
      \end{array}
      \]
  \end{example}
\end{frame}

\section{Positive normal form}
\subsection{Motivation}

\begin{frame}{Example: Freecell}
\begin{center}
    \includegraphics[width=6cm]{figures/freecell}
\end{center}
 \begin{example}[good and bad effects]
   If we move a card $c$ to a free tableau position, the \alert{good
     effect} is that the card formerly below $c$ is now available.

   The \alert{bad effect} is that we lose one free tableau position.
 \end{example}
\end{frame}

\begin{frame}{What is a good or bad effect?}
  \hilite{Question:} Which operator effects are good, and which are
  bad?

  \medskip
  
  Difficult to answer in general, because it depends on
  context:
  \begin{itemize}
  \item Locking the entrance door is \alert{good} if we want to keep
    burglars out.
  \item Locking the entrance door is \alert{bad} if we want to enter.
  \end{itemize}

  We will now consider a reformulation of planning tasks that makes
  the distinction between good and bad effects obvious.
\end{frame}

\subsection{Definition \& algorithm}

\begin{frame}{Positive normal form}
  \begin{definition}[operators in positive normal form]
    An operator $o = \langle \chi, e \rangle$ is in \alert{positive normal
      form} if it is in effect normal form, no negation symbols appear in
      $\chi$, and no negation symbols appear in any effect condition in $e$.
  \end{definition}

  \begin{definition}[planning tasks in positive normal form]
    A planning task $\langle A,I,O,\gamma\rangle$ is in \alert{positive
      normal form} if all operators in $O$ are in positive normal form
    and no negation symbols occur in the goal $\gamma$.
  \end{definition}
\end{frame}

\begin{frame}{Positive normal form: existence}
  \begin{theorem}[positive normal form]
    Every planning task $\Pi$ has an equivalent planning task $\Pi'$
    in positive normal form.

    Moreover, $\Pi'$ can be computed from $\Pi$ in polynomial time.
  \end{theorem}

  \hilite{Note:} Equivalence here means that the represented
  transition systems of $\Pi$ and $\Pi'$, limited to the states that
  can be reached from the initial state, are isomorphic.

  \bigskip

  We prove the theorem by describing a suitable algorithm. \\
  (However, we do not prove its correctness or complexity.)
\end{frame}

\begin{frame}{Positive normal form: algorithm}
 \begin{block}{Transformation of $\langle A, I, O, \gamma\rangle$ to
     positive normal form}
   Convert all operators $o \in O$ to effect normal form. \\
   Convert all conditions to negation normal form (NNF). \\
   \textbf{while} any condition contains a negative literal $\neg a$:
   \\
   {}\qquad Let $a$ be a variable which occurs negatively in a
   condition. \\
   {}\qquad $A := A \cup \{\hat{a}\}$ \text{~for some new state
     variable $\hat{a}$} \\
   {}\qquad $I(\hat{a}) := 1 - I(a)$ \\
   {}\qquad Replace the effect $a$ by
   $(a \land \neg \hat{a})$ in all operators $o \in O$. \\
   {}\qquad Replace the effect $\neg a$ by
   $(\neg a \land \hat{a})$ in all operators $o \in O$. \\
   {}\qquad Replace $\neg a$ by $\hat{a}$ in all conditions. \\
   Convert all operators $o \in O$ to effect normal form (again).
 \end{block}

 Here, \emph{all conditions} refers to all operator preconditions,
 operator effect conditions and the goal.

 %% Note: It appears that conversion to normal form at the start isn't
 %% actually necessary; just converting the conditions to NNF should
 %% be good enough even without normal form. So the algorithm can be
 %% simplified by only converting to normal form at the end.
 %%
 %% It is not clear if we actually need normal form for anything, so
 %% alternatively, we might define "positive form" to not require
 %% normal form, and optionally define "positive normal form" as the
 %% combination of positive form and normal form.
\end{frame}

\subsection{Example}

\begin{frame}{Positive normal form: example}
 \begin{example}[transformation to positive normal form]
   \setmywidth{${} = \{\, \langle
     \var{bike} \land \var{bike-unlocked},
     \var{bike-locked} \land \neg \var{bike-unlocked}\rangle,$}
   \begin{align*}
     A & = \{\var{home}, \var{uni}, \var{lecture}, \var{bike},
     \var{bike-locked}
     \only<all:3-8>{\alert<all:3>{, \var{bike-unlocked}}}
     \} \\
     I & = \{\var{home} \mapsto 1, \var{bike} \mapsto 1,
     \var{bike-locked} \mapsto 1, \\
     & \phantom{ = \{\,} \var{uni} \mapsto 0, \var{lecture}
     \mapsto 0
     \only<all:3-8>{\alert<all:3>{, \var{bike-unlocked} \mapsto 0}}
     \} \\
     O & = \{
     \langle
     \var{home} \land \var{bike} \land
     \only<all:-6>{\alert<all:2,6>{\neg\var{bike-locked}}}
     \only<all:7-8>{\alert<all:7>{\var{bike-unlocked}}},
     \neg \var{home} \land \var{uni}
     \rangle, \\
     & \phantom{= \{\,}
     \langle
     \var{bike} \land \var{bike-locked},
     \alert<all:4-5>{\neg \var{bike-locked}}
     \only<all:5-8>{\alert<all:5>{\land \var{bike-unlocked}}}
     \rangle, \\
     & \phantom{ = \{\,}
     \langle
     \var{bike} \land
     \only<all:-6>{\alert<all:2,6>{\neg \var{bike-locked}}}
     \only<all:7-8>{\alert<all:7>{\var{bike-unlocked}}},
     \alert<all:4-5>{\var{bike-locked}}
     \only<all:5-8>{\alert<all:5>{\land \neg\var{bike-unlocked}}}
     \rangle, \\
     & \phantom{ = \{\,}
     \langle
     \var{uni},
     \var{lecture} \land
     ((\var{bike} \land
     \only<all:-6>{\alert<all:2,6>{\neg \var{bike-locked}}}
     \only<all:7-8>{\alert<all:7>{\var{bike-unlocked}}}
     )
     \CEF \neg\var{bike})
     \rangle
     \} \\
     \gamma & = \usemywidthmath{\var{lecture} \land \var{bike}}
   \end{align*}
 \end{example}
 \begin{overprint}
   \onslide<all:2>
   \alert{Identify state variable $a$ occurring negatively
     in conditions.}
   \onslide<all:3>
   \alert{Introduce new variable $\hat a$ with complementary initial
     value.}
   \onslide<all:4>
   \alert{Identify effects on variable $a$.}
   \onslide<all:5>
   \alert{Introduce complementary effects for $\hat a$.}
   \onslide<all:6>
   \alert{Identify negative conditions for $a$.}
   \onslide<all:7>
   \alert{Replace by positive condition $\hat a$.}
 \end{overprint}
\end{frame}

\subsection{Advantage}

\begin{frame}{Why positive normal form is interesting}
  In positive normal form, good and bad effects are easy to distinguish:
  \begin{itemize}
  \item Effects that make state variables true are good \\
    (\alert{add effects}).
  \item Effects that make state variables false are bad \\
    (\alert{delete effects}).
  \end{itemize}
  \bigskip

  This is of high relevance for some planning techniques that we will see 
  later in this course.
\end{frame}

\section[STRIPS]{STRIPS operators}
\subsection{Definition}
\begin{frame}{STRIPS operators}
  \begin{definition}
    An operator $\langle \chi,e\rangle$ is a \alert{STRIPS operator} if
    \begin{itemize}
    \item $\chi$ is a conjunction of atoms, and
    \item $e$ is a conjunction of atomic effects.
    \end{itemize}
  \end{definition}

  Hence every STRIPS operator is of the form
  \[
    \langle a_1\land\dots\land a_n,\ \ l_1\land\dots\land l_m\rangle
    \]
  where $a_i$ are atoms and $l_j$ are atomic effects.
  \medskip

  \hilite{Note:} Sometimes we allow conjunctions of \alert{literals}
    as preconditions. We denote this as \alert{STRIPS with negative
    preconditions}.
\end{frame}

\subsection{Properties}
\begin{frame}{Why STRIPS is interesting}
  \begin{itemize}
  \item STRIPS operators are \alert{particularly simple}, yet
    expressive enough to capture general planning problems.
  \item In particular, STRIPS planning is \alert{no easier} than
    general planning problems.
  \item Most algorithms in the planning literature are \alert{only
    presented for STRIPS operators} (generalization is often, but not
    always, obvious).
  \end{itemize}
  \begin{block}{STRIPS}
    STanford Research Institute Planning System \\
    (Fikes \& Nilsson, 1971)
  \end{block}
\end{frame}

\begin{frame}{Transformation to STRIPS}
  \begin{itemize}
  \item Not every operator is equivalent to a STRIPS operator.
  \item However, each operator can be transformed into a \alert{set}
    of STRIPS operators whose ``combination'' is equivalent to the
    original operator. (How?)
  \item However, this transformation may exponentially increase the
    number of required operators. There are planning tasks for which
    such a blow-up is unavoidable.
  \item There are polynomial transformations of planning tasks to
    STRIPS, but these do not preserve the structure of the transition
    system (\eg, length of shortest plans may change).
  \end{itemize}
\end{frame}

\section*{Summary}
\begin{frame}{Summary}
  \begin{itemize}
  \item \alert{Effect normal form} simplifies the structure of 
    the operator effects: conditional effects contain only atomic effects, and
    there is at most one occurrence of any atomic effect.
  \item \alert{Positive normal form} allows to distinguish good
    and bad effects.
  \item The form of \alert{STRIPS operators} is even more restrictive than
    effect normal form, forbidding complex preconditions and conditional
    effects.
  \item All three forms are expressive enough to capture general planning
    problems.
  \item Transformation to effect normal form and
    positive normal form can be done with a small polynomial size increase.
  \item Structure preserving transformations of planning tasks to 
    STRIPS can increase the number of operators exponentially.
  \end{itemize}
\end{frame}

\end{document}
