\documentclass{gkibeamer}

\usepackage{tikz}[2008/02/13]

\title[AI Planning]{Principles of AI Planning}
\author[B.~Nebel,\\R.~Mattm\"uller]{Bernhard Nebel and Robert Mattm\"uller}
\date[]{Winter Semester 2011/2012}

%% Use \alert instead of \textred where it makes sense.
%% (\textred makes sense when you *really* want the color to be red,
%% independent of the beamer style settings.)
\newcommand{\hilite}[1]{{\color{structure} #1}}
\newcommand{\textred}[1]{{\color{red} #1}}
\newcommand{\textblue}[1]{{\color{blue} #1}}
\newcommand{\textgreen}[1]{{\color{green} #1}}
\newcommand{\textbrown}[1]{{\color{brown} #1}}
\newcommand{\textviolet}[1]{{\color{violet} #1}}
\newcommand{\textbg}[1]{{\color{bg} #1}}

\newcommand{\eg}{e.\,g.}
\newcommand{\ie}{i.\,e.}

%% State variables and values for state variables
\newcommand{\var}[1]{\textit{#1}}
\newcommand{\val}[1]{\text{#1}}

%% Various stuff
\newcommand{\astar}{A${}^*$}
\newcommand{\idastar}{IDA${}^*$}

%% Blocksworld stuff
\newcommand{\AONB}{\var{A-on-B}}
\newcommand{\AONC}{\var{A-on-C}}
\newcommand{\AOND}{\var{A-on-D}}
\newcommand{\BONA}{\var{B-on-A}}
\newcommand{\BONC}{\var{B-on-C}}
\newcommand{\BOND}{\var{B-on-D}}
\newcommand{\CONA}{\var{C-on-A}}
\newcommand{\CONB}{\var{C-on-B}}
\newcommand{\COND}{\var{C-on-D}}
\newcommand{\DONA}{\var{D-on-A}}
\newcommand{\DONB}{\var{D-on-B}}
\newcommand{\DONC}{\var{D-on-C}}
\newcommand{\AONTABLE}{\var{A-on-T}}
\newcommand{\BONTABLE}{\var{B-on-T}}
\newcommand{\CONTABLE}{\var{C-on-T}}
\newcommand{\CLEARA}{\var{A-clear}}
\newcommand{\CLEARB}{\var{B-clear}}
\newcommand{\CLEARC}{\var{C-clear}}

\newcommand{\applyop}[2]{\textit{app}_{#1}(#2)}
\newcommand{\applyops}[2]{\textit{app}_{#1}(#2)}
\newcommand{\applyplan}[2]{\textit{app}_{#1}(#2)}
\newcommand{\regr}[2]{\textit{regr}_{#1}(#2)}
\newcommand{\regrstrips}[2]{\textit{sregr}_{#1}(#2)}
\newcommand{\eprecon}[2]{\textit{EPC}_{#1}(#2)}

\newcommand{\compl}[1]{\overline{#1}}

\newcommand{\changes}[2]{\lbrack #1\rbrack_{#2}}
\newcommand{\CEF}{\vartriangleright}

\newcommand{\relaxation}[1]{{#1}^+}
\newcommand{\onset}[1]{\textit{on}(#1)}
\newcommand{\hplus}{\ensuremath{h^+}}
\newcommand{\hmax}{\ensuremath{h_{\text{max}}}}
\newcommand{\hadd}{\ensuremath{h_{\text{add}}}}
\newcommand{\hlmcut}{\ensuremath{h_{\text{LM-cut}}}}
\newcommand{\hff}{\ensuremath{h_{\text{FF}}}}
\newcommand{\hcs}{\ensuremath{h_{\text{cs}}}}
\newcommand{\hsa}{\ensuremath{h_{\text{sa}}}}
\newcommand{\hhhh}{\ensuremath{h_{\text{HHH}}}}

\newcommand{\sasplus}{SAS${}^+$}

\newcommand{\graphequiv}{\stackrel{\textup{G}}{\sim}}

\newcommand{\cg}{\ensuremath{\textit{CG}}}

%% Turing machines and complexity theory
\newcommand{\accept}{{\textsf{Y}}}

\newcommand{\easier}{\ensuremath{\le_{\text{p}}}}

\newcommand{\decisionclass}[1]{\ensuremath{\textsf{\textup{#1}}}}
\newcommand{\dtime}{\decisionclass{DTIME}}
\newcommand{\ntime}{\decisionclass{NTIME}}
\newcommand{\dspace}{\decisionclass{DSPACE}}
\newcommand{\nspace}{\decisionclass{NSPACE}}

\newcommand{\ptime}{\decisionclass{P}}
\newcommand{\np}{\decisionclass{NP}}
\newcommand{\pspace}{\decisionclass{PSPACE}}
\newcommand{\npspace}{\decisionclass{NPSPACE}}
\newcommand{\exptime}{\decisionclass{EXP}}
\newcommand{\nexptime}{\decisionclass{NEXP}}
\newcommand{\expspace}{\decisionclass{EXPSPACE}}
\newcommand{\dblexptime}{\decisionclass{2EXP}}

\newcommand{\planex}{\textsc{PlanEx}}
\newcommand{\planlen}{\textsc{PlanLen}}

\newcommand{\ms}[1]{\mathchoice{\mbox{\normalsize\it #1}}{\mbox{\normalsize\it #1}}{\mbox{\scriptsize\it #1}}{\mbox{\tiny\it #1}}}
\newcommand{\image}[2]{\ms{img}_{#1}(#2)}
\newcommand{\spreimage}[2]{\ms{spreimg}_{#1}(#2)}
\newcommand{\wpreimage}[2]{\ms{wpreimg}_{#1}(#2)}
\newcommand{\disbwd}[2]{\delta^{\ms{bwd}}_{#1}(#2)}
\newcommand{\ats}[1]{\ms{scope}(#1)}
\newcommand{\ndopcpc}[2]{\tau^{\ms{nd}}_{#1}(#2)}
\newcommand{\choice}{\mathop{|}}
\newcommand{\opcpc}[2]{\tau_{#1}(#2)}

%% Don't use \begin{proof}/\end{proof} for multi-part proofs because
%% this places a QED symbol.
\newenvironment{proofstart}{\begin{block}{Proof.}}{\end{block}}
\newenvironment{proofmid}{\begin{block}{Proof (ctd.)}}{\end{block}}
\newenvironment{proofend}{\begin{proof}[Proof (ctd.)]}{\end{proof}}

\newtheorem{remark}[theorem]{Remark}

%% Macros to align the width of things.
\newlength{\mywidth}
\newcommand{\setmywidth}[1]{\settowidth{\mywidth}{#1}}
\newcommand{\usemywidth}[1]{\makebox[\mywidth][l]{#1}}
\newcommand{\usemywidthmath}[1]{\usemywidth{\ensuremath{#1}}}

%% A tighter "align" environment.
\newenvironment{tightalign}[1][c]{\par\(\begin{array}[#1]{@{}r@{}l}}
  {\end{array}\)\par}
\newenvironment{wrappedmath}[1][t]{\begin{array}[#1]{@{}l}}{\end{array}}

%% Nondeterministic transition systems with connector arcs
\newcommand{\arc}[4]{
  \begin{pgfscope}
    \pgfsetlinewidth{1pt}
    \pgfpathmoveto{\pgfpointanchor{#3}{center}}
    \pgfpathlineto{\pgfpointanchor{#1}{#2}}
    \pgfpathlineto{\pgfpointanchor{#4}{center}}
    \pgfusepath{clip}
    \pgfpathcircle{\pgfpointanchor{#1}{#2}}{1.8mm}
    \pgfusepath{draw}
  \end{pgfscope}
}

\newcommand{\drawblock}[5]{% PARAMETERS: COLOR, CORNERCOORDS, SIZE
% TOP SIDE
\color{#1!55}
\pgfmoveto{\pgfrelative{\pgfxyz(#2,#3,#4)}{\pgfxyz(0,#5,0)}}
\pgflineto{\pgfrelative{\pgfxyz(#2,#3,#4)}{\pgfxyz(#5,#5,0)}}
\pgflineto{\pgfrelative{\pgfxyz(#2,#3,#4)}{\pgfxyz(#5,#5,#5)}}
\pgflineto{\pgfrelative{\pgfxyz(#2,#3,#4)}{\pgfxyz(0,#5,#5)}}
\pgflineto{\pgfrelative{\pgfxyz(#2,#3,#4)}{\pgfxyz(0,#5,0)}}
\pgffill
\color{black}
\pgfmoveto{\pgfrelative{\pgfxyz(#2,#3,#4)}{\pgfxyz(0,#5,0)}}
\pgflineto{\pgfrelative{\pgfxyz(#2,#3,#4)}{\pgfxyz(#5,#5,0)}}
\pgflineto{\pgfrelative{\pgfxyz(#2,#3,#4)}{\pgfxyz(#5,#5,#5)}}
\pgflineto{\pgfrelative{\pgfxyz(#2,#3,#4)}{\pgfxyz(0,#5,#5)}}
\pgflineto{\pgfrelative{\pgfxyz(#2,#3,#4)}{\pgfxyz(0,#5,0)}}
\pgfstroke
% RIGHT SIDE
\color{#1!65}
\pgfmoveto{\pgfrelative{\pgfxyz(#2,#3,#4)}{\pgfxyz(#5,0,0)}}
\pgflineto{\pgfrelative{\pgfxyz(#2,#3,#4)}{\pgfxyz(#5,#5,0)}}
\pgflineto{\pgfrelative{\pgfxyz(#2,#3,#4)}{\pgfxyz(#5,#5,#5)}}
\pgflineto{\pgfrelative{\pgfxyz(#2,#3,#4)}{\pgfxyz(#5,0,#5)}}
\pgflineto{\pgfrelative{\pgfxyz(#2,#3,#4)}{\pgfxyz(#5,0,0)}}
\pgffill
\color{black}
\pgfmoveto{\pgfrelative{\pgfxyz(#2,#3,#4)}{\pgfxyz(#5,0,0)}}
\pgflineto{\pgfrelative{\pgfxyz(#2,#3,#4)}{\pgfxyz(#5,#5,0)}}
\pgflineto{\pgfrelative{\pgfxyz(#2,#3,#4)}{\pgfxyz(#5,#5,#5)}}
\pgflineto{\pgfrelative{\pgfxyz(#2,#3,#4)}{\pgfxyz(#5,0,#5)}}
\pgflineto{\pgfrelative{\pgfxyz(#2,#3,#4)}{\pgfxyz(#5,0,0)}}
\pgfstroke
%FRONT
\color{#1!99}
\pgfmoveto{\pgfxyz(#2,#3,#4)}
\pgflineto{\pgfrelative{\pgfxyz(#2,#3,#4)}{\pgfxyz(#5,0,0)}}
\pgflineto{\pgfrelative{\pgfxyz(#2,#3,#4)}{\pgfxyz(#5,#5,0)}}
\pgflineto{\pgfrelative{\pgfxyz(#2,#3,#4)}{\pgfxyz(0,#5,0)}}
\pgflineto{\pgfxyz(#2,#3,#4)}
\pgffill
\color{black}
\pgfmoveto{\pgfxyz(#2,#3,#4)}
\pgflineto{\pgfrelative{\pgfxyz(#2,#3,#4)}{\pgfxyz(#5,0,0)}}
\pgflineto{\pgfrelative{\pgfxyz(#2,#3,#4)}{\pgfxyz(#5,#5,0)}}
\pgflineto{\pgfrelative{\pgfxyz(#2,#3,#4)}{\pgfxyz(0,#5,0)}}
\pgflineto{\pgfxyz(#2,#3,#4)}
\pgfstroke
}

\newcommand{\RsGsB}{\begin{pgfpicture}{0.5mm}{0.5mm}{12mm}{5mm}
\pgfsetxvec{\pgfpoint{0.4cm}{0cm}}
\pgfsetyvec{\pgfpoint{0cm}{0.4cm}}
\pgfsetzvec{\pgfpoint{0.15cm}{0.15cm}}
\drawblock{red}{0}{0}{0}{1}
\drawblock{green}{1}{0}{0}{1}
\drawblock{blue}{2}{0}{0}{1}
\end{pgfpicture}
}

\newcommand{\twostacksA}[3]{\begin{pgfpicture}{0.5mm}{0.5mm}{0.8cm}{0.9cm}
\pgfsetxvec{\pgfpoint{0.4cm}{0cm}}
\pgfsetyvec{\pgfpoint{0cm}{0.4cm}}
\pgfsetzvec{\pgfpoint{0.15cm}{0.15cm}}
\drawblock{#2}{0}{0}{0}{1}
\drawblock{#1}{0}{1}{0}{1}
\drawblock{#3}{1}{0}{0}{1}
\end{pgfpicture}
}

\newcommand{\twostacksB}[3]{\begin{pgfpicture}{0.5mm}{0.5mm}{0.8cm}{0.9cm}
\pgfsetxvec{\pgfpoint{0.4cm}{0cm}}
\pgfsetyvec{\pgfpoint{0cm}{0.4cm}}
\pgfsetzvec{\pgfpoint{0.15cm}{0.15cm}}
\drawblock{#1}{0}{0}{0}{1}
\drawblock{#3}{1}{0}{0}{1}
\drawblock{#2}{1}{1}{0}{1}
\end{pgfpicture}
}

\newcommand{\onestack}[3]{\begin{pgfpicture}{0.5mm}{0.5mm}{4mm}{13mm}
\pgfsetxvec{\pgfpoint{0.4cm}{0cm}}
\pgfsetyvec{\pgfpoint{0cm}{0.4cm}}
\pgfsetzvec{\pgfpoint{0.15cm}{0.15cm}}
\drawblock{#3}{0}{0}{0}{1}
\drawblock{#2}{0}{1}{0}{1}
\drawblock{#1}{0}{2}{0}{1}
\end{pgfpicture}
}

\newcommand{\RGsB}{\twostacksA{red}{green}{blue}}

\newcommand{\RBsG}{\twostacksB{green}{red}{blue}}
\newcommand{\RBsGr}{\twostacksA{red}{blue}{green}}

\newcommand{\BRsG}{\twostacksA{blue}{red}{green}}
\newcommand{\BRsGr}{\twostacksB{green}{blue}{red}}

\newcommand{\RsBG}{\twostacksB{red}{blue}{green}}

\newcommand{\GRsB}{\twostacksA{green}{red}{blue}}
\newcommand{\RsGB}{\twostacksB{red}{green}{blue}}

\newcommand{\RGB}{\onestack{red}{green}{blue}}
\newcommand{\RBG}{\onestack{red}{blue}{green}}
\newcommand{\GRB}{\onestack{green}{red}{blue}}
\newcommand{\GBR}{\onestack{green}{blue}{red}}
\newcommand{\BRG}{\onestack{blue}{red}{green}}
\newcommand{\BGR}{\onestack{blue}{green}{red}}


\begin{document}
\subtitle{13.~Nondeterministic planning}
\date{January 20th, 2012}
\maketitles

\section{Motivation}

\begin{frame}
  \frametitle{Nondeterministic planning}

  \begin{itemize}
  \item The world is not predictable.
  \item 
    AI robotics:
    \begin{itemize}
    \item imprecise movement of the robot
    \item other robots
    \item human beings, animals
    \item machines (cars, trains, airplanes, lawn-mowers, \dots)
    \item natural phenomena (wind, water, snow, temperature, \dots)
    \end{itemize}
  \item 
    Games: other players are outside our control.
    \begin{itemize}
    \item To win a game (reaching a goal state) with certainty, all
      possible actions by the other players have to be anticipated (a
      \alert{winning strategy} of a game).
    \item The world is not predictable because it is unknown:
      we cannot \alert{observe} everything.
    \end{itemize}
  \end{itemize}

  \smallskip

  In this lecture, we will only deal with uncertain operator outcomes,
  not with partial observability.
\end{frame}

\begin{frame}<handout:1>[label=game-example]
  \frametitle{Nondeterminism in games}
  \begin{center}
    \begin{pgfpicture}{1cm}{0.5cm}{9cm}{7.5cm}

      \pgfnodecircle{N0}[fill]{\pgfxy(5,7)}{1.1mm}
      \only<all:1>{\pgfnodecircle{N00}[fill]{\pgfxy(2.5,4)}{1.1mm}}
      \pgfnodecircle{N000}[fill]{\pgfxy(1.5,1)}{1.1mm}
      \pgfnodecircle{N001}[fill]{\pgfxy(3.5,1)}{1.1mm}
      \only<all:1>{\pgfnodecircle{N01}[fill]{\pgfxy(7.5,4)}{1.1mm}}
      \pgfnodecircle{N010}[fill]{\pgfxy(6.5,1)}{1.1mm}
      \pgfnodecircle{N011}[fill]{\pgfxy(8.5,1)}{1.1mm}
      
      \pgfputat{\pgfxy(5,5)}{\pgfbox[center,center]{\Large our actions}}
      \only<all:1>{\pgfputat{\pgfxy(5,3)}{\pgfbox[center,center]{\Large opponent actions}}}
      
      \pgfsetlinewidth{1pt}
      \pgfsetendarrow{\pgfarrowtriangle{6pt}}

      \only<all:1>{
        \color{red}
        \pgfnodeconnline{N0}{N00}
        \color{blue}
        \pgfnodeconnline{N0}{N01}
        \color{violet}
        \pgfnodeconnline{N00}{N000}
        \color{green}
        \pgfnodeconnline{N00}{N001}
        \color{violet}
        \pgfnodeconnline{N01}{N010}
        \color{green}
        \pgfnodeconnline{N01}{N011}
      }
      
      \only<all:2>{
        \color{red}
        \pgfnodeconnline{N0}{N000}
        \pgfnodeconnline{N0}{N001}
        \color{blue}
        \pgfnodeconnline{N0}{N010}
        \pgfnodeconnline{N0}{N011}
      }
    \end{pgfpicture}
  \end{center}
\end{frame}

\only<handout>{
  \againframe<handout:2>{game-example}
}

\begin{frame}
  \frametitle{Nondeterministic planning}

  \begin{itemize}
  \item In \alert{deterministic planning} we have assumed that the
    only changes taking place in the world are those caused by us and
    that we can \alert{exactly predict} the results of our actions.
  \item 
    \alert{Other agents} and processes, beyond our control,
    are formalized as \alert{nondeterminism}.
  \item
    Implications:
    \begin{enumerate}
    \item The future state of the world cannot be predicted.
    \item We cannot reliably plan ahead: no single operator sequence
      achieves the goals.
    \item In some cases it is not possible to achieve the goals with
      certainty no matter which outcomes the actions have, but only
      under certain fairness assumptions.
    \end{enumerate}
  \end{itemize}
\end{frame}

\section{Transition systems and planning tasks}

\subsection{Transition systems}

\begin{frame}{Transition systems}
  \frametitle{Transition systems with nondeterminism (cf.~Chapter~2)}
  \begin{definition}[transition system]
    A \alert{nondeterministic transition system} is a 5-tuple
    $\mathcal T = \langle S, L, T, s_0, S_\star\rangle$ where
    \begin{itemize}
    \item $S$ is a finite set of \alert{states},
    \item $L$ is a finite set of (transition) \alert{labels},
    \item $T \subseteq S \times L \times S$ is the \alert{transition
      relation},
    \item $s_0 \in S$ is the \alert{initial state}, and
    \item $S_\star \subseteq S$ is the set of \alert{goal states}.
    \end{itemize}
  \end{definition}

  \structure{Note:} $T \subseteq S \times L \times S$ allows
  \alert{nondeterministic operators} with more than one possible
  outcome.
\end{frame}

\subsection{Operators}

\begin{frame}
  \frametitle{Nondeterministic operators}
  \begin{definition}[nondeterministic operator]
    Let $V$ be a set of finite-domain state variables.  A
    nondeterministic operator in unary nondeterminism normal
    form with conjunctive precondition and unconditional effects, or
    \alert{nondeterministic operator} for short, is a pair $o = \langle
    \chi, E \rangle$, where
    \begin{itemize}
    \item $\chi$ is a conjunction of atoms over $V$ (the
      \alert{precondition}), and
    \item $E = \{ e_1, \dots, e_n \}$ is a finite set of possible
      \alert{effects} of $o$, each $e_i$ being a conjunction of atomic
      finite-domain effects over $V$.
    \end{itemize}
  \end{definition}
\end{frame}

\begin{frame}
  \frametitle{Nondeterministic operators}
  \begin{definition}[nondeterministic operator application]
    Let $o = \langle \chi, E \rangle$ be a nondeterministic operator
    and $s$ a state.

    \medskip

    Applicability of $o$ in $s$ is definied as in the deterministic
    case, i.e., $o$ is \alert{applicable} in $s$ iff $s \models \chi$
    and the change set of each effect $e \in E$ is consistent.

    \medskip

    If $o$ is applicable in $s$, then the \alert{application} of $o$
    in $s$ leads to one of the states in the set $\applyop{o}{s} := \{
    \applyop{\langle \chi, e \rangle}{s} \,|\, e \in E \}$
    nondeterministically.
  \end{definition}
\end{frame}

\begin{frame}
  \frametitle{Nondeterministic operators}
  \begin{example}
    $\textit{put-on-block}(A,B) = \langle \chi, \{ e_1, e_2 \} \rangle$ where
    \begin{itemize}
    \item $\chi = \{ \textit{handempty} \mapsto \textit{false},
      \textit{clear-B} \mapsto \textit{true}, \textit{pos-A} \mapsto
      \textit{hand} \}$,
    \item $e_1 = \{ \textit{handempty} \mapsto \textit{true},
      \textit{clear-B} \mapsto \textit{false}, \textit{pos-A} \mapsto
      \textit{on-B} \}$,
    \item $e_2 = \{ \textit{handempty} \mapsto \textit{true},
      \textit{posn-A} \mapsto \textit{table} \}$.
    \end{itemize}
    
    \medskip

    Applied to a state where the agent is holding block $A$ and block
    $B$ is clear, this operator leads to one of two possible successor
    states. Either $A$ gets stacked on $B$ successfully, or $A$ is
    dropped to the table.
  \end{example}
\end{frame}

\subsection{Planning tasks}

\begin{frame}{Nondeterministic planning task}
  \begin{definition}[nondeterministic planning task]
    A (fully observable) \alert{nondeterministic planning task} is a 4-tuple
    $\Pi = \langle V,I,O,\gamma\rangle$ where
    \begin{itemize}
    \item $V$ is a finite set of \alert{finite-domain state
      variables},
    \item $I$ is an \alert{initial state} over $V$,
    \item $O$ is a finite set of \alert{nondeterministic operators} over
      $V$, and
    \item $\gamma$ is a conjunctions of atoms over $V$ describing the
      \alert{goal states}.
    \end{itemize}
  \end{definition}

  \smallskip

  \structure{Remark:} In the following, we will always assume that our
  nondeterministic planning tasks are fully observable and omit the
  qualification.

\end{frame}

\begin{frame}{Mapping planning tasks to transition systems}
  \begin{definition}[induced transition system]
    Every nondeterministic planning task $\Pi = \langle
    V,I,O,\gamma\rangle$ induces a corresponding nondeterministic
    transition system $\alert{\mathcal T(\Pi)} = \langle S, L, T, s_0,
    S_\star\rangle$:
    \begin{itemize}
    \item $S$ is the set of all states over $V$,
    \item $L$ is the set of operators $O$,
    \item $T = \{ \langle s, o, s'\rangle \mid
      s \in S,~ o \text{~applicable in~}s,~ s' \in \applyop{o}{s}\}$,
    \item $s_0 = I$, and
    \item $S_\star = \{ s \in S \mid s \models \gamma\}$
    \end{itemize}
  \end{definition}
\end{frame}

\section{Plans}

\subsection{Motivation}

\begin{frame}
  \frametitle{What is a plan?}
  
  In nondeterministic planning, plans are more complicated
  objects than in the deterministic case:

  \medskip

  The best action to take may \alert{depend} on nondeterministic
  effects of previous operators.

  \medskip

  Nondeterministic plans thus often require \alert{branching}.
  Sometimes, they even require \alert{looping}\\
  (we will likely only cover branching in this course).
\end{frame}

\begin{frame}
  \frametitle{What is a plan?}

  \begin{example}[Branching]
    (Part of) a plan for winning the game \alert{Connect Four} can be
    described as follows:
    \begin{itemize}
    \item Place a tile in the 4th column.
      \begin{itemize}
      \item If opponent places a tile in the 1st, 4th or 7th column, \\
        place a tile in the 4th column.
      \item If opponent places a tile in the 2nd or 5th column, \\
        place a tile in the 2nd column.
      \item If opponent places a tile in the 3rd or 6th column, \\
        place a tile in the 6th column.
      \end{itemize}
    \end{itemize}
    
    There is no \alert{non-branching} plan that solves the task\\
    (= is guaranteed to win the game).
  \end{example}
\end{frame}

\begin{frame}
  \frametitle{What is a plan?}

  \begin{example}[Looping]
    A plan for building a card house can be described as follows:
    \begin{enumerate}
    \item Build a wall with two cards. \\
      If the structure falls apart, redo from start.
    \item Build a second wall with two cards. \\
      If the structure falls apart, redo from start.
    \item Build a ceiling on top of the walls with a fifth card. \\
      If the structure falls apart, redo from start.
    \item Build a wall on top of the ceiling with two cards. \\
      If the structure falls apart, redo from start.
    \end{enumerate}
    
    There is no \alert{non-looping} plan that solves the task \\
    (unless the planning agent is very dextrous).
  \end{example}
\end{frame}

\begin{frame}
  \frametitle{What is a plan?}

  \begin{itemize}
  \item Plans should be allowed to \alert{branch}. Otherwise, most
    interesting nondeterministic planning tasks cannot be solved.
  \item We may or may not allow plans to \alert{loop}.
    \begin{itemize}
    \item Non-looping plans are preferable because they
      \alert{guarantee} that the goal is reached within a bounded
      number of steps.
    \item Where non-looping plans are not possible, looping plans
      may be adequate because they at least guarantee that the goal
      will be reached \alert{eventually} unless nature is
      \alert{unfair}.
    \end{itemize}
  \end{itemize}
  We will now introduce the formal concepts necessary to define
  branching (and looping) plans.
\end{frame}

\subsection{Definition}

\begin{frame}
  \frametitle{Nondeterministic plans: formal definition}

  \begin{definition}[strategy]
    Let $\Pi = \langle V, I, O, \gamma\rangle$ be a nondeterministic
    planning task with state set $S$ and goal states $S_\star$.

    \medskip

    A \alert{strategy} for $\Pi$ is a function $\pi: S_\pi \to O$ for
    some subset $S_\pi \subseteq S$ such that $\pi(s)$ is applicable in
    $s$ for all $s \in S_\pi$.

    \medskip

    The set of states reachable in $\mathcal T(\Pi)$ starting in state
    $s$ and following $\pi$ is denoted by $S_\pi(s)$.
  \end{definition}
\end{frame}

\begin{frame}
  \frametitle{Nondeterministic plans: formal definition}

  \begin{definition}[weak, closed, proper, and acyclic strategies]
    Let $\Pi = \langle V, I, O, \gamma\rangle$ be a nondeterministic
    planning task with state set $S$ and goal states $S_\star$, and let
    $\pi$ be a strategy for $\Pi$.

    Then $\pi$ is called
    \begin{itemize}
    \item \alert{weak} iff $S_\pi(s_0) \cap S_\star \neq \emptyset$,
    \item \alert{closed} iff $S_\pi(s_0) \subseteq S_\pi$,
    \item \alert{proper} iff $S_\pi(s') \cap S_\star \neq \emptyset$
      for all $s' \in S_\pi(s_0)$, and
    \item \alert{acyclic} iff there is no state $s' \in
      S_\pi(s_0)$ such that $s'$ is reachable from $s'$ following
      $\pi$ in a strictly positive number of steps.
    \end{itemize}
  \end{definition}
\end{frame}

\begin{frame}
  \frametitle{Nondeterministic plans: formal definition}
  \begin{itemize}
  \item \alert{Strategies} in nondeterministic planning correspond to
    \alert{applicable operator sequences} in deterministic planning.
  \item In deterministic planning, a \alert{plan} is an applicable
    operator sequence that results in a goal state.
  \item In nondeterministic planning, we define different notions of
    ``resulting in a goal state''.
  \end{itemize}
\end{frame}

\begin{frame}
  \frametitle{Nondeterministic plans: formal definition}

  \begin{definition}
    Let $\Pi = \langle V, I, O, \gamma\rangle$ be a nondeterministic
    planning task with state set $S$ and goal states $S_\star$.

    \begin{itemize}
    \item A strategy for $\Pi$ is called a \alert{weak plan} for $\Pi$ \\
      iff it is weak.
    \item A strategy for $\Pi$ is called a \alert{strong cyclic plan}
      for $\Pi$ \\
      iff it is closed and proper.
    \item A strong cyclic plan for $\Pi$ is called a \alert{strong
      plan} for $\Pi$ \\
      iff it is acyclic.
    \end{itemize}
  \end{definition}
\end{frame}

% \section{Summary}

\begin{frame}
  \frametitle{Summary and outlook}
  
  We extended the deterministic (\alert{classical}) planning formalism:
  \begin{itemize}
  \item \alert{operators} can be nondeterministic
  \end{itemize}
  \structure{Remark:} We could also introduce nondeterminism in the
  initial situation by allowing more than one initial state, but this
  can be easily compiled into our formalism.

  \bigskip

  As a consequence, \alert{plans} can contain
  \begin{itemize}
  \item \alert{branches} and
  \item \alert{loops}.
  \end{itemize}

  \bigskip

  In the following chapter, we consider the \alert{strong planning}
  problem (and maybe strong cyclic planning, if time permits) and
  discuss some algorithms.
\end{frame}

\end{document}
