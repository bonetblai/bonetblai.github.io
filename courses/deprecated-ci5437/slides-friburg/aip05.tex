\documentclass{gkibeamer}

\usepackage{tabularx}
\usepackage{tikz}
\usetikzlibrary{shapes}

\usepackage{tikz}[2008/02/13]

\title[AI Planning]{Principles of AI Planning}
\author[B.~Nebel,\\R.~Mattm\"uller]{Bernhard Nebel and Robert Mattm\"uller}
\date[]{Winter Semester 2011/2012}

%% Use \alert instead of \textred where it makes sense.
%% (\textred makes sense when you *really* want the color to be red,
%% independent of the beamer style settings.)
\newcommand{\hilite}[1]{{\color{structure} #1}}
\newcommand{\textred}[1]{{\color{red} #1}}
\newcommand{\textblue}[1]{{\color{blue} #1}}
\newcommand{\textgreen}[1]{{\color{green} #1}}
\newcommand{\textbrown}[1]{{\color{brown} #1}}
\newcommand{\textviolet}[1]{{\color{violet} #1}}
\newcommand{\textbg}[1]{{\color{bg} #1}}

\newcommand{\eg}{e.\,g.}
\newcommand{\ie}{i.\,e.}

%% State variables and values for state variables
\newcommand{\var}[1]{\textit{#1}}
\newcommand{\val}[1]{\text{#1}}

%% Various stuff
\newcommand{\astar}{A${}^*$}
\newcommand{\idastar}{IDA${}^*$}

%% Blocksworld stuff
\newcommand{\AONB}{\var{A-on-B}}
\newcommand{\AONC}{\var{A-on-C}}
\newcommand{\AOND}{\var{A-on-D}}
\newcommand{\BONA}{\var{B-on-A}}
\newcommand{\BONC}{\var{B-on-C}}
\newcommand{\BOND}{\var{B-on-D}}
\newcommand{\CONA}{\var{C-on-A}}
\newcommand{\CONB}{\var{C-on-B}}
\newcommand{\COND}{\var{C-on-D}}
\newcommand{\DONA}{\var{D-on-A}}
\newcommand{\DONB}{\var{D-on-B}}
\newcommand{\DONC}{\var{D-on-C}}
\newcommand{\AONTABLE}{\var{A-on-T}}
\newcommand{\BONTABLE}{\var{B-on-T}}
\newcommand{\CONTABLE}{\var{C-on-T}}
\newcommand{\CLEARA}{\var{A-clear}}
\newcommand{\CLEARB}{\var{B-clear}}
\newcommand{\CLEARC}{\var{C-clear}}

\newcommand{\applyop}[2]{\textit{app}_{#1}(#2)}
\newcommand{\applyops}[2]{\textit{app}_{#1}(#2)}
\newcommand{\applyplan}[2]{\textit{app}_{#1}(#2)}
\newcommand{\regr}[2]{\textit{regr}_{#1}(#2)}
\newcommand{\regrstrips}[2]{\textit{sregr}_{#1}(#2)}
\newcommand{\eprecon}[2]{\textit{EPC}_{#1}(#2)}

\newcommand{\compl}[1]{\overline{#1}}

\newcommand{\changes}[2]{\lbrack #1\rbrack_{#2}}
\newcommand{\CEF}{\vartriangleright}

\newcommand{\relaxation}[1]{{#1}^+}
\newcommand{\onset}[1]{\textit{on}(#1)}
\newcommand{\hplus}{\ensuremath{h^+}}
\newcommand{\hmax}{\ensuremath{h_{\text{max}}}}
\newcommand{\hadd}{\ensuremath{h_{\text{add}}}}
\newcommand{\hlmcut}{\ensuremath{h_{\text{LM-cut}}}}
\newcommand{\hff}{\ensuremath{h_{\text{FF}}}}
\newcommand{\hcs}{\ensuremath{h_{\text{cs}}}}
\newcommand{\hsa}{\ensuremath{h_{\text{sa}}}}
\newcommand{\hhhh}{\ensuremath{h_{\text{HHH}}}}

\newcommand{\sasplus}{SAS${}^+$}

\newcommand{\graphequiv}{\stackrel{\textup{G}}{\sim}}

\newcommand{\cg}{\ensuremath{\textit{CG}}}

%% Turing machines and complexity theory
\newcommand{\accept}{{\textsf{Y}}}

\newcommand{\easier}{\ensuremath{\le_{\text{p}}}}

\newcommand{\decisionclass}[1]{\ensuremath{\textsf{\textup{#1}}}}
\newcommand{\dtime}{\decisionclass{DTIME}}
\newcommand{\ntime}{\decisionclass{NTIME}}
\newcommand{\dspace}{\decisionclass{DSPACE}}
\newcommand{\nspace}{\decisionclass{NSPACE}}

\newcommand{\ptime}{\decisionclass{P}}
\newcommand{\np}{\decisionclass{NP}}
\newcommand{\pspace}{\decisionclass{PSPACE}}
\newcommand{\npspace}{\decisionclass{NPSPACE}}
\newcommand{\exptime}{\decisionclass{EXP}}
\newcommand{\nexptime}{\decisionclass{NEXP}}
\newcommand{\expspace}{\decisionclass{EXPSPACE}}
\newcommand{\dblexptime}{\decisionclass{2EXP}}

\newcommand{\planex}{\textsc{PlanEx}}
\newcommand{\planlen}{\textsc{PlanLen}}

\newcommand{\ms}[1]{\mathchoice{\mbox{\normalsize\it #1}}{\mbox{\normalsize\it #1}}{\mbox{\scriptsize\it #1}}{\mbox{\tiny\it #1}}}
\newcommand{\image}[2]{\ms{img}_{#1}(#2)}
\newcommand{\spreimage}[2]{\ms{spreimg}_{#1}(#2)}
\newcommand{\wpreimage}[2]{\ms{wpreimg}_{#1}(#2)}
\newcommand{\disbwd}[2]{\delta^{\ms{bwd}}_{#1}(#2)}
\newcommand{\ats}[1]{\ms{scope}(#1)}
\newcommand{\ndopcpc}[2]{\tau^{\ms{nd}}_{#1}(#2)}
\newcommand{\choice}{\mathop{|}}
\newcommand{\opcpc}[2]{\tau_{#1}(#2)}

%% Don't use \begin{proof}/\end{proof} for multi-part proofs because
%% this places a QED symbol.
\newenvironment{proofstart}{\begin{block}{Proof.}}{\end{block}}
\newenvironment{proofmid}{\begin{block}{Proof (ctd.)}}{\end{block}}
\newenvironment{proofend}{\begin{proof}[Proof (ctd.)]}{\end{proof}}

\newtheorem{remark}[theorem]{Remark}

%% Macros to align the width of things.
\newlength{\mywidth}
\newcommand{\setmywidth}[1]{\settowidth{\mywidth}{#1}}
\newcommand{\usemywidth}[1]{\makebox[\mywidth][l]{#1}}
\newcommand{\usemywidthmath}[1]{\usemywidth{\ensuremath{#1}}}

%% A tighter "align" environment.
\newenvironment{tightalign}[1][c]{\par\(\begin{array}[#1]{@{}r@{}l}}
  {\end{array}\)\par}
\newenvironment{wrappedmath}[1][t]{\begin{array}[#1]{@{}l}}{\end{array}}

%% Nondeterministic transition systems with connector arcs
\newcommand{\arc}[4]{
  \begin{pgfscope}
    \pgfsetlinewidth{1pt}
    \pgfpathmoveto{\pgfpointanchor{#3}{center}}
    \pgfpathlineto{\pgfpointanchor{#1}{#2}}
    \pgfpathlineto{\pgfpointanchor{#4}{center}}
    \pgfusepath{clip}
    \pgfpathcircle{\pgfpointanchor{#1}{#2}}{1.8mm}
    \pgfusepath{draw}
  \end{pgfscope}
}

\newcommand{\drawblock}[5]{% PARAMETERS: COLOR, CORNERCOORDS, SIZE
% TOP SIDE
\color{#1!55}
\pgfmoveto{\pgfrelative{\pgfxyz(#2,#3,#4)}{\pgfxyz(0,#5,0)}}
\pgflineto{\pgfrelative{\pgfxyz(#2,#3,#4)}{\pgfxyz(#5,#5,0)}}
\pgflineto{\pgfrelative{\pgfxyz(#2,#3,#4)}{\pgfxyz(#5,#5,#5)}}
\pgflineto{\pgfrelative{\pgfxyz(#2,#3,#4)}{\pgfxyz(0,#5,#5)}}
\pgflineto{\pgfrelative{\pgfxyz(#2,#3,#4)}{\pgfxyz(0,#5,0)}}
\pgffill
\color{black}
\pgfmoveto{\pgfrelative{\pgfxyz(#2,#3,#4)}{\pgfxyz(0,#5,0)}}
\pgflineto{\pgfrelative{\pgfxyz(#2,#3,#4)}{\pgfxyz(#5,#5,0)}}
\pgflineto{\pgfrelative{\pgfxyz(#2,#3,#4)}{\pgfxyz(#5,#5,#5)}}
\pgflineto{\pgfrelative{\pgfxyz(#2,#3,#4)}{\pgfxyz(0,#5,#5)}}
\pgflineto{\pgfrelative{\pgfxyz(#2,#3,#4)}{\pgfxyz(0,#5,0)}}
\pgfstroke
% RIGHT SIDE
\color{#1!65}
\pgfmoveto{\pgfrelative{\pgfxyz(#2,#3,#4)}{\pgfxyz(#5,0,0)}}
\pgflineto{\pgfrelative{\pgfxyz(#2,#3,#4)}{\pgfxyz(#5,#5,0)}}
\pgflineto{\pgfrelative{\pgfxyz(#2,#3,#4)}{\pgfxyz(#5,#5,#5)}}
\pgflineto{\pgfrelative{\pgfxyz(#2,#3,#4)}{\pgfxyz(#5,0,#5)}}
\pgflineto{\pgfrelative{\pgfxyz(#2,#3,#4)}{\pgfxyz(#5,0,0)}}
\pgffill
\color{black}
\pgfmoveto{\pgfrelative{\pgfxyz(#2,#3,#4)}{\pgfxyz(#5,0,0)}}
\pgflineto{\pgfrelative{\pgfxyz(#2,#3,#4)}{\pgfxyz(#5,#5,0)}}
\pgflineto{\pgfrelative{\pgfxyz(#2,#3,#4)}{\pgfxyz(#5,#5,#5)}}
\pgflineto{\pgfrelative{\pgfxyz(#2,#3,#4)}{\pgfxyz(#5,0,#5)}}
\pgflineto{\pgfrelative{\pgfxyz(#2,#3,#4)}{\pgfxyz(#5,0,0)}}
\pgfstroke
%FRONT
\color{#1!99}
\pgfmoveto{\pgfxyz(#2,#3,#4)}
\pgflineto{\pgfrelative{\pgfxyz(#2,#3,#4)}{\pgfxyz(#5,0,0)}}
\pgflineto{\pgfrelative{\pgfxyz(#2,#3,#4)}{\pgfxyz(#5,#5,0)}}
\pgflineto{\pgfrelative{\pgfxyz(#2,#3,#4)}{\pgfxyz(0,#5,0)}}
\pgflineto{\pgfxyz(#2,#3,#4)}
\pgffill
\color{black}
\pgfmoveto{\pgfxyz(#2,#3,#4)}
\pgflineto{\pgfrelative{\pgfxyz(#2,#3,#4)}{\pgfxyz(#5,0,0)}}
\pgflineto{\pgfrelative{\pgfxyz(#2,#3,#4)}{\pgfxyz(#5,#5,0)}}
\pgflineto{\pgfrelative{\pgfxyz(#2,#3,#4)}{\pgfxyz(0,#5,0)}}
\pgflineto{\pgfxyz(#2,#3,#4)}
\pgfstroke
}

\newcommand{\RsGsB}{\begin{pgfpicture}{0.5mm}{0.5mm}{12mm}{5mm}
\pgfsetxvec{\pgfpoint{0.4cm}{0cm}}
\pgfsetyvec{\pgfpoint{0cm}{0.4cm}}
\pgfsetzvec{\pgfpoint{0.15cm}{0.15cm}}
\drawblock{red}{0}{0}{0}{1}
\drawblock{green}{1}{0}{0}{1}
\drawblock{blue}{2}{0}{0}{1}
\end{pgfpicture}
}

\newcommand{\twostacksA}[3]{\begin{pgfpicture}{0.5mm}{0.5mm}{0.8cm}{0.9cm}
\pgfsetxvec{\pgfpoint{0.4cm}{0cm}}
\pgfsetyvec{\pgfpoint{0cm}{0.4cm}}
\pgfsetzvec{\pgfpoint{0.15cm}{0.15cm}}
\drawblock{#2}{0}{0}{0}{1}
\drawblock{#1}{0}{1}{0}{1}
\drawblock{#3}{1}{0}{0}{1}
\end{pgfpicture}
}

\newcommand{\twostacksB}[3]{\begin{pgfpicture}{0.5mm}{0.5mm}{0.8cm}{0.9cm}
\pgfsetxvec{\pgfpoint{0.4cm}{0cm}}
\pgfsetyvec{\pgfpoint{0cm}{0.4cm}}
\pgfsetzvec{\pgfpoint{0.15cm}{0.15cm}}
\drawblock{#1}{0}{0}{0}{1}
\drawblock{#3}{1}{0}{0}{1}
\drawblock{#2}{1}{1}{0}{1}
\end{pgfpicture}
}

\newcommand{\onestack}[3]{\begin{pgfpicture}{0.5mm}{0.5mm}{4mm}{13mm}
\pgfsetxvec{\pgfpoint{0.4cm}{0cm}}
\pgfsetyvec{\pgfpoint{0cm}{0.4cm}}
\pgfsetzvec{\pgfpoint{0.15cm}{0.15cm}}
\drawblock{#3}{0}{0}{0}{1}
\drawblock{#2}{0}{1}{0}{1}
\drawblock{#1}{0}{2}{0}{1}
\end{pgfpicture}
}

\newcommand{\RGsB}{\twostacksA{red}{green}{blue}}

\newcommand{\RBsG}{\twostacksB{green}{red}{blue}}
\newcommand{\RBsGr}{\twostacksA{red}{blue}{green}}

\newcommand{\BRsG}{\twostacksA{blue}{red}{green}}
\newcommand{\BRsGr}{\twostacksB{green}{blue}{red}}

\newcommand{\RsBG}{\twostacksB{red}{blue}{green}}

\newcommand{\GRsB}{\twostacksA{green}{red}{blue}}
\newcommand{\RsGB}{\twostacksB{red}{green}{blue}}

\newcommand{\RGB}{\onestack{red}{green}{blue}}
\newcommand{\RBG}{\onestack{red}{blue}{green}}
\newcommand{\GRB}{\onestack{green}{red}{blue}}
\newcommand{\GBR}{\onestack{green}{blue}{red}}
\newcommand{\BRG}{\onestack{blue}{red}{green}}
\newcommand{\BGR}{\onestack{blue}{green}{red}}


\begin{document}

\lectureno{5}
\subtitle{Planning as search: progression and regression}
\date{November 4th, 2011}
\maketitles

\section[Search]{Planning as (classical) search}
\subsection{Introduction}

\begin{frame}{What do we mean by search?}
  \begin{itemize}
  \item \alert{Search} is a very generic term.
  \item[$\leadsto$] Every algorithm that tries out various
    alternatives can be said to ``search'' in some way.
  \item Here, we mean \alert{classical search} algorithms.
    \begin{itemize}
    \item \alert{Search nodes} are \alert{expanded} to
      generate \alert{successor nodes}.
    \item \hilite{Examples:} breadth-first search, \astar,
      hill-climbing, \dots
    \end{itemize}
  \item To be brief, we just say \alert{search} in the following \\
    (not ``classical search'').
  \end{itemize}
\end{frame}

\begin{frame}{Do you know this stuff already?}
  \begin{itemize}
  \item We \alert{assume prior knowledge} of basic search algorithms:
    \begin{itemize}
    \item uninformed vs.\ informed
    \item systematic vs.\ local
    \end{itemize}
  \item There will be a small refresher in the next chapter.
  \item \hilite{Background:} Russell \& Norvig, Artificial
    Intelligence -- \\ A Modern Approach, Ch.~3 (all of it), Ch.~4 (local
    search)
  \end{itemize}
\end{frame}

\begin{frame}{Search in planning}
  \begin{itemize}
  \item \alert{search:} one of the \alert{big success stories} of AI
  \item many planning algorithms based on classical AI search \\
    (we'll see some other algorithms later, though)
  \item will be the focus of this and the following chapters \\
    (the majority of the course)
  \end{itemize}
\end{frame}

\begin{frame}{Satisficing or optimal planning?}
  Must carefully distinguish two different problems:
  \begin{itemize}
  \item \alert{satisficing planning:} any solution is OK \\
    (although shorter solutions typically preferred)
  \item \alert{optimal planning:} plans must have shortest possible
    length
  \end{itemize}
  
  \bigskip

  Both are often solved by search, but:
  \begin{itemize}
  \item details are \alert{very different}
  \item almost \alert{no overlap} between good techniques for
    satisficing planning and good techniques for optimal planning
  \item many problems that are trivial for satisficing planners
    are impossibly hard for optimal planners
  \end{itemize}
\end{frame}

\subsection[Classification]{Classification of search-based planners}

\begin{frame}{Planning by search}
  How to apply search to planning?
  $\leadsto$ \alert{many choices to make!}
  
  \begin{overlayarea}{\textwidth}{4.5cm}
    \begin{block}{
        \only<all:1>{Choice 1: Search direction}
        \only<all:2>{Choice 2: Search space representation}
        \only<all:3>{Choice 3: Search algorithm}
        \only<all:4>{Choice 4: Search control}
      }
      \begin{itemize}
        \only<all:1>{
        \item \alert{progression:} forward from initial
          state to goal
        \item \alert{regression:} backward from goal
          states to initial state
        \item \alert{bidirectional search}
        }
        \only<all:2>{
        \item search nodes are associated with \alert{states} \\
          ($\leadsto$ \alert{state-space search})
        \item search nodes are associated with \alert{sets of states}
        }
        \only<all:3>{
        \item \alert{uninformed search:} \\
          depth-first, breadth-first, iterative depth-first, \dots
        \item \alert{heuristic search (systematic):} \\
          greedy best-first, \astar, Weighted \astar, \idastar, \dots
        \item \alert{heuristic search (local):} \\
          hill-climbing, simulated annealing, beam search, \dots
        }
        \only<all:4>{
        \item \alert{heuristics} for informed search algorithms
        \item \alert{pruning techniques:} invariants, symmetry
          elimination, partial-order reduction, helpful actions pruning, \dots
        }
      \end{itemize}
    \end{block}
  \end{overlayarea}
\end{frame}

\begin{frame}{Search-based satisficing planners}
  \begin{block}{FF (Hoffmann \& Nebel, 2001)}
    \begin{itemize}
    \item \hilite{search direction:} forward search
    \item \hilite{search space representation:} single states
    \item \hilite{search algorithm:} enforced hill-climbing (informed
      local)
    \item \hilite{heuristic:} FF heuristic (inadmissible)
    \item \hilite{pruning technique:} helpful actions (incomplete)
    \end{itemize}
  \end{block}
  $\leadsto$ one of the best satisficing planners
\end{frame}

\begin{frame}{Search-based optimal planners}
  \begin{block}{Fast Downward Stone Soup (Helmert et al., 2011)}
    \begin{itemize}
    \item \hilite{search direction:} forward search
    \item \hilite{search space representation:} single states
    \item \hilite{search algorithm:} \astar\ (informed
      systematic)
    \item \hilite{heuristic:} multiple admissible heuristics combined
      into a heuristic portfolio (LM-cut, M\&S, blind, \dots)
    \item \hilite{pruning technique:} none
    \end{itemize}
  \end{block}
  $\leadsto$ one of the best optimal planners
\end{frame}

\begin{frame}{Our plan for the next lectures}
  Choices to make:
  \begin{enumerate}
  \item search direction: progression/regression/both
    \\ $\leadsto$ \alert{this chapter}
  \item search space representation: states/sets of states
    \\ $\leadsto$ \alert{this chapter}
  \item search algorithm: uninformed/heuristic; systematic/local
    \\ $\leadsto$ \alert{next chapter}
  \item search control: heuristics, pruning techniques
    \\ $\leadsto$ \alert{following chapters}
  \end{enumerate}
\end{frame}

\section{Progression}
\subsection{Overview}

\begin{frame}{Planning by forward search: progression}
  \alert{Progression:} Computing the successor state $\applyop{o}{s}$
  of a state $s$ with respect to an operator $o$.

  \medskip

  \alert{Progression planners} find solutions by forward search:
  \begin{itemize}
  \item start from initial state
  \item iteratively pick a previously generated state and
    \alert{progress it} through an operator, generating a new state
  \item solution found when a goal state generated
  \end{itemize}

  \medskip

  \hilite{pro:} very easy and efficient to implement
\end{frame}

\begin{frame}{Search space representation in progression planners}
  Two alternative search spaces for progression planners:
  \begin{enumerate}
  \item \alert{search nodes correspond to states}
    \begin{itemize}
    \item when the same state is generated along different paths, \\
      it is not considered again (\alert{duplicate detection})
    \item \hilite{pro:} save time to consider same state again
    \item \hilite{con:} memory intensive (must maintain \alert{closed
      list})
    \end{itemize}
  \item \alert{search nodes correspond to operator sequences}
    \begin{itemize}
    \item different operator sequences may lead to identical states
      (\alert{transpositions}); search does not notice this
    \item \hilite{pro:} can be very memory-efficient
    \item \hilite{con:} much wasted work (often exponentially slower)
    \end{itemize}
  \end{enumerate}
  $\leadsto$ first alternative usually preferable in planning \\
  (\alert{unlike} many classical search benchmarks like 15-puzzle)
\end{frame}

\subsection{Example}

\begin{frame}{Progression planning example (depth-first search)}
  \hilite{Example} where search nodes correspond to operator sequences
  \\ (no duplicate detection)
  \begin{center}
    \begin{tikzpicture}
      \tikzstyle node style=[fill, circle]

      \node[node style] (N00) at (0:6mm) {};
      \node[node style] (N01) at (180:6mm) {};

      \node[node style] (N02) at (0:22mm) {};
      \node[node style] (N04) at (60:22mm) {};
      \node[node style] (N05) at (120:22mm) {};
      \node[node style] (N06) at (180:22mm) {};
      
      \node[node style] (N03) at (0:37mm) {};
      \node[node style] (N07) at (30:37mm) {};
      \node[node style] (N08) at (60:37mm) {};
      \node[node style] (N09) at (90:37mm) {};
      \node[node style] (N10) at (120:37mm)[label=110:$s_0$] {};
      \node[node style] (N11) at (150:37mm) {};
      \node[node style] (N12) at (180:37mm) {};

      \tikzstyle blue arc=[->, thick, >=latex, blue]
      \tikzstyle red arc=[->, thick, >=latex, red]

      \uncover<all:1>{\draw[blue arc] (N02) -- (N04);}
      \uncover<all:1>{\draw[blue arc] (N04) -- (N00);}
      \uncover<all:1>{\draw[blue arc] (N03) -- (N02);}
      \uncover<all:1>{\draw[blue arc] (N07) -- (N04);}
      \uncover<all:1>{\draw[blue arc, bend left=40] (N08) to (N04);}
      \uncover<all:1>{\draw[blue arc] (N09) -- (N04);}
      \uncover<all:1,10>{\draw[blue arc] (N05) -- (N01);}
      \uncover<all:1,5,9>{\draw[blue arc] (N12) -- (N11);}
      \uncover<all:1>{\draw[blue arc] (N10) -- (N09);}
      \uncover<all:1,6,7,8,9,10>{\draw[blue arc] (N11) -- (N05);}

      \uncover<all:1>{\draw[red arc] (N02) -- (N00);}
      \uncover<all:1>{\draw[red arc] (N07) -- (N03);}
      \uncover<all:1>{\draw[red arc] (N04) -- (N08);}
      \uncover<all:1>{\draw[red arc] (N08) -- (N07);}
      \uncover<all:1>{\draw[red arc] (N09) -- (N05);}
      \uncover<all:1>{\draw[red arc] (N01) -- (N06);}
      \uncover<all:1,7,8,9>{\draw[red arc] (N05) -- (N06);}
      \uncover<all:1,2,3,4,5,6,7,8,9,10>{\draw[red arc] (N10) -- (N11);}
      \uncover<all:1,3,4,5>{\draw[red arc] (N11) -- (N06);}
      \uncover<all:1,4,5,8,9>{\draw[red arc] (N06) -- (N12);}

      \node[draw, ellipse, dashed, minimum height=8mm, minimum width=25mm] (goal) at (0,0) [label=below:$S_\star$] {};
    \end{tikzpicture}
  \end{center}
\end{frame}

\section{Regression}
\subsection{Overview}

\begin{frame}{Forward search vs.\ backward search}
  Going through a transition graph in forward and backward directions
  is \alert{not symmetric}:
  \begin{itemize}
  \item forward search starts from a \alert{single}
    initial state; \\
    backward search starts from a \alert{set} of goal states
  \item when applying an operator $o$ in a state $s$ in forward
    direction, there is a \alert{unique successor state} $s'$; \\
    if we applied operator $o$ to end up in state $s'$, \\
    there can be \alert{several possible predecessor states} $s$
  \end{itemize}

  $\leadsto$ most natural representation for backward search in
  planning associates \alert{sets of states} with search nodes
\end{frame}

\begin{frame}{Planning by backward search: regression}
  \alert{Regression:} Computing the possible predecessor states
  $\regr{o}{G}$ of a set of states $G$ with respect to the last
  operator $o$ that was applied.

  \medskip

  \alert{Regression planners} find solutions by backward search:
  \begin{itemize}
  \item start from set of goal states
  \item iteratively pick a previously generated state set and \\
    \alert{regress it} through an operator, generating a new state set
  \item solution found when a generated state set includes the initial
    state
  \end{itemize}

  \medskip

  \hilite{Pro:} can handle many states simultaneously \\
  \hilite{Con:} basic operations complicated and expensive
\end{frame}

\begin{frame}{Search space representation in regression planners}
  identify state sets with \alert{logical formulae} (again):
  \begin{itemize}
  \item \alert{search nodes correspond to state sets}
  \item each state set is represented by a \alert{logical formula}: \\
    $\varphi$ represents $\{s \in S \mid s \models \varphi\}$
  \item many basic search operations like detecting duplicates
    are NP-hard or coNP-hard
  \end{itemize}
\end{frame}

\subsection{Example}

\begin{frame}{Regression planning example (depth-first search)}
  \begin{center}
    \begin{tikzpicture}
      \tikzstyle node style=[fill, circle]

      \uncover<all:1,2>{\node[node style] (N00) at (0:6mm) {};}
      \uncover<all:1,2>{\node[node style] (N01) at (180:6mm) {};}

      \uncover<all:1,4>{\node[node style] (N02) at (0:22mm) {};}
      \uncover<all:1,3,5>{\node[node style] (N04) at (60:22mm) {};}
      \uncover<all:1,3>{\node[node style] (N05) at (120:22mm) {};}
      \uncover<all:1>{\node[node style] (N06) at (180:22mm) {};}

      \uncover<all:1>{\node[node style] (N03) at (0:37mm) {};}
      \uncover<all:1,4>{\node[node style] (N07) at (30:37mm) {};}
      \uncover<all:1,4,5>{\node[node style] (N08) at (60:37mm) {};}
      \uncover<all:1,4>{\node[node style] (N09) at (90:37mm) {};}
      \uncover<all:1,5>{\node[node style] (N10) at (120:37mm)[label=110:$I$] {};}
      \uncover<all:1,4>{\node[node style] (N11) at (150:37mm) {};}
      \uncover<all:1>{\node[node style] (N12) at (180:37mm) {};}
      
      \tikzstyle blue arc=[->, thick, >=latex, blue]
      \tikzstyle red arc=[->, thick, >=latex, red]

      \uncover<all:1,3->{\draw[blue arc] (N04) -- (N00);}
      \uncover<all:1,4->{\draw[blue arc] (N02) -- (N04);}
      \uncover<all:1>{\draw[blue arc] (N03) -- (N02);}
      \uncover<all:1,4->{\draw[blue arc] (N07) -- (N04);}
      \uncover<all:1,4->{\draw[blue arc, bend left=40] (N08) to (N04);}
      \uncover<all:1,4->{\draw[blue arc] (N09) -- (N04);}
      \uncover<all:1,3->{\draw[blue arc] (N05) -- (N01);}
      \uncover<all:1>{\draw[blue arc] (N12) -- (N11);}
      \uncover<all:1>{\draw[blue arc] (N10) -- (N09);}
      \uncover<all:1,4->{\draw[blue arc] (N11) -- (N05);}

      \uncover<all:1>{\draw[red arc] (N02) -- (N00);}
      \uncover<all:1>{\draw[red arc] (N07) -- (N03);}
      \uncover<all:1,5->{\draw[red arc] (N04) -- (N08);}
      \uncover<all:1,5->{\draw[red arc] (N08) -- (N07);}
      \uncover<all:1>{\draw[red arc] (N09) -- (N05);}
      \uncover<all:1>{\draw[red arc] (N01) -- (N06);}
      \uncover<all:1>{\draw[red arc] (N05) -- (N06);}
      \uncover<all:1,5->{\draw[red arc] (N10) -- (N11);}
      \uncover<all:1>{\draw[red arc] (N11) -- (N06);}
      \uncover<all:1>{\draw[red arc] (N06) -- (N12);}
      
      \node[draw, ellipse, dashed, minimum height=8mm, minimum width=25mm] (goal) at (0,0) [label=below:$\gamma$] {};
      
      %% depiction of the plan
      \visible<all:2->{\node (P1) at (4,5.5) {$\gamma$};}
      \visible<all:3->{\node (P2) at (2.5,5.5) {$\varphi_1$};}
      \visible<all:3->{\node[anchor=west] at (-4.5,5.5) {$\varphi_1=\regr{\textblue{\longrightarrow}}{\gamma}$};}
      \visible<all:4->{\node (P3) at (1,5.5) {$\varphi_2$};}
      \visible<all:4->{\node[anchor=west] at (-4.5,5.0) {$\varphi_2=\regr{\textblue{\longrightarrow}}{\varphi_1}$};}
      \visible<all:5->{\node (P4) at (-0.5,5.5) {$\varphi_3$};}
      \visible<all:5->{\node[anchor=west] at (-4.5,4.5) {$\varphi_3=\regr{\textred{\longrightarrow}}{\varphi_2}, I\models\varphi_3$};}
      
      \visible<all:3->{\draw[blue arc] (P2)--(P1);}
      \visible<all:4->{\draw[blue arc] (P3)--(P2);}
      \visible<all:5->{\draw[red arc] (P4)--(P3);}
    \end{tikzpicture}
  \end{center}
\end{frame}

\subsection[STRIPS]{Regression for STRIPS tasks}

\begin{frame}{Regression for STRIPS planning tasks}
  \begin{definition}[STRIPS planning task]
    A planning task is a \alert{STRIPS planning task} if all operators
    are STRIPS operators and the goal is a conjunction of atoms.
  \end{definition}

  \medskip

  Regression \alert{for STRIPS planning tasks} is very simple:
  \begin{itemize}
  \item Goals are conjunctions of atoms $a_1 \land \dots \land a_n$.
  \item \alert{First step}: Choose an operator that makes
    none of $a_1, \dots, a_n$ false.
  \item \alert{Second step}: Remove goal atoms achieved by the
    operator (if any) and add its preconditions.
  \item[$\leadsto$] Outcome of regression is again conjunction of
    atoms.
  \end{itemize}
  \hilite{Optimization:} only consider operators making some $a_i$
  true
\end{frame}

\begin{frame}{STRIPS regression}
  \begin{definition}[STRIPS regression]
    Let $\varphi = \varphi_1 \land \dots \land \varphi_n$ be a
    conjunction of atoms, and \\ let $o = \langle \chi, e\rangle$ be a
    STRIPS operator which adds the atoms $a_1, \dots, a_k$ and deletes
    the atoms $d_1, \dots, d_l$.

    \medskip

    The \alert{STRIPS regression} of $\varphi$ with respect to $o$ is
    \[
      \regrstrips{o}{\varphi} := \begin{cases}
        \bot \qquad \text{if~} a_i = d_j \text{~for some~} i, j \\
        \bot \qquad \text{if~} \varphi_i = d_j \text{~for some~} i, j \\
        \chi \land \bigwedge (\{\varphi_1, \dots,
        \varphi_n\} \setminus \{a_1, \dots, a_k\})
        & \text{otherwise}
      \end{cases}
    \]
  \end{definition}
  \hilite{Note:} $\regrstrips{o}{\varphi}$ is again a conjunction of
  atoms, or $\bot$.
\end{frame}

\begin{frame}{STRIPS regression example}
  \begin{center}
    \begin{pgfpicture}{5mm}{-2mm}{75mm}{15mm}
      \pgfputat{\pgfxy(0,-0.5)}{\pgfbox[left,left]{\footnotesize
          \alert{Note}: Predecessor states are in general not unique.}}
      \pgfputat{\pgfxy(0,-0.9)}{\pgfbox[left,left]{\footnotesize
          This picture is just for illustration purposes.}}
      \pgfputat{\pgfxy(7,0)}{\pgfbox[center,right]{\RGB}}
      \pgfputat{\pgfxy(6,0)}{\pgfbox[center,right]{$o_3$}}
      \pgfputat{\pgfxy(5,0)}{\pgfbox[center,right]{\RsGB}}
      \pgfputat{\pgfxy(4,0)}{\pgfbox[center,right]{$o_2$}}
      \pgfputat{\pgfxy(3,0)}{\pgfbox[center,right]{\GRsB}}
      \pgfputat{\pgfxy(2,0)}{\pgfbox[center,right]{$o_1$}}
      \pgfputat{\pgfxy(1,0)}{\pgfbox[center,right]{\BGR}}
    \end{pgfpicture}
  \end{center}

  \newcommand{\CLEARR}{\textred{\blacksquare}\var{clr}}
  \newcommand{\CLEARG}{\textgreen{\blacksquare}\var{clr}}
  \newcommand{\CLEARBl}{\textblue{\blacksquare}\var{clr}}
  \newcommand{\RONTABLE}{\textred{\blacksquare}\var{onT}}
  \newcommand{\GONTABLE}{\textgreen{\blacksquare}\var{onT}}
  \newcommand{\BlONTABLE}{\textblue{\blacksquare}\var{onT}}
  \newcommand{\RONG}{\textred{\blacksquare}\var{on}\textgreen{\blacksquare}}
  \newcommand{\RONB}{\textred{\blacksquare}\var{on}\textblue{\blacksquare}}
  \newcommand{\BONG}{\textblue{\blacksquare}\var{on}\textgreen{\blacksquare}}
  \newcommand{\BONR}{\textblue{\blacksquare}\var{on}\textred{\blacksquare}}
  \newcommand{\GONB}{\textgreen{\blacksquare}\var{on}\textblue{\blacksquare}}
  \newcommand{\GONR}{\textgreen{\blacksquare}\var{on}\textred{\blacksquare}}
  
  \[
  \begin{array}{@{}r@{\,}c@{\,}l@{\quad}l@{}}
      o_1 & = & \langle \BONG \land \CLEARBl,
      & \neg \BONG \land \BlONTABLE \land \CLEARG \rangle \\
      o_2 & = & \langle \GONR \land \CLEARG \land \CLEARBl,
      & \neg \CLEARBl \land \neg \GONR \land \GONB \land \CLEARR \rangle \\
      o_3 & = & \langle \RONTABLE \land \CLEARR \land \CLEARG,
      & \neg \CLEARG \land \neg \RONTABLE \land \RONG \rangle \\[\medskipamount]
      \gamma & = & \RONG \land \GONB \\
      \varphi_1 & = &
      \multicolumn{2}{l}{
        \regrstrips{o_3}{\gamma} =
        \RONTABLE \land \CLEARR \land \CLEARG \land \GONB
      } \\
      \varphi_2 & = &
      \multicolumn{2}{l}{
        \regrstrips{o_2}{\varphi_1} =
        \GONR \land \CLEARG \land \CLEARBl \land \RONTABLE
      } \\
      \varphi_3 & = &
      \multicolumn{2}{l}{
        \regrstrips{o_1}{\varphi_2} =
        \BONG \land \CLEARBl \land \GONR \land \RONTABLE
      }
    \end{array}
    \]
\end{frame}

\subsection[General case]{Regression for general planning tasks}

\begin{frame}{Regression for general planning tasks}
  \begin{itemize}
  \item With disjunctions and conditional effects, things become more
    tricky. How to regress $a \lor (b \land c)$ with respect to $\langle
    q, d \CEF b\rangle$?
  \item The story about goals and subgoals and fulfilling subgoals, as
    in the STRIPS case, is no longer useful.
  \item We present a general method for doing regression for any
    formula and any operator.
  \item Now we extensively use the idea of representing sets of states
    as formulae.
  \end{itemize}
\end{frame}

\begin{frame}{Effect preconditions}
  \begin{definition}[effect precondition]
    The \alert{effect precondition $\eprecon{l}{e}$} for literal $l$
    and effect $e$ is defined as follows:
    \[
      \begin{array}{rcl}
        \eprecon{l}{l} & = & \top \\
        \eprecon{l}{l'} & = & \bot \text{~if~} l \neq l'
        \quad(\text{for literals~} l') \\
        \eprecon{l}{e_1 \land \dots \land e_n} & = &
        \eprecon{l}{e_1} \lor \dots \lor \eprecon{l}{e_n} \\
        \eprecon{l}{\chi \CEF e} & = & \eprecon{l}{e} \land \chi \\
      \end{array}
    \]
  \end{definition}
  %% TODO: We might want to replace the right-hand side of the last
  %% rule by c \land \eprecon{l}{e}, because then the relationship
  %% between e and EPC_l(e) would be very clear syntactically:
  %% replace l by \top, replace all other literals by \bot,
  %% replace \land by \lor, replace \rhd by \land.
  \hilite{Intuition:} $\eprecon{l}{e}$ describes the situations in which
  effect $e$ causes literal $l$ to become true.
\end{frame}

\begin{frame}{Effect precondition examples}
  \begin{example}
    \[
      \begin{array}{rcl}
        \eprecon{a}{b \land c} & =
        & \bot \lor \bot \equiv \bot \\
        \eprecon{a}{a \land (b \CEF a)} & =
        & \top \lor (\top \land b) \equiv \top \\
        \eprecon{a}{(c \CEF a) \land (b \CEF a)} & = &
        (\top \land c) \lor (\top \land b) \equiv c \lor b \\
      \end{array}
    \]
  \end{example}
\end{frame}

\begin{frame}{Effect preconditions: connection to change sets}
  \begin{lemma}[A]
    Let $s$ be a state, $l$ a literal and $e$ an effect.

    Then $l \in \changes{e}{s}$ if and only if $s \models
    \eprecon{l}{e}$.
  \end{lemma}

  \begin{proofstart}
    Induction on the structure of the effect $e$.
    \pause

    Base case 1, $e = l$:
    $l \in \changes{l}{s} = \{l\}$ by definition,
    and $s \models \eprecon{l}{l} = \top$ by definition.
    Both sides of the equivalence are true.
    \pause

    Base case 2, $e = l'$ for some literal $l'\neq l$:
    $l \notin \changes{l'}{s} = \{l'\}$ by definition,
    and $s \not\models \eprecon{l}{l'} = \bot$ by definition.
    Both sides are false.
  \end{proofstart}
\end{frame}

\begin{frame}{Effect preconditions: connection to change sets}
  \begin{proofend}
    Inductive case 1, $e = e_1 \land \dots \land e_n$: \\
    \begin{tabularx}{\textwidth}{l@{\,}X}
      $l \in \changes{e}{s}$
      & iff
      $l \in \changes{e_1}{s} \cup \dots \cup \changes{e_n}{s}$
      \hfill (\alert{Def $\changes{e_1 \land \dots \land e_n}{s}$}) \\
      & iff
      $l \in \changes{e'}{s}$ for some $e'\in \{e_1, \dots, e_n\}$ \\
      & iff
      $s \models \eprecon{l}{e'}$ for some $e' \in \{e_1, \dots, e_n\}$
      \hfill (\alert{IH}) \\
      & iff
      $s \models \eprecon{l}{e_1} \lor \dots \lor \eprecon{l}{e_n}$ \\
      & iff
      $s \models \eprecon{l}{e_1 \land \dots \land e_n}$.
      \hfill (\alert{Def $\textit{EPC}$})
    \end{tabularx}
    \pause
    \medskip

    Inductive case 2, $e = \chi \CEF e'$: \\
    \begin{tabularx}{\textwidth}{l@{\,}X}
      $l \in \changes{\chi \CEF e'}{s}$
      & iff
      $l \in \changes{e'}{s}$ and $s \models \chi$
      \hfill (\alert{Def $\changes{\chi \CEF e'}{s}$}) \\
      & iff
      $s \models \eprecon{l}{e'}$ and $s \models \chi$
      \hfill (\alert{IH}) \\
      & iff
      $s \models \eprecon{l}{e'} \land \chi$ \\
      & iff
      $s \models \eprecon{l}{\chi \CEF e'}$.
      \hfill (\alert{Def $\textit{EPC}$})
    \end{tabularx}
  \end{proofend}
\end{frame}

\begin{frame}{Effect preconditions: connection to normal form}
  \begin{block}{Remark: $\textit{EPC}$ vs.\ effect normal form}
    Notice that in terms of $\eprecon{a}{e}$, any operator $\langle
    \chi,e\rangle$ can be expressed in effect normal form as
    \[
      \left\langle \chi, \bigwedge_{a \in A} \left(\left(
      \eprecon{a}{e} \CEF a \right) \land \left(
      \eprecon{\neg a}{e} \CEF \neg a\right) \right) \right\rangle,
    \]
    where $A$ is the set of all state variables.
  \end{block}
\end{frame}

\begin{frame}{Regressing state variables}
  The formula
  $\eprecon{a}{e} \lor (a \land \neg \eprecon{\neg a}{e})$
  expresses \\
  the \alert{value of state variable $a \in A$ after applying $o$} \\
  in terms of \alert{values of state variables before applying $o$}.
  
  \bigskip
    
  Either:
  \begin{itemize}
  \item $a$ became true, or
  \item $a$ was true before and it did not become false.
  \end{itemize}
\end{frame}

\begin{frame}{Regressing state variables: examples}
  \begin{example}
    Let $e = (b \CEF a) \land (c \CEF \neg a) \land b \land \neg d$.
    \[
      \begin{array}{l|l}
        \text{variable~} x
        & \eprecon{x}{e} \lor (x \land
        \neg \eprecon{\neg x}{e}) \\ \hline
        a & b \lor (a \land \neg c) \\
        b & \top \lor(b \land \neg \bot) \equiv \top \\
        c & \bot \lor(c \land \neg \bot) \equiv c \\
        d & \bot \lor(d \land \neg \top) \equiv \bot
      \end{array}
    \]
  \end{example}
\end{frame}

\begin{frame}{Regressing state variables: correctness}
  \begin{lemma}[B]
    Let $a$ be a state variable, $o = \langle \chi, e\rangle$
    an operator, \\
    $s$ a state, and $s' = \applyop{o}{s}$.

    Then $s \models \eprecon{a}{e} \lor (a \land \neg \eprecon{\neg
      a}{e})$ if and only if $s'\models a$.
  \end{lemma}

  \pause

  \begin{proofstart}
    \hilite{$(\Rightarrow)$:} Assume $s \models \eprecon{a}{e} \lor (a
    \land \neg \eprecon{\neg a}{e})$.

    Do a case analysis on the two disjuncts.
    \pause

    \begin{enumerate}
    \item Assume that $s \models \eprecon{a}{e}$.
      By Lemma A, we have $a \in \changes{e}{s}$ and hence $s' \models
      a$.
      \pause
    \item Assume that $s \models a \land \neg \eprecon{\neg a}{e}$.
      By Lemma A, we have $\neg a \notin \changes{e}{s}$.
      Hence $a$ remains true in $s'$.
    \end{enumerate}
  \end{proofstart}
\end{frame}

\begin{frame}{Regressing state variables: correctness}
  \begin{proofend}
    \hilite{$(\Leftarrow)$:} We showed that if the formula is
    \alert{true} in $s$, then $a$ is \alert{true} in $s'$.
    For the second part, we show that if the formula is \alert{false}
    in $s$, then $a$ is \alert{false} in $s'$. \pause

    \begin{itemize}
    \item So assume
      $s \not\models \eprecon{a}{e} \lor (a \land \neg \eprecon{\neg
      a}{e})$. \pause
    \item Then $s \models \neg \eprecon{a}{e} \land (\neg a \lor
      \eprecon{\neg a}{e})$ (de Morgan). \pause
    \item Case distinction: $a$ is true or $a$ is false in $s$.
      \pause
      \begin{enumerate}
      \item Assume that $s \models a$.
        Now $s \models \eprecon{\neg a}{e}$ because $s \models \neg a
        \lor \eprecon{\neg a}{e}$.

        Hence by Lemma A $\neg a \in \changes{e}{s}$ and we get $s'
        \not\models a$. \pause
      \item Assume that $s\not\models a$.
        Because $s \models \neg \eprecon{a}{e}$, by Lemma A
        we get $a \notin \changes{e}{s}$ and hence $s' \not\models a$.
      \end{enumerate}
      \pause
      Therefore in both cases $s' \not\models a$.
    \end{itemize}
  \end{proofend}
\end{frame}

\begin{frame}{Regression: general definition}
  We base the definition of regression on formulae $\eprecon{l}{e}$.

  \begin{definition}[general regression]
    Let $\varphi$ be a propositional formula and $o = \langle \chi,
    e\rangle$ an operator.

    The \alert{regression of $\varphi$ with respect to $o$} is
    \[ \regr{o}{\varphi} = \chi \land \varphi_{\text{r}} \land \kappa \]
    where
    \begin{enumerate}
    \item $\varphi_{\text{r}}$ is obtained from $\varphi$ by replacing
      each $a \in A$ by $\eprecon{a}{e} \lor
      (a \land \neg \eprecon{\neg a}{e})$, and
    \item $\kappa = \bigwedge_{a \in A} \neg (\eprecon{a}{e} \land 
      \eprecon{\neg a}{e})$.
    \end{enumerate}
  \end{definition}

  The formula $\kappa$ expresses that operators are only applicable in
  states where their change sets are consistent.
\end{frame}

\begin{frame}{Regression examples}
  \begin{itemize}
  \item $\regr{\langle a,b\rangle}{b}
    \equiv a \land (\top \lor (b \land \neg \bot)) \land \top
    \equiv a$
  \item $\regr{\langle a,b\rangle}{b \land c \land d}$ \\
    $\equiv a \land (\top \lor (b \land \neg \bot)) \land (\bot \lor
    (c \land \neg \bot)) \land (\bot \lor (d \land \neg \bot)) \land
    \top$ \\
    $\equiv a \land c \land d$
  \item $\regr{\langle a, c \CEF b\rangle}{b}
    \equiv a \land (c \lor (b \land \neg \bot)) \land \top
    \equiv a \land (c \lor b)$
  \item $\regr{\langle a,(c \CEF b) \land (b \CEF \neg b)\rangle}{b}
    \equiv a \land (c \lor (b \land \neg b)) \land \neg (c \land b)$ \\
    $\equiv a \land c \land \neg b$
  \item $\regr{\langle a, (c \CEF b) \land (d \CEF \neg b)\rangle}{b}
    \equiv a \land (c \lor (b \land \neg d)) \land \neg (c \land d)$ \\
    $\equiv a \land (c \lor b) \land (c \lor \neg d) \land (\neg c
    \lor \neg d)$ \\
    $\equiv a \land (c \lor b) \land \neg d$
  \end{itemize}
\end{frame}

\begin{frame}{Regression example: binary counter}
  \[
    \begin{array}{r@{\,}c@{\,}l}
      (\neg b_0 & \CEF & b_0) \land{} \\
      ((\neg b_1 \land b_0) & \CEF & (b_1 \land \neg b_0)) \land{} \\
      ((\neg b_2 \land b_1 \land b_0) & \CEF & (b_2 \land \neg b_1
      \land \neg b_0))
    \end{array}
  \]

  \begin{small}
    \[
      \begin{array}{r@{\,}c@{\,}l}
        \eprecon{b_2}{e} & = & \neg b_2 \land b_1 \land b_0 \\
        \eprecon{b_1}{e} & = & \neg b_1 \land b_0 \\
        \eprecon{b_0}{e} & = & \neg b_0 \\
        \eprecon{\neg b_2}{e} & = & \bot \\
        \eprecon{\neg b_1}{e} & = & \neg b_2 \land b_1 \land b_0 \\
        \eprecon{\neg b_0}{e} & = &
        (\neg b_1 \land b_0) \lor (\neg b_2 \land b_1 \land b_0)
        \equiv (\neg b_1 \lor \neg b_2) \land b_0
      \end{array}
    \]
  \end{small}
  Regression replaces state variables as follows:
  \[
    \begin{array}{rcl}
      b_2 & \text{by} &
      (\neg b_2 \land b_1 \land b_0) \lor (b_2 \land \neg \bot)
      \equiv \alert{(b_1 \land b_0) \lor b_2} \\
      b_1 & \text{by} &
      (\neg b_1 \land b_0)
      \lor
      (b_1 \land \neg (\neg b_2 \land b_1 \land b_0)) \\
      && \equiv \alert{
        (\neg b_1 \land b_0)
        \lor
        (b_1 \land(b_2 \lor \neg b_0))} \\
      b_0 & \text{by} &
      \neg b_0
      \lor
      (b_0 \land \neg ((\neg b_1 \lor \neg b_2) \land b_0))
      \equiv \alert{\neg b_0 \lor (b_1 \land b_2)}
    \end{array}
    \]
\end{frame}

\begin{frame}{General regression: correctness}
  \begin{theorem}[correctness of $\regr{o}{\varphi}$]
    Let $\varphi$ be a formula,
    $o$ an operator and
    $s$ a state.

    Then $s \models \regr{o}{\varphi}$
    iff $o$ is applicable in $s$ and $\applyop{o}{s} \models \varphi$.
  \end{theorem}
  \pause

  \begin{proofstart}
    Let $o = \langle \chi, e\rangle$. Recall that $\regr{o}{\varphi} =
    \chi \land \varphi_{\text{r}} \land \kappa$, where
    $\varphi_{\text{r}}$ and $\kappa$ are as defined previously.

    \medskip
    \pause

    If $o$ is inapplicable in $s$, then $s \not\models \chi \land
    \kappa$, both sides of the ``iff'' condition are false, and we are
    done. Hence, we only further consider states $s$ where $o$ is
    applicable. Let $s' := \applyop{o}{s}$.

    \medskip
    \pause

    We know that $s \models \chi \land \kappa$ (because $o$ is
    applicable), so the ``iff'' condition we need to prove simplifies
    to:
    \[s \models \varphi_{\text{r}} \text{~iff~} s' \models \varphi.\]
  \end{proofstart}
\end{frame}

\begin{frame}{General regression: correctness}
  \begin{proofmid}
    \hilite{To show:} $s \models \varphi_{\text{r}}$
    iff $s' \models \varphi$.

    \medskip
    \pause

    We show that for all formulae $\psi$, $s \models \psi_{\text{r}}$
    iff $s' \models \psi$, where $\psi_{\text{r}}$ is $\psi$ with
    every $a \in A$ replaced by $\eprecon{a}{e} \lor (a \land \neg
    \eprecon{\neg a}{e})$.

    \pause
    \medskip

    The proof is by structural induction on $\psi$.

    \begin{description}[Induction hypothesis:]
    \item[Induction hypothesis]
      $s \models \psi_{\text{r}}$ if and only if $s' \models \psi$.
      \pause
    \item[Base cases 1 \& 2] $\psi = \top$ or $\psi = \bot$:
      trivial, as $\psi_{\text{r}} = \psi$.
      \pause
    \item[Base case 3] $\psi = a$ for some $a \in A$: \\
      Then $\psi_{\text{r}} = \eprecon{a}{e} \lor
      (a \land \neg \eprecon{\neg a}{e})$. \\
      By Lemma B, $s \models \psi_{\text{r}}$ iff $s' \models \psi$.
    \end{description}
  \end{proofmid}
\end{frame}

\begin{frame}{General regression: correctness}
  \begin{proofend}
    \begin{description}[Inductive case 3]
    \item[Inductive case 1] $\psi = \neg \psi'$:
      \vspace{-3mm}
      \begin{align*}
        s \models \psi_{\text{r}}
        & \text{ iff } s \models (\neg \psi')_{\text{r}}
        \text{ iff } s \models \neg(\psi'_{\text{r}})
        \text{ iff } s \not\models \psi'_{\text{r}}\\
        & \text{ iff (\alert{IH}) } s' \not\models \psi'
        \text{ iff } s' \models \neg\psi'
        \text{ iff } s' \models \psi
      \end{align*}
      \vspace{-8mm}
      \pause
    \item[Inductive case 2] $\psi = \psi' \lor \psi''$:
      \vspace{-3mm}
      \begin{align*}
        s \models \psi_{\text{r}}
        & \text{ iff } s \models (\psi' \lor \psi'')_{\text{r}}
        \text{ iff } s \models \psi'_{\text{r}} \lor \psi''_{\text{r}}\\
        & \text{ iff } s \models \psi'_{\text{r}} \text{ or } s \models \psi''_{\text{r}}\\
        & \text{ iff (\alert{IH, twice}) } s' \models \psi' \text{ or } s' \models \psi''\\
        & \text{ iff } s' \models \psi' \lor \psi''
        \text{ iff } s' \models \psi
      \end{align*}
      \vspace{-8mm}
      \pause
    \item[Inductive case 3] $\psi = \psi' \land \psi''$: Very similar
      to inductive case 2, just with $\land$ instead of $\lor$ and
      ``and'' instead of ``or''.
    \end{description}
  \end{proofend}
  %% TODO (Kommentare von Jan):
  %% Folie 69: Ich finde den Beweis nicht ganz optimal:
  %% Die Einführung von psi anstelle von phi erscheint mir
  %% überflüssig.
  %% "over subformulae psi' of" ebenso. Ich würde einfach schreiben:
  %% The proof is by structural induction on phi. [erledigt (RM, 2011-11-10)]
  %% Die inductive cases 2 und 3 sind aus zwei Gründen etwas unglücklich:
  %% erstens ist hier auf einmal von einem psi'' die Rede wo doch vorher
  %% nur von psi' die Rede war [hat sich mit der aenderung von oben erledigt
  %% (RM, 2011-11-10)],
  %% zweitens benutzt Du ' zum einen für
  %% Subformeln von psi, zum zweiten für den Nachfolgerzustand s'. 
\end{frame}

\subsection{Practical issues}

\begin{frame}{Emptiness and subsumption testing}
  The following two tests are useful when performing regression
  searches to avoid exploring unpromising branches:
  
  \begin{itemize}
  \item Test that $\regr{o}{\varphi}$ does not represent the
    empty set \\ (which would mean that search is in a dead end).
    
    For example, $\regr{\langle a, \neg p\rangle}{p}
    \equiv a \land \bot \equiv \bot$.
  \item Test that $\regr{o}{\varphi}$ does not represent a subset of
    $\varphi$ \\
    (which would make the problem harder than before).

    For example, $\regr{\langle b, c\rangle}{a} \equiv a \land b$.
  \end{itemize}

  Both of these problems are \alert{NP-hard}.
\end{frame}

\begin{frame}{Formula growth}
  The formula
  $\regr{o_1}{\regr{o_2}{\dots \regr{o_{n-1}}{\regr{o_n}{\varphi}}}}$
  may have size $O(|\varphi||o_1||o_2|\dots|o_{n-1}||o_n|)$, \ie,
  the product of the sizes of $\varphi$ and the operators.

  $\leadsto$ worst-case \alert{exponential} size $O(m^n)$

  \begin{block}{Logical simplifications}
    \begin{itemize}
    \item $\bot \land \varphi \equiv \bot$,
      $\top \land \varphi \equiv \varphi$,
      $\bot \lor \varphi \equiv\varphi$,
      $\top \lor \varphi \equiv\top$
    \item $a \lor \varphi \equiv a \lor \varphi[\bot/a]$,
      $\neg a \lor \varphi \equiv \neg a \lor \varphi[\top/a]$,
      $a \land \varphi \equiv a \land \varphi[\top/a]$,
      $\neg a \land \varphi \equiv \neg a \land \varphi[\bot/a]$
    \item idempotency, absorption, commutativity, associativity,
      $\dots$
    \end{itemize}
  \end{block}
\end{frame}

\begin{frame}{Restricting formula growth in search trees}
  \begin{description}[Problem]
  \item[Problem] very big formulae obtained by regression
  \item[Cause] \alert{disjunctivity} in the (NNF) formulae

    (formulae \alert{without disjunctions} easily convertible to small
    formulae $l_1 \land \dots \land l_n$ where $l_i$ are literals and
    $n$ is at most the number of state variables.)
  \item[Idea] handle disjunctivity when generating search trees
  \end{description}
\end{frame}

\begin{frame}{Unrestricted regression: search tree example}
  \alert{Unrestricted regression:} do not treat disjunctions specially

  \bigskip

  Goal $\gamma = a \land b$, initial state $I = \{ a \mapsto 0, b \mapsto 0,
    c \mapsto 0\}$.
  \begin{pgfpicture}{0cm}{0cm}{10.2cm}{5cm}
    \pgfsetlinewidth{0.7pt}
    
    \pgfnodebox{N1}[virtual]{\pgfxy(9.3,3)}{
      $\gamma = a\land b$}{1pt}{1pt}
    \pgfnodebox{N1o}[virtual]{\pgfxy(8.7,3.2)}{}{1pt}{1pt}
    \pgfnodebox{N1u}[virtual]{\pgfxy(8.7,2.8)}{}{1pt}{1pt}
    \pgfnodebox{N2}[virtual]{\pgfxy(5.5,4.5)}{
      \alert{$\neg a\land a$}}{1pt}{1pt}
    \pgfnodebox{N3}[virtual]{\pgfxy(5.5,1.5)}{
      $(\neg c\lor a)\land b$}{1pt}{1pt}
    \pgfnodebox{N4}[virtual]{\pgfxy(1.2,0.5)}{
      \textgreen{$(\neg c\lor a)\land\neg a$}}{1pt}{1pt}
    \pgfnodebox{N5}[virtual]{\pgfxy(1.2,2.5)}{
      $(\neg c\lor a)\land b$}{1pt}{1pt}
    
    \pgfsetendarrow{\pgfarrowtriangle{6pt}}
    
    \pgfnodeconnline{N2}{N1o}
    \pgfnodeconnline{N3}{N1u}
    \pgfnodeconnline{N4}{N3}
    \pgfnodeconnline{N5}{N3}
    
    \pgfnodelabelrotated{N2}{N1}[0.5][1.5mm]{\pgfbox[center,base]{
        $\langle \neg a,b\rangle$}}
    \pgfnodelabelrotated{N3}{N1}[0.5][1.5mm]{\pgfbox[center,base]{
        $\langle b,\neg c\CEF a\rangle$}}
    \pgfnodelabelrotated{N4}{N3}[0.5][1.5mm]{\pgfbox[center,base]{
        $\langle \neg a,b\rangle$}}
    \pgfnodelabelrotated{N5}{N3}[0.5][1.5mm]{\pgfbox[center,base]{
        $\langle b,\neg c\CEF a\rangle$}}
  \end{pgfpicture}
\end{frame}

\begin{frame}{Full splitting: search tree example}
  \alert{Full splitting:} always remove all disjunctivity

  \bigskip

  Goal $\gamma = a \land b$, initial state $I = \{ a \mapsto 0, b \mapsto 0,
    c \mapsto 0\}$.

  $(\neg c \lor a) \land b$ in DNF:
  $(\neg c \land b) \lor (a \land b)$ \\
  $\leadsto$ split into $\neg c\land b$ and $a\land b$

  \begin{pgfpicture}{0cm}{0cm}{10.2cm}{5cm}
    \pgfsetlinewidth{0.7pt}

    \pgfnodebox{N1}[virtual]{\pgfxy(9.3,3)}{$\gamma = a\land b$}{1pt}{1pt}
    \pgfnodebox{N1o}[virtual]{\pgfxy(8.5,3.2)}{}{1pt}{1pt}
    \pgfnodebox{N1u}[virtual]{\pgfxy(8.5,2.8)}{}{1pt}{1pt}
    \pgfnodebox{N2}[virtual]{\pgfxy(5.5,4.5)}{\alert{$\neg a\land a$}}{1pt}{1pt}
    \pgfnodebox{N3a}[virtual]{\pgfxy(5.5,1)}{$\neg c\land b$}{1pt}{1pt}
    \pgfnodebox{N3b}[virtual]{\pgfxy(4,3)}{(duplicate of $\gamma$) $a\land b$}{1pt}{1pt}
    \pgfnodebox{N4a}[virtual]{\pgfxy(0.7,0.5)}{\textgreen{$\neg c\land\neg a$}}{1pt}{1pt}
    \pgfnodebox{N5a}[virtual]{\pgfxy(0.7,1.5)}{$\neg c\land b$}{1pt}{1pt}
    %\pgfnodebox{N4b}[virtual]{\pgfxy(0.7,2.5)}{\alert{$a\land\neg a$}}{1pt}{1pt}
    %\pgfnodebox{N5ba}[virtual]{\pgfxy(0.7,3.5)}{$a\land b$}{1pt}{1pt}
    %\pgfnodebox{N5bb}[virtual]{\pgfxy(0.7,4.5)}{$\neg c\land b$}{1pt}{1pt}
    
    \pgfsetendarrow{\pgfarrowtriangle{6pt}}
    
    \pgfnodeconnline{N2}{N1o}
    \pgfnodeconnline{N3a}{N1u}
    \pgfnodeconnline{N3b}{N1}
    \pgfnodeconnline{N4a}{N3a}
    %\pgfnodeconnline{N4b}{N3b}
    \pgfnodeconnline{N5a}{N3a}
    %\pgfnodeconnline{N5ba}{N3b}
    %\pgfnodeconnline{N5bb}{N3b}
    
    \pgfnodelabelrotated{N2}{N1}[0.35][1mm]{\pgfbox[center,base]{
        $\langle \neg a,b\rangle$}}
    \pgfnodelabelrotated{N3a}{N1}[0.35][1mm]{\pgfbox[center,base]{
        $\langle b,\neg c\CEF a\rangle$}}
    \pgfnodelabelrotated{N3b}{N1}[0.35][1mm]{\pgfbox[center,base]{
        $\langle b,\neg c\CEF a\rangle$}}
    \pgfnodelabelrotated{N4a}{N3a}[0.3][1mm]{\pgfbox[center,base]{
        $\langle \neg a,b\rangle$}}
    %\pgfnodelabelrotated{N4b}{N3b}[0.3][1mm]{\pgfbox[center,base]{
    %    $\langle \neg a,b\rangle$}}
    \pgfnodelabelrotated{N5a}{N3a}[0.3][1mm]{\pgfbox[center,base]{
        $\langle b,\neg c\CEF a\rangle$}}
    %\pgfnodelabelrotated{N5ba}{N3b}[0.3][1mm]{\pgfbox[center,base]{
    %    $\langle b,\neg c\CEF a\rangle$}}
    %\pgfnodelabelrotated{N5bb}{N3b}[0.3][1mm]{\pgfbox[center,base]{
    %    $\langle b,\neg c\CEF a\rangle$}}
  \end{pgfpicture}
\end{frame}

\begin{frame}{General splitting strategies}
  Alternatives:
  \begin{enumerate}
  \item Do nothing (\alert{unrestricted regression}).
  \item Always eliminate all disjunctivity
    (\alert{full splitting}).
  \item Reduce disjunctivity if formula becomes too big.
  \end{enumerate}

  \medskip

  Discussion:
  \begin{itemize}
  \item \alert{With unrestricted regression} the formulae may have
    \alert{size that is exponential} in the number of state variables.
  \item \alert{With full splitting} search tree can be
    \alert{exponentially bigger} than without splitting.
  \item The third option lies between these two extremes.
  \end{itemize}
\end{frame}

\section*{Summary}

\begin{frame}{Summary}
  \begin{itemize}
  \item (Classical) \alert{search} is a very important planning
    approach.
  \item Search-based planning algorithms differ along many dimensions,
    including
    \begin{itemize}
    \item \alert{search direction} (forward, backward)
    \item \alert{what each search node represents} \\
      (a state, a set of states, an operator sequence)
    \end{itemize}
  \item \alert{Progression search} proceeds forwards from the
    initial state.%
    \begin{itemize}
    \item If we use duplicate detection, each search node corresponds
      to a unique \alert{state}.
    \item If we do not use duplicate detection, each search node
      corresponds to a unique \alert{operator sequence}.
    \end{itemize}
  \end{itemize}
\end{frame}

\begin{frame}{Summary (ctd.)}
  \begin{itemize}
  \item \alert{Regression search} proceeds backwards from the goal.
    \begin{itemize}
    \item Each search node corresponds to a \alert{set of states}
      represented by a \alert{formula}.
    \item Regression is simple for \alert{STRIPS} operators.
    \item The theory for \alert{general regression} is more complex.
    \item When applying regression in practice, additional
      considerations such as when and how to perform \alert{splitting}
      come into play.
    \end{itemize}
  \end{itemize}
\end{frame}

\end{document}
