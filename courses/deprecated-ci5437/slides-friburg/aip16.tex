\documentclass{gkibeamer}

\usepackage{tikz}[2008/02/13]

\title[AI Planning]{Principles of AI Planning}
\author[B.~Nebel,\\R.~Mattm\"uller]{Bernhard Nebel and Robert Mattm\"uller}
\date[]{Winter Semester 2011/2012}

%% Use \alert instead of \textred where it makes sense.
%% (\textred makes sense when you *really* want the color to be red,
%% independent of the beamer style settings.)
\newcommand{\hilite}[1]{{\color{structure} #1}}
\newcommand{\textred}[1]{{\color{red} #1}}
\newcommand{\textblue}[1]{{\color{blue} #1}}
\newcommand{\textgreen}[1]{{\color{green} #1}}
\newcommand{\textbrown}[1]{{\color{brown} #1}}
\newcommand{\textviolet}[1]{{\color{violet} #1}}
\newcommand{\textbg}[1]{{\color{bg} #1}}

\newcommand{\eg}{e.\,g.}
\newcommand{\ie}{i.\,e.}

%% State variables and values for state variables
\newcommand{\var}[1]{\textit{#1}}
\newcommand{\val}[1]{\text{#1}}

%% Various stuff
\newcommand{\astar}{A${}^*$}
\newcommand{\idastar}{IDA${}^*$}

%% Blocksworld stuff
\newcommand{\AONB}{\var{A-on-B}}
\newcommand{\AONC}{\var{A-on-C}}
\newcommand{\AOND}{\var{A-on-D}}
\newcommand{\BONA}{\var{B-on-A}}
\newcommand{\BONC}{\var{B-on-C}}
\newcommand{\BOND}{\var{B-on-D}}
\newcommand{\CONA}{\var{C-on-A}}
\newcommand{\CONB}{\var{C-on-B}}
\newcommand{\COND}{\var{C-on-D}}
\newcommand{\DONA}{\var{D-on-A}}
\newcommand{\DONB}{\var{D-on-B}}
\newcommand{\DONC}{\var{D-on-C}}
\newcommand{\AONTABLE}{\var{A-on-T}}
\newcommand{\BONTABLE}{\var{B-on-T}}
\newcommand{\CONTABLE}{\var{C-on-T}}
\newcommand{\CLEARA}{\var{A-clear}}
\newcommand{\CLEARB}{\var{B-clear}}
\newcommand{\CLEARC}{\var{C-clear}}

\newcommand{\applyop}[2]{\textit{app}_{#1}(#2)}
\newcommand{\applyops}[2]{\textit{app}_{#1}(#2)}
\newcommand{\applyplan}[2]{\textit{app}_{#1}(#2)}
\newcommand{\regr}[2]{\textit{regr}_{#1}(#2)}
\newcommand{\regrstrips}[2]{\textit{sregr}_{#1}(#2)}
\newcommand{\eprecon}[2]{\textit{EPC}_{#1}(#2)}

\newcommand{\compl}[1]{\overline{#1}}

\newcommand{\changes}[2]{\lbrack #1\rbrack_{#2}}
\newcommand{\CEF}{\vartriangleright}

\newcommand{\relaxation}[1]{{#1}^+}
\newcommand{\onset}[1]{\textit{on}(#1)}
\newcommand{\hplus}{\ensuremath{h^+}}
\newcommand{\hmax}{\ensuremath{h_{\text{max}}}}
\newcommand{\hadd}{\ensuremath{h_{\text{add}}}}
\newcommand{\hlmcut}{\ensuremath{h_{\text{LM-cut}}}}
\newcommand{\hff}{\ensuremath{h_{\text{FF}}}}
\newcommand{\hcs}{\ensuremath{h_{\text{cs}}}}
\newcommand{\hsa}{\ensuremath{h_{\text{sa}}}}
\newcommand{\hhhh}{\ensuremath{h_{\text{HHH}}}}

\newcommand{\sasplus}{SAS${}^+$}

\newcommand{\graphequiv}{\stackrel{\textup{G}}{\sim}}

\newcommand{\cg}{\ensuremath{\textit{CG}}}

%% Turing machines and complexity theory
\newcommand{\accept}{{\textsf{Y}}}

\newcommand{\easier}{\ensuremath{\le_{\text{p}}}}

\newcommand{\decisionclass}[1]{\ensuremath{\textsf{\textup{#1}}}}
\newcommand{\dtime}{\decisionclass{DTIME}}
\newcommand{\ntime}{\decisionclass{NTIME}}
\newcommand{\dspace}{\decisionclass{DSPACE}}
\newcommand{\nspace}{\decisionclass{NSPACE}}

\newcommand{\ptime}{\decisionclass{P}}
\newcommand{\np}{\decisionclass{NP}}
\newcommand{\pspace}{\decisionclass{PSPACE}}
\newcommand{\npspace}{\decisionclass{NPSPACE}}
\newcommand{\exptime}{\decisionclass{EXP}}
\newcommand{\nexptime}{\decisionclass{NEXP}}
\newcommand{\expspace}{\decisionclass{EXPSPACE}}
\newcommand{\dblexptime}{\decisionclass{2EXP}}

\newcommand{\planex}{\textsc{PlanEx}}
\newcommand{\planlen}{\textsc{PlanLen}}

\newcommand{\ms}[1]{\mathchoice{\mbox{\normalsize\it #1}}{\mbox{\normalsize\it #1}}{\mbox{\scriptsize\it #1}}{\mbox{\tiny\it #1}}}
\newcommand{\image}[2]{\ms{img}_{#1}(#2)}
\newcommand{\spreimage}[2]{\ms{spreimg}_{#1}(#2)}
\newcommand{\wpreimage}[2]{\ms{wpreimg}_{#1}(#2)}
\newcommand{\disbwd}[2]{\delta^{\ms{bwd}}_{#1}(#2)}
\newcommand{\ats}[1]{\ms{scope}(#1)}
\newcommand{\ndopcpc}[2]{\tau^{\ms{nd}}_{#1}(#2)}
\newcommand{\choice}{\mathop{|}}
\newcommand{\opcpc}[2]{\tau_{#1}(#2)}

%% Don't use \begin{proof}/\end{proof} for multi-part proofs because
%% this places a QED symbol.
\newenvironment{proofstart}{\begin{block}{Proof.}}{\end{block}}
\newenvironment{proofmid}{\begin{block}{Proof (ctd.)}}{\end{block}}
\newenvironment{proofend}{\begin{proof}[Proof (ctd.)]}{\end{proof}}

\newtheorem{remark}[theorem]{Remark}

%% Macros to align the width of things.
\newlength{\mywidth}
\newcommand{\setmywidth}[1]{\settowidth{\mywidth}{#1}}
\newcommand{\usemywidth}[1]{\makebox[\mywidth][l]{#1}}
\newcommand{\usemywidthmath}[1]{\usemywidth{\ensuremath{#1}}}

%% A tighter "align" environment.
\newenvironment{tightalign}[1][c]{\par\(\begin{array}[#1]{@{}r@{}l}}
  {\end{array}\)\par}
\newenvironment{wrappedmath}[1][t]{\begin{array}[#1]{@{}l}}{\end{array}}

%% Nondeterministic transition systems with connector arcs
\newcommand{\arc}[4]{
  \begin{pgfscope}
    \pgfsetlinewidth{1pt}
    \pgfpathmoveto{\pgfpointanchor{#3}{center}}
    \pgfpathlineto{\pgfpointanchor{#1}{#2}}
    \pgfpathlineto{\pgfpointanchor{#4}{center}}
    \pgfusepath{clip}
    \pgfpathcircle{\pgfpointanchor{#1}{#2}}{1.8mm}
    \pgfusepath{draw}
  \end{pgfscope}
}


\begin{document}
\subtitle{13.~Computational complexity of classical planning}
\date{July 20th, 2010}
\maketitles

%% NOTE: This chapter was drastically cut compared to WS 2006/2007.
%% For example, we removed:
%% * the foundational definitions and theorems needed for
%%   nondeterministic planning complexity (ATMs, Chandra's theorem)
%% * proofs for planning with first-order representations
%% * proofs for domain-dependent planning
%%
%% So if later courses want to deal with these topics in more depth,
%% it may make sense to base them on the 2006/2007 slides instead of
%% these.
%%
%% Note though that we also applied many stylistic changes in
%% 2008/2009 to get the chapter more in line with the rest of the
%% course. For example:
%% * Use theorem and definition environments instead of plain blocks.
%% * Don't capitalize headings.
%% * Use \langle\rangle for tuples.
%% * Separate the statement of theorems from their proofs.
%%
%% It's probably less work to base future revisions of this chapter on
%% this one and re-add the old material than to take the old version
%% and redo the style conversions.

\section{Motivation}

\begin{frame}{How hard is planning?}
  \begin{itemize}
  \item We have seen that planning can be done in time
    \alert{polynomial} in the size of the \alert{transition system}.
  \item However, we have not seen algorithms which are polynomial
    in the \alert{input size} (size of the task description).
  \item[$\leadsto$] What is the precise \alert{computational complexity} of
    the \alert{planning problem}?
  \end{itemize}
\end{frame}

\begin{frame}{Why computational complexity?}
  \begin{itemize}
  \item \alert{understand} the problem
  \item know what is \alert{not} possible
  \item find interesting \alert{subproblems} that are easier to solve
  \item distinguish \alert{essential features} from \alert{syntactic
    sugar}
    \begin{itemize}
    \item Is STRIPS planning easier than general planning?
    \item Is planning for FDR tasks harder than for propositional
      tasks?
    \end{itemize}
  \end{itemize}
\end{frame}

\section{Background}
\subsection{Turing machines}

\begin{frame}{Nondeterministic Turing machines}
  \begin{definition}[nondeterministic Turing machine]
    A \alert{nondeterministic Turing machine (NTM)} is a 6-tuple
    $\langle \Sigma, \Box, Q, q_0, q_\accept, \delta\rangle$ with the
    following components:
    \begin{itemize}
    \item \alert{input alphabet} $\Sigma$ and \alert{blank symbol}
      $\Box \notin \Sigma$
      \begin{itemize}
      \item alphabets always nonempty and finite
      \item \alert{tape alphabet} $\Sigma_\Box = \Sigma \cup \{\Box\}$
      \end{itemize}
    \item finite set $Q$ of \alert{internal states} with \alert{initial
      state} $q_0 \in Q$ and \alert{accepting state} $q_\accept \in Q$
      \begin{itemize}
      \item \alert{nonterminal states} $Q' := Q \setminus \{q_\accept\}$
      \end{itemize}
    \item \alert{transition relation}
      $\delta \subseteq (Q' \times \Sigma_\Box) \times (Q \times
      \Sigma_\Box \times \{-1,+1\})$
    \end{itemize}
  \end{definition}
\end{frame}

\begin{frame}{Deterministic Turing machines}
  \begin{definition}[deterministic Turing machine]
    A \alert{deterministic Turing machine (DTM)}
    is an NTM
    where the transition relation is \alert{functional}, \ie,
    for all $\langle q, a\rangle \in Q' \times \Sigma_\Box$, there is
    exactly one triple $\langle q', a', \Delta\rangle$ with
    $\langle\langle q,a\rangle, \langle q', a',\Delta\rangle\rangle
    \in \delta$.

    \medskip

    \hilite{Notation:} We write $\delta(q, a)$ for the unique triple
    $\langle q', a', \Delta\rangle$ such that $\langle\langle q,
    a\rangle, \langle q', a', \Delta\rangle\rangle \in \delta$.
  \end{definition}
\end{frame}

\begin{frame}{Turing machine configurations}
  \begin{definition}[Configuration]
    Let $M = \langle \Sigma, \Box, Q, q_0, q_\accept, \delta\rangle$
    be an NTM.

    \smallskip

    A \alert{configuration} of $M$ is a triple
    $\langle w, q, x\rangle \in \Sigma_\Box^* \times Q \times \Sigma_\Box^+$.
    \begin{itemize}
    \item $w$: tape contents before tape head
    \item $q$: current state
    \item $x$: tape contents after and including tape head
    \end{itemize}
  \end{definition}
\end{frame}

\begin{frame}{Turing machine transitions}
  \begin{definition}[yields relation]
    Let $M = \langle \Sigma, \Box, Q, q_0, q_\accept, \delta\rangle$
    be an NTM.

    \smallskip

    A configuration $c$ of $M$ \alert{yields}
    a configuration $c'$ of $M$, \\
    in symbols \alert{$c \vdash c'$}, as defined by the following rules, \\
    where $a, a', b \in \Sigma_\Box$,
    $w,x \in \Sigma_\Box^*$,
    $q,q' \in Q$ and
    $\langle\langle q, a\rangle, \langle q',a',\Delta\rangle\rangle
    \in \delta$:
    \begin{alignat*}{2}
      \langle w, q, ax\rangle
      & \vdash \langle wa', q', x\rangle
      & \qquad & \text{if~} \Delta = +1, |x| \ge 1  \\
      \langle w, q, a\rangle
      & \vdash \langle wa', q', \Box\rangle
      && \text{if~} \Delta = +1  \\
      \langle wb, q, ax\rangle
      & \vdash \langle w, q', ba'x\rangle
      && \text{if~} \Delta = -1  \\
      \langle\epsilon, q, ax\rangle
      & \vdash
      \langle\epsilon, q', \Box a'x\rangle
      && \text{if~} \Delta = -1 \\
    \end{alignat*}
  \end{definition}
\end{frame}

\begin{frame}{Accepting configurations}
  \begin{definition}[accepting configuration, time]
    Let $M = \langle \Sigma, \Box, Q, q_0, q_\accept, \delta\rangle$
    be an NTM, \\
    let $c = \langle w, q, x\rangle$ be a configuration of $M$,
    and let $n \in \mathbb N_0$.
    \begin{itemize}
    \item If $q = q_\accept$, $M$ \alert{accepts $c$ in time $n$}.
    \item If $q \neq q_\accept$ and $M$ accepts some $c'$ with $c
      \vdash c'$ in time $n$, then $M$ \alert{accepts $c$ in time $n +
        1$}.
    \end{itemize}
  \end{definition}
  \begin{definition}[accepting configuration, space]
    Let $M = \langle \Sigma, \Box, Q, q_0, q_\accept, \delta\rangle$
    be an NTM, \\
    let $c = \langle w, q, x\rangle$ be a configuration of $M$,
    and let $n \in \mathbb N_0$.
    \begin{itemize}
    \item If $q = q_\accept$ and $|w| + |x| \le n$, $M$ \alert{accepts
      $c$ in space $n$}. 
    \item If $q \neq q_\accept$ and $M$ accepts some $c'$ with $c
      \vdash c'$ in space $n$, then $M$ \alert{accepts $c$ in space
        $n$}.
    \end{itemize}
  \end{definition}
\end{frame}
%% TODO: This is a bit unintuitive, terminology-wise. "Accepts in
%% time/space n" here really means something like "Accepts in
%% time/space at most n". We might want to change either the name or
%% the definition so that it is more intuitive. If we change the
%% definition, make sure to capture nondeterminism properly and make
%% sure to adjust the following definitions that depend on this one.

\begin{frame}{Accepting words and languages}
  \begin{definition}[accepting words]
    Let $M = \langle \Sigma, \Box, Q, q_0, q_\accept, \delta\rangle$
    be an NTM.

    \smallskip

    $M$ \alert{accepts the word $w \in \Sigma^*$
      in time (space) $n \in \mathbb N_0$} \\
    iff $M$ accepts $\langle\epsilon, q_0, w\rangle$ in time (space) $n$.
    \begin{itemize}
    \item Special case: $M$ accepts $\epsilon$ in time (space)
      $n \in \mathbb N_0$ \\
      iff $M$ accepts $\langle\epsilon, q_0, \Box\rangle$ in time
      (space) $n$.
    \end{itemize}
  \end{definition}

  \begin{definition}[accepting languages]
    Let $M = \langle \Sigma, \Box, Q, q_0, q_\accept, \delta\rangle$
    be an NTM, and let $f: \mathbb N_0 \to \mathbb N_0$.

    \smallskip
    
    $M$ \alert{accepts the language $L \subseteq \Sigma^*$
      in time (space) $f$} \\
    iff $M$ accepts each word $w \in L$ in time (space) $f(|w|)$, \\
    and $M$ does not accept any word $w \notin L$ (in any time/space).
  \end{definition}
\end{frame}

\subsection{Complexity classes}

\begin{frame}{Time and space complexity classes}
    \begin{definition}[\dtime, \ntime, \dspace, \nspace]
    Let $f: \mathbb N_0 \to \mathbb N_0$.

    \smallskip

    Complexity class \alert{$\dtime(f)$} contains all languages \\
    accepted in time $f$ by some DTM.

    \smallskip

    Complexity class \alert{$\ntime(f)$} contains all languages \\
    accepted in time $f$ by some NTM.

    \smallskip

    Complexity class \alert{$\dspace(f)$} contains all languages \\
    accepted in space $f$ by some DTM.

    \smallskip

    Complexity class \alert{$\nspace(f)$} contains all languages \\
    accepted in space $f$ by some NTM.
    \end{definition}
\end{frame}

\begin{frame}{Polynomial time and space classes}
  Let $\mathcal P$ be the set of polynomials $p: \mathbb N_0 \to
  \mathbb N_0$ whose coefficients are natural numbers.
  \begin{definition}[\ptime, \np, \pspace, \npspace]
    \renewcommand{\arraycolsep}{1pt}
    $\begin{array}{rl}
      \ptime & = \bigcup_{p \in \mathcal P} \dtime(p) \\
      \np & = \bigcup_{p \in \mathcal P} \ntime(p) \\
      \pspace & = \bigcup_{p \in \mathcal P} \dspace(p) \\
      \npspace & = \bigcup_{p \in \mathcal P} \nspace(p)
    \end{array}$
  \end{definition}
\end{frame}

\begin{frame}{Polynomial complexity class relationships}
  \begin{theorem}[complexity class hierarchy]
    $\ptime \subseteq \np \subseteq \pspace = \npspace$
  \end{theorem}
  \begin{proof}
    $\ptime \subseteq \np$ and $\pspace \subseteq \npspace$ is
    obvious because deterministic Turing machines are a special case
    of nondeterministic ones.

    \smallskip

    $\np \subseteq \npspace$ holds because a Turing machine can
    only visit polynomially many tape cells within polynomial time.

    \smallskip

    $\pspace = \npspace$ is a special case of a classical result known
    as Savitch's theorem (Savitch 1970).
  \end{proof}
\end{frame}

\section[Complexity of planning]{Complexity of propositional planning}
\subsection[(Bounded) plan existence]{Plan existence and bounded plan existence}

\begin{frame}{The propositional planning problem}
  \begin{definition}[plan existence]
    The \alert{plan existence} problem (\planex) \\
    is the following decision problem:
    \smallskip

    \begin{tabular}{@{}ll}
      \textsc{Given:} &
      Planning task $\Pi$ \\
      \textsc{Question:} &
      Is there a plan for $\Pi$?
    \end{tabular}
  \end{definition}
  $\leadsto$ decision problem analogue of \alert{satisficing planning}

  \medskip

  \begin{definition}[bounded plan existence]
    The \alert{bounded plan existence} problem (\planlen) \\
    is the following decision problem:
    \smallskip

    \begin{tabular}{@{}ll}
      \textsc{Given:} &
      Planning task $\Pi$, length bound $K \in \mathbb N_0$ \\
      \textsc{Question:} &
      Is there a plan for $\Pi$ of length at most $K$?
    \end{tabular}
  \end{definition}
  $\leadsto$ decision problem analogue of \alert{optimal planning}
\end{frame}

\begin{frame}{Plan existence vs.\ bounded plan existence}
  \begin{theorem}[reduction from {\planex} to {\planlen}]
    $\planex \easier \planlen$
  \end{theorem}

  \begin{proof}
    A propositional planning task with $n$ state variables has a plan
    \\ iff it has a plan of length at most $2^n - 1$.

    \smallskip

    $\leadsto$ map instance $\Pi$ of {\planex}
    to instance $\langle \Pi, 2^n - 1\rangle$ of {\planlen}, where $n$
    is the number of $n$ state variables of $\Pi$

    \smallskip

    $\leadsto$ polynomial reduction
  \end{proof}
\end{frame}

\subsection{PSPACE-completeness}

\begin{frame}{Membership in \pspace}
  \begin{theorem}[{\pspace} membership for \planlen]
    $\planlen \in \pspace$
  \end{theorem}
  \begin{proof}
    Show $\planlen \in \npspace$ and use Savitch's theorem.

    Nondeterministic algorithm:

    \smallskip
    \small

    \textbf{def} plan($\langle A, I, O, G\rangle$, $K$): \\
    \hspace*{1cm} $s := I$ \\
    \hspace*{1cm} $k := K$ \\
    \hspace*{1cm} \textbf{while} $s \not\models G$: \\
    \hspace*{2cm} \textbf{guess} $o \in O$ \\
    \hspace*{2cm} \textbf{fail} if $o$ not applicable in $s$
    \textbf{or} k = 0\\
    \hspace*{2cm} $s := \applyop{o}{s}$ \\
    \hspace*{2cm} $k := k-1$ \\
    \hspace*{1cm} \textbf{accept} \\
  \end{proof}
\end{frame}

\begin{frame}{Hardness for {\pspace}}
  \hilite{Idea:} \alert{generic reduction}

  \begin{itemize}
  \item For an \alert{arbitrary fixed DTM $M$} with space bound
    polynomial $p$ and input $w$, generate planning task which is
    solvable iff $M$ accepts $w$ in space $p(|w|)$.
  \item For simplicity, restrict to TMs which never move to the left
    of the initial head position (no loss of generality).
  \end{itemize}
\end{frame}

\begin{frame}{Reduction: state variables}
  Let $M = \langle \Sigma, \Box, Q, q_0, q_\accept, \delta\rangle$ be the
  fixed DTM and let $p$ be its space-bound polynomial.

  \smallskip

  Given input $w_1 \dots w_n$, define \alert{relevant tape positions}
  $X := \{1, \dots, p(n)\}$.

  \begin{block}{State variables}
    \begin{itemize}
    \item $\text{state}_q$ for all $q \in Q$
    \item $\text{head}_i$ for all $i \in X \cup \{0, p(n) + 1\}$
    \item $\text{content}_{i,a}$ for all $i \in X$, $a \in \Sigma_\Box$
    \end{itemize}
    $\leadsto$ allows encoding a Turing machine configuration
  \end{block}
\end{frame}

\begin{frame}{Reduction: initial state}
  Let $M = \langle \Sigma, \Box, Q, q_0, q_\accept, \delta\rangle$ be the
  fixed DTM and let $p$ be its space-bound polynomial.

  \smallskip

  Given input $w_1 \dots w_n$, define \alert{relevant tape positions}
  $X := \{1, \dots, p(n)\}$.

  \begin{block}{Initial state}
    Initially true:
    \begin{itemize}
    \item $\text{state}_{q_0}$
    \item $\text{head}_1$
    \item $\text{content}_{i,w_i}$ for all $i \in \{1, \dots, n\}$
    \item $\text{content}_{i,\Box}$ for all $i \in X \setminus \{1, \dots,
      n\}$
    \end{itemize}
    Initially false:
    \begin{itemize}
    \item all others
    \end{itemize}
  \end{block}
\end{frame}

\begin{frame}{Reduction: operators}
  Let $M = \langle \Sigma, \Box, Q, q_0, q_\accept, \delta\rangle$ be the
  fixed DTM and let $p$ be its space-bound polynomial.

  \smallskip

  Given input $w_1 \dots w_n$, define \alert{relevant tape positions}
  $X := \{1, \dots, p(n)\}$.

  \begin{block}{Operators}
    One operator for each transition rule
    $\delta(q,a) = \langle q',a',\Delta\rangle$ \\
    and each cell position $i \in X$:
    \begin{itemize}
    \item precondition:
      $\text{state}_q \land \text{head}_i \land
        \text{content}_{i,a}$
    \item effect: $\begin{wrappedmath}
      \neg \text{state}_q
      \land \neg \text{head}_i
      \land \neg \text{content}_{i,a} \\
            {} \land \text{state}_{q'}
        \land \text{head}_{i + \Delta}
        \land \text{content}_{i,a'}
      \end{wrappedmath}$

      \begin{itemize}
      \item If $q = q'$ and/or $a = a'$, omit the effects on
        $\text{state}_q$ and/or $\text{content}_{i,a}$, to avoid
        consistency condition issues.
      \end{itemize}
    \end{itemize}
  \end{block}
\end{frame}

\begin{frame}{Reduction: goal}
  Let $M = \langle \Sigma, \Box, Q, q_0, q_\accept, \delta\rangle$ be the
  fixed DTM and let $p$ be its space-bound polynomial.

  \smallskip

  Given input $w_1 \dots w_n$, define \alert{relevant tape positions}
  $X := \{1, \dots, p(n)\}$.

  \begin{block}{Goal}
    $\text{state}_{q_\accept}$
  \end{block}
\end{frame}

\begin{frame}{\pspace-completeness for STRIPS plan existence}
  \begin{theorem}[\pspace-completeness; Bylander, 1994]
    {\planex} and {\planlen} are \pspace-complete.

    This is true even when restricting to STRIPS tasks.
  \end{theorem}

  \begin{proof}
    Membership for {\planlen} was already shown.

    \smallskip

    Hardness for {\planex} follows because we just presented a
    polynomial reduction from an arbitrary problem in {\pspace} to
    {\planex}. (Note that the reduction only generates STRIPS tasks.)

    \smallskip

    Membership for {\planex} and hardness for {\planlen} follows from
    the polynomial reduction from {\planex} to {\planlen}.
  \end{proof}
\end{frame}

\section{More complexity results}

\begin{frame}{More complexity results}
  In addition to the basic complexity result presented in this
  chapter, there are many special cases, generalizations, variations
  and related problems studied in the literature:

  \begin{itemize}
  \item different \alert{planning formalisms}
    \begin{itemize}
    \item \eg, finite-domain representation, nondeterministic effects,
      partial observability, schematic operators, numerical state
      variables
    \end{itemize}
  \item \alert{syntactic restrictions} of planning tasks
    \begin{itemize}
    \item \eg, without preconditions, without conjunctive effects,
      STRIPS without delete effects
    \end{itemize}
  \item \alert{semantic restrictions} of planning task
    \begin{itemize}
    \item \eg, restricting to certain classes of causal graphs
    \end{itemize}
  \item \alert{particular planning domains}
    \begin{itemize}
    \item \eg, Blocksworld, Logistics, FreeCell
    \end{itemize}
  \end{itemize}
\end{frame}

\begin{frame}{Complexity results for different planning formalisms}
  Some results for different planning formalisms:
  \begin{itemize}
  \item \alert{FDR tasks:}
    \begin{itemize}
    \item same complexity as for propositional tasks (``folklore'')
    \item also true for the {\sasplus} special case
    \end{itemize}
  \item \alert{nondeterministic effects:}
    \begin{itemize}
    \item fully observable: \exptime-complete (Littman, 1997)
    \item unobservable: \expspace-complete (Haslum \& Jonsson, 1999)
    \item partially observable: \dblexptime-complete (Rintanen, 2004)
    \end{itemize}
  \item \alert{schematic operators:}
    \begin{itemize}
    \item usually adds one exponential level to \planex\ complexity
    \item \eg, classical case \expspace-complete (Erol et al., 1995)
    \end{itemize}
  \item \alert{numerical state variables:}
    \begin{itemize}
    \item undecidable in most variations (Helmert, 2002)
    \end{itemize}
  \end{itemize}
\end{frame}

\section*{Summary}
\begin{frame}{Summary}
  \begin{itemize}
    \item \alert{Propositional planning} is \alert{PSPACE-complete}.
    \item The hardness proof is a polynomial reduction that translates
        an \alert{arbitrary polynomial-space DTM} into a \alert{STRIPS task}:
        \begin{itemize}
            \item Configurations of the DTM are encoded by propositional
                variables.
            \item Operators simulate transistions of the DTM.
            \item The DTM accepts an input iff there is a plan for the
               corresponding STRIPS task.
        \end{itemize}
    \item This implies that there is \alert{no polynomial algorithm} for
        classical planning unless \ptime=\pspace.
    \item It also means that classical planning is not polynomially
        reducible to any problem in \np\ unless \np=\pspace.
  \end{itemize}
\end{frame}
\end{document}
