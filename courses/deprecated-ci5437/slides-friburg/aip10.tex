\documentclass{gkibeamer}

\usepackage{tikz}[2008/02/13]

\title[AI Planning]{Principles of AI Planning}
\author[B.~Nebel,\\R.~Mattm\"uller]{Bernhard Nebel and Robert Mattm\"uller}
\date[]{Winter Semester 2011/2012}

%% Use \alert instead of \textred where it makes sense.
%% (\textred makes sense when you *really* want the color to be red,
%% independent of the beamer style settings.)
\newcommand{\hilite}[1]{{\color{structure} #1}}
\newcommand{\textred}[1]{{\color{red} #1}}
\newcommand{\textblue}[1]{{\color{blue} #1}}
\newcommand{\textgreen}[1]{{\color{green} #1}}
\newcommand{\textbrown}[1]{{\color{brown} #1}}
\newcommand{\textviolet}[1]{{\color{violet} #1}}
\newcommand{\textbg}[1]{{\color{bg} #1}}

\newcommand{\eg}{e.\,g.}
\newcommand{\ie}{i.\,e.}

%% State variables and values for state variables
\newcommand{\var}[1]{\textit{#1}}
\newcommand{\val}[1]{\text{#1}}

%% Various stuff
\newcommand{\astar}{A${}^*$}
\newcommand{\idastar}{IDA${}^*$}

%% Blocksworld stuff
\newcommand{\AONB}{\var{A-on-B}}
\newcommand{\AONC}{\var{A-on-C}}
\newcommand{\AOND}{\var{A-on-D}}
\newcommand{\BONA}{\var{B-on-A}}
\newcommand{\BONC}{\var{B-on-C}}
\newcommand{\BOND}{\var{B-on-D}}
\newcommand{\CONA}{\var{C-on-A}}
\newcommand{\CONB}{\var{C-on-B}}
\newcommand{\COND}{\var{C-on-D}}
\newcommand{\DONA}{\var{D-on-A}}
\newcommand{\DONB}{\var{D-on-B}}
\newcommand{\DONC}{\var{D-on-C}}
\newcommand{\AONTABLE}{\var{A-on-T}}
\newcommand{\BONTABLE}{\var{B-on-T}}
\newcommand{\CONTABLE}{\var{C-on-T}}
\newcommand{\CLEARA}{\var{A-clear}}
\newcommand{\CLEARB}{\var{B-clear}}
\newcommand{\CLEARC}{\var{C-clear}}

\newcommand{\applyop}[2]{\textit{app}_{#1}(#2)}
\newcommand{\applyops}[2]{\textit{app}_{#1}(#2)}
\newcommand{\applyplan}[2]{\textit{app}_{#1}(#2)}
\newcommand{\regr}[2]{\textit{regr}_{#1}(#2)}
\newcommand{\regrstrips}[2]{\textit{sregr}_{#1}(#2)}
\newcommand{\eprecon}[2]{\textit{EPC}_{#1}(#2)}

\newcommand{\compl}[1]{\overline{#1}}

\newcommand{\changes}[2]{\lbrack #1\rbrack_{#2}}
\newcommand{\CEF}{\vartriangleright}

\newcommand{\relaxation}[1]{{#1}^+}
\newcommand{\onset}[1]{\textit{on}(#1)}
\newcommand{\hplus}{\ensuremath{h^+}}
\newcommand{\hmax}{\ensuremath{h_{\text{max}}}}
\newcommand{\hadd}{\ensuremath{h_{\text{add}}}}
\newcommand{\hlmcut}{\ensuremath{h_{\text{LM-cut}}}}
\newcommand{\hff}{\ensuremath{h_{\text{FF}}}}
\newcommand{\hcs}{\ensuremath{h_{\text{cs}}}}
\newcommand{\hsa}{\ensuremath{h_{\text{sa}}}}
\newcommand{\hhhh}{\ensuremath{h_{\text{HHH}}}}

\newcommand{\sasplus}{SAS${}^+$}

\newcommand{\graphequiv}{\stackrel{\textup{G}}{\sim}}

\newcommand{\cg}{\ensuremath{\textit{CG}}}

%% Turing machines and complexity theory
\newcommand{\accept}{{\textsf{Y}}}

\newcommand{\easier}{\ensuremath{\le_{\text{p}}}}

\newcommand{\decisionclass}[1]{\ensuremath{\textsf{\textup{#1}}}}
\newcommand{\dtime}{\decisionclass{DTIME}}
\newcommand{\ntime}{\decisionclass{NTIME}}
\newcommand{\dspace}{\decisionclass{DSPACE}}
\newcommand{\nspace}{\decisionclass{NSPACE}}

\newcommand{\ptime}{\decisionclass{P}}
\newcommand{\np}{\decisionclass{NP}}
\newcommand{\pspace}{\decisionclass{PSPACE}}
\newcommand{\npspace}{\decisionclass{NPSPACE}}
\newcommand{\exptime}{\decisionclass{EXP}}
\newcommand{\nexptime}{\decisionclass{NEXP}}
\newcommand{\expspace}{\decisionclass{EXPSPACE}}
\newcommand{\dblexptime}{\decisionclass{2EXP}}

\newcommand{\planex}{\textsc{PlanEx}}
\newcommand{\planlen}{\textsc{PlanLen}}

\newcommand{\ms}[1]{\mathchoice{\mbox{\normalsize\it #1}}{\mbox{\normalsize\it #1}}{\mbox{\scriptsize\it #1}}{\mbox{\tiny\it #1}}}
\newcommand{\image}[2]{\ms{img}_{#1}(#2)}
\newcommand{\spreimage}[2]{\ms{spreimg}_{#1}(#2)}
\newcommand{\wpreimage}[2]{\ms{wpreimg}_{#1}(#2)}
\newcommand{\disbwd}[2]{\delta^{\ms{bwd}}_{#1}(#2)}
\newcommand{\ats}[1]{\ms{scope}(#1)}
\newcommand{\ndopcpc}[2]{\tau^{\ms{nd}}_{#1}(#2)}
\newcommand{\choice}{\mathop{|}}
\newcommand{\opcpc}[2]{\tau_{#1}(#2)}

%% Don't use \begin{proof}/\end{proof} for multi-part proofs because
%% this places a QED symbol.
\newenvironment{proofstart}{\begin{block}{Proof.}}{\end{block}}
\newenvironment{proofmid}{\begin{block}{Proof (ctd.)}}{\end{block}}
\newenvironment{proofend}{\begin{proof}[Proof (ctd.)]}{\end{proof}}

\newtheorem{remark}[theorem]{Remark}

%% Macros to align the width of things.
\newlength{\mywidth}
\newcommand{\setmywidth}[1]{\settowidth{\mywidth}{#1}}
\newcommand{\usemywidth}[1]{\makebox[\mywidth][l]{#1}}
\newcommand{\usemywidthmath}[1]{\usemywidth{\ensuremath{#1}}}

%% A tighter "align" environment.
\newenvironment{tightalign}[1][c]{\par\(\begin{array}[#1]{@{}r@{}l}}
  {\end{array}\)\par}
\newenvironment{wrappedmath}[1][t]{\begin{array}[#1]{@{}l}}{\end{array}}

%% Nondeterministic transition systems with connector arcs
\newcommand{\arc}[4]{
  \begin{pgfscope}
    \pgfsetlinewidth{1pt}
    \pgfpathmoveto{\pgfpointanchor{#3}{center}}
    \pgfpathlineto{\pgfpointanchor{#1}{#2}}
    \pgfpathlineto{\pgfpointanchor{#4}{center}}
    \pgfusepath{clip}
    \pgfpathcircle{\pgfpointanchor{#1}{#2}}{1.8mm}
    \pgfusepath{draw}
  \end{pgfscope}
}


\newcommand{\cliques}{\ensuremath{\textrm{cliques}}}

\newcommand{\bigarrow}[2]{%
  \begin{pgfscope}
    \pgfsetlinewidth{2pt}
    \pgfsetendarrow{\pgfarrowsingle{1pt}}
    \pgfline{\pgfxy(#1)}{\pgfrelative{\pgfxy(#1)}{\pgfxy(#2,0)}}
  \end{pgfscope}
}

\newcommand{\borderrect}[3]{%
  {\color{#1}\pgfrect[fill]{\pgfxy(#2)}{\pgfxy(#3)}}%
  {\color{black}\pgfrect[stroke]{\pgfxy(#2)}{\pgfxy(#3)}}%
}

\newcommand{\tile}[3][orange!30]{
  \pgfputat{\pgfxy(#2)}{
    \borderrect{#1}{-0.5,-0.5}{1,1}
    \pgfbox[center,center]{#3}
  }
}

\newcommand{\abstractedtile}[1]{
  \tile[orange!20]{#1}{}
}

\newcommand{\abstractedtileanimated}[3][orange!30]{
  \only<all:1>{\tile[#1]{#2}{#3}}
  \only<all:2->{\abstractedtile{#2}}
}






\tikzstyle{vertex interior}=[fill,minimum size=0.80cm,circle,
    inner sep=0pt,outer sep=0.2pt,color=red!20]
\tikzstyle{vertex border}=[draw,thin,minimum size=0.80cm,circle,
    inner sep=0pt,outer sep=0.2pt,color=black]
\tikzstyle{vertex hidden}=[draw,minimum size=0.80cm,circle,
    inner sep=0pt,outer sep=0.2pt, color=gray, text=gray]
\tikzstyle{subvertex}=[color=gray]

\tikzstyle{bigvertex interior}=[fill=red!20]
\tikzstyle{bigvertex border}=[]

\tikzstyle{transition}=[semithick]

\newcommand{\vertex}[3][]{
    \node[vertex interior] at (#2) (#3) {};
    \node[vertex border,#1] at (#3) {\small #3};
}

\newcommand{\numberedvertex}[4][]{
    \node[vertex interior] at (#2) (#3) {};
    \node[vertex border,#1] at (#3) {\small #4};
}

\newcommand{\vertexnolabel}[3][]{
    \node[vertex interior] at (#2) (#3) {};
    \node[vertex border,#1] at (#3) {};
}

\newcommand{\wildcard}{\ensuremath{\star}}

% Usage: \if\blank{#1}...\else...\fi
\catcode`\@=11 % as in plain.tex
\long\def\blank#1{\bl@nk#1@@..\bl@nk}%
\long\def\bl@nk#1#2@#3#4\bl@nk{#3#4}
\catcode`\@=12

\long\def\test#1{\begingroup \toks0{[#1]}%
    \newlinechar`\/\message{/\the\toks0:
        \if\blank{#1}EMPTY\else NOT empty\fi%
    }\endgroup}

\newcommand{\hiddenvertex}[1]{
    \if \blank{#1}
    \else
        \node[vertex hidden] at (#1) {#1};
    \fi
}

\newcommand{\hiddenvertices}[4]{
    \hiddenvertex{#1}
    \hiddenvertex{#2}
    \hiddenvertex{#3}
    \hiddenvertex{#4}
}

\newcommand{\verticalvertexborder}[3][]{
    \draw[bigvertex border, #1] (#2.west) arc(180:0:2.9mm) -- 
        (#3.east) arc(0:-180:2.9mm) -- (#2.west);
}

\newcommand{\verticalvertex}[5][]{
    \draw[bigvertex interior, #1] (#2.west) arc(180:0:2.9mm) -- 
        (#3.east) arc(0:-180:2.9mm) -- (#2.west);
	\hiddenvertices{#2}{#3}{#4}{#5}
	\verticalvertexborder{#2}{#3}
}

\newcommand{\verticalgoalvertex}{\verticalvertex[color=green!40]}

\newcommand{\horizontalvertexborder}[3][]{
    \draw[bigvertex border, #1] (#2.south) arc(-90:-270:2.9mm) -- 
        (#3.north) arc(90:-90:2.9mm) -- (#2.south);
}

\newcommand{\horizontalvertex}[3][]{
    \draw[bigvertex interior, #1] (#2.south) arc(-90:-270:2.9mm) -- 
        (#3.north) arc(90:-90:2.9mm) -- (#2.south);
	\hiddenvertices{#2}{#3}{}{}
	\horizontalvertexborder{#2}{#3}
}

\newcommand{\trianglevertexleftborder}[4][]{
    \draw[bigvertex border,#1] (#2.south west) arc(-135:-225:2.9mm) -- 
        (#3.north west) arc(135:0:2.9mm) -- 
        (#4.east) arc(0:-135:2.9mm) -- 
        (#2.south west);
}

\newcommand{\trianglevertexleft}[5][]{    
    \draw[bigvertex interior, #1] (#2.south west) arc(-135:-225:2.9mm) -- 
        (#3.north west) arc(135:0:2.9mm) -- 
        (#4.east) arc(0:-135:2.9mm) -- 
        (#2.south west); 
	\hiddenvertices{#2}{#3}{#4}{#5}
	\trianglevertexleftborder{#2}{#3}{#4}
}

\newcommand{\trianglevertexrightborder}[4][]{
    \draw[bigvertex border, #1] (#2.south east) arc(-45:45:2.9mm) -- 
        (#3.north east) arc(45:180:2.9mm) -- 
        (#4.west) arc(180:315:2.9mm) -- 
        (#2.south east);
}

\newcommand{\trianglevertexright}[5][]{   
    \draw[bigvertex interior, #1] (#2.south east) arc(-45:45:2.9mm) -- 
        (#3.north east) arc(45:180:2.9mm) -- 
        (#4.west) arc(180:315:2.9mm) -- 
        (#2.south east); 
	\hiddenvertices{#2}{#3}{#4}{#5}
	\trianglevertexrightborder{#2}{#3}{#4}
}

\newcommand{\squarevertexborder}[5][]{
    \draw[bigvertex border] (#2.west) arc(180:90:2.9mm) -- 
        (#3.north) arc(90:0:2.9mm) -- 
        (#4.east) arc(0:-90:2.9mm) -- 
        (#5.south) arc(-90:-180:2.9mm) -- 
        (#2.west);
}

\newcommand{\squarevertex}[5][]{    
    \draw[bigvertex interior, #1] (#2.west) arc(180:90:2.9mm) -- 
        (#3.north) arc(90:0:2.9mm) -- 
        (#4.east) arc(0:-90:2.9mm) -- 
        (#5.south) arc(-90:-180:2.9mm) -- 
        (#2.west); 
	\hiddenvertices{#2}{#3}{#4}{#5}
	\squarevertexborder{#2}{#3}{#4}{#5}
}

\newcommand{\gravisvertexborder}[3][]{
    \draw[bigvertex border, #1] (#2.south west) arc(225:45:2.9mm) -- 
        (#3.north east) arc(45:-135:2.9mm) -- 
        (#2.south west);
}

\newcommand{\gravisvertex}[3][]{    
    \draw[bigvertex interior,#1] (#2.south west) arc(225:45:2.9mm) -- 
        (#3.north east) arc(45:-135:2.9mm) -- 
        (#2.south west); 
	\hiddenvertices{#2}{#3}{}{}
	\gravisvertexborder{#2}{#3}
}

\newcommand{\akutvertexborder}[3][]{
    \draw[bigvertex border, #1] (#2.north west) arc(135:-45:2.9mm) -- 
        (#3.south east) arc(-45:-225:2.9mm) -- 
        (#2.north west);
}

\newcommand{\akutvertex}[3][]{    
    \draw[bigvertex interior,#1] (#2.north west) arc(135:-45:2.9mm) -- 
        (#3.south east) arc(-45:-225:2.9mm) -- 
        (#2.north west); 
	\hiddenvertices{#2}{#3}{}{}
	\akutvertexborder{#2}{#3}
}

\newcommand{\goalvertex}[3][]{
    \node[vertex interior,color=green!40] at (#2) (#3) {};
    \node[vertex border,#1] at (#3) {\small #3};
}

\newcommand{\numberedgoalvertex}[4][]{
    \node[vertex interior,color=green!40] at (#2) (#3) {};
    \node[vertex border,#1] at (#3) {\small #4};
}

\newcommand{\goalvertexnolabel}[3][]{
    \node[vertex interior,color=green!40] at (#2) (#3) {};
    \node[vertex border,#1] at (#3) {};
}

\newcommand{\transition}[5][]{\draw[<->,>=stealth,transition,#1] (#2)--(#3);}

\newcommand{\circletransition}{140:1.25mm}
\newcommand{\selftransition}[2][0:0mm]{
    \draw[->,>=stealth,transition] (#2) +(#1) arc(-45:225:1.5mm);}

\newcommand{\picfulltransitiongraphbase}[2]{%
  \begin{tikzpicture}[scale=1.4]
    #1{0,0}{LRR}
    %
    \draw[transition,->,>=stealth](-0.6,0) -- (LRR);
    %
    #1{1,0}{LLL}
    #1{1,1}{LLR}
    #1{1,-1}{LRL}
    %
    #1{2,1.5}{ALR}
    #1{2,0.5}{ALL}
    #1{2,-0.5}{BLL}
    #1{2,-1.5}{BRL}
    %
    #1{3,1.5}{ARL}
    #1{3,0.5}{ARR}
    #1{3,-0.5}{BRR}
    #1{3,-1.5}{BLR}
    %
    #2{4,0}{RRR}
    #2{4,1}{RRL}
    #2{4,-1}{RLR}
    %
    #2{5,0}{RLL}
    %
    \transition{LLL}{LRL}{move(A,R)}{move(A,L)}
    \transition{LLL}{LLR}{move(B,R)}{move(B,L)}
    \transition{LRL}{LRR}{move(B,R)}{move(B,L)}
    \transition{LLR}{LRR}{move(A,R)}{move(A,L)}
    %
    \transition{LLL}{ALL}{pick(A,L)}{drop(A,L)}
    \transition{LLL}{BLL}{pick(B,L)}{drop(B,L)}
    \transition{LLR}{ALR}{pick(A,L)}{drop(A,L)}
    \transition{LRL}{BRL}{pick(B,L)}{drop(B,L)}
    \transition{ALL}{ALR}{move(A,R)}{move(A,L)}
    \transition{BLL}{BRL}{move(B,R)}{move(B,L)}
    %
    \transition{ALR}{ARR}{move(A,R)}{move(A,L)}
    \transition{ALL}{ARL}{move(A,R)}{move(A,L)}
    \transition{BRL}{BRR}{move(B,R)}{move(B,L)}
    \transition{BLL}{BLR}{move(B,R)}{move(B,L)}
    %
    \transition{RRR}{ARR}{pick(A,R)}{drop(A,R)}
    \transition{RRR}{BRR}{pick(B,R)}{drop(B,R)}
    \transition{RRL}{ARL}{pick(A,R)}{drop(A,R)}
    \transition{RLR}{BLR}{pick(B,R)}{drop(B,R)}
    \transition{ARR}{ARL}{move(A,L)}{move(A,R)}
    \transition{BRR}{BLR}{move(B,L)}{move(B,R)}
    %
    \transition{RRR}{RLR}{move(A,L)}{move(A,R)}
    \transition{RRR}{RRL}{move(B,L)}{move(B,R)}
    \transition{RLR}{RLL}{move(B,L)}{move(B,R)}
    \transition{RRL}{RLL}{move(A,L)}{move(A,R)}
  \end{tikzpicture}%
}

\newcommand{\picfulltransitiongraph}{%
  \picfulltransitiongraphbase{\vertex}{\goalvertex}}

\newcommand{\picfulltransitiongraphnolabels}{%
  \picfulltransitiongraphbase{\vertexnolabel}{\goalvertexnolabel}}

\newcommand{\picperfectabstraction}{%
  \begin{tikzpicture}[scale=1.4]
    \vertex[text=gray]{0,0}{LRR}
    %
    \draw[transition,->,>=stealth](-0.6,0) -- (LRR);
    %
    \vertex{1,1}{LLR}
    \vertex{1,0}{LLL}
    \vertex{1,-1}{LRL}
    \verticalvertex{LLR}{LRL}{LLL}{}
    %
    \vertex{2,1.5}{ALR}
    \vertex{2,0.5}{ALL}
    \vertex{2,-0.5}{BLL}
    \vertex{2,-1.5}{BRL}
    \verticalvertex{ALR}{BRL}{ALL}{BLL}
    %
    \vertex{3,1.5}{ARL}
    \vertex{3,0.5}{ARR}
    \vertex{3,-0.5}{BRR}
    \vertex{3,-1.5}{BLR}
    \verticalvertex{ARL}{BLR}{ARR}{BRR}
    %
    %
    \vertex{4,0}{RRR}
    \vertex{4,1}{RRL}
    \vertex{4,-1}{RLR}
    \vertex{5,0}{RLL}
    \trianglevertexright[fill=green!40]{RLL}{RRL}{RLR}{RRR}
    %
    \transition{LLL}{LRR}{}{}
    %
    \transition{LLL}{LLL-|ALL.west}{}{}
    %
    \transition{LLL-|ALL.east}{RRR-|ARL.west}{}{}
    %
    \transition{RRR}{RRR-|ARL.east}{}{}
    %
    \selftransition[\circletransition]{LLR.north east}
    \selftransition[\circletransition]{ALR.north east}
    \selftransition[\circletransition]{ARL.north east}
    \selftransition[\circletransition]{RRL.north east}
  \end{tikzpicture}%
}

\newcommand{\piconestateabstraction}{%
  \begin{tikzpicture}[scale=1.4]
    \vertex[text=gray]{0,0}{LRR}
    %
    \draw[transition,->,>=stealth](-0.6,0) -- (LRR);
    %
    \vertex{1,1}{LLR}
    \vertex{1,0}{LLL}
    \vertex{1,-1}{LRL}
    \vertex{2,1.5}{ALR}
    \vertex{2,0.5}{ALL}
    \vertex{2,-0.5}{BLL}
    \vertex{2,-1.5}{BRL}
    \vertex{3,1.5}{ARL}
    \vertex{3,0.5}{ARR}
    \vertex{3,-0.5}{BRR}
    \vertex{3,-1.5}{BLR}
    \vertex{4,0}{RRR}
    \vertex{4,1}{RRL}
    \vertex{4,-1}{RLR}
    \vertex{5,0}{RLL}
	
    \draw[bigvertex interior,color=green!40] (LRR.230) arc(-130:-230:2.9mm)
    -- (LLR.140) arc(-220:-240:2.9mm)
    -- (ALR.120) arc(-240:-270:2.9mm) -- (ARL.north) arc(90:60:2.9mm)
    -- (RRL.60) arc (-300:-320:2.9mm) -- (RLL.50) arc(50:-50:2.9mm)
	-- (RLR.320) arc (-40:-60:2.9mm) -- (BLR.300) arc (-60:-90:2.9mm)
	-- (BRL.south) arc (-90:-120:2.9mm) -- (LRL.240) arc (-120:-140:2.9mm)
	-- (LRR.230);
	\hiddenvertices{LRR}{LLR}{LLL}{LRL}
	\hiddenvertices{ALR}{ALL}{BLL}{BRL}
	\hiddenvertices{ARL}{ARR}{BRR}{BLR}
	\hiddenvertices{RRR}{RRL}{RLR}{RLL}
    \draw[bigvertex border] (LRR.230) arc(-130:-230:2.9mm)
    -- (LLR.140) arc(-220:-240:2.9mm)
    -- (ALR.120) arc(-240:-270:2.9mm) -- (ARL.north) arc(90:60:2.9mm)
    -- (RRL.60) arc (-300:-320:2.9mm) -- (RLL.50) arc(50:-50:2.9mm)
	-- (RLR.320) arc (-40:-60:2.9mm) -- (BLR.300) arc (-60:-90:2.9mm)
	-- (BRL.south) arc (-90:-120:2.9mm) -- (LRL.240) arc (-120:-140:2.9mm)
	-- (LRR.230);
	
    \selftransition[\circletransition]{ARL.north east}
  \end{tikzpicture}%
}

\newcommand{\picexampleabstraction}[1][\verticalvertex]{%
  \begin{tikzpicture}[scale=1.4]
    \vertex[text=gray]{0,0}{LRR}
    %
    \draw[transition,->,>=stealth](-0.6,0) -- (LRR);
    %
    \vertex{1,1}{LLR}
    \vertex{1,0}{LLL}
    \vertex{1,-1}{LRL}
    #1{LLR}{LRL}{LLL}{}
    %
    \vertex{2,1.5}{ALR}
    \vertex{3,1.5}{ARL}
    \vertex{2,0.5}{ALL}
    \vertex{3,0.5}{ARR}
    \vertex{2,-0.5}{BLL}
    \vertex{2,-1.5}{BRL}
    \vertex{3,-0.5}{BRR}
    \vertex{3,-1.5}{BLR}
    \squarevertex{ALR}{ARL}{BLR}{BRL}
	\hiddenvertices{ALL}{ARR}{BLL}{BRR}
    %
    \vertex{4,0}{RRR}
    \vertex{4,1}{RRL}
    \vertex{4,-1}{RLR}
    \vertex{5,0}{RLL}
    \trianglevertexright[fill=green!40]{RLL}{RRL}{RLR}{RRR}
    %
    \draw[transition,->,>=stealth](-0.6,0) -- (LRR);
    %
    \transition{LRR}{LLL}{}{}
    \transition{LLL}{LLL-|ALL.west}{}{}
    \transition{RRR}{RRR-|ARL.east}{}{}
    %
    \selftransition[\circletransition]{LLR.north east}
    \selftransition[\circletransition]{RRL.north east}
    \selftransition[0:6mm]{ALR.north}
  \end{tikzpicture}
}

\newcommand{\picprojectionpackage}{%
  \begin{tikzpicture}[scale=1.4]
    \vertex{0,0}{LRR}
    \vertex{1,0}{LLL}
    \vertex{1,1}{LLR}
    \vertex{1,-1}{LRL}
    \trianglevertexleft{LRR}{LLR}{LRL}{LLL}
    %
    \vertex{2,1.5}{ALR}
    \vertex{3,1.5}{ARL}
    \vertex{2,0.5}{ALL}
    \vertex{3,0.5}{ARR}
    \squarevertex{ALR}{ARL}{ARR}{ALL}
    %
    \vertex{2,-0.5}{BLL}
    \vertex{2,-1.5}{BRL}
    \vertex{3,-0.5}{BRR}
    \vertex{3,-1.5}{BLR}
    \squarevertex{BLL}{BRR}{BLR}{BRL}
    %
    \vertex{4,0}{RRR}
    \vertex{4,1}{RRL}
    \vertex{4,-1}{RLR}
    \vertex{5,0}{RLL}
    \trianglevertexright[fill=green!40]{RLL}{RRL}{RLR}{RRR}
    %
    \draw[transition,->,>=stealth](-0.6,0) -- (LRR);
    %
    \transition{LLR}{LLR-|ALR.west}{}{}
    \transition{LRL}{LRL-|BRL.west}{}{}
    \transition{RRL}{RRL-|ARL.east}{}{}
    \transition{RLR}{RLR-|ARL.east}{}{}
    \selftransition[\circletransition]{LLR.north east}
    \selftransition[\circletransition]{RRL.north east}
    \selftransition[0:6mm]{ALR.north}
    \selftransition[0:6mm]{BLL.north}
  \end{tikzpicture}
}

\newcommand{\picprojectionpackagetruck}{
  \begin{tikzpicture}[scale=1.4]
    \vertex{0,0}{LRR}
    \vertex{1,-1}{LRL}
    \gravisvertex{LRR}{LRL}
    %
    \vertex{1,0}{LLL}
    \vertex{1,1}{LLR}
    \verticalvertex{LLR}{LLL}{}{}
    %
    \vertex{2,1.5}{ALR}
    \vertex{2,0.5}{ALL}
    \verticalvertex{ALR}{ALL}{}{}
    %
    \vertex{3,1.5}{ARL}
    \vertex{3,0.5}{ARR}
    \verticalvertex{ARL}{ARR}{}{}
    %
    \only<beamer:1|handout:0>{
      \vertex{3,-1.5}{BLR}
      \vertex{2,-0.5}{BLL}
      \vertex{3,-0.5}{BRR}
      \vertex{2,-1.5}{BRL}
      \gravisvertex{BLL}{BLR}
      \akutvertex{BRR}{BRL}
    }%
    \only<2>{
      \vertex{3,-1.5}{BRR}
      \vertex{2,-0.5}{BLL}
      \vertex{3,-0.5}{BLR}
      \vertex{2,-1.5}{BRL}
      \horizontalvertex{BLL}{BLR}
      \horizontalvertex{BRL}{BRR}
    }%
    %
    \vertex{4,0}{RRR}
    \vertex{4,1}{RRL}
    \verticalvertex[fill=green!40]{RRL}{RRR}{}{}
    %
    \vertex{4,-1}{RLR}
    \vertex{5,0}{RLL}
    \akutvertex[fill=green!40]{RLL}{RLR}
    %
    \draw[transition,->,>=stealth](-0.6,0) -- (LRR);
    \transition{intersection of LRR.north east--LRL.north east and LRR.east--LLL.west}
      {LLL.north-|LLR.west}{}{}
    \transition{ALL.north west-|LLR.east}{LLR.east-|ALL.west}{}{}
    \transition{LLR.east-|ALL.east}{LLR.east-|ARR.west}{}{}
    \transition{LLR.east-|ARL.east}{ARR.north east-|RRL.west}{}{}
    \transition{RRR.north-|RRR.east}
      {intersection of RLL.north west--RLR.north west and RLL.west--RRR.east}{}{}
    %
    \transition{LLL.north-|LLL.east}{BLL}{}{}
    \transition{LRL}{BRL}{}{}
    \only<beamer:1|handout:0>{\transition{BRR}{RRR.west|-RRR.north}{}{}}
    \only<2>{\transition{BRR.north east}{RRR.south west}{}{}}
    \transition{BLR}{RLR}{}{}
    \only<beamer:1|handout:0>{\transition{BRL.south east}{BLR.south west}{}{}}
    \only<2>{\transition{BLL.south-|BLL.east}{BRL.north-|BRL.east}{}{}}
    \selftransition[\circletransition]{LRR.north east}
    \selftransition[\circletransition]{LLR.north east}
    \selftransition[\circletransition]{ALR.north east}
    \selftransition[\circletransition]{ARL.north east}
    \selftransition[\circletransition]{RRL.north east}
    \selftransition[\circletransition]{RLL.north east}
    \only<beamer:1|handout:0>{
      \selftransition[\circletransition]{BLL.north east}
      \selftransition[\circletransition]{BRR.north east}
    }%
    \only<2>{
      \selftransition[1:8mm]{BLL.north}
      \selftransition[1:8mm]{BRL.north}
    }%
  \end{tikzpicture}%
}

\newcommand{\picrefinement}{%
  \begin{tikzpicture}[scale=1.4]
    \vertex{0,0}{LRR}
    \vertex{1,0}{LLL}
    \vertex{1,1}{LLR}
    \vertex{1,-1}{LRL}
    \trianglevertexleft{LRR}{LLR}{LRL}{LLL}
    \gravisvertexborder[subvertex]{LRR}{LRL}
    \verticalvertexborder[subvertex]{LLR}{LLL}
    \trianglevertexleftborder{LRR}{LLR}{LRL}{LLL}
    %
    \vertex{2,1.5}{ALR}
    \vertex{3,1.5}{ARL}
    \vertex{2,0.5}{ALL}
    \vertex{3,0.5}{ARR}
    \squarevertex{ALR}{ARL}{ARR}{ALL}
    \verticalvertexborder[subvertex]{ALR}{ALL}
    \verticalvertexborder[subvertex]{ARL}{ARR}
    \squarevertexborder{ALR}{ARL}{ARR}{ALL}
    %
    \vertex{2,-0.5}{BLL}
    \vertex{2,-1.5}{BRL}
    \vertex{3,-0.5}{BLR}
    \vertex{3,-1.5}{BRR}
    \squarevertex{BLL}{BLR}{BRR}{BRL}
    \horizontalvertexborder[subvertex]{BLL}{BLR}
    \horizontalvertexborder[subvertex]{BRL}{BRR}
    \squarevertexborder{BLL}{BLR}{BRR}{BRL}
    %
    \vertex{4,0}{RRR}
    \vertex{4,1}{RRL}
    \vertex{4,-1}{RLR}
    \vertex{5,0}{RLL}
    \trianglevertexright[fill=green!40]{RLL}{RRL}{RLR}{RRR}
    \verticalvertexborder[subvertex]{RRL}{RRR}
    \akutvertexborder[subvertex]{RLL}{RLR}
    \trianglevertexrightborder{RLL}{RRL}{RLR}{RRR}
	%
    \draw[transition,->,>=stealth](-0.6,0) -- (LRR);
    %
    \transition{LLR}{LLR-|ALR.west}{}{}
    \transition{LRL}{LRL-|BRL.west}{}{}
    \transition{RRL}{RRL-|ARL.east}{}{}
    \transition{RLR}{RLR-|ARL.east}{}{}
    \selftransition[\circletransition]{LLR.north east}
    \selftransition[\circletransition]{RRL.north east}
    \selftransition[0:6mm]{ALR.north}
    \selftransition[0:6mm]{BLL.north}
  \end{tikzpicture}
}


\newcommand{\pictwopackages}{%
  \begin{tikzpicture}[scale=1.28]
    \vertex{0,0}{LLR}
    \vertex{1,0}{LLL}
    \vertex{2,.5}{ILL}
    \vertex{2,-.5}{LIL}
    \vertex{3,0}{IIL}
    \vertex{4,0}{IIR}
    \vertex{5,.5}{RIR}
    \vertex{5,-.5}{IRR}
    \goalvertex{6,0}{RRR}
    \goalvertex{7,0}{RRL}
    \vertex{2.3,1.2}{ILR}
    \vertex{3.1,1.5}{RLR}
    \vertex{3.85,1.5}{RLL}
    \vertex{4.7,1.2}{RIL}
    \vertex{2.3,-1.2}{LIR}
    \vertex{3.1,-1.5}{LRR}
    \vertex{3.85,-1.5}{LRL}
    \vertex{4.7,-1.2}{IRL}

    \draw[transition,->,>=stealth](-0.6,0) -- (LLR);
    \transition{LLR}{LLL}{}{}
    \transition{LLL}{ILL}{}{}
    \transition{LLL}{LIL}{}{}
    \transition{ILL}{IIL}{}{}
    \transition{LIL}{IIL}{}{}
    \transition{IIL}{IIR}{}{}
    \transition{IIR}{RIR}{}{}
    \transition{IIR}{IRR}{}{}
    \transition{RIR}{RRR}{}{}
    \transition{IRR}{RRR}{}{}
    \transition{RRR}{RRL}{}{}
    \transition{ILL}{ILR}{}{}
    \transition{RLR}{ILR}{}{}
    \transition{RLR}{RLL}{}{}
    \transition{RIL}{RLL}{}{}
    \transition{RIL}{RIR}{}{}
    \transition{LIL}{LIR}{}{}
    \transition{LRR}{LIR}{}{}
    \transition{LRR}{LRL}{}{}
    \transition{LRL}{IRL}{}{}
    \transition{IRR}{IRL}{}{}
  \end{tikzpicture}
}

\newlength{\vertexround}
\setlength{\vertexround}{3.14mm}
\newcommand{\pictwopackagesabstractpacktwo}{%
  \begin{tikzpicture}[scale=1.28]
    \vertex{0,0}{LLR}
    \vertex{1,0}{LLL}
    \vertex{2,.5}{ILL}
    \vertex{2,-.5}{LIL}
    \vertex{3,0}{IIL}
    \vertex{4,0}{IIR}
    \vertex{5,.5}{RIR}
    \vertex{5,-.5}{IRR}
    \goalvertex{6,0}{RRR}
    \goalvertex{7,0}{RRL}
    \vertex{2.3,1.2}{ILR}
    \vertex{3.1,1.5}{RLR}
    \vertex{3.85,1.5}{RLL}
    \vertex{4.7,1.2}{RIL}
    \vertex{2.3,-1.2}{LIR}
    \vertex{3.1,-1.5}{LRR}
    \vertex{3.85,-1.5}{LRL}
    \vertex{4.7,-1.2}{IRL}
    
    \draw[bigvertex interior] (LLR.south) arc(-90:-240:\vertexround)
    -- (ILR.120) arc(120:100:\vertexround) 
	-- (RLR.110) arc (110:90:\vertexround)
    -- (RLL.north) arc(90:-60:\vertexround) -- (ILL.-60)
    -- (LLL.300) arc (-60:-90:\vertexround) -- (LLR.south);
	\hiddenvertices{LLR}{LLL}{ILL}{ILR}
	\hiddenvertices{RLR}{RLL}{}{}
    \draw[bigvertex border] (LLR.south) arc(-90:-240:\vertexround)
    -- (ILR.120) arc(120:100:\vertexround) 
	-- (RLR.120) arc (120:90:\vertexround)
    -- (RLL.north) arc(90:-60:\vertexround) -- (ILL.-60)
    -- (LLL.300) arc (-60:-90:\vertexround) -- (LLR.south);
    
    \draw[bigvertex interior] (LIR.310) arc(-50:-165:\vertexround)
    -- (LIL.200) arc(-160:-240:\vertexround) 
    -- (RIL.120) arc(120:10:\vertexround)
    -- (RIR.20) arc(20:-50:\vertexround)
    -- (LIR.310);
	\hiddenvertices{LIR}{LIL}{IIL}{IIR}
	\hiddenvertices{RIR}{RIL}{}{}
    \draw[bigvertex border] (LIR.310) arc(-50:-165:\vertexround)
    -- (LIL.200) arc(-160:-240:\vertexround) 
    -- (RIL.120) arc(120:10:\vertexround)
    -- (RIR.20) arc(20:-50:\vertexround)
    -- (LIR.310);

    \draw[bigvertex interior, color=green!40] (RRL.north) arc(90:-60:\vertexround)
    -- (IRL.-60) arc(-60:-80:\vertexround)
	-- (LRL.-70) arc (-70:-90:\vertexround)
    -- (LRR.south) arc(-90:-240:\vertexround) -- (IRR.120)
    -- (RRR.120) arc (120:90:\vertexround) -- (RRL.north);
	\hiddenvertices{LRR}{LRL}{IRL}{IRR}
	\hiddenvertices{RRR}{RRL}{}{}
    \draw[bigvertex border] (RRL.north) arc(90:-60:\vertexround)
    -- (IRL.-60) arc(-60:-80:\vertexround)
	-- (LRL.-70) arc (-70:-90:\vertexround)
    -- (LRR.south) arc(-90:-240:\vertexround) -- (IRR.120)
    -- (RRR.120) arc (120:90:\vertexround) -- (RRL.north);

    \draw[transition,->,>=stealth](-0.6,0) -- (LLR);
    \transition{ILL.-60}{intersection of LIL.-240 -- RIL.120 and ILR.180 -- LRR.180}{}{}
    \transition{IRR.120}{intersection of RIR.-50 -- LIR.310 and IRL.0 -- RLL.0}{}{}
    \selftransition[-.5mm:-3mm]{RRL.north}
    \selftransition[-.1mm:1mm]{RIL.north}
    \begin{scope}[yscale=-1]
      \selftransition[-.5mm:-3mm]{LLL.south}
    \end{scope}

    \only<all:5>{
      \transition[gray,->]{LLR}{LLL}{}{}
      \transition[gray,->]{LLL}{ILL}{}{}
      \transition[gray,->]{ILL}{IIL}{}{}
      \transition[gray,->]{IIL}{IIR}{}{}
      \transition[gray,->]{IIR}{RIR}{}{}
      \transition[gray,->]{RIR}{RRR}{}{}
    }
  \end{tikzpicture}
}

\newcommand{\pictwopackagesabstractpackone}{%
  \begin{tikzpicture}[scale=1.28]
    \vertex{0,0}{LLR}
    \vertex{1,0}{LLL}
    \vertex{2,.5}{ILL}
    \vertex{2,-.5}{LIL}
    \vertex{3,0}{IIL}
    \vertex{4,0}{IIR}
    \vertex{5,.5}{RIR}
    \vertex{5,-.5}{IRR}
    \goalvertex{6,0}{RRR}
    \goalvertex{7,0}{RRL}
    \vertex{2.3,1.2}{ILR}
    \vertex{3.1,1.5}{RLR}
    \vertex{3.85,1.5}{RLL}
    \vertex{4.7,1.2}{RIL}
    \vertex{2.3,-1.2}{LIR}
    \vertex{3.1,-1.5}{LRR}
    \vertex{3.85,-1.5}{LRL}
    \vertex{4.7,-1.2}{IRL}
    
    \draw[bigvertex interior] (LLR.north) arc(90:240:\vertexround)
    -- (LIR.-120) arc(-120:-100:\vertexround) 
    -- (LRR.-110) arc (-110:-90:\vertexround)
    -- (LRL.south) arc(-90:60:\vertexround) -- (LIL.60)
    -- (LLL.-300) arc (60:90:\vertexround) -- (LLR.north);
    \hiddenvertices{LLR}{LLL}{LIL}{LIR}
    \hiddenvertices{LRR}{LRL}{}{}
    \draw[bigvertex border] (LLR.north) arc(90:240:\vertexround)
    -- (LIR.-120) arc(-120:-100:\vertexround) 
    -- (LRR.-110) arc (-110:-90:\vertexround)
    -- (LRL.south) arc(-90:60:\vertexround) -- (LIL.60)
    -- (LLL.-300) arc (60:90:\vertexround) -- (LLR.north);
    
    \draw[bigvertex interior] (ILR.-310) arc(50:165:\vertexround)
    -- (ILL.-200) arc(160:240:\vertexround) 
    -- (IRL.-120) arc(-120:-10:\vertexround)
    -- (IRR.-20) arc(-20:50:\vertexround)
    -- (ILR.-310);
    \hiddenvertices{ILR}{ILL}{IIL}{IIR}
    \hiddenvertices{IRR}{IRL}{}{}
    \draw[bigvertex border] (ILR.-310) arc(50:165:\vertexround)
    -- (ILL.-200) arc(160:240:\vertexround) 
    -- (IRL.-120) arc(-120:-10:\vertexround)
    -- (IRR.-20) arc(-20:50:\vertexround)
    -- (ILR.-310);

    \draw[bigvertex interior, color=green!40] (RRL.south) arc(-90:60:\vertexround)
    -- (RIL.60) arc(60:80:\vertexround)
    -- (RLL.70) arc (70:90:\vertexround)
    -- (RLR.north) arc(90:240:\vertexround) -- (RIR.-120)
    -- (RRR.-120) arc (-120:-90:\vertexround) -- (RRL.south);
    \hiddenvertices{RLR}{RLL}{RIL}{RIR}
    \hiddenvertices{RRR}{RRL}{}{}
    \draw[bigvertex border] (RRL.south) arc(-90:60:\vertexround)
    -- (RIL.60) arc(60:80:\vertexround)
    -- (RLL.70) arc (70:90:\vertexround)
    -- (RLR.north) arc(90:240:\vertexround) -- (RIR.-120)
    -- (RRR.-120) arc (-120:-90:\vertexround) -- (RRL.south);

    \draw[transition,->,>=stealth](-0.6,0) -- (LLR);
    \transition{LIL.60}{intersection of ILL.240 -- IRL.240 and LIR.180 -- RLR.180}{}{}
    \transition{LIL.60}{intersection of ILL.240 -- IRL.-120 and LIR.-180 -- RLR.-180}{}{}
    \transition{RIR.-120}{intersection of IRR.50 -- ILR.-310 and RIL.0 -- LRL.0}{}{}
    \selftransition[-.5mm:-3mm]{LLL.north}
    \selftransition[-.1mm:1mm]{ILR.north}
    \begin{scope}[yscale=-1]
      \selftransition[-.5mm:-3mm]{RRL.south}
    \end{scope}

    \only<all:3>{
      \transition[gray,->]{LLR}{LLL}{}{}
      \transition[gray,->]{LLL}{ILL}{}{}
      \transition[gray,->]{ILL}{IIL}{}{}
      \transition[gray,->]{IIL}{IIR}{}{}
      \transition[gray,->]{IIR}{RIR}{}{}
      \transition[gray,->]{RIR}{RRR}{}{}
    }
  \end{tikzpicture}
}

\newcommand{\bendtransition}[6][]{\draw[->,>=stealth,transition,#1] (#2) edge [#6] node[above=-1mm,sloped,#5] {\scriptsize #4} (#3);}
\newcommand{\bendtransitionnolabel}[6][]{\draw[->,>=stealth,transition,#1] (#2) edge [#6] node[above=-1mm,sloped,#5] {} (#3);}
\newcommand{\picatomicprojectionpackage}{%
  \begin{tikzpicture}[->,>=stealth,looseness=.5,bend angle=20]
       	\vertex{0,0}{L}
	\vertex{2,1}{A}
	\vertex{2,-1}{B}
	\goalvertex{4,0}{R}

	\draw[transition,->,>=stealth](-0.8,0) -- (L);
	\bendtransition{L}{L}{M{\wildcard}{\wildcard}{\wildcard}}{above left}{loop above}
	\bendtransition{L}{A}{PAL}{}{bend left}
	\bendtransition{A}{L}{DAL}{}{bend left}
	\bendtransition{A}{A}{M{\wildcard}{\wildcard}{\wildcard}}{above=1mm}{loop above}
	\bendtransition{A}{R}{DAR}{}{bend left}
	\bendtransition{R}{A}{PAR}{}{bend left}
	\bendtransition{R}{R}{M{\wildcard}{\wildcard}{\wildcard}}{above right}{loop above}
	\bendtransition{R}{B}{PBR}{below=-2mm}{bend left}
	\bendtransition{B}{R}{DBR}{below=-2mm}{bend left}
	\bendtransition{B}{B}{M{\wildcard}{\wildcard}{\wildcard}}{below=-1mm}{loop below}
	\bendtransition{B}{L}{DBL}{below=-2mm}{bend left}
	\bendtransition{L}{B}{PBL}{below=-2mm}{bend left}
  \end{tikzpicture}
}

\newcommand{\picatomicprojectionpackagenumberedstates}{%
  \begin{tikzpicture}[->,>=stealth,looseness=.5,bend angle=20]
        \numberedvertex{0,0}{L}{0}
	\numberedvertex{2,1}{A}{2}
	\numberedvertex{2,-1}{B}{3}
	\numberedgoalvertex{4,0}{R}{1}

	\draw[transition,->,>=stealth](-0.8,0) -- (L);
	\bendtransition{L}{L}{M{\wildcard}{\wildcard}{\wildcard}}{above left}{loop above}
	\bendtransition{L}{A}{PAL}{}{bend left}
	\bendtransition{A}{L}{DAL}{}{bend left}
	\bendtransition{A}{A}{M{\wildcard}{\wildcard}{\wildcard}}{above=1mm}{loop above}
	\bendtransition{A}{R}{DAR}{}{bend left}
	\bendtransition{R}{A}{PAR}{}{bend left}
	\bendtransition{R}{R}{M{\wildcard}{\wildcard}{\wildcard}}{above right}{loop above}
	\bendtransition{R}{B}{PBR}{below=-2mm}{bend left}
	\bendtransition{B}{R}{DBR}{below=-2mm}{bend left}
	\bendtransition{B}{B}{M{\wildcard}{\wildcard}{\wildcard}}{below=-1mm}{loop below}
	\bendtransition{B}{L}{DBL}{below=-2mm}{bend left}
	\bendtransition{L}{B}{PBL}{below=-2mm}{bend left}
  \end{tikzpicture}
}

\newcommand{\picatomicprojectiontrucka}{%
  \begin{tikzpicture}[->,>=stealth,looseness=.5]
	\goalvertex{0,0}{L}
	\goalvertex{4,0}{R}

	\draw[transition,->,>=stealth](4.8,0) -- (R);
	\bendtransition{L}{L}{\parbox{22mm}{PAL,DAL,MB{\wildcard}{\wildcard},\\PB{\wildcard},DB{\wildcard}}}{above}{loop above}
	\bendtransition{L}{R}{MALR}{}{bend left}
	\bendtransition{R}{L}{MARL}{}{bend left}
	\bendtransition{R}{R}{\parbox{22mm}{PAR,DAR,MB{\wildcard}{\wildcard},\\PB{\wildcard},DB{\wildcard}}}{above}{loop above}
  \end{tikzpicture}
}

\newcommand{\picatomicprojectiontruckb}{%
  \begin{tikzpicture}[->,>=stealth,looseness=.5]
	\goalvertex{0,0}{L}
	\goalvertex{4,0}{R}

	\draw[transition,->,>=stealth](4.8,0) -- (R);
	\bendtransition{L}{L}{\parbox{22mm}{PBL,DBL,MA{\wildcard}{\wildcard},\\PA{\wildcard},DA{\wildcard}}}{above}{loop above}
	\bendtransition{L}{R}{MBLR}{}{bend left}
	\bendtransition{R}{L}{MBRL}{}{bend left}
	\bendtransition{R}{R}{\parbox{22mm}{PBR,DBR,MA{\wildcard}{\wildcard},\\PA{\wildcard},DA{\wildcard}}}{above}{loop above}
  \end{tikzpicture}
}

\newcommand{\picproductpackagetrucka}[1][]{%
  \begin{tikzpicture}[->,>=stealth,looseness=.5,bend angle=20,#1]
  	\vertex{0,0}{LL}
  	\vertex{2,0}{LR}
	\vertex{3,2}{AL}
	\vertex{5,2}{AR}
	\vertex{3,-2}{BL}
	\vertex{5,-2}{BR}
  	\goalvertex{6,0}{RL}
  	\goalvertex{8,0}{RR}
	
	\draw[transition,->,>=stealth](2.6,0.6) -- (LR);
	\bendtransition{LL}{LR}{MALR}{}{bend left}
	\bendtransition{LR}{LL}{MARL}{below=-2mm}{bend left}
	\bendtransition{AL}{AR}{MALR}{}{bend left}
	\bendtransition{AR}{AL}{MARL}{below=-2mm}{bend left}
	\bendtransition{BL}{BR}{MALR}{}{bend left}
	\bendtransition{BR}{BL}{MARL}{below=-2mm}{bend left}
	\bendtransition{RL}{RR}{MALR}{}{bend left}
	\bendtransition{RR}{RL}{MARL}{below=-2mm}{bend left}
	
	\bendtransition{LL}{AL}{PAL}{}{bend left=50}
	\bendtransition{AL}{LL}{DAL}{}{bend right}
	\bendtransition{AR}{RR}{DAR}{}{bend left=50}
	\bendtransition{RR}{AR}{PAR}{}{bend right}
	\bendtransition{RR}{BR}{PBR}{}{bend left=50}
	\bendtransition{BR}{RR}{DBR}{}{bend right}
	\bendtransition{BL}{LL}{DBL}{}{bend left=50}
	\bendtransition{LL}{BL}{PBL}{}{bend right}
	
	\bendtransition{LR}{BR}{PBL}{near start}{bend left=50}
	\bendtransition{BR}{LR}{DBL}{near end,below=-2mm}{bend right}
	\bendtransition{BL}{RL}{DBR}{near end}{bend left=50}
	\bendtransition{RL}{BL}{PBR}{near start,below=-2mm}{bend right}
	
	\bendtransition{LL}{LL}{MB{\wildcard}{\wildcard}}{above=2mm,pos=.9,sloped=false}{loop left}
	\bendtransition{RR}{RR}{MB{\wildcard}{\wildcard}}{above=2mm,pos=.1,sloped=false}{loop right}
	\bendtransition{AL}{AL}{MB{\wildcard}{\wildcard}}{}{loop above}
	\bendtransition{AR}{AR}{MB{\wildcard}{\wildcard}}{}{loop above}
	\bendtransition{LR}{LR}{MB{\wildcard}{\wildcard}}{}{loop above}
	\bendtransition{RL}{RL}{MB{\wildcard}{\wildcard}}{}{loop above}
	\bendtransition{BL}{BL}{MB{\wildcard}{\wildcard}}{below=-1mm}{loop below}
	\bendtransition{BR}{BR}{MB{\wildcard}{\wildcard}}{below=-1mm}{loop below}
  \end{tikzpicture}
}

\newcommand{\picproductpackagetruckanumberedstates}[1][]{%
  \begin{tikzpicture}[->,>=stealth,looseness=.5,bend angle=20,#1]
  	\numberedvertex{0,0}{LL}{0}
  	\numberedvertex{2,0}{LR}{1}
	\numberedvertex{3,2}{AL}{4}
	\numberedvertex{5,2}{AR}{5}
	\numberedvertex{3,-2}{BL}{6}
	\numberedvertex{5,-2}{BR}{7}
  	\numberedgoalvertex{6,0}{RL}{2}
  	\numberedgoalvertex{8,0}{RR}{3}
	
	\draw[transition,->,>=stealth](2.6,0.6) -- (LR);
	\bendtransition{LL}{LR}{MALR}{}{bend left}
	\bendtransition{LR}{LL}{MARL}{below=-2mm}{bend left}
	\bendtransition{AL}{AR}{MALR}{}{bend left}
	\bendtransition{AR}{AL}{MARL}{below=-2mm}{bend left}
	\bendtransition{BL}{BR}{MALR}{}{bend left}
	\bendtransition{BR}{BL}{MARL}{below=-2mm}{bend left}
	\bendtransition{RL}{RR}{MALR}{}{bend left}
	\bendtransition{RR}{RL}{MARL}{below=-2mm}{bend left}
	
	\bendtransition{LL}{AL}{PAL}{}{bend left=50}
	\bendtransition{AL}{LL}{DAL}{}{bend right}
	\bendtransition{AR}{RR}{DAR}{}{bend left=50}
	\bendtransition{RR}{AR}{PAR}{}{bend right}
	\bendtransition{RR}{BR}{PBR}{}{bend left=50}
	\bendtransition{BR}{RR}{DBR}{}{bend right}
	\bendtransition{BL}{LL}{DBL}{}{bend left=50}
	\bendtransition{LL}{BL}{PBL}{}{bend right}
	
	\bendtransition{LR}{BR}{PBL}{near start}{bend left=50}
	\bendtransition{BR}{LR}{DBL}{near end,below=-2mm}{bend right}
	\bendtransition{BL}{RL}{DBR}{near end}{bend left=50}
	\bendtransition{RL}{BL}{PBR}{near start,below=-2mm}{bend right}
	
	\bendtransition{LL}{LL}{MB{\wildcard}{\wildcard}}{above=2mm,pos=.9,sloped=false}{loop left}
	\bendtransition{RR}{RR}{MB{\wildcard}{\wildcard}}{above=2mm,pos=.1,sloped=false}{loop right}
	\bendtransition{AL}{AL}{MB{\wildcard}{\wildcard}}{}{loop above}
	\bendtransition{AR}{AR}{MB{\wildcard}{\wildcard}}{}{loop above}
	\bendtransition{LR}{LR}{MB{\wildcard}{\wildcard}}{}{loop above}
	\bendtransition{RL}{RL}{MB{\wildcard}{\wildcard}}{}{loop above}
	\bendtransition{BL}{BL}{MB{\wildcard}{\wildcard}}{below=-1mm}{loop below}
	\bendtransition{BR}{BR}{MB{\wildcard}{\wildcard}}{below=-1mm}{loop below}
  \end{tikzpicture}
}

\newcommand{\subpicatomicprojectionpackagesmall}{%
  \begin{scope}[->,>=stealth,looseness=.5,bend angle=20,scale=.7,transform shape]
	\vertex{0,0}{L}
	\vertex{2,1}{A}
	\vertex{2,-1}{B}
	\goalvertex{4,0}{R}

	\draw[transition,->,>=stealth](-0.8,0) -- (L);
	\bendtransition{L}{L}{M{\wildcard}{\wildcard}{\wildcard}}{above left}{loop above}
	\bendtransition{L}{A}{PAL}{}{bend left}
	\bendtransition{A}{L}{DAL}{}{bend left}
	\bendtransition{A}{A}{M{\wildcard}{\wildcard}{\wildcard}}{above=1mm}{loop above}
	\bendtransition{A}{R}{DAR}{}{bend left}
	\bendtransition{R}{A}{PAR}{}{bend left}
	\bendtransition{R}{R}{M{\wildcard}{\wildcard}{\wildcard}}{above right}{loop above}
	\bendtransition{R}{B}{PBR}{below=-2mm}{bend left}
	\bendtransition{B}{R}{DBR}{below=-2mm}{bend left}
	\bendtransition{B}{B}{M{\wildcard}{\wildcard}{\wildcard}}{below=-1mm}{loop below}
	\bendtransition{B}{L}{DBL}{below=-2mm}{bend left}
	\bendtransition{L}{B}{PBL}{below=-2mm}{bend left}
	
	\only<all:2>{\vertex[draw=red,thick]{2,1}{A}}
	\only<all:3>{\vertex[draw=red,thick]{0,0}{L}}
	\only<all:3>{\draw[transition,->,>=stealth,red](-0.8,0) -- (L);}
	\only<all:4>{\goalvertex[fill=green]{4,0}{R}}
	\only<all:5>{\bendtransition[red]{L}{A}{PAL}{}{bend left}}
	\only<all:6>{\bendtransition[red]{L}{L}{M{\wildcard}{\wildcard}{\wildcard}}{above left}{loop above}}
	\only<all:7>{\bendtransition[red]{L}{B}{PBL}{below=-2mm}{bend left}}
	\only<all:8>{\bendtransition[red]{A}{A}{M{\wildcard}{\wildcard}{\wildcard}}{above=1mm}{loop above}}
  \end{scope}
}

\newcommand{\subpicatomicprojectiontruckasmall}{%
  \begin{scope}[->,>=stealth,looseness=.5,scale=.7,transform shape]
	\goalvertex{0,0}{L}
	\goalvertex{4,0}{R}

	\draw[transition,->,>=stealth](4.8,0) -- (R);
	\bendtransition{L}{L}{\parbox{22mm}{PAL,DAL,MB{\wildcard}{\wildcard},\\PB{\wildcard},DB{\wildcard}}}{above}{loop above}
	\bendtransition{L}{R}{MALR}{}{bend left}
	\bendtransition{R}{L}{MARL}{}{bend left}
	\bendtransition{R}{R}{\parbox{22mm}{PAR,DAR,MB{\wildcard}{\wildcard},\\PB{\wildcard},DB{\wildcard}}}{above}{loop above}

	\only<all:2>{\goalvertex[draw=red,thick]{0,0}{L}}
	\only<all:3>{\goalvertex[draw=red,thick]{4,0}{R}}
	\only<all:3>{\draw[transition,->,>=stealth,red](4.8,0) -- (R);}
	\only<all:4>{\goalvertex[fill=green]{0,0}{L}}
	\only<all:5>{\bendtransition[red]{L}{L}{\parbox{22mm}{PAL\color{black},DAL,MB{\wildcard}{\wildcard},\\PB{\wildcard},DB{\wildcard}}}{above}{loop above}}
	\only<all:6>{\bendtransition[red]{L}{R}{MALR}{}{bend left}}
	\only<all:7>{\bendtransition[red]{R}{R}{\parbox{22mm}{\color{black}PAR,DAR,MB{\wildcard}{\wildcard},\\{\color{red}PB{\wildcard}},DB{\wildcard}}}{above}{loop above}}
	\only<all:8>{\bendtransition[red]{L}{L}{\parbox{22mm}{\color{black}PAL,DAL,{\color{red}MB{\wildcard}{\wildcard}},\\PB{\wildcard},DB{\wildcard}}}{above}{loop above}}
  \end{scope}
}

\newcommand{\subpicproductpackagetruckasmall}{%
  \begin{scope}[->,>=stealth,looseness=.5,,bend angle=20,scale=.7,transform shape]
  	\vertex{0,0}{LL}
  	\vertex{2,0}{LR}
	\vertex{3,2}{AL}
	\vertex{5,2}{AR}
	\vertex{3,-2}{BL}
	\vertex{5,-2}{BR}
  	\goalvertex{6,0}{RL}
  	\goalvertex{8,0}{RR}
	
	\draw[transition,->,>=stealth](2.6,0.6) -- (LR);
	\bendtransition{LL}{LR}{MALR}{}{bend left}
	\bendtransition{LR}{LL}{MARL}{below=-2mm}{bend left}
	\bendtransition{AL}{AR}{MALR}{}{bend left}
	\bendtransition{AR}{AL}{MARL}{below=-2mm}{bend left}
	\bendtransition{BL}{BR}{MALR}{}{bend left}
	\bendtransition{BR}{BL}{MARL}{below=-2mm}{bend left}
	\bendtransition{RL}{RR}{MALR}{}{bend left}
	\bendtransition{RR}{RL}{MARL}{below=-2mm}{bend left}
	
	\bendtransition{LL}{AL}{PAL}{}{bend left=50}
	\bendtransition{AL}{LL}{DAL}{}{bend right}
	\bendtransition{AR}{RR}{DAR}{}{bend left=50}
	\bendtransition{RR}{AR}{PAR}{}{bend right}
	\bendtransition{RR}{BR}{PBR}{}{bend left=50}
	\bendtransition{BR}{RR}{DBR}{}{bend right}
	\bendtransition{BL}{LL}{DBL}{}{bend left=50}
	\bendtransition{LL}{BL}{PBL}{}{bend right}
	
	\bendtransition{LR}{BR}{PBL}{near start}{bend left=50}
	\bendtransition{BR}{LR}{DBL}{near end,below=-2mm}{bend right}
	\bendtransition{BL}{RL}{DBR}{near end}{bend left=50}
	\bendtransition{RL}{BL}{PBR}{near start,below=-2mm}{bend right}
	
	\bendtransition{LL}{LL}{MB{\wildcard}{\wildcard}}{above=2mm,sloped=false,pos=.9}{loop left}
	\bendtransition{RR}{RR}{MB{\wildcard}{\wildcard}}{above=2mm,sloped=false,pos=.1}{loop right}
	\bendtransition{AL}{AL}{MB{\wildcard}{\wildcard}}{}{loop above}
	\bendtransition{AR}{AR}{MB{\wildcard}{\wildcard}}{}{loop above}
	\bendtransition{LR}{LR}{MB{\wildcard}{\wildcard}}{}{loop above}
	\bendtransition{RL}{RL}{MB{\wildcard}{\wildcard}}{}{loop above}
	\bendtransition{BL}{BL}{MB{\wildcard}{\wildcard}}{below=-1mm}{loop below}
	\bendtransition{BR}{BR}{MB{\wildcard}{\wildcard}}{below=-1mm}{loop below}
	
	\only<all:2>{\vertex[draw=red,thick]{3,2}{AL}}
	\only<all:3>{\vertex[draw=red,thick]{2,0}{LR}}
	\only<all:3>{\draw[transition,->,>=stealth,red](2.6,0.6) -- (LR);}
  	\only<all:4>{\goalvertex[fill=green]{6,0}{RL}}
	\only<all:5>{\bendtransition[red]{LL}{AL}{PAL}{}{bend left=50}}
	\only<all:6>{\bendtransition[red]{LL}{LR}{MALR}{}{bend left}}
	\only<all:7>{\bendtransition[red]{LR}{BR}{PBL}{near start}{bend left=50}}
	\only<all:8>{\bendtransition[red]{AL}{AL}{MB{\wildcard}{\wildcard}}{}{loop above}}
  \end{scope}
}

\newcommand{\picproductpackagetruckaillustration}{%
  \begin{tikzpicture}[->,>=stealth,looseness=.5,,bend angle=20,scale=0.98]
  	\subpicatomicprojectionpackagesmall
	\node (otimes) at (4,0) {\LARGE $\otimes$};	
	\begin{scope}[xshift=5cm]
  	\subpicatomicprojectiontruckasmall
	\end{scope}
	\node (otimes) at (0,-3) {\LARGE $=$};	
	\begin{scope}[xshift=2cm,yshift=-3cm]
  	\subpicproductpackagetruckasmall
	\end{scope}
  \end{tikzpicture}
}


\newcommand{\picshrinkexampleone}[1][]{%
  \begin{tikzpicture}[->,>=stealth,looseness=.5,,bend angle=20,#1]
  	\vertex{0,0}{LL}
  	\vertex{2,0}{LR}
	\only<all:1-4>{\vertex{3,2}{AL}}
	\only<all:1-4>{\vertex{5,2}{AR}}
	\only<all:4>{\vertex[draw=red,thick]{3,2}{AL}}
	\only<all:4>{\vertex[draw=red,thick]{5,2}{AR}}
	\only<all:5-8>{\vertex{4,2}{A}}
	\only<all:8>{\vertex[draw=red,thick]{4,2}{A}}

	\only<all:1-6>{\vertex{3,-2}{BL}}
	\only<all:1-6>{\vertex{5,-2}{BR}}
	\only<all:6>{\vertex[draw=red,thick]{3,-2}{BL}}
	\only<all:6>{\vertex[draw=red,thick]{5,-2}{BR}}
	\only<all:7-8>{\vertex{4,-2}{B}}
	\only<all:8>{\vertex[draw=red,thick]{4,-2}{B}}
	\only<all:9->{\vertex{4,0}{I}}

	\only<all:1>{\goalvertex{6,0}{RL}}
  	\only<all:1>{\goalvertex{8,0}{RR}}
  	\only<all:2>{\goalvertex[draw=red,thick]{6,0}{RL}}
  	\only<all:2>{\goalvertex[draw=red,thick]{8,0}{RR}}
	\only<all:3->{\goalvertex{7,0}{R}}

	
	\draw[transition,->,>=stealth](2.6,0.6) -- (LR);
	\bendtransition{LL}{LR}{MALR}{}{bend left}
	\only<all:1-8>{\bendtransition{LR}{LL}{MARL}{below=-2mm}{bend left}}
	\only<all:9->{\bendtransition{LR}{LL}{MARL}{}{bend left}}
	\only<all:1-4>{\bendtransition{AL}{AR}{MALR}{}{bend left}}
	\only<all:1-4>{\bendtransition{AR}{AL}{MARL}{below=-2mm}{bend left}}
	\only<all:1-6>{\bendtransition{BL}{BR}{MALR}{}{bend left}}
	\only<all:1-6>{\bendtransition{BR}{BL}{MARL}{below=-2mm}{bend left}}
	\only<all:1-2>{\bendtransition{RL}{RR}{MALR}{}{bend left}}
	\only<all:1-2>{\bendtransition{RR}{RL}{MARL}{below=-2mm}{bend left}}
	
	\only<all:1-4>{\bendtransition{LL}{AL}{PAL}{}{bend left=50}}
	\only<all:1-4>{\bendtransition{AL}{LL}{DAL}{}{bend right}}
	\only<all:5-8>{\bendtransition{LL}{A}{PAL}{}{bend left=50}}
	\only<all:5-8>{\bendtransition{A}{LL}{DAL}{}{bend right}}
	\only<all:1-2>{\bendtransition{AR}{RR}{DAR}{}{bend left=50}}
	\only<all:1-2>{\bendtransition{RR}{AR}{PAR}{}{bend right}}
	\only<all:3-4>{\bendtransition{AR}{R}{DAR}{}{bend left=50}}
	\only<all:3-4>{\bendtransition{R}{AR}{PAR}{}{bend right}}
	\only<all:5-8>{\bendtransition{A}{R}{DAR}{}{bend left=50}}
	\only<all:5-8>{\bendtransition{R}{A}{PAR}{}{bend right}}
	\only<all:1-2>{\bendtransition{RR}{BR}{PBR}{}{bend left=50}}
	\only<all:1-2>{\bendtransition{BR}{RR}{DBR}{}{bend right}}
	\only<all:3-6>{\bendtransition{R}{BR}{PBR}{}{bend left=50}}
	\only<all:3-6>{\bendtransition{BR}{R}{DBR}{}{bend right}}
	\only<all:7-8>{\bendtransition{R}{B}{PBR}{}{bend left=50}}
	\only<all:7-8>{\bendtransition{B}{R}{DBR}{}{bend right}}
	\only<all:1-6>{\bendtransition{BL}{LL}{DBL}{}{bend left=50}}
	\only<all:1-6>{\bendtransition{LL}{BL}{PBL}{}{bend right}}
	\only<all:7-8>{\bendtransition{B}{LL}{DBL}{}{bend left=50}}
	\only<all:7-8>{\bendtransition{LL}{B}{PBL}{}{bend right}}
	
	\only<all:1-6>{\bendtransition{LR}{BR}{PBL}{near start}{bend left=50}}
	\only<all:1-6>{\bendtransition{BR}{LR}{DBL}{near end,below=-2mm}{bend right}}
	\only<all:7-8>{\bendtransition{LR}{B}{PBL}{near start}{bend left=50}}
	\only<all:7-8>{\bendtransition{B}{LR}{DBL}{near end,below=-2mm}{bend right}}
	\only<all:1-2>{\bendtransition{BL}{RL}{DBR}{near end}{bend left=50}}%
	\only<all:1-2>{\bendtransition{RL}{BL}{PBR}{near start,below=-2mm}{bend right}}%
	\only<all:3-6>{\bendtransition{BL}{R}{DBR}{near end}{bend left=50}}%
	\only<all:3-6>{\bendtransition{R}{BL}{PBR}{near start,below=-2mm}{bend right}}%
	
	\bendtransition{LL}{LL}{MB{\wildcard}{\wildcard}}{above=2mm,sloped=false,pos=.9}{loop left}
	\only<all:1-2>{\bendtransition{RR}{RR}{MB{\wildcard}{\wildcard}}{above=2mm,sloped=false,pos=.1}{loop right}}
	\only<all:1-4>{\bendtransition{AL}{AL}{MB{\wildcard}{\wildcard}}{}{loop above}}
	\only<all:1-4>{\bendtransition{AR}{AR}{MB{\wildcard}{\wildcard}}{}{loop above}}
	\only<all:5-8>{\bendtransition{A}{A}{M{\wildcard}{\wildcard}{\wildcard}}{}{loop above}}
	\bendtransition{LR}{LR}{MB{\wildcard}{\wildcard}}{}{loop above}
	\only<all:1-2>{\bendtransition{RL}{RL}{MB{\wildcard}{\wildcard}}{}{loop above}}
	\only<all:1-6>{\bendtransition{BL}{BL}{MB{\wildcard}{\wildcard}}{below=-1mm}{loop below}}
	\only<all:7-8>{\bendtransition{B}{B}{M{\wildcard}{\wildcard}{\wildcard}}{below=-1mm}{loop below}}
	\only<all:3->{\bendtransition{R}{R}{M{\wildcard}{\wildcard}{\wildcard}}{above=2mm,sloped=false}{loop right}}
	\only<all:1-6>{\bendtransition{BR}{BR}{MB{\wildcard}{\wildcard}}{below=-1mm}{loop below}}

	\only<all:9->{\bendtransition{I}{R}{D{\wildcard}R}{}{bend left}}
	\only<all:9->{\bendtransition{R}{I}{P{\wildcard}R}{}{bend left}}
	\only<all:9->{\bendtransition{I}{I}{M{\wildcard}{\wildcard}{\wildcard}}{}{loop above}}
	\only<all:9->{\bendtransition{LR}{I}{PBL}{}{bend left}}
	\only<all:9->{\bendtransition{I}{LR}{DBL}{}{bend left}}
	\only<all:9->{\bendtransition{LL}{I}{P{\wildcard}L}{}{bend right=55}}
	\only<all:9->{\bendtransition{I}{LL}{D{\wildcard}L}{below=-2mm}{bend left=80}}
  \end{tikzpicture}
}

\newcommand{\picshrinkexampleonenumberedstates}[1][]{%
  \begin{tikzpicture}[->,>=stealth,looseness=.5,,bend angle=20,#1]
        \useasboundingbox (-1.5,-3.5) rectangle (9.5,3.5);
        %% \draw (-1.5,-3.5) rectangle (9.5,3.5);
  	\numberedvertex{0,0}{LL}{0}
  	\numberedvertex{2,0}{LR}{1}
	\only<all:1-5>{\numberedvertex{3,2}{AL}{4}}
	\only<all:1-5>{\numberedvertex{5,2}{AR}{5}}
	\only<all:5>{\numberedvertex[draw=red,thick]{3,2}{AL}{4}}
	\only<all:5>{\numberedvertex[draw=red,thick]{5,2}{AR}{5}}
	\only<all:6-9>{\numberedvertex{4,2}{A}{4}}
	\only<all:9>{\numberedvertex[draw=red,thick]{4,2}{A}{4}}

	\only<all:1-7>{\numberedvertex{3,-2}{BL}{6}}
	\only<all:1-7>{\numberedvertex{5,-2}{BR}{7}}
	\only<all:7>{\numberedvertex[draw=red,thick]{3,-2}{BL}{6}}
	\only<all:7>{\numberedvertex[draw=red,thick]{5,-2}{BR}{7}}
	\only<all:8-9>{\numberedvertex{4,-2}{B}{6}}
	\only<all:9>{\numberedvertex[draw=red,thick]{4,-2}{B}{6}}
	\only<all:10-11>{\numberedvertex{4,0}{I}{4}}
        \only<all:12>{\numberedvertex{4,0}{I}{\alert{4$\mapsto$3}}}
        \only<all:13->{\numberedvertex{4,0}{I}{\alert<all:13>{3}}}

	\only<all:1-2>{\numberedgoalvertex{6,0}{RL}{2}}
  	\only<all:1-2>{\numberedgoalvertex{8,0}{RR}{3}}
  	\only<all:3>{\numberedgoalvertex[draw=red,thick]{6,0}{RL}{2}}
  	\only<all:3>{\numberedgoalvertex[draw=red,thick]{8,0}{RR}{3}}
	\only<all:4->{\numberedgoalvertex{7,0}{R}{2}}

	
	\draw[transition,->,>=stealth](2.6,0.6) -- (LR);
	\bendtransition{LL}{LR}{MALR}{}{bend left}
	\only<all:1-9>{\bendtransition{LR}{LL}{MARL}{below=-2mm}{bend left}}
	\only<all:10->{\bendtransition{LR}{LL}{MARL}{}{bend left}}
	\only<all:1-5>{\bendtransition{AL}{AR}{MALR}{}{bend left}}
	\only<all:1-5>{\bendtransition{AR}{AL}{MARL}{below=-2mm}{bend left}}
	\only<all:1-7>{\bendtransition{BL}{BR}{MALR}{}{bend left}}
	\only<all:1-7>{\bendtransition{BR}{BL}{MARL}{below=-2mm}{bend left}}
	\only<all:1-3>{\bendtransition{RL}{RR}{MALR}{}{bend left}}
	\only<all:1-3>{\bendtransition{RR}{RL}{MARL}{below=-2mm}{bend left}}
	
	\only<all:1-5>{\bendtransition{LL}{AL}{PAL}{}{bend left=50}}
	\only<all:1-5>{\bendtransition{AL}{LL}{DAL}{}{bend right}}
	\only<all:6-9>{\bendtransition{LL}{A}{PAL}{}{bend left=50}}
	\only<all:6-9>{\bendtransition{A}{LL}{DAL}{}{bend right}}
	\only<all:1-3>{\bendtransition{AR}{RR}{DAR}{}{bend left=50}}
	\only<all:1-3>{\bendtransition{RR}{AR}{PAR}{}{bend right}}
	\only<all:4-5>{\bendtransition{AR}{R}{DAR}{}{bend left=50}}
	\only<all:4-5>{\bendtransition{R}{AR}{PAR}{}{bend right}}
	\only<all:6-9>{\bendtransition{A}{R}{DAR}{}{bend left=50}}
	\only<all:6-9>{\bendtransition{R}{A}{PAR}{}{bend right}}
	\only<all:1-3>{\bendtransition{RR}{BR}{PBR}{}{bend left=50}}
	\only<all:1-3>{\bendtransition{BR}{RR}{DBR}{}{bend right}}
	\only<all:4-7>{\bendtransition{R}{BR}{PBR}{}{bend left=50}}
	\only<all:4-7>{\bendtransition{BR}{R}{DBR}{}{bend right}}
	\only<all:8-9>{\bendtransition{R}{B}{PBR}{}{bend left=50}}
	\only<all:8-9>{\bendtransition{B}{R}{DBR}{}{bend right}}
	\only<all:1-7>{\bendtransition{BL}{LL}{DBL}{}{bend left=50}}
	\only<all:1-7>{\bendtransition{LL}{BL}{PBL}{}{bend right}}
	\only<all:8-9>{\bendtransition{B}{LL}{DBL}{}{bend left=50}}
	\only<all:8-9>{\bendtransition{LL}{B}{PBL}{}{bend right}}
	
	\only<all:1-7>{\bendtransition{LR}{BR}{PBL}{near start}{bend left=50}}
	\only<all:1-7>{\bendtransition{BR}{LR}{DBL}{near end,below=-2mm}{bend right}}
	\only<all:8-9>{\bendtransition{LR}{B}{PBL}{near start}{bend left=50}}
	\only<all:8-9>{\bendtransition{B}{LR}{DBL}{near end,below=-2mm}{bend right}}
	\only<all:1-3>{\bendtransition{BL}{RL}{DBR}{near end}{bend left=50}}%
	\only<all:1-3>{\bendtransition{RL}{BL}{PBR}{near start,below=-2mm}{bend right}}%
	\only<all:4-7>{\bendtransition{BL}{R}{DBR}{near end}{bend left=50}}%
	\only<all:4-7>{\bendtransition{R}{BL}{PBR}{near start,below=-2mm}{bend right}}%
	
	\bendtransition{LL}{LL}{MB{\wildcard}{\wildcard}}{above=2mm,sloped=false,pos=.9}{loop left}
	\only<all:1-3>{\bendtransition{RR}{RR}{MB{\wildcard}{\wildcard}}{above=2mm,sloped=false,pos=.1}{loop right}}
	\only<all:1-5>{\bendtransition{AL}{AL}{MB{\wildcard}{\wildcard}}{}{loop above}}
	\only<all:1-5>{\bendtransition{AR}{AR}{MB{\wildcard}{\wildcard}}{}{loop above}}
	\only<all:6-9>{\bendtransition{A}{A}{M{\wildcard}{\wildcard}{\wildcard}}{}{loop above}}
	\bendtransition{LR}{LR}{MB{\wildcard}{\wildcard}}{}{loop above}
	\only<all:1-3>{\bendtransition{RL}{RL}{MB{\wildcard}{\wildcard}}{}{loop above}}
	\only<all:1-7>{\bendtransition{BL}{BL}{MB{\wildcard}{\wildcard}}{below=-1mm}{loop below}}
	\only<all:8-9>{\bendtransition{B}{B}{M{\wildcard}{\wildcard}{\wildcard}}{below=-1mm}{loop below}}
	\only<all:4->{\bendtransition{R}{R}{M{\wildcard}{\wildcard}{\wildcard}}{above=2mm,sloped=false}{loop right}}
	\only<all:1-7>{\bendtransition{BR}{BR}{MB{\wildcard}{\wildcard}}{below=-1mm}{loop below}}

	\only<all:10->{\bendtransition{I}{R}{D{\wildcard}R}{}{bend left}}
	\only<all:10->{\bendtransition{R}{I}{P{\wildcard}R}{}{bend left}}
	\only<all:10->{\bendtransition{I}{I}{M{\wildcard}{\wildcard}{\wildcard}}{}{loop above}}
	\only<all:10->{\bendtransition{LR}{I}{PBL}{}{bend left}}
	\only<all:10->{\bendtransition{I}{LR}{DBL}{}{bend left}}
	\only<all:10->{\bendtransition{LL}{I}{P{\wildcard}L}{}{bend right=55}}
	\only<all:10->{\bendtransition{I}{LL}{D{\wildcard}L}{below=-2mm}{bend left=80}}
  \end{tikzpicture}
}

\newcommand{\picshrinkexampleoneresult}{%
  \begin{tikzpicture}[->,>=stealth,looseness=.5,,bend angle=20]
  	\vertex{0,0}{LL}
  	\vertex{2,0}{LR}
	\vertex{4,0}{I}
	\goalvertex{7,0}{R}
	
	\draw[transition,->,>=stealth](2.6,0.6) -- (LR);
	\bendtransition{LL}{LR}{MALR}{}{bend left}
	\bendtransition{LR}{LL}{MARL}{}{bend left}
	\bendtransition{LL}{LL}{MB{\wildcard}{\wildcard}}{above=2mm,sloped=false}{loop left}
	\bendtransition{LR}{LR}{MB{\wildcard}{\wildcard}}{}{loop above}
	\bendtransition{R}{R}{M{\wildcard}{\wildcard}{\wildcard}}{above=2mm,sloped=false}{loop right}
	\bendtransition{I}{R}{D{\wildcard}R}{}{bend left}
	\bendtransition{R}{I}{P{\wildcard}R}{}{bend left}
	\bendtransition{I}{I}{M{\wildcard}{\wildcard}{\wildcard}}{}{loop above}
	\bendtransition{LR}{I}{PBL}{}{bend left}
	\bendtransition{I}{LR}{DBL}{}{bend left}
	\bendtransition{LL}{I}{P{\wildcard}L}{}{bend right=55}
	\bendtransition{I}{LL}{D{\wildcard}L}{below=-2mm}{bend left=80}
  \end{tikzpicture}
}

\newcommand{\picshrinkexampleoneresultdifferent}{%
  \begin{tikzpicture}[->,>=stealth,looseness=.5,,bend angle=20]
  	\vertex{0,-1}{LL}
  	\vertex{0,1}{LR}
	\vertex{3,0}{I}
	\goalvertex{6,0}{R}
	
	\draw[transition,->,>=stealth](0.6,1.6) -- (LR);
	\bendtransition{LL}{LR}{MALR}{}{bend left}
	\bendtransition{LR}{LL}{MARL}{}{bend left}
	\bendtransition{LL}{LL}{MB{\wildcard}{\wildcard}}{above=2mm,sloped=false}{loop left}
	\bendtransition{LR}{LR}{MB{\wildcard}{\wildcard}}{above=2mm,sloped=false}{loop left}
	\bendtransition{R}{R}{M{\wildcard}{\wildcard}{\wildcard}}{above=2mm,sloped=false}{loop right}
	\bendtransition{I}{R}{D{\wildcard}R}{}{bend left}
	\bendtransition{R}{I}{P{\wildcard}R}{}{bend left}
	\bendtransition{I}{I}{M{\wildcard}{\wildcard}{\wildcard}}{}{loop above}
	\bendtransition{LR}{I}{PBL}{}{bend left=40}
	\bendtransition{I}{LR}{DBL}{below=-2mm}{bend right}
	\bendtransition{LL}{I}{P{\wildcard}L}{}{bend right}
	\bendtransition{I}{LL}{D{\wildcard}L}{below=-2mm}{bend left=40}
  \end{tikzpicture}
}

\newcommand{\picproductfinal}[1][]{%
  \begin{tikzpicture}[->,>=stealth,looseness=.5,,bend angle=20,#1]
  	\vertex{-.5,2.5}{LRL}
  	\vertex{1,1.5}{LRR}
  	\vertex{-.5,-1.5}{LLL}
  	\vertex{1,-2.5}{LLR}
  	\vertex{3,0.5}{IL}
  	\vertex{4.5,-.5}{IR}
  	\goalvertex{6.5,0.5}{RL}
  	\goalvertex{8,-.5}{RR}

	\draw[transition,->,>=stealth](1.6,1.1) -- (LRR);

	\bendtransition{LRL}{LRR}{MBLR}{below=-2mm}{bend left}
	\bendtransition{LRR}{LRL}{MBRL}{below=-2mm}{bend left}
	\bendtransition{LLL}{LLR}{MBLR}{below=-2mm}{bend left}
	\bendtransition{LLR}{LLL}{MBRL}{below=-2mm}{bend left}
	\bendtransition{IL}{IR}{MBLR}{}{bend left}
	\bendtransition{IR}{IL}{MBRL}{}{bend left}
	\bendtransition{RL}{RR}{MBLR}{}{bend left}
	\bendtransition{RR}{RL}{MBRL}{}{bend left}

	\bendtransition{IL}{RL}{DAR}{}{bend left=40}
	\bendtransition{RL}{IL}{PAR}{below=-2mm}{bend right}
	\bendtransition{IR}{RR}{D{\wildcard}R}{}{bend right}
	\bendtransition{RR}{IR}{P{\wildcard}R}{below=-2mm}{bend left=40}
	\bendtransition{LLL}{IL}{P{\wildcard}L}{near end,below=-2mm}{bend left}
	\bendtransition{IL}{LLL}{D{\wildcard}L}{below=-2mm,near start}{bend left}
	\bendtransition{LLR}{IR}{PAL}{}{bend right}
	\bendtransition{IR}{LLR}{DAL}{below=-2mm}{bend left=40}

	\bendtransition{LLL}{LRL}{MALR}{}{bend left=50}
	\bendtransition{LRL}{LLL}{MARL}{rotate=180,below=0mm}{bend right}
	\bendtransition{LLR}{LRR}{MALR}{near end}{bend left}
	\bendtransition{LRR}{LLR}{MARL}{rotate=180,above=-2mm,near start}{bend left}

	\bendtransition{LRL}{IL}{PBL}{}{bend left=60}
	\bendtransition{IL}{LRL}{DBL}{}{bend right=30}
	\bendtransition{IL}{IL}{MA{\wildcard}{\wildcard}}{below=-1mm}{loop below}
	\bendtransition{IR}{IR}{MA{\wildcard}{\wildcard}}{below=-1mm}{loop below}
	\bendtransition{RR}{RR}{MA{\wildcard}{\wildcard}}{below=-1mm}{loop below}
	\bendtransition{RL}{RL}{MA{\wildcard}{\wildcard}}{}{loop above}
  \end{tikzpicture}
}

\newcommand{\picshrinkexampletworesult}{%
  \begin{tikzpicture}[->,>=stealth,looseness=.5,,bend angle=20]
  	\vertex{0,0}{LRR}
    \node[vertex interior] at (2,0) (LLL) {};
    \node[vertex border] at (2,0) {\parbox{4mm}{\tiny LLL\\LRL\\LLR}};
	\vertex{4,0}{I}
	\goalvertex{6,0}{R}
	
	\draw[transition,->,>=stealth](-0.8,0) -- (LRR);
	\bendtransition{I}{I}{M{\wildcard}{\wildcard}{\wildcard}}{}{loop above}
	\bendtransition{R}{R}{M{\wildcard}{\wildcard}{\wildcard}}{}{loop above}
	\bendtransition{LLL}{LLL}{M{\wildcard}{\wildcard}{\wildcard}}{}{loop above}

	\bendtransition{LRR}{LLL}{M{\wildcard}RL}{}{bend left}
	\bendtransition{LLL}{LRR}{M{\wildcard}LR}{below=-2mm}{bend left}
	\bendtransition{LLL}{I}{P{\wildcard}L}{}{bend left}
	\bendtransition{I}{LLL}{D{\wildcard}L}{below=-2mm}{bend left}
	\bendtransition{I}{R}{D{\wildcard}R}{}{bend left}
	\bendtransition{R}{I}{P{\wildcard}R}{below=-2mm}{bend left}
  \end{tikzpicture}
}

%%\author[M.~Helmert, G.~R\"oger]{Malte Helmert and Gabriele R\"oger}
%% abstraction pictures by Gabi

\begin{document}

\lectureno{10}
\subtitle{Planning as search: abstractions}
\date{December 6th, 2011}
\maketitles

\section{Abstractions: informally}
\subsection{Introduction}

\begin{frame}{Coming up with heuristics in a principled way}
  \begin{block}{General procedure for obtaining a heuristic}
    Solve an easier version of the problem.
  \end{block}

  Two common methods:
  \begin{itemize}
  \item \alert{relaxation:} consider \alert{less constrained} version
    of the problem
  \item \alert{abstraction:} consider \alert{smaller} version of real
    problem
  \end{itemize}

  In previous chapters, we have studied \alert{relaxation}, which has
  been very successfully applied to \alert{satisficing planning}.

  \smallskip

  Now, we study \alert{abstraction}, which is one of the most
  prominent techniques for \alert{optimal planning}.
\end{frame}

\begin{frame}{Abstracting a transition system}
  Abstracting a transition system means \alert{dropping some
    distinctions} between states, while \alert{preserving the
  transition behaviour} as much as possible.

  \begin{itemize}
  \item An abstraction of a transition system $\mathcal T$ is defined
    by an \alert{abstraction mapping $\alpha$} that defines which
    states of $\mathcal T$ should be distinguished and which ones
    should not.
  \item From $\mathcal T$ and $\alpha$, we compute an \alert{abstract
    transition system $\mathcal T'$} which is similar to $\mathcal T$,
    but smaller.
  \item The \alert{abstract goal distances} (goal distances in
    $\mathcal T'$) are used as heuristic estimates for goal distances
    in $\mathcal T$.
  \end{itemize}
\end{frame}

\begin{frame}{Abstracting a transition system: example}
  \begin{example}[15-puzzle]
    A \alert{15-puzzle} state is given by a permutation
    $\langle b, t_1, \dots, t_{15}\rangle$ of $\{1, \dots, 16\}$,
    where $b$ denotes the blank position and the other components
    denote the positions of the 15 tiles.

    \smallskip

    One possible \alert{abstraction mapping} ignores the precise
    location of tiles 8--15, \ie, two states are distinguished iff
    they differ in the position of the blank or one of the tiles 1--7:
    \[\alpha(\langle b, t_1, \dots, t_{15}\rangle) =
      \langle b, t_1, \dots, t_7\rangle\]

    \medskip

    The heuristic values for this abstraction correspond to the cost
    of moving tiles 1--7 to their goal positions.
  \end{example}
\end{frame}

\begin{frame}{Abstraction example: 15-puzzle}
  \begin{center}
    \begin{pgfpicture}{-0.5cm}{-0.5cm}{9.5cm}{3.5cm}
      \abstractedtileanimated{0,3}{9}
      \tile{1,3}{2}
      \abstractedtileanimated{2,3}{12}
      \tile{3,3}{6}
      \tile{0,2}{5}
      \tile{1,2}{7}
      \abstractedtileanimated{2,2}{14}
      \abstractedtileanimated{3,2}{13}
      \tile{0,1}{3}
      \tile{1,1}{4}
      \tile{2,1}{1}
      \abstractedtileanimated{3,1}{11}
      \abstractedtileanimated{0,0}{15}
      \abstractedtileanimated{1,0}{10}
      \abstractedtileanimated{2,0}{8}
      \tile[black]{3,0}{}
      %
      \bigarrow{4,1.5}{1}
      %
      \tile{6,3}{1}
      \tile{7,3}{2}
      \tile{8,3}{3}
      \tile{9,3}{4}
      \tile{6,2}{5}
      \tile{7,2}{6}
      \tile{8,2}{7}
      \abstractedtileanimated{9,2}{8}
      \abstractedtileanimated{6,1}{9}
      \abstractedtileanimated{7,1}{10}
      \abstractedtileanimated{8,1}{11}
      \abstractedtileanimated{9,1}{12}
      \abstractedtileanimated{6,0}{13}
      \abstractedtileanimated{7,0}{14}
      \abstractedtileanimated{8,0}{15}
      \tile[black]{9,0}{}
    \end{pgfpicture}
  \end{center}
  \begin{overprint}
    \onslide<all:1>
    \alert{real state space}
    \begin{itemize}
    \item $16! = 20922789888000 \approx 2 \cdot 10^{13}$ states
    \item $\frac{16!}{2} = 10461394944000 \approx 10^{13}$ reachable
      states
    \end{itemize}
    \onslide<all:2>
    \alert{abstract state space}
    \begin{itemize}
    \item $16 \cdot 15 \cdot \ldots \cdot 9 = 518918400 \approx 5
      \cdot 10^8$ states
    \item $16 \cdot 15 \cdot \ldots \cdot 9 = 518918400 \approx 5
      \cdot 10^8$ reachable states
    \end{itemize}
  \end{overprint}
\end{frame}

\begin{frame}{Computing the abstract transition system}
  Given $\mathcal T$ and $\alpha$, how do we compute $\mathcal T'$?

  \begin{block}{Requirement}
    We want to obtain an \alert{admissible heuristic}. 

    Hence, $h^*(\alpha(s))$ (in the abstract state space $\mathcal T'$)
    should never overestimate $h^*(s)$ (in the concrete state space
    $\mathcal T$).
  \end{block}

  \medskip

  An easy way to achieve this is to ensure that \alert{all solutions
    in $\mathcal T$ also exist in $\mathcal T'$}:
  \begin{itemize}
  \item If $s$ is a goal state in $\mathcal T$, then $\alpha(s)$ is a
    goal state in $\mathcal T'$.
  \item If $\mathcal T$ has a transition from $s$ to $t$, then
    $\mathcal T'$ has a transition from $\alpha(s)$ to $\alpha(t)$.
  \end{itemize}
\end{frame}

\begin{frame}{Computing the abstract transition system: example}
  \begin{example}[15-puzzle]
    In the running example:
    \begin{itemize}
    \item $\mathcal T$ has the unique goal state
      $\langle 16, 1, 2, \dots, 15\rangle$.
      \begin{itemize}
      \item[$\leadsto$] $\mathcal T'$ has the unique goal state
        $\langle 16, 1, 2, \dots, 7\rangle$.
      \end{itemize}
    \item Let $x$ and $y$ be neighboring positions in the $4 \times 4$
      grid. \\
      $\mathcal T$ has a transition from
      $\langle x, t_1, \dots, t_{i-1}, y, t_{i+1}, \dots,
      t_{15}\rangle$ \\
      to $\langle y, t_1, \dots, t_{i-1}, x, t_{i+1}, \dots,
      t_{15}\rangle$ for all $i \in \{1, \dots, 15\}$.
      \begin{itemize}
      \item[$\leadsto$]
        $\mathcal T'$ has a transition from
        $\langle x, t_1, \dots, t_{i-1}, y, t_{i+1}, \dots,
        t_7\rangle$ \\
        to $\langle y, t_1, \dots, t_{i-1}, x, t_{i+1}, \dots,
        t_7\rangle$ for all $i \in \{1, \dots, 7\}$.
      \item[$\leadsto$]
        Moreover, $\mathcal T'$ has a transition from
        $\langle x, t_1, \dots, t_7\rangle$ to 
        $\langle y, t_1, \dots, t_7\rangle$ if
        $y \notin \{t_1, \dots, t_7\}$.
      \end{itemize}
    \end{itemize}
  \end{example}
\end{frame}

\subsection{Practical requirements}

\begin{frame}{Practical requirements for abstractions}
  To be useful in practice, an abstraction heuristic must be efficiently
  computable. This gives us two requirements for $\alpha$:
  \begin{itemize}
  \item For a given state $s$, the \alert{abstract state}
    $\alpha(s)$ must be efficiently computable.
  \item For a given abstract state $\alpha(s)$, the \alert{abstract
    goal distance} $h^*(\alpha(s))$ must be efficiently computable.
  \end{itemize}

  \medskip

  There are different ways of achieving these requirements:
  \begin{itemize}
  \item \alert{pattern database heuristics}
    (Culberson \& Schaeffer, 1996)
  \item \alert{merge-and-shrink abstractions}
    (Dr\"ager, Finkbeiner \& Podelski, 2006)
  \item \alert{structural patterns}
    (Katz \& Domshlak, 2008b)
    \begin{itemize}
    \item not covered in this course
    \end{itemize}
  \end{itemize}
\end{frame}

\begin{frame}{Practical requirements for abstractions: example}
  \begin{example}[15-puzzle]
    In our running example, $\alpha$ can be very efficiently
    computed: \\
    just project the given $16$-tuple to its first $8$
    components.

    \medskip

    To compute abstract goal distances efficiently during search, most
    common algorithms precompute \alert{all abstract goal distances}
    prior to search by performing a backward breadth-first search from
    the goal state(s). The distances are then stored in a table
    (requires about 495 MB of RAM).

    During search, computing $h^*(\alpha(s))$ is just a table lookup.
  \end{example}

  This heuristic is an example of a \alert{pattern database
    heuristic}.
\end{frame}

\subsection{Multiple abstractions}

\begin{frame}{Multiple abstractions}
  \begin{itemize}
  \item One important practical question is how to come up with
    a suitable abstraction mapping $\alpha$.
  \item Indeed, there is usually a \alert{huge number of
    possibilities}, and it is important to pick good abstractions
    (\ie, ones that lead to informative heuristics).
  \item However, it is generally \alert{not necessary to commit to a
    single abstraction}.
  \end{itemize}
\end{frame}

\begin{frame}{Combining multiple abstractions}
  \alert{Maximizing} several abstractions:
  \begin{itemize}
  \item Each abstraction mapping gives rise to an admissible
    heuristic.
  \item By computing the \alert{maximum} of several admissible
    heuristics, we obtain another admissible heuristic which
    \alert{dominates} the component heuristics.
  \item Thus, we can always compute several abstractions and maximize
    over the individual abstract goal distances.
  \end{itemize}

  \medskip

  \alert{Adding} several abstractions:
  \begin{itemize}
  \item In some cases, we can even compute the \alert{sum} of
    individual estimates and still stay admissible.
  \item Summation often leads to \alert{much higher estimates} than
    maximization, so it is \alert{important to understand when it is
      admissible}.
  \end{itemize}
\end{frame}

\begin{frame}{Maximizing several abstractions: example}
  \begin{example}[15-puzzle]
    \begin{itemize}
    \item mapping to tiles 1--7 was arbitrary \\
      $\leadsto$ can use \alert{any subset} of tiles
    \item with the same amount of memory required for the tables for
      the mapping to tiles 1--7, we could store the tables for
      \alert{nine different abstractions} to six tiles and the blank
    \item use \alert{maximum} of individual estimates
    \end{itemize}
  \end{example}
\end{frame}

\begin{frame}<all:1-3>{Adding several abstractions: example}
  \begin{center}
    \begin{pgfpicture}{-0.5cm}{-0.5cm}{9.5cm}{3.5cm}
      \abstractedtileanimated{0,3}{9}
      \tile{1,3}{2}
      \abstractedtileanimated{2,3}{12}
      \tile{3,3}{6}
      \tile{0,2}{5}
      \tile{1,2}{7}
      \abstractedtileanimated{2,2}{14}
      \abstractedtileanimated{3,2}{13}
      \tile{0,1}{3}
      \tile{1,1}{4}
      \tile{2,1}{1}
      \abstractedtileanimated{3,1}{11}
      \abstractedtileanimated{0,0}{15}
      \abstractedtileanimated{1,0}{10}
      \abstractedtileanimated{2,0}{8}
      \only<all:1-2>{\tile[black]{3,0}{}}
      \only<all:3>{\abstractedtile{3,0}}
      %
      \tile{6,3}{9}
      \abstractedtileanimated{7,3}{2}
      \tile{8,3}{12}
      \abstractedtileanimated{9,3}{6}
      \abstractedtileanimated{6,2}{5}
      \abstractedtileanimated{7,2}{7}
      \tile{8,2}{14}
      \tile{9,2}{13}
      \abstractedtileanimated{6,1}{3}
      \abstractedtileanimated{7,1}{4}
      \abstractedtileanimated{8,1}{1}
      \tile{9,1}{11}
      \tile{6,0}{15}
      \tile{7,0}{10}
      \tile{8,0}{8}
      \only<all:1-2>{\tile[black]{9,0}{}}
      \only<all:3>{\abstractedtile{9,0}}
    \end{pgfpicture}
  \end{center}
  \begin{itemize}
  \item \alert{1st abstraction:} ignore precise location of 8--15
    \only<all:3>{\alert{and blank}}
  \item \alert{2nd abstraction:} ignore precise location of 1--7
    \only<all:3>{\alert{and blank}}
  \item[$\leadsto$]
    \only<all:1>{Is the \alert{sum} of the abstraction heuristics
      \alert{admissible}?}
    \only<all:2->{The \alert{sum} of the abstraction heuristics is
      \alert{\only<all:2>{not }admissible}.}
  \end{itemize}
\end{frame}

\subsection{Outlook}

\begin{frame}{Our plan for the next lectures}
  In the following, we take a deeper look at abstractions and their
  use for admissible heuristics.
  \begin{itemize}
  \item In the rest of \alert{this chapter}, we \alert{formally
    introduce} abstractions and abstraction heuristics and study
    some of their most important properties.
  \item In the \alert{following chapters}, we discuss some particular
    classes of abstraction heuristics in detail, namely \alert{pattern
      database heuristics} and \alert{merge-and-shrink abstractions}.
  \end{itemize}
\end{frame}

\section{Abstractions: formally}
\subsection{Transition systems}

\begin{frame}{Transition systems}
  \hilite{Reminder from Chapter 2:}
  \begin{definition}[transition system]
    A \alert{transition system} is a 5-tuple $\mathcal T = \langle S,
    L, T, s_0, S_\star\rangle$ where
    \begin{itemize}
    \item $S$ is a finite set of \alert{states},
    \item $L$ is a finite set of (transition) \alert{labels},
    \item $T \subseteq S \times L \times S$ is the \alert{transition
      relation},
    \item $s_0 \in S$ is the \alert{initial state}, and
    \item $S_\star \subseteq S$ is the set of \alert{goal states}.
    \end{itemize}
    We say that $\mathcal T$ \alert{has the transition} $\langle s, \ell,
    s'\rangle$ if $\langle s, \ell, s'\rangle \in T$.

    We also write this \alert{$s \xrightarrow{\ell} s'$}, or \alert{$s
      \rightarrow s'$} when not interested in $\ell$.
  \end{definition}
\end{frame}

\begin{frame}{Transition systems: example}
  \begin{center}
    \picfulltransitiongraphnolabels
  \end{center}
  \hilite{Note:} To reduce clutter, our figures usually omit arc
  labels and collapse transitions between identical states. However,
  these are important for the formal definition of the transition
  system.
\end{frame}

\begin{frame}{Transition systems of FDR planning tasks}
  \begin{definition}[induced transition system of an FDR planning task]
    Let $\Pi = \langle V, I, O, \gamma\rangle$ be an FDR planning task.

    The \alert{induced transition system of $\Pi$}, in symbols \alert{$\mathcal
    T(\Pi)$}, is the transition system $\mathcal T(\Pi) =
    \langle S, L, T, s_0, S_\star\rangle$, where
    \begin{itemize}
    \item $S$ is the set of states over $V$,
    \item $L = O$,
    \item $T = \{\langle s, o, t\rangle \in
      S \times L \times S \mid
      \applyop{o}{s} = t\}$,
    \item $s_0 = I$, and
    \item $S_\star = \{s \in S \mid s \models \gamma\}$.
    \end{itemize}
  \end{definition}
\end{frame}

\begin{frame}{Example task: one package, two trucks}
  \begin{example}[one package, two trucks]
    Consider the following FDR planning task $\langle V, I, O,
    \gamma\rangle$:
    \begin{itemize}
     \item $V = \{p, t_{\val{A}}, t_{\val{B}}\}$ with
       \begin{itemize}
       \item $\mathcal D_p = \{\val{L}, \val{R}, \val{A}, \val{B}\}$
       \item $\mathcal D_{t_{\val{A}}} = \mathcal D_{t_{\val{B}}} = \{\val{L},
         \val{R}\}$
       \end{itemize}
     \item $I = \{
       p \mapsto \val{L},
       t_{\val{A}} \mapsto \val{R},
       t_{\val{B}} \mapsto \val{R}\}$
     \item
       \begin{tightalign}[t]
         O ={} &
       \{ \text{pickup}_{i,j} \mid i \in \{\val{A}, \val{B}\}, j \in
           \{\val{L}, \val{R}\}\} \\ {}\cup{} &
       \{\text{drop}_{i,j} \mid i \in \{\val{A}, \val{B}\}, j \in
           \{\val{L}, \val{R}\}\} \\ {}\cup{} &
       \{\text{move}_{i,j,j'} \mid i \in \{\val{A}, \val{B}\}, j, j' \in
           \{\val{L}, \val{R}\}, j \neq j'\}\text{, where}
       \end{tightalign}
       \begin{itemize}
       \item $\text{pickup}_{i,j} = \langle
         t_i = j \land p = j,
         p := i\rangle$
       \item $\text{drop}_{i,j} = \langle
         t_i = j \land p = i,
         p := j\rangle$
       \item $\text{move}_{i,j,j'} = \langle
         t_i = j, t_i := j'
         \rangle$
       \end{itemize}
    \item $\gamma = (p = \val{R})$
    \end{itemize}
  \end{example}
\end{frame}

\begin{frame}{Transition system of example task}
  \begin{center}
    \picfulltransitiongraph
  \end{center}
  \begin{itemize}
  \item State $\{p \mapsto i, t_{\val{A}} \mapsto j, t_{\val{B}} \mapsto
    k\}$ is depicted as $ijk$.
  \item Transition labels are again not shown. For example, the
    transition from LLL to ALL has the label $\text{pickup}_{\val{A},
      \val{L}}$.
  \end{itemize}
\end{frame}

\subsection{Abstractions}

\begin{frame}{Abstractions}
  \begin{definition}[abstraction, abstraction mapping]
    Let $\mathcal T = \langle S, L, T, s_0, S_\star\rangle$ and
    $\mathcal T' = \langle S', L', T', s'_0, S'_\star\rangle$ \\
    be transition systems with the same label set $L = L'$, \\
    and let $\alpha: S \to S'$ be a \alert{surjective} function.

    \smallskip

    We say that $\mathcal T'$ is \alert{an abstraction of $\mathcal
      T$ with abstraction mapping $\alpha$} (or: \alert{abstraction
      function $\alpha$}) if
    \begin{itemize}
    \item $\alpha(s_0) = s'_0$,
    \item for all $s \in S_\star$, we have $\alpha(s) \in S'_\star$, and
    \item for all $\langle s, \ell, t\rangle \in T$, we have
      $\langle \alpha(s), \ell, \alpha(t)\rangle \in T'$.
    \end{itemize}
  \end{definition}
\end{frame}

\begin{frame}{Abstractions: terminology}
  Let $\mathcal T$ and $\mathcal T'$ be transition systems and
  $\alpha$ a function such that $\mathcal T'$ is an abstraction of
  $\mathcal T$ with abstraction mapping $\alpha$.
  \begin{itemize}
  \item $\mathcal T$ is called the \alert{concrete transition system}.
  \item $\mathcal T'$ is called the \alert{abstract transition
    system}.
  \item Similarly: \alert{concrete/abstract state space},
    \alert{concrete/abstract transition}, etc.
  \end{itemize}
  We say that:
  \begin{itemize}
  \item \alert{$\mathcal T'$ is an abstraction of $\mathcal T$}
    (without mentioning $\alpha$)
  \item \alert{$\alpha$ is an abstraction mapping on $\mathcal T$}
    (without mentioning $\mathcal T'$)
  \end{itemize}

  \medskip

  \hilite{Note:} For a given $\mathcal T$ and $\alpha$, there can be
  multiple abstractions $\mathcal T'$, and for a given $\mathcal T$
  and $\mathcal T'$, there can be multiple abstraction mappings
  $\alpha$.
\end{frame}

\begin{frame}{Abstraction: example}
  \begin{center}
    %% TODO: It would be nice if the states didn't jump around
    %%       when abstracting, but that appears to be a bit of work.
    \only<all:1>{\alert{concrete transition system}

      \picfulltransitiongraph}
    \only<all:2>{\alert{abstract transition system}

      \picexampleabstraction}
  \end{center}
  \only<all:2>{\hilite{Note:} Most arcs represent many parallel
    transitions.}
\end{frame}

\begin{frame}{Induced abstractions}
  \begin{definition}[induced abstractions]
    Let $\mathcal T = \langle S, L, T, s_0, S_\star\rangle$ be a transition
    system, and let $\alpha: S \to S'$ be a surjective function.

    \smallskip

    The \alert{abstraction (of $\mathcal T$) induced by $\alpha$},
    in symbols $\mathcal T^\alpha$, is the transition system $\mathcal
    T^\alpha = \langle S', L, T', s'_0, S'_\star\rangle$ defined by:
    \begin{itemize}
    \item $T' = \{\langle \alpha(s), \ell, \alpha(t)\rangle \mid
      \langle s, \ell, t\rangle \in T\}$
    \item $s'_0 = \alpha(s_0)$
    \item $S'_\star = \{ \alpha(s) \mid s \in S_\star \}$
    \end{itemize}
  \end{definition}
  \hilite{Note:} It is easy to see that $\mathcal T^\alpha$ is
  an abstraction of $\mathcal T$. It is the ``\alert{smallest}''
  abstraction of $\mathcal T$ with abstraction mapping $\alpha$.
\end{frame}

\begin{frame}{Induced abstractions: terminology}
  Let $\mathcal T$ and $\mathcal T'$ be transition systems and
  $\alpha$ be a function such that $\mathcal T' = \mathcal T^\alpha$
  (\ie, $\mathcal T'$ is the abstraction of $\mathcal T$ induced by
  $\alpha$).
  \begin{itemize}
  \item $\alpha$ is called a \alert{strict homomorphism}
    from $\mathcal T$ to $\mathcal T'$, and \\
    $\mathcal T'$ is called a \alert{strictly homomorphic abstraction} of
    $\mathcal T$.
  \item If $\alpha$ is bijective, it is called an \alert{isomorphism}
    between $\mathcal T$ and $\mathcal T'$, and
    the two transition systems are called \alert{isomorphic}.
  \end{itemize}
\end{frame}

\begin{frame}{Strictly homomorphic abstractions: example}
  \begin{center}
    \only<all:1>{\picexampleabstraction}
    \only<all:2>{\picexampleabstraction[\verticalgoalvertex]}
  \end{center}

  \only<all:1>{This abstraction is a strictly homomorphic abstraction of the
    concrete transition system $\mathcal T$.}

  \only<all:2>{If we add any goal states or transitions, it is still
    \\ an abstraction of $\mathcal T$, but no longer a strictly
    homomorphic one.}
\end{frame}

\subsection{Abstraction heuristics}

\begin{frame}{Abstraction heuristics}
  \begin{definition}[abstraction heuristic induced by an abstraction]
    Let $\Pi$ be an FDR planning task with state space $S$, and let
    $\mathcal A$ be an abstraction of $\mathcal T(\Pi)$ with
    abstraction mapping $\alpha$.

    \smallskip

    The \alert{abstraction heuristic induced by $\mathcal A$ and
      $\alpha$}, \alert{$h^{\mathcal A, \alpha}$}, is the heuristic
    function $h^{\mathcal A, \alpha}: S \to \mathbb N_0 \cup
    \{\infty\}$ which maps each state $s \in S$ to
    $h^*_{\mathcal A}(\alpha(s))$ (the goal distance of $\alpha(s)$ in
    $\mathcal A$).
  \end{definition}
  \hilite{Note:} $h^{\mathcal A, \alpha}(s) = \infty$ if no goal state
  of $\mathcal A$ is reachable from $\alpha(s)$

  \begin{definition}[abstraction heur.\ induced by strict homomorphism]
    Let $\Pi$ be an FDR planning task and $\alpha$ a
    strict homomorphism on $\mathcal T(\Pi)$.
    The \alert{abstraction heuristic induced by $\alpha$},
    \alert{$h^\alpha$}, is the abstraction heuristic induced by
    $\mathcal T(\Pi)^\alpha$ and $\alpha$, \ie, $h^\alpha := h^{\mathcal
      T(\Pi)^\alpha, \alpha}$.
  \end{definition}
\end{frame}

\begin{frame}{Abstraction heuristics: example}
  \begin{center}%
    \only<all:1>{\picexampleabstraction[\verticalgoalvertex]}%
    \only<all:2>{\picexampleabstraction}%

    \medskip

    \alert{$h^{\only<all:1>{\mathcal A, }\alpha}(
      \{p \mapsto \val{L}, t_\val{A} \mapsto \val{R},
        t_\val{B} \mapsto \val{R}\}) = \only<all:1>{1}\only<all:2>{3}$}
  \end{center}
\end{frame}

\begin{frame}{Consistency of abstraction heuristics}
  \begin{theorem}[consistency and admissibility of $h^{\mathcal A, \alpha}$]
    Let $\Pi$ be an FDR planning task, and let $\mathcal A$ be an
    abstraction of $\mathcal T(\Pi)$ with abstraction mapping
    $\alpha$.

    Then $h^{\mathcal A, \alpha}$ is safe, goal-aware, admissible and
    consistent.
  \end{theorem}

  \begin{proofstart}
    We prove goal-awareness and consistency; \\
    the other properties follow from these two.

    \smallskip

    Let $\mathcal T = \mathcal T(\Pi) = \langle S, L, T, s_0, S_\star\rangle$
    and $\mathcal A = \langle S', L', T', s'_0, S'_\star\rangle$.

    \smallskip

    \pause

    \hilite{Goal-awareness:} We need to show that $h^{\mathcal A,
      \alpha}(s) = 0$ for all $s \in S_\star$, so let $s \in S_\star$.
    Then $\alpha(s) \in S'_\star$ by the definition of abstractions and
    abstraction mappings, and hence $h^{\mathcal A, \alpha}(s) =
    h^*_{\mathcal A}(\alpha(s)) = 0$.
  \end{proofstart}
\end{frame}

\begin{frame}{Consistency of abstraction heuristics (ctd.)}
  \begin{proofend}
    \hilite{Consistency:} Let $s, t \in S$ such that $t$ is a
    successor of $s$. We need to prove that
    $h^{\mathcal A, \alpha}(s) \le h^{\mathcal A, \alpha}(t) + 1$.

    \pause
    
    Since $t$ is a successor of $s$, there exists an operator $o$ with
    $\applyop{o}{s} = t$ and hence $\langle s, o, t\rangle \in T$.

    \pause

    By the definition of abstractions and abstraction mappings,
    we get $\langle \alpha(s), o, \alpha(t)\rangle \in T'$ $\leadsto$
    $\alpha(t)$ is a successor of $\alpha(s)$ in $\mathcal A$.

    \pause

    Therefore,
    $h^{\mathcal A, \alpha}(s)
    = h^*_{\mathcal A}(\alpha(s))
    \le h^*_{\mathcal A}(\alpha(t)) + 1
    = h^{\mathcal A, \alpha}(t) + 1$, where the inequality holds because
    the shortest path from $\alpha(s)$ to the goal in $\mathcal A$
    cannot be longer than the shortest path from $\alpha(s)$ to the
    goal via $\alpha(t)$.
  \end{proofend}
\end{frame}

\subsection[Additivity]{Additive abstraction heuristics}

\begin{frame}{Orthogonality of abstraction mappings}
  \begin{definition}[orthogonal abstraction mappings]
    Let $\alpha_1$ and $\alpha_2$ be abstraction mappings on $\mathcal
    T$.

    \medskip

    We say that $\alpha_1$ and $\alpha_2$ are \alert{orthogonal}
    if for all transitions $\langle s, \ell, t\rangle$ of $\mathcal T$,
    we have $\alpha_i(s) = \alpha_i(t)$ for at least one $i \in \{1, 2\}$.
  \end{definition}
\end{frame}

\begin{frame}{Affecting transition labels}
  \begin{definition}[affecting transition labels]
    Let $\mathcal T$ be a transition system, and let $\ell$ be one of its
    labels.

    We say that $\ell$ \alert{affects $\mathcal T$} if $\mathcal
    T$ has a transition $\langle s, \ell, t\rangle$ with $s \neq t$.
  \end{definition}

  \medskip

  \begin{theorem}[affecting labels vs.\ orthogonality]
    Let $\mathcal A_1$ be an abstraction of $\mathcal T$ with
    abstraction mapping $\alpha_1$. \\
    Let $\mathcal A_2$ be an abstraction of $\mathcal T$ with
    abstraction mapping $\alpha_2$.

    \smallskip

    If no label of $\mathcal T$ affects both $\mathcal A_1$
    and $\mathcal A_2$, then $\alpha_1$ and $\alpha_2$ are orthogonal.
  \end{theorem}

  (Easy proof omitted.)
\end{frame}

\begin{frame}{Orthogonal abstraction mappings: example}
  \begin{center}
    \begin{pgfpicture}{-0.5cm}{-0.5cm}{9.5cm}{3.5cm}
      \abstractedtile{0,3}
      \tile{1,3}{2}
      \abstractedtile{2,3}
      \tile{3,3}{6}
      \tile{0,2}{5}
      \tile{1,2}{7}
      \abstractedtile{2,2}
      \abstractedtile{3,2}
      \tile{0,1}{3}
      \tile{1,1}{4}
      \tile{2,1}{1}
      \abstractedtile{3,1}
      \abstractedtile{0,0}
      \abstractedtile{1,0}
      \abstractedtile{2,0}
      \only<all:1>{\tile[black]{3,0}{}}
      \only<all:2>{\abstractedtile{3,0}}
      %
      \tile{6,3}{9}
      \abstractedtile{7,3}
      \tile{8,3}{12}
      \abstractedtile{9,3}
      \abstractedtile{6,2}
      \abstractedtile{7,2}
      \tile{8,2}{14}
      \tile{9,2}{13}
      \abstractedtile{6,1}
      \abstractedtile{7,1}
      \abstractedtile{8,1}
      \tile{9,1}{11}
      \tile{6,0}{15}
      \tile{7,0}{10}
      \tile{8,0}{8}
      \only<all:1>{\tile[black]{9,0}{}}
      \only<all:2>{\abstractedtile{9,0}}
    \end{pgfpicture}
  \end{center}
  \alert{Are the abstraction mappings orthogonal?}
\end{frame}

\begin{frame}{Orthogonality and additivity}
  \begin{theorem}[additivity for orthogonal abstraction mappings]
    Let $h^{\mathcal A_1, \alpha_1}, \dots, h^{\mathcal A_n,
    \alpha_n}$ be abstraction heuristics for the same planning task $\Pi$
    such that $\alpha_i$ and $\alpha_j$ are orthogonal for all
    $i \neq j$.

    Then $\sum_{i=1}^n h^{\mathcal A_i, \alpha_i}$ is a safe,
    goal-aware, admissible and consistent heuristic for $\Pi$.
  \end{theorem}
\end{frame}

\begin{frame}<beamer:1-5|handout:1,3,5>{Orthogonality and additivity: example}
  \begin{center}
    \only<all:1>{\pictwopackages}
    \only<beamer:2,3|handout:3>{\pictwopackagesabstractpackone}
    \only<beamer:4,5|handout:5>{\pictwopackagesabstractpacktwo}

    \bigskip

    \only<all:1>{\alert{transition system $\mathcal T$}

      state variables: first package, second package, truck}
    \only<beamer:2,3|handout:3>{\alert{abstraction $\mathcal A_1$}

      mapping: only consider state of first package}
    \only<beamer:4,5|handout:5>{\alert{abstraction $\mathcal A_2$}
      (orthogonal to $\mathcal A_1$)

      mapping: only consider state of second package}
  \end{center}
\end{frame}

\begin{frame}{Orthogonality and additivity: proof}
  \begin{proofstart}
    We prove goal-awareness and consistency; \\
    the other properties follow from these two.

    \smallskip

    Let $\mathcal T = \mathcal T(\Pi) = \langle S, L, T, s_0, S_\star\rangle$.

    \pause
    \smallskip

    \hilite{Goal-awareness:} For goal states $s \in S_\star$,
    $\sum_{i=1}^n h^{\mathcal A_i, \alpha_i}(s) = \sum_{i=1}^n 0 = 0$
    because all individual abstractions are goal-aware.
  \end{proofstart}
\end{frame}

\begin{frame}{Orthogonality and additivity: proof (ctd.)}
  \begin{proofmid}
    \hilite{Consistency:} Let $s, t \in S$ such that $t$ is a
    successor of $s$.

    Let $L := \sum_{i=1}^n h^{\mathcal A_i, \alpha_i}(s)$ and
    $R := \sum_{i=1}^n h^{\mathcal A_i, \alpha_i}(t)$.

    We need to prove that $L \le R + 1$.

    \pause
    \smallskip

    Since $t$ is a successor of $s$, there exists an operator $o$ with
    $\applyop{o}{s} = t$ and hence $\langle s, o, t\rangle \in T$.

    Because the abstraction mappings are orthogonal, $\alpha_i(s) \neq
    \alpha_i(t)$ for \alert{at most one} $i \in \{1, \dots, n\}$.

    \pause
    \smallskip

    \hilite{Case 1: $\alpha_i(s) = \alpha_i(t)$ for all $i \in
      \{1, \dots, n\}$.}

    \pause
    \begin{tightalign}
      \text{Then~} L & {} = \sum_{i=1}^n h^{\mathcal A_i, \alpha_i}(s) \\
    & {} = \sum_{i=1}^n h^*_{\mathcal A_i}(\alpha_i(s)) \\
    & {} = \sum_{i=1}^n h^*_{\mathcal A_i}(\alpha_i(t)) \\
    & {} = \sum_{i=1}^n h^{\mathcal A_i, \alpha_i}(t) \\
    & {} = R \le R + 1.
    \end{tightalign}
  \end{proofmid}
\end{frame}

\begin{frame}{Orthogonality and additivity: proof (ctd.)}
  \begin{proofend}
    \hilite{Case 2: $\alpha_i(s) \neq \alpha_i(t)$ for exactly one $i
      \in \{1, \dots, n\}$.}

    Let $k \in \{1, \dots, n\}$ such that $\alpha_k(s) \neq
      \alpha_k(t)$.

    \pause
    \smallskip
    \begin{tightalign}
      \text{Then~} L & {} = \sum_{i=1}^n h^{\mathcal A_i, \alpha_i}(s) \\
    & {} = \sum_{i \in \{1, \dots, n\} \setminus \{k\}}
      h^*_{\mathcal A_i}(\alpha_i(s))
      + h^{\mathcal A_k, \alpha_k}(s) \\
    & {} \le \sum_{i \in \{1, \dots, n\} \setminus \{k\}}
      h^*_{\mathcal A_i}(\alpha_i(t))
      + h^{\mathcal A_k, \alpha_k}(t) + 1 \\
    & {} = \sum_{i=1}^n h^{\mathcal A_i, \alpha_i}(t) + 1 \\
    & {} = R + 1,
    \end{tightalign}
    where the inequality holds because $\alpha_i(s) = \alpha_i(t)$ for
    all $i \neq k$ and $h^{\mathcal A_k, \alpha_k}$ is consistent.
  \end{proofend}
\end{frame}

\subsection[Refinements]{Coarsenings and refinements}

\begin{frame}{Abstractions of abstractions}
  \begin{theorem}[transitivity of abstractions]
    Let $\mathcal T$, $\mathcal T'$ and $\mathcal T''$ be transition
    systems.

    \begin{itemize}
    \item If $\mathcal T'$ is an abstraction of $\mathcal T$ \\
      and $\mathcal T''$ is an abstraction of $\mathcal T'$, \\
      then $\mathcal T''$ is an abstraction of $\mathcal T$.
    \item If $\mathcal T'$ is a strictly homomorphic abstraction of
      $\mathcal T$ \\
      and $\mathcal T''$ is a strictly homomorphic abstraction of
      $\mathcal T'$, \\
      then $\mathcal T''$ is a strictly homomorphic abstraction of
      $\mathcal T$.
    \end{itemize}
  \end{theorem}
\end{frame}

%% The following three frames belong together, but we must put them
%% into separate frames because \picprojectionpackagetruck uses
%% overlay numbers 1 and 2 internally. There is probably a better way
%% of doing this.
\begin{frame}{Abstractions of abstractions: example}
  \begin{center}
    \picfulltransitiongraph

    \bigskip

    \alert{transition system $\mathcal T$}
  \end{center}
\end{frame}

\begin{frame}{Abstractions of abstractions: example}
  \begin{center}
    \picprojectionpackagetruck

    \bigskip

    \alert{Transition system $\mathcal T'$ as an abstraction of
      $\mathcal T$}
  \end{center}
\end{frame}

\begin{frame}{Abstractions of abstractions: example}
  \begin{center}
    \only<all:1>{\picrefinement}
    \only<all:2>{\picprojectionpackage}

    \bigskip
    
    \alert{Transition system $\mathcal T''$ as an abstraction of
      \only<all:1>{$\mathcal T'$}\only<all:2>{$\mathcal T$}}
  \end{center}
\end{frame}

\begin{frame}{Abstractions of abstractions (proof)}
  \begin{proofstart}
    Let $\mathcal T = \langle S, L, T, s_0, S_\star\rangle$,
    let $\mathcal T' = \langle S', L, T', s'_0, S'_\star\rangle$ be an
    abstraction of $\mathcal T$ with abstraction mapping $\alpha$,
    and let $\mathcal T'' = \langle S'', L, T'', s''_0, S''_\star\rangle$ be
    an abstraction of $\mathcal T'$ with abstraction mapping
    $\alpha'$.

    \smallskip

    We show that $\mathcal T''$ is an abstraction of $\mathcal T$ with
    abstraction mapping $\beta := \alpha' \circ \alpha$, \ie, that
    \begin{enumerate}
    \item $\beta(s_0) = s''_0$,
    \item for all $s \in S_\star$, we have $\beta(s) \in S''_\star$, and
    \item for all $\langle s, \ell, t\rangle \in T$, we have $\langle
      \beta(s), \ell, \beta(t)\rangle \in T''$.
    \end{enumerate}

    Moreover, we show that if $\alpha$ and $\alpha'$ are strict
    homomorphisms, \\ then $\beta$ is also a strict homomorphism.

    \medskip

    \dots
  \end{proofstart}
\end{frame}

\begin{frame}{Abstractions of abstractions: proof}
  \begin{proofmid}
    \hilite{1. $\beta(s_0) = s''_0$}
    
    Because $\mathcal T'$ is an abstraction of $\mathcal T$ with
    mapping $\alpha$, we have $\alpha(s_0) = s'_0$. Because $\mathcal
    T''$ is an abstraction of $\mathcal T'$ with mapping $\alpha'$, we
    have $\alpha'(s'_0) = s''_0$.

    Hence $\beta(s_0) = \alpha'(\alpha(s_0)) = \alpha'(s'_0) = s''_0$.

    \medskip

    \dots
  \end{proofmid}
\end{frame}

\begin{frame}{Abstractions of abstractions: proof (ctd.)}
  \begin{proofmid}
    \hilite{2. For all $s \in S_\star$, we have $\beta(s) \in S''_\star$:}
    
    Let $s \in S_\star$. Because $\mathcal T'$ is an abstraction of
    $\mathcal T$ with mapping $\alpha$, we have $\alpha(s) \in S'_\star$.
    Because $\mathcal T''$ is an abstraction of $\mathcal T'$ with
    mapping $\alpha'$ and $\alpha(s) \in S'_\star$, we have
    $\alpha'(\alpha(s)) \in S''_\star$.

    Hence $\beta(s) = \alpha'(\alpha(s)) \in S''_\star$.

    \pause
    \medskip

    \hilite{Strict homomorphism if $\alpha$ and $\alpha'$
      strict homomorphisms:}
    
    Let $s'' \in S''_\star$. Because $\alpha'$ is a strict
    homomorphism, there exists a state $s' \in S'_\star$ such that
    $\alpha'(s') = s''$. Because $\alpha$ is a strict homomorphism,
    there exists a state $s \in S_\star$ such that $\alpha(s) = s'$.

    Thus $s'' = \alpha'(\alpha(s)) = \beta(s)$ for some $s \in S_\star$.

    \medskip

    \dots
  \end{proofmid}
\end{frame}

\begin{frame}{Abstractions of abstractions: proof (ctd.)}
  \begin{proofend}
    \hilite{3. For all $\langle s, \ell, t\rangle \in T$, we have
      $\langle \beta(s), \ell, \beta(t)\rangle \in T''$}
    
    Let $\langle s, \ell, t\rangle \in T$. Because $\mathcal T'$ is an
    abstraction of $\mathcal T$ with mapping $\alpha$, we have
    $\langle \alpha(s), \ell, \alpha(t)\rangle \in T'$. 
    Because $\mathcal T''$ is an abstraction of $\mathcal T'$ with
    mapping $\alpha'$ and $\langle \alpha(s), \ell, \alpha(t)\rangle \in
    T'$, we have $\langle \alpha'(\alpha(s)), \ell,
    \alpha'(\alpha(t))\rangle \in T''$.

    Hence $\langle \beta(s), \ell, \beta(t)\rangle =
    \langle \alpha'(\alpha(s)), \ell, \alpha'(\alpha(t))\rangle \in T''$.

    \pause
    \medskip

    \hilite{Strict homomorphism if $\alpha$ and $\alpha'$
      strict homomorphisms:}

    Let $\langle s'', \ell, t''\rangle \in T''$. Because $\alpha'$ is a
    strict homomorphism, there exists a transition $\langle s', \ell, t'\rangle
    \in T'$ such that $\alpha'(s') = s''$ and $\alpha'(t') = t''$.
    Because $\alpha$ is a strict homomorphism, there exists a transition
    $\langle s, \ell, t\rangle \in T$ such that $\alpha(s) = s'$ and
    $\alpha(t) = t'$.

    Thus $\langle s'', \ell, t''\rangle = \langle \alpha'(\alpha(s)), \ell,
    \alpha'(\alpha(t))\rangle = \langle \beta(s), \ell, \beta(t)\rangle$
    for some $\langle s, \ell, t\rangle \in T$.
  \end{proofend}
\end{frame}

\begin{frame}{Coarsenings and refinements}
  \hilite{Terminology:}
  Let $\mathcal T$ be a transition system, \\
  let $\mathcal T'$ be an abstraction of $\mathcal T$ with abstraction
  mapping $\alpha$,
  and \\ let $\mathcal T''$ be an abstraction of $\mathcal T'$ with
  abstraction mapping $\alpha'$.

  \smallskip

  Then:
  \begin{itemize}
  \item $\langle \mathcal T'', \alpha' \circ \alpha\rangle$ is called a
    \alert{coarsening} of $\langle \mathcal T', \alpha\rangle$, and
  \item $\langle \mathcal T', \alpha\rangle$ is called a
    \alert{refinement} of $\langle \mathcal T'', \alpha' \circ
    \alpha\rangle$.
  \end{itemize}
\end{frame}

\begin{frame}{Heuristic quality of refinements}
  \begin{theorem}[heuristic quality of refinements]
    Let $h^{\mathcal A, \alpha}$ and $h^{\mathcal B, \beta}$
    be abstraction heuristics for the same planning task $\Pi$
    such that $\langle \mathcal A, \alpha\rangle$ is a refinement of
    $\langle\mathcal B, \beta\rangle$.

    Then $h^{\mathcal A, \alpha}$ dominates $h^{\mathcal B, \beta}$.
  \end{theorem}

  In other words, $h^{\mathcal A, \alpha}(s) \ge h^{\mathcal B,
    \beta}(s)$ for all states $s$ of $\Pi$.

  \pause

  \begin{proofstart}
    Since $\langle \mathcal A, \alpha\rangle$ is a refinement of
    $\langle\mathcal B, \beta\rangle$, there exists a mapping
    $\alpha'$ such that $\beta = \alpha' \circ \alpha$ and
    $\mathcal B$ is an abstraction of $\mathcal A$ with abstraction
    mapping $\alpha'$.

    For any state $s$ of $\Pi$, we get
    $h^{\mathcal B, \beta}(s)
    = h^*_{\mathcal B}(\beta(s))
    = h^*_{\mathcal B}(\alpha'(\alpha(s)))
    = h^{\mathcal B, \alpha'}(\alpha(s))
    \le h^*_{\mathcal A}(\alpha(s))
    = h^{\mathcal A, \alpha}(s)$, where the inequality holds because
    $h^{\mathcal B, \alpha'}$ is an admissible heuristic in the
    transition system $\mathcal A$.
  \end{proofstart}
  %% TODO: We're abusing notation here when we write
  %% h^{\mathcal B, \alpha'}(\alpha(s))$, because this is a heuristic
  %% estimate with respect to to the transition system \mathcal A,
  %% which is not the transition system for a planning task.
  %% We have only defined abstraction heuristics for planning tasks
  %% (which was a bad idea, but we're stuck with it for now). In
  %% future courses, this should be cleaned up by defining abstraction
  %% heuristics for transition systems in general, which doesn't
  %% really change anything -- all the relevant definitions and
  %% theorems don't depend on properties of planning tasks at all.
\end{frame}

%% TODO: This subsection probably belongs somewhere much earlier,
%%       maybe even into Chapter 1.
\subsection[Equivalence]{Equivalent transition systems}

\begin{frame}{Isomorphic transition systems}
  \begin{definition}[isomorphic transition systems]
    Let $\mathcal T = \langle S, L, T, s_0, S_\star\rangle$ and
    $\mathcal T' = \langle S', L', T', s'_0, S'_\star\rangle$ be transition
    systems.

    We say that \alert{$\mathcal T$ is isomorphic to
      $\mathcal T'$}, in symbols \alert{$\mathcal T \sim \mathcal
      T'$}, if there exist bijective functions $\varphi: S \to S'$
    and $\psi: L \to L'$ such that:
    \begin{itemize}
    \item $\varphi(s_0) = s'_0$,
    \item $s \in S_\star$ iff $\varphi(s) \in S'_\star$, and
    \item $\langle s, \ell, t\rangle \in T$ iff $\langle \varphi(s),
      \psi(\ell), \varphi(t)\rangle \in T'$.
    \end{itemize}
  \end{definition}
\end{frame}

\begin{frame}{Graph-equivalent transition systems}
  \begin{definition}[graph-equivalent transition systems]
    Let $\mathcal T = \langle S, L, T, s_0, S_\star\rangle$ and
    $\mathcal T' = \langle S', L', T', s'_0, S'_\star\rangle$ be transition
    systems.

    We say that \alert{$\mathcal T$ is graph-equivalent to $\mathcal
      T'$}, in symbols \alert{$\mathcal T \graphequiv \mathcal T'$},
    if there exists a bijective function $\varphi: S \to S'$ such
    that:
    \begin{itemize}
    \item $\varphi(s_0) = s'_0$,
    \item $s \in S_\star$ iff $\varphi(s) \in S'_\star$, and
    \item $\langle s, \ell, t\rangle \in T$ for some $\ell \in L$ iff
      $\langle \varphi(s), \ell', \varphi(t)\rangle \in T'$ for some $\ell'
      \in L'$.
    \end{itemize}
  \end{definition}
  \hilite{Note:} There is no requirement that the labels of $\mathcal
  T$ and $\mathcal T'$ correspond in any way. For example, it is
  permitted that all transitions of $\mathcal T$ have different labels
  and all transitions of $\mathcal T'$ have the same label.
\end{frame}

\begin{frame}{Isomorphism vs.\ graph equivalence}
  \begin{itemize}
  \item $(\sim)$ and $(\graphequiv)$ are equivalence relations.
  \item Two isomorphic transition systems are interchangeable for all
    practical intents and purposes.
  \item Two graph-equivalent transition systems are interchangeable
    for most intents and purposes. \\ In particular, their state
    distances are identical, so they define the same abstraction
    heuristic for corresponding abstraction functions.
  \item Isomorphism implies graph equivalence, but not vice versa.
  \end{itemize}
\end{frame}

\subsection[Practice]{Abstraction heuristics in practice}

\begin{frame}{Using abstraction heuristics in practice}
  In practice, there are conflicting goals for abstractions:
  \begin{itemize}
  \item we want to obtain an \alert{informative heuristic}, but
  \item want to keep its \alert{representation small}.
  \end{itemize}

  \medskip

  Abstractions have small representations if they have
  \begin{itemize}
  \item \alert{few abstract states} and
  \item a \alert{succinct encoding for $\alpha$}.
  \end{itemize}
\end{frame}

\begin{frame}{Counterexample: one-state abstraction}
  \begin{center}
    \piconestateabstraction
  \end{center}

  \alert{One-state abstraction:} $\alpha(s) := \text{const}$.
  \begin{itemize}
  \item[$+$] \alert{very few abstract states}
    and \alert{succinct encoding for $\alpha$}
  \item[$-$] \alert{completely uninformative heuristic}
  \end{itemize}
\end{frame}

\begin{frame}{Counterexample: identity abstraction}
  \begin{center}
    \picfulltransitiongraph
  \end{center}

  \alert{Identity abstraction:} $\alpha(s) := s$.
  \begin{itemize}
  \item[$+$] \alert{perfect heuristic}
    and \alert{succinct encoding for $\alpha$}
  \item[$-$] \alert{too many abstract states}
  \end{itemize}
\end{frame}

\begin{frame}{Counterexample: perfect abstraction}
  \begin{center}
    \picperfectabstraction
  \end{center}

  \alert{Perfect abstraction:} $\alpha(s) := h^*(s)$. \\
  \begin{itemize}
  \item[$+$] \alert{perfect heuristic} and
    usually \alert{few abstract states}
  \item[$-$] usually \alert{no succinct encoding for $\alpha$}
  \end{itemize}
\end{frame}

\begin{frame}{Automatically deriving good abstraction heuristics}
  \begin{block}{Abstraction heuristics for planning: main research
      problem}
    \alert{Automatically derive effective abstraction heuristics} \\
    for planning tasks.
  \end{block}

  \begin{itemize}
  \item[$\leadsto$] we will study two state-of-the-art approaches \\
    in the next two chapters
  \end{itemize}
\end{frame}

\section*{Summary}

\begin{frame}{Summary}
  \begin{itemize}
  \item An \alert{abstraction} relates a transition system $\mathcal
    T$ (\eg\ of a planning task) to another (usually smaller)
    transition system $\mathcal T'$ via an \alert{abstraction mapping}
    $\alpha$.
  \item Abstraction \alert{preserves all important aspects} of
    $\mathcal T$: \\ initial state, goal states and (labeled)
    transitions.
  \item Hence, they can be used to define \alert{heuristics} for the
    original system $\mathcal T$: estimate the goal distance of $s$ in
    $\mathcal T$ by the optimal goal distance of $\alpha(s)$ in
    $\mathcal T'$.
  \item Such \alert{abstraction heuristics} are \alert{safe},
    \alert{goal-aware}, \alert{admissible} and \alert{consistent}.
  \end{itemize}
\end{frame}

\begin{frame}{Summary (ctd.)}
  \begin{itemize}
  \item \alert{Strictly homomorphic abstractions} are desirable as
    they do not include ``unnecessary'' abstract goal states or
    transitions (which could lower heuristic values).
  \item Any surjection from the states of $\mathcal T$ to any set
    induces a strictly homomorphic abstraction in a natural way.
  \item Multiple abstraction heuristics can be added without losing
    properties like admissibility if the underlying abstraction
    mappings are \alert{orthogonal}.
  \item One sufficient condition for orthogonality is that
    abstractions are \alert{affected} by disjoint sets of labels.
  \end{itemize}
\end{frame}

\begin{frame}{Summary (ctd.)}
  \begin{itemize}
  \item The process of abstraction is \alert{transitive}: an
    abstraction can be abstracted further to yield another
    abstraction.
  \item Based on this notion, we can define abstractions that are
    \alert{coarsenings} or \alert{refinements} of others.
  \item A refinement can never lead to a worse heuristic.
  \item Practically useful abstractions are those which give
    \alert{informative heuristics}, yet have a \alert{small
      representation}.
  \end{itemize}
\end{frame}

\end{document}
