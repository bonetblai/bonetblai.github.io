\documentclass{gkibeamer}

\usepackage{tikz}[2008/02/13]

\title[AI Planning]{Principles of AI Planning}
\author[B.~Nebel,\\R.~Mattm\"uller]{Bernhard Nebel and Robert Mattm\"uller}
\date[]{Winter Semester 2011/2012}

%% Use \alert instead of \textred where it makes sense.
%% (\textred makes sense when you *really* want the color to be red,
%% independent of the beamer style settings.)
\newcommand{\hilite}[1]{{\color{structure} #1}}
\newcommand{\textred}[1]{{\color{red} #1}}
\newcommand{\textblue}[1]{{\color{blue} #1}}
\newcommand{\textgreen}[1]{{\color{green} #1}}
\newcommand{\textbrown}[1]{{\color{brown} #1}}
\newcommand{\textviolet}[1]{{\color{violet} #1}}
\newcommand{\textbg}[1]{{\color{bg} #1}}

\newcommand{\eg}{e.\,g.}
\newcommand{\ie}{i.\,e.}

%% State variables and values for state variables
\newcommand{\var}[1]{\textit{#1}}
\newcommand{\val}[1]{\text{#1}}

%% Various stuff
\newcommand{\astar}{A${}^*$}
\newcommand{\idastar}{IDA${}^*$}

%% Blocksworld stuff
\newcommand{\AONB}{\var{A-on-B}}
\newcommand{\AONC}{\var{A-on-C}}
\newcommand{\AOND}{\var{A-on-D}}
\newcommand{\BONA}{\var{B-on-A}}
\newcommand{\BONC}{\var{B-on-C}}
\newcommand{\BOND}{\var{B-on-D}}
\newcommand{\CONA}{\var{C-on-A}}
\newcommand{\CONB}{\var{C-on-B}}
\newcommand{\COND}{\var{C-on-D}}
\newcommand{\DONA}{\var{D-on-A}}
\newcommand{\DONB}{\var{D-on-B}}
\newcommand{\DONC}{\var{D-on-C}}
\newcommand{\AONTABLE}{\var{A-on-T}}
\newcommand{\BONTABLE}{\var{B-on-T}}
\newcommand{\CONTABLE}{\var{C-on-T}}
\newcommand{\CLEARA}{\var{A-clear}}
\newcommand{\CLEARB}{\var{B-clear}}
\newcommand{\CLEARC}{\var{C-clear}}

\newcommand{\applyop}[2]{\textit{app}_{#1}(#2)}
\newcommand{\applyops}[2]{\textit{app}_{#1}(#2)}
\newcommand{\applyplan}[2]{\textit{app}_{#1}(#2)}
\newcommand{\regr}[2]{\textit{regr}_{#1}(#2)}
\newcommand{\regrstrips}[2]{\textit{sregr}_{#1}(#2)}
\newcommand{\eprecon}[2]{\textit{EPC}_{#1}(#2)}

\newcommand{\compl}[1]{\overline{#1}}

\newcommand{\changes}[2]{\lbrack #1\rbrack_{#2}}
\newcommand{\CEF}{\vartriangleright}

\newcommand{\relaxation}[1]{{#1}^+}
\newcommand{\onset}[1]{\textit{on}(#1)}
\newcommand{\hplus}{\ensuremath{h^+}}
\newcommand{\hmax}{\ensuremath{h_{\text{max}}}}
\newcommand{\hadd}{\ensuremath{h_{\text{add}}}}
\newcommand{\hlmcut}{\ensuremath{h_{\text{LM-cut}}}}
\newcommand{\hff}{\ensuremath{h_{\text{FF}}}}
\newcommand{\hcs}{\ensuremath{h_{\text{cs}}}}
\newcommand{\hsa}{\ensuremath{h_{\text{sa}}}}
\newcommand{\hhhh}{\ensuremath{h_{\text{HHH}}}}

\newcommand{\sasplus}{SAS${}^+$}

\newcommand{\graphequiv}{\stackrel{\textup{G}}{\sim}}

\newcommand{\cg}{\ensuremath{\textit{CG}}}

%% Turing machines and complexity theory
\newcommand{\accept}{{\textsf{Y}}}

\newcommand{\easier}{\ensuremath{\le_{\text{p}}}}

\newcommand{\decisionclass}[1]{\ensuremath{\textsf{\textup{#1}}}}
\newcommand{\dtime}{\decisionclass{DTIME}}
\newcommand{\ntime}{\decisionclass{NTIME}}
\newcommand{\dspace}{\decisionclass{DSPACE}}
\newcommand{\nspace}{\decisionclass{NSPACE}}

\newcommand{\ptime}{\decisionclass{P}}
\newcommand{\np}{\decisionclass{NP}}
\newcommand{\pspace}{\decisionclass{PSPACE}}
\newcommand{\npspace}{\decisionclass{NPSPACE}}
\newcommand{\exptime}{\decisionclass{EXP}}
\newcommand{\nexptime}{\decisionclass{NEXP}}
\newcommand{\expspace}{\decisionclass{EXPSPACE}}
\newcommand{\dblexptime}{\decisionclass{2EXP}}

\newcommand{\planex}{\textsc{PlanEx}}
\newcommand{\planlen}{\textsc{PlanLen}}

\newcommand{\ms}[1]{\mathchoice{\mbox{\normalsize\it #1}}{\mbox{\normalsize\it #1}}{\mbox{\scriptsize\it #1}}{\mbox{\tiny\it #1}}}
\newcommand{\image}[2]{\ms{img}_{#1}(#2)}
\newcommand{\spreimage}[2]{\ms{spreimg}_{#1}(#2)}
\newcommand{\wpreimage}[2]{\ms{wpreimg}_{#1}(#2)}
\newcommand{\disbwd}[2]{\delta^{\ms{bwd}}_{#1}(#2)}
\newcommand{\ats}[1]{\ms{scope}(#1)}
\newcommand{\ndopcpc}[2]{\tau^{\ms{nd}}_{#1}(#2)}
\newcommand{\choice}{\mathop{|}}
\newcommand{\opcpc}[2]{\tau_{#1}(#2)}

%% Don't use \begin{proof}/\end{proof} for multi-part proofs because
%% this places a QED symbol.
\newenvironment{proofstart}{\begin{block}{Proof.}}{\end{block}}
\newenvironment{proofmid}{\begin{block}{Proof (ctd.)}}{\end{block}}
\newenvironment{proofend}{\begin{proof}[Proof (ctd.)]}{\end{proof}}

\newtheorem{remark}[theorem]{Remark}

%% Macros to align the width of things.
\newlength{\mywidth}
\newcommand{\setmywidth}[1]{\settowidth{\mywidth}{#1}}
\newcommand{\usemywidth}[1]{\makebox[\mywidth][l]{#1}}
\newcommand{\usemywidthmath}[1]{\usemywidth{\ensuremath{#1}}}

%% A tighter "align" environment.
\newenvironment{tightalign}[1][c]{\par\(\begin{array}[#1]{@{}r@{}l}}
  {\end{array}\)\par}
\newenvironment{wrappedmath}[1][t]{\begin{array}[#1]{@{}l}}{\end{array}}

%% Nondeterministic transition systems with connector arcs
\newcommand{\arc}[4]{
  \begin{pgfscope}
    \pgfsetlinewidth{1pt}
    \pgfpathmoveto{\pgfpointanchor{#3}{center}}
    \pgfpathlineto{\pgfpointanchor{#1}{#2}}
    \pgfpathlineto{\pgfpointanchor{#4}{center}}
    \pgfusepath{clip}
    \pgfpathcircle{\pgfpointanchor{#1}{#2}}{1.8mm}
    \pgfusepath{draw}
  \end{pgfscope}
}


\begin{document}
\lectureno{7}
\subtitle{Planning as search: relaxed planning tasks}
\date{November 18th, 2011}
\maketitles

\section[Obtaining heuristics]{How to obtain a heuristic}
\subsection[STRIPS heuristic]{The STRIPS heuristic}

\begin{frame}{A simple heuristic for deterministic planning}
  STRIPS (Fikes \& Nilsson, 1971) used the number of state
  variables that differ in current state $s$ and a
  STRIPS goal $a_1 \land \dots \land a_n$:
  \[ h(s) := |\{ i \in \{1, \dots, n\} \mid s \not\models a_i
      \}|. \]

  \hilite{Intuition:} more true goal literals $\leadsto$ closer to the
  goal

  \medskip

  $\leadsto$ \alert{STRIPS heuristic (a.k.a.\ goal-count heuristic)} (properties?)

  \bigskip

  \hilite{Note:} From now on, for convenience we usually write
  heuristics as functions of states (as above), not nodes.

  Node heuristic $h'$ is defined from state heuristic $h$ as
  $h'(\sigma) := h(\textit{state}(\sigma))$.
\end{frame}

\begin{frame}{Criticism of the STRIPS heuristic}
  What is wrong with the STRIPS heuristic?
  \begin{itemize}
  \item quite \alert{uninformative}: \\
    the range of heuristic values in a given task is small; \\
    typically, most successors have the same estimate
  \item very sensitive to \alert{reformulation}: \\
    can easily transform any planning task into an equivalent one
    where $h(s) = 1$ for all non-goal states (how?)
  \item ignores almost all \alert{problem structure}: \\
    heuristic value does not depend on the set of operators!
  \end{itemize}

  $\leadsto$ need a better, principled way of coming up with
  heuristics
\end{frame}

\subsection{Relaxation and abstraction}

\begin{frame}{Coming up with heuristics in a principled way}
  \begin{block}{General procedure for obtaining a heuristic}
    Solve an easier version of the problem.
  \end{block}

  Two common methods:
  \begin{itemize}
  \item \alert{relaxation:} consider \alert{less constrained} version
    of the problem
  \item \alert{abstraction:} consider \alert{smaller} version of real
    problem
  \end{itemize}

  Both have been very successfully applied in planning. \\
  We consider both in this course, beginning with \alert{relaxation}.
\end{frame}

\begin{frame}{Relaxing a problem}
  How do we relax a problem?

 \begin{example}[Route planning for a road network]
   The road network is formalized as a weighted graph over points in
   the Euclidean plane. The weight of an edge is the \alert{road
   distance} between two locations.
 \end{example}

 A relaxation \alert{drops constraints} of the original problem.

 \begin{example}[Relaxation for route planning]
 
   Use the \alert{Euclidean distance} $\sqrt{|x_1 - x_2|^2 + |y_1 -
   y_2|^2}$ as a heuristic for the road distance between
   $\langle x_1, y_1\rangle$ and $\langle x_2, y_2\rangle$

   This is \alert{a lower bound} on the road distance ($\leadsto$
   admissible).
 \end{example}

 $\leadsto$ We drop the constraint of having to travel on roads.
\end{frame}

\newcommand<>{\city}[3]{
  \pgfnodecircle{#1}[fill]{\pgfxy(#3,#2)}{1.10mm}
  \pgfputat{\pgfrelative{\pgfxy(0,-0.15)}{\pgfnodecenter{#1}}}
	   {\pgfbox[left,center]{\small\alert#4{#1}}}}
\newcommand<>{\citya}[3]{\pgfnodecircle{#1}[fill]{\pgfxy(#3,#2)}{1.10mm}
  \pgfputat{\pgfrelative{\pgfxy(0,0.15)}{\pgfnodecenter{#1}}}
	   {\pgfbox[left,center]{\small\alert#4{#1}}}}
\newcommand<>{\cityl}[3]{\pgfnodecircle{#1}[fill]{\pgfxy(#3,#2)}{1.10mm}
  \pgfputat{\pgfrelative{\pgfxy(-0.15,0.0)}{\pgfnodecenter{#1}}}
	   {\pgfbox[right,center]{\small\alert#4{#1}}}}

\begin{frame}{{\astar} using the Euclidean distance heuristic}
 \begin{pgfpicture}{11.0cm}{119cm}{21cm}{126cm}
   \pgfsetxvec{\pgfpoint{1.5cm}{0cm}}
   \pgfsetyvec{\pgfpoint{0cm}{2.5cm}}
   \pgfsetlinewidth{0.7pt}
   
   %% Apparently, some of these coordinates were originally taken
   %% from a map, but at least some (Wurzburg?) were moved around
   %% a bit for visual clarity.
   \cityl<all:3,4>{Frankfurt}{50.12}{8.68}
   \city<all:1>{Freiburg}{47.98}{7.85}
   \cityl<all:2>{Karlsruhe}{49.00}{8.40}
   \city<all:7>{Munich}{48.13}{11.58}
   \citya<all:6,7,8>{Nuremberg}{49.45}{11.1}
   \city<all:0>{Passau}{48.57}{13.43}
   \citya<all:0>{Regensburg}{49.02}{12.10}
   \citya<all:3>{Stuttgart}{48.80}{9.20}
   \city<all:4,5,6>{Ulm}{48.40}{9.97}
   \citya<all:5>{Wurzburg}{50.00}{9.9}

   \alert<all:8>{\pgfnodeconnline{Freiburg}{Karlsruhe}}
   \alert<all:8>{\pgfnodeconnline{Karlsruhe}{Frankfurt}}
   \pgfnodeconnline{Karlsruhe}{Stuttgart}
   \pgfnodeconnline{Stuttgart}{Ulm}
   \pgfnodeconnline{Ulm}{Munich}
   \pgfnodeconnline{Munich}{Passau}
   \pgfnodeconnline{Munich}{Regensburg}
   \pgfnodeconnline{Regensburg}{Passau}
   \pgfnodeconnline{Nuremberg}{Regensburg}
   \alert<all:8>{\pgfnodeconnline{Wurzburg}{Nuremberg}}
   \alert<all:8>{\pgfnodeconnline{Frankfurt}{Wurzburg}}

   \pgfnodelabelrotated{Freiburg}{Karlsruhe}[0.5][1pt]{
     \pgfbox[center,base]{\footnotesize 120 km}}
   \pgfnodelabelrotated{Karlsruhe}{Frankfurt}[0.5][1pt]{
     \pgfbox[center,base]{\footnotesize 120 km}}
   \pgfnodelabelrotated{Karlsruhe}{Stuttgart}[0.5][1pt]{
     \pgfbox[center,base]{\footnotesize 100 km}}
   \pgfnodelabelrotated{Stuttgart}{Ulm}[0.5][1pt]{
     \pgfbox[center,base]{\footnotesize 100 km}}
   \pgfnodelabelrotated{Ulm}{Munich}[0.5][1pt]{
     \pgfbox[center,base]{\footnotesize 100 km}}
   \pgfnodelabelrotated{Munich}{Passau}[0.5][1pt]{
     \pgfbox[center,base]{\footnotesize 120 km}}
   \pgfnodelabelrotated{Munich}{Regensburg}[0.5][1pt]{
     \pgfbox[center,base]{\footnotesize 80 km}}
   \pgfnodelabelrotated{Regensburg}{Passau}[0.5][1pt]{
     \pgfbox[center,base]{\footnotesize 160 km}}
   \pgfnodelabelrotated{Nuremberg}{Regensburg}[0.5][1pt]{
     \pgfbox[center,base]{\footnotesize 100 km}}
   \pgfnodelabelrotated{Frankfurt}{Wurzburg}[0.5][1pt]{
     \pgfbox[center,base]{\footnotesize 100 km}}
   \pgfnodelabelrotated{Wurzburg}{Nuremberg}[0.75][1pt]{
     \pgfbox[center,base]{\footnotesize 120 km}}
   
   \pgfsetdash{{3pt}{3pt}}{0pt}
   \color{violet}

   \visible<all:1>{
     \pgfnodeconnline{Freiburg}{Nuremberg}
     \pgfnodelabelrotated{Freiburg}{Nuremberg}[0.8][2pt]{
       \pgfbox[center,base]{\footnotesize\textblue{200 km}}}
     \pgfnodelabelrotated{Freiburg}{Nuremberg}[0.8][-2pt]{
       \pgfbox[center,top]{\footnotesize 200 km}}
   }
   \visible<all:2>{
     \pgfnodeconnline{Karlsruhe}{Nuremberg}
     \pgfnodelabelrotated{Karlsruhe}{Nuremberg}[0.5][2pt]{
       \pgfbox[center,base]{\footnotesize\textblue{270 km}}}
     \pgfnodelabelrotated{Karlsruhe}{Nuremberg}[0.5][-2pt]{
       \pgfbox[center,top]{\footnotesize 150 km}}
   }
   \visible<all:3,4>{
     \pgfnodeconnline{Frankfurt}{Nuremberg}
     \pgfnodelabelrotated{Frankfurt}{Nuremberg}[0.5][2pt]{
       \pgfbox[center,base]{\footnotesize\textblue{420 km}}}
     \pgfnodelabelrotated{Frankfurt}{Nuremberg}[0.5][-2pt]{
       \pgfbox[center,top]{\footnotesize 180 km}}
   }
   \visible<all:3>{
     \pgfnodeconnline{Stuttgart}{Nuremberg}
     \pgfnodelabelrotated{Stuttgart}{Nuremberg}[0.6][2pt]{
       \pgfbox[center,base]{\footnotesize\textblue{340 km}}}
     \pgfnodelabelrotated{Stuttgart}{Nuremberg}[0.6][-2pt]{
       \pgfbox[center,top]{\footnotesize 120 km}}
   }
   \visible<all:4,5,6>{
     \pgfnodeconnline{Ulm}{Nuremberg}
     \pgfnodelabelrotated{Ulm}{Nuremberg}[0.5][2pt]{
       \pgfbox[center,base]{\footnotesize\textblue{450 km}}}
     \pgfnodelabelrotated{Ulm}{Nuremberg}[0.5][-2pt]{
       \pgfbox[center,top]{\footnotesize 130 km}}
   }
   \visible<all:5>{
     \pgfnodeconnline{Wurzburg}{Nuremberg}
     \pgfnodelabelrotated{Wurzburg}{Nuremberg}[0.25][2pt]{
       \pgfbox[center,base]{\footnotesize\textblue{440 km}}}
     \pgfnodelabelrotated{Wurzburg}{Nuremberg}[0.25][-2pt]{
       \pgfbox[center,top]{\footnotesize 100 km}}
   }
   \visible<all:7>{
     \pgfnodeconnline{Nuremberg}{Munich}
     \pgfnodelabelrotated{Nuremberg}{Munich}[0.5][2pt]{
       \pgfbox[center,base]{\footnotesize\textblue{540 km}}}
     \pgfnodelabelrotated{Nuremberg}{Munich}[0.5][-2pt]{
       \pgfbox[center,top]{\footnotesize 120 km}}
   }
   \visible<all:6->{\pgfputat{\pgfxy(11.1,49.75)}{
       \pgfbox[left,center]{\footnotesize\textblue{460 km}}}}
 \end{pgfpicture}
\end{frame}

\section{Relaxed planning tasks}
\subsection{Definition}

\begin{frame}{Relaxed planning tasks: idea}
  In \alert{positive normal form} (remember?), good and bad effects
  are easy to distinguish:
  \begin{itemize}
  \item Effects that make state variables true are good \\
    (\alert{add effects}).
  \item Effects that make state variables false are bad \\
    (\alert{delete effects}).
  \end{itemize}
  
  Idea for the heuristic: Ignore all delete effects.
\end{frame}

\begin{frame}{Relaxed planning tasks}
  \begin{definition}[relaxation of operators]
    The \alert{relaxation} $\relaxation{o}$ of an operator
    $o = \langle \chi, e\rangle$ in positive normal form is the operator
    which is obtained by replacing all negative effects $\neg a$
    within $e$ by the do-nothing effect $\top$.
  \end{definition}
  
  \begin{definition}[relaxation of planning tasks]
    The \alert{relaxation} $\relaxation{\Pi}$ of a planning task
    $\Pi = \langle A, I, O, \gamma\rangle$ in positive normal form is the
    planning task $\relaxation{\Pi} := \langle A, I,
    \{\relaxation{o} \mid o \in O\}, \gamma\rangle$.
  \end{definition}
  
  \begin{definition}[relaxation of operator sequences]
    The \alert{relaxation} of an operator sequence $\pi = o_1 \dots
    o_n$ is the operator sequence $\relaxation{\pi} :=
    \relaxation{o_1} \dots \relaxation{o_n}$.
  \end{definition}
\end{frame}

\begin{frame}{Relaxed planning tasks: terminology}
  \begin{itemize}
  \item Planning tasks in positive normal form without delete effects
    are called \alert{relaxed planning tasks}.
  \item Plans for relaxed planning tasks are called \alert{relaxed
    plans}.
  \item If $\Pi$ is a planning task in positive normal form and
    $\relaxation{\pi}$ is a plan for $\relaxation{\Pi}$, then
    $\relaxation{\pi}$ is called a \alert{relaxed plan for $\Pi$}.
  \end{itemize}
\end{frame}

\subsection{The relaxation lemma}

\begin{frame}{Dominating states}
  The \alert{on-set} $\onset{s}$ of a state $s$ is the set of
  true state variables in $s$, i.e.\ $\onset{s} = s^{-1}(\{1\})$.
  
  A state $s'$ \alert{dominates} another state $s$ iff
  $\onset{s} \subseteq \onset{s'}$.

  \begin{lemma}[domination]
    Let $s$ and $s'$ be valuations of a set of propositional variables
    $A$ and let $\chi$ be a propositional formula over $A$ which does not
    contain negation symbols.
    
    If $s \models \chi$ and $s'$ dominates $s$, then $s' \models \chi$.
  \end{lemma}

  \begin{proofstart}
    Proof by induction over the structure of $\chi$.
    \begin{itemize}
    \item Base case $\chi = \top$: then $s' \models \top$.
    \item Base case $\chi = \bot$: then $s \not\models \bot$.
    \end{itemize}
  \end{proofstart}
\end{frame}

\begin{frame}{Dominating states (ctd.)}
  \begin{proofend}
    \begin{itemize}
    \item Base case $\chi = a \in A$: assume $s \models a$ and
      $\mathit{on}(s) \subseteq \mathit{on}(s')$. With $a \in
      \mathit{on}(s)$ we get $a \in \mathit{on}(s')$, hence $s'
      \models a$.
    \item Inductive case $\chi = \chi_1 \wedge \chi_2$: by induction
      hypothesis, our claim holds for the proper subformulas $\chi_1$
      and $\chi_2$ of $\chi$.
      \begin{eqnarray*}
        s\phantom{'} \models \chi & \Longleftrightarrow & s\phantom{'} \models
        \chi_1 \wedge \chi_2\\ 
        & \Longleftrightarrow & s\phantom{'} \models \chi_1 \text{
          and } s\phantom{'} \models \chi_2\\ 
        & \stackrel{\textrm{I.H. (twice)}}{\Longrightarrow} &
        s' \models \chi_1 \text{ and } s' \models \chi_2\\
        & \Longleftrightarrow & s' \models \chi_1 \wedge \chi_2\\
        & \Longleftrightarrow & s' \models \chi.
      \end{eqnarray*}
    \item Inductive case $\chi = \chi_1 \vee \chi_2$: Analogous.
    \end{itemize}
  \end{proofend}
\end{frame}

\begin{frame}{The relaxation lemma}
  For the rest of this chapter, we assume that all planning
  tasks are in positive normal form.
  \begin{lemma}[relaxation]
    Let $s$ be a state, let $s'$ be a state that dominates $s$, \\
    and let $\pi$ be an operator sequence which is applicable in $s$.
    
    Then $\relaxation{\pi}$ is applicable in $s'$ and
    $\applyplan{\relaxation{\pi}}{s'}$ dominates
    $\applyplan{\pi}{s}$.

    Moreover, if $\pi$ leads to a goal state from $s$, then
    $\relaxation{\pi}$ leads to a goal state from $s'$.
  \end{lemma}
  
  \begin{proofstart}
    The ``moreover'' part follows from the rest by the domination
    lemma. Prove the rest by induction over the length of $\pi$.

    \pause
    \smallskip

    \textblue{Base case:} $\pi = \epsilon$

    $\applyplan{\relaxation{\pi}}{s'} = s'$
    dominates $\applyplan{\pi}{s} = s$ by assumption.
  \end{proofstart}
\end{frame}

\begin{frame}{The relaxation lemma (ctd.)}
  \begin{proofend}
    \textblue{Inductive case:} $\pi = o_1\dots o_{n+1}$

    By the induction hypothesis,
    $\relaxation{o_1}\dots\relaxation{o_n}$
    is applicable in $s'$, and
    $t' = \applyplan{\relaxation{o_1}\dots\relaxation{o_n}}{s'}$
    dominates
    $t = \applyplan{o_1 \dots o_n}{s}$.
    
    \pause
    \smallskip
    
    Let $o := o_{n+1} = \langle \chi,e\rangle$
    and $\relaxation{o} = \langle \chi, e^+\rangle$.
    By assumption, $o$ is applicable in $t$, and thus
    $t \models \chi$. By the domination lemma, we get $t' \models \chi$
    and hence $\relaxation{o}$ is applicable in $t'$.
    Therefore, $\relaxation{\pi}$ is applicable in $s'$.
    
    \pause
    \smallskip
    
    Because $o$ is in positive normal form, all effect conditions
    satisfied by $t$ are also satisfied by $t'$ (by the domination
    lemma). Therefore, $(\changes{e}{t} \cap A) \subseteq
    \changes{e^+}{t'}$ (where $A$ is the set of state variables, or
    positive literals).
   
    \pause
    \smallskip
    
    We get
    $\onset{\applyplan{\pi}{s}}
    \subseteq \onset{t} \cup (\changes{e}{t} \cap A)
    \subseteq \onset{t'} \cup \changes{e^+}{t'}
    = \onset{\applyplan{\relaxation{\pi}}{s'}}$,
    and thus
    $\applyplan{\relaxation{\pi}}{s'}$ dominates
    $\applyplan{\pi}{s}$.
  \end{proofend}
\end{frame}

\begin{frame}{Consequences of the relaxation lemma}
  \begin{corollary}[relaxation leads to dominance and preserves plans]
    Let $\pi$ be an operator sequence which is applicable in state
    $s$.
    
    Then $\relaxation{\pi}$ is applicable in $s$ and
    $\applyplan{\relaxation{\pi}}{s}$ dominates
    $\applyplan{\pi}{s}$.

    If $\pi$ is a plan for $\Pi$, then $\relaxation{\pi}$ is a plan
    for $\relaxation{\Pi}$.
  \end{corollary}

  \begin{proof}
    Apply relaxation lemma with $s' = s$.
  \end{proof}

  \begin{itemize}
  \item[$\leadsto$] Relaxations of plans are relaxed plans.
  \item[$\leadsto$] Relaxations are no harder to solve than the
    original task.
  \item[$\leadsto$] Optimal relaxed plans are never longer
    than optimal plans for original tasks.
  \end{itemize}
\end{frame}

\begin{frame}{Consequences of the relaxation lemma (ctd.)}
  \begin{corollary}[relaxation preserves dominance]
    Let $s$ be a state, let $s'$ be a state that dominates $s$, \\
    and let $\relaxation{\pi}$ be a relaxed operator sequence
    applicable in $s$.
    
    Then $\relaxation{\pi}$ is applicable in $s'$ and
    $\applyplan{\relaxation{\pi}}{s'}$ dominates
    $\applyplan{\relaxation{\pi}}{s}$.
  \end{corollary}

  \begin{proof}
    Apply relaxation lemma with $\relaxation{\pi}$ for $\pi$,
    noting that $\relaxation{(\relaxation{\pi})} = \relaxation{\pi}$.
  \end{proof}

  \begin{itemize}
  \item[$\leadsto$] If there is a relaxed plan starting from state
    $s$, the same plan can be used starting from a dominating state
    $s'$.
  \item[$\leadsto$] Making a transition to a dominating state never
    hurts in relaxed planning tasks.
  \end{itemize}
\end{frame}

\subsection{Greedy algorithm}

\begin{frame}{Monotonicity of relaxed planning tasks}
  We need one final property before we can provide an algorithm for
  solving relaxed planning tasks.
  \begin{lemma}[monotonicity]
    Let $\relaxation{o} = \langle \chi, \relaxation{e}\rangle$ be a
    relaxed operator and let $s$ be a state in which $\relaxation{o}$
    is applicable.
    
    Then $\applyop{\relaxation{o}}{s}$ dominates $s$.
  \end{lemma}

  \begin{proof}
    Since relaxed operators only have positive effects,
    we have
    $\onset{s} \subseteq \onset{s} \cup \changes{\relaxation{e}}{s} = 
    \onset{\applyop{\relaxation{o}}{s}}$.
  \end{proof}

  \begin{itemize}
  \item[$\leadsto$] Together with our previous results, this means
    that making a transition in a relaxed planning task \alert{never}
    hurts.
  \end{itemize}
\end{frame}

\begin{frame}{Greedy algorithm for relaxed planning tasks}
  The relaxation and monotonicity lemmas suggest the following
  algorithm for solving relaxed planning tasks:
  \begin{block}{Greedy planning algorithm for
      $\langle A, I, \relaxation{O}, \gamma\rangle$}
    $s := I$ \\
    $\relaxation{\pi} := \epsilon$ \\
    \textbf{forever}: \\
    {}\qquad \textbf{if} $s \models \gamma$: \\
    {}\qquad\qquad \textbf{return} $\relaxation{\pi}$ \\
    {}\qquad \textbf{else if} there is an operator $\relaxation{o} \in
    \relaxation{O}$ applicable in $s$ \\
    {}\qquad \phantom{\textbf{else if}} with
    $\applyop{\relaxation{o}}{s} \neq s$: \\
    {}\qquad\qquad Append such an operator $\relaxation{o}$ to $\relaxation{\pi}$. \\
    {}\qquad\qquad $s := \applyop{\relaxation{o}}{s}$ \\
    {}\qquad \textbf{else}: \\
    {}\qquad\qquad \textbf{return} unsolvable
  \end{block}
\end{frame}

\begin{frame}{Correctness of the greedy algorithm}
  The algorithm is \alert{sound}:
  \begin{itemize}
  \item If it returns a plan, this is indeed a correct solution.
  \item If it returns ``unsolvable'', the task is indeed unsolvable
    \begin{itemize}
    \item Upon termination, there clearly is no relaxed plan from
      $s$.
    \item By iterated application of the monotonicity lemma, $s$
      dominates $I$.
    \item By the relaxation lemma, there is no solution from $I$.
    \end{itemize}
  \end{itemize}
  
  \bigskip
  
  What about \alert{completeness} (termination) and \alert{runtime}?
  \begin{itemize}
  \item Each iteration of the loop adds at least one atom to
    $\onset{s}$.
  \item This guarantees termination after at most $|A|$ iterations.
  \item Thus, the algorithm can clearly be implemented to run in
    polynomial time.
    \begin{itemize}
    \item A good implementation runs in $O(\|\Pi\|)$.
    \end{itemize}
  \end{itemize}
\end{frame}

\subsection{Optimality}

\begin{frame}{Using the greedy algorithm as a heuristic}
  We can apply the greedy algorithm within heuristic search:
  \begin{itemize}
  \item In a search node $\sigma$, solve the relaxation of the
    planning task with $\textit{state}(\sigma)$ as the initial state.
  \item Set $h(\sigma)$ to the length of the generated relaxed plan.
  \end{itemize}
  
  \medskip

  Is this an \alert{admissible} heuristic?
  \begin{itemize}
  \item Yes if the relaxed plans are \alert{optimal} (due to the
    plan preservation corollary).
  \item However, usually they are not, because our greedy planning
    algorithm is very poor.
  \end{itemize}

  (What about safety? Goal-awareness? Consistency?)
\end{frame}

\begin{frame}{The set cover problem}
  To obtain an admissible heuristic, we need to generate optimal
  relaxed plans. Can we do this efficiently?

  \medskip

  This question is related to the following problem:
  \begin{problem}[set cover]
    \textblue{Given:} a finite set $U$, a collection of subsets
    $C = \{C_1, \dots, C_n\}$ with $C_i \subseteq U$ for all
    $i \in \{1, \dots, n\}$, and a natural number $K$.
    
    \smallskip

    \textblue{Question:} Does there exist a set cover of size at most
    $K$, \ie, a subcollection $S = \{S_1, \dots, S_m\} \subseteq C$
    with $S_1 \cup \dots \cup S_m = U$ and $m \leq K$?
  \end{problem}
  
  \medskip

  The following is a classical result from complexity theory:
  \begin{theorem}[Karp 1972]
    The set cover problem is NP-complete.
  \end{theorem}
\end{frame}

\begin{frame}{Hardness of optimal relaxed planning}
  \begin{theorem}[optimal relaxed planning is hard]
    The problem of deciding whether a given relaxed planning task
    has a plan of length at most $K$ is NP-complete.
  \end{theorem}

  \begin{proofstart}
    For \textblue{membership in NP}, guess a plan and verify.
    It is sufficient to check plans of length at most $|A|$, so this
    can be done in nondeterministic polynomial time.

    \medskip
    
    For \textblue{hardness}, we reduce from the set cover problem.
  \end{proofstart}
\end{frame}

\begin{frame}{Hardness of optimal relaxed planning (ctd.)}
  \begin{proofend}
    Given a set cover instance $\langle U, C, K\rangle$, we generate
    the following relaxed planning task
    $\relaxation{\Pi} = \langle A, I, \relaxation{O}, \gamma\rangle$:
    \begin{itemize}
    \item $A = U$
    \item $I = \{a \mapsto 0 \mid a \in A\}$
    \item $\relaxation{O} =
      \{\langle \top, \bigwedge_{a \in C_i} a \rangle
      \mid C_i \in C\}$
    \item $\gamma = \bigwedge_{a \in U} a$
    \end{itemize}

    If $S$ is a set cover, the corresponding operators form a plan.
    Conversely, each plan induces a set cover by taking the subsets
    corresponding to the operators. Clearly, there exists a plan of
    length at most $K$ iff there exists a set cover of size $K$.

    \smallskip

    Moreover, $\relaxation{\Pi}$ can be generated from the set cover
    instance in polynomial time, so this is a polynomial reduction.
  \end{proofend}
\end{frame}

\subsection{Discussion}

\begin{frame}{Using relaxations in practice}
  How can we use relaxations for heuristic planning in practice?

  \medskip

  Different possibilities:
  \begin{itemize}
  \item Implement an \alert{optimal planner} for relaxed planning
    tasks and use its solution lengths as an estimate, even though it
    is NP-hard.

    $\leadsto$ \alert{{\hplus} heuristic}
  \item Do not actually solve the relaxed planning task, but compute
    an estimate of its difficulty in a different way.

    $\leadsto$ \alert{{\hmax} heuristic}, \alert{{\hadd} heuristic},
    \alert{{\hlmcut} heuristic}
  \item Compute a solution for relaxed planning tasks which is not
    necessarily optimal, but ``reasonable''.

    $\leadsto$ \alert{{\hff} heuristic}
  \end{itemize}
\end{frame}

\section*{Summary}

\begin{frame}{Summary}
  \begin{itemize}
  \item Two general methods for coming up with heuristics:
    \begin{itemize}
    \item \alert{relaxation}: solve a \alert{less constrained}
      problem
    \item \alert{abstraction}: solve a \alert{small} problem
    \end{itemize}
  \item Here, we consider the \alert{delete relaxation}, which
    requires tasks in positive normal form and ignores delete
    effects.
  \item Delete-relaxed tasks have a \alert{domination} property: \\
    it is always beneficial to make more fluents true.
  \item They also have a \alert{monotonicity} property: \\
    applying operators leads to dominating states.
  \item Because of these two properties, \alert{finding some relaxed
    plan} greedily is \alert{easy} (polynomial).
  \item For an informative heuristic, we would ideally want to find \\
    \alert{optimal relaxed plans}. This is \alert{NP-complete}.
  \end{itemize}
\end{frame}

\end{document}
