\documentclass{gkibeamer}

%% NOTE: This is a much abbreviated version of the older (WS
%% 2008/2009) chapter on invariant synthesis, which had to be skipped
%% for time reasons. This is here mainly so that we can properly
%% introduce tasks in finite-domain representation, which are the
%% basis for the chapters on invariant synthesis.

\usepackage{tikz}[2008/02/13]

\title[AI Planning]{Principles of AI Planning}
\author[B.~Nebel,\\R.~Mattm\"uller]{Bernhard Nebel and Robert Mattm\"uller}
\date[]{Winter Semester 2011/2012}

%% Use \alert instead of \textred where it makes sense.
%% (\textred makes sense when you *really* want the color to be red,
%% independent of the beamer style settings.)
\newcommand{\hilite}[1]{{\color{structure} #1}}
\newcommand{\textred}[1]{{\color{red} #1}}
\newcommand{\textblue}[1]{{\color{blue} #1}}
\newcommand{\textgreen}[1]{{\color{green} #1}}
\newcommand{\textbrown}[1]{{\color{brown} #1}}
\newcommand{\textviolet}[1]{{\color{violet} #1}}
\newcommand{\textbg}[1]{{\color{bg} #1}}

\newcommand{\eg}{e.\,g.}
\newcommand{\ie}{i.\,e.}

%% State variables and values for state variables
\newcommand{\var}[1]{\textit{#1}}
\newcommand{\val}[1]{\text{#1}}

%% Various stuff
\newcommand{\astar}{A${}^*$}
\newcommand{\idastar}{IDA${}^*$}

%% Blocksworld stuff
\newcommand{\AONB}{\var{A-on-B}}
\newcommand{\AONC}{\var{A-on-C}}
\newcommand{\AOND}{\var{A-on-D}}
\newcommand{\BONA}{\var{B-on-A}}
\newcommand{\BONC}{\var{B-on-C}}
\newcommand{\BOND}{\var{B-on-D}}
\newcommand{\CONA}{\var{C-on-A}}
\newcommand{\CONB}{\var{C-on-B}}
\newcommand{\COND}{\var{C-on-D}}
\newcommand{\DONA}{\var{D-on-A}}
\newcommand{\DONB}{\var{D-on-B}}
\newcommand{\DONC}{\var{D-on-C}}
\newcommand{\AONTABLE}{\var{A-on-T}}
\newcommand{\BONTABLE}{\var{B-on-T}}
\newcommand{\CONTABLE}{\var{C-on-T}}
\newcommand{\CLEARA}{\var{A-clear}}
\newcommand{\CLEARB}{\var{B-clear}}
\newcommand{\CLEARC}{\var{C-clear}}

\newcommand{\applyop}[2]{\textit{app}_{#1}(#2)}
\newcommand{\applyops}[2]{\textit{app}_{#1}(#2)}
\newcommand{\applyplan}[2]{\textit{app}_{#1}(#2)}
\newcommand{\regr}[2]{\textit{regr}_{#1}(#2)}
\newcommand{\regrstrips}[2]{\textit{sregr}_{#1}(#2)}
\newcommand{\eprecon}[2]{\textit{EPC}_{#1}(#2)}

\newcommand{\compl}[1]{\overline{#1}}

\newcommand{\changes}[2]{\lbrack #1\rbrack_{#2}}
\newcommand{\CEF}{\vartriangleright}

\newcommand{\relaxation}[1]{{#1}^+}
\newcommand{\onset}[1]{\textit{on}(#1)}
\newcommand{\hplus}{\ensuremath{h^+}}
\newcommand{\hmax}{\ensuremath{h_{\text{max}}}}
\newcommand{\hadd}{\ensuremath{h_{\text{add}}}}
\newcommand{\hlmcut}{\ensuremath{h_{\text{LM-cut}}}}
\newcommand{\hff}{\ensuremath{h_{\text{FF}}}}
\newcommand{\hcs}{\ensuremath{h_{\text{cs}}}}
\newcommand{\hsa}{\ensuremath{h_{\text{sa}}}}
\newcommand{\hhhh}{\ensuremath{h_{\text{HHH}}}}

\newcommand{\sasplus}{SAS${}^+$}

\newcommand{\graphequiv}{\stackrel{\textup{G}}{\sim}}

\newcommand{\cg}{\ensuremath{\textit{CG}}}

%% Turing machines and complexity theory
\newcommand{\accept}{{\textsf{Y}}}

\newcommand{\easier}{\ensuremath{\le_{\text{p}}}}

\newcommand{\decisionclass}[1]{\ensuremath{\textsf{\textup{#1}}}}
\newcommand{\dtime}{\decisionclass{DTIME}}
\newcommand{\ntime}{\decisionclass{NTIME}}
\newcommand{\dspace}{\decisionclass{DSPACE}}
\newcommand{\nspace}{\decisionclass{NSPACE}}

\newcommand{\ptime}{\decisionclass{P}}
\newcommand{\np}{\decisionclass{NP}}
\newcommand{\pspace}{\decisionclass{PSPACE}}
\newcommand{\npspace}{\decisionclass{NPSPACE}}
\newcommand{\exptime}{\decisionclass{EXP}}
\newcommand{\nexptime}{\decisionclass{NEXP}}
\newcommand{\expspace}{\decisionclass{EXPSPACE}}
\newcommand{\dblexptime}{\decisionclass{2EXP}}

\newcommand{\planex}{\textsc{PlanEx}}
\newcommand{\planlen}{\textsc{PlanLen}}

\newcommand{\ms}[1]{\mathchoice{\mbox{\normalsize\it #1}}{\mbox{\normalsize\it #1}}{\mbox{\scriptsize\it #1}}{\mbox{\tiny\it #1}}}
\newcommand{\image}[2]{\ms{img}_{#1}(#2)}
\newcommand{\spreimage}[2]{\ms{spreimg}_{#1}(#2)}
\newcommand{\wpreimage}[2]{\ms{wpreimg}_{#1}(#2)}
\newcommand{\disbwd}[2]{\delta^{\ms{bwd}}_{#1}(#2)}
\newcommand{\ats}[1]{\ms{scope}(#1)}
\newcommand{\ndopcpc}[2]{\tau^{\ms{nd}}_{#1}(#2)}
\newcommand{\choice}{\mathop{|}}
\newcommand{\opcpc}[2]{\tau_{#1}(#2)}

%% Don't use \begin{proof}/\end{proof} for multi-part proofs because
%% this places a QED symbol.
\newenvironment{proofstart}{\begin{block}{Proof.}}{\end{block}}
\newenvironment{proofmid}{\begin{block}{Proof (ctd.)}}{\end{block}}
\newenvironment{proofend}{\begin{proof}[Proof (ctd.)]}{\end{proof}}

\newtheorem{remark}[theorem]{Remark}

%% Macros to align the width of things.
\newlength{\mywidth}
\newcommand{\setmywidth}[1]{\settowidth{\mywidth}{#1}}
\newcommand{\usemywidth}[1]{\makebox[\mywidth][l]{#1}}
\newcommand{\usemywidthmath}[1]{\usemywidth{\ensuremath{#1}}}

%% A tighter "align" environment.
\newenvironment{tightalign}[1][c]{\par\(\begin{array}[#1]{@{}r@{}l}}
  {\end{array}\)\par}
\newenvironment{wrappedmath}[1][t]{\begin{array}[#1]{@{}l}}{\end{array}}

%% Nondeterministic transition systems with connector arcs
\newcommand{\arc}[4]{
  \begin{pgfscope}
    \pgfsetlinewidth{1pt}
    \pgfpathmoveto{\pgfpointanchor{#3}{center}}
    \pgfpathlineto{\pgfpointanchor{#1}{#2}}
    \pgfpathlineto{\pgfpointanchor{#4}{center}}
    \pgfusepath{clip}
    \pgfpathcircle{\pgfpointanchor{#1}{#2}}{1.8mm}
    \pgfusepath{draw}
  \end{pgfscope}
}


\begin{document}
\lectureno{9}
\subtitle{Interlude: Finite-domain representation}
\date{December 6th, 2011}
\maketitles

\section{Invariants}

\subsection{Introduction}

\begin{frame}{Invariants}
  \begin{itemize}
  \item When we as humans reason about planning tasks, we implicitly
    make use of ``obvious'' properties of these tasks.
    \begin{itemize}
    \item \hilite{Example:} we are never in two places at the same
      time
    \end{itemize}
  \item We can express this as a logical formula $\varphi$ that is
    \\ \alert{true in all reachable states}.
    \begin{itemize}
    \item \hilite{Example:} $\varphi = \neg (\textit{at-uni} \land
      \textit{at-home})$
    \end{itemize}
  \item Such formulae are called \alert{invariants} of the task.
  \end{itemize}
\end{frame}

\subsection{Computing invariants}

\begin{frame}{Computing invariants}
  How does an \alert{automated} planner come up with invariants?
  \begin{itemize}
  \item Theoretically, testing if an arbitrary formula $\varphi$ \\
    is an invariant is \alert{as hard as planning} itself.
  \item Still, many practical invariant synthesis algorithms exist.
  \item To remain efficient (= polynomial-time), these algorithms only
    compute a \alert{subset} of all useful invariants.
  \item Empirically, they tend to at least find the ``obvious''
    invariants of a planning task.
  \end{itemize}
\end{frame}

\begin{frame}{Invariant synthesis algorithms}
  Most algorithms for generating invariants are based on a
  \alert{generate-test-repair} paradigm:
  \begin{itemize}
  \item \hilite{Generate:} Suggest some invariant candidates, \eg,
    by enumerating all possible formulas $\varphi$ of a certain
    size.
  \item \hilite{Test:} Try to prove that $\varphi$ is indeed an
    invariant. \\
    Usually done \alert{inductively}:
    \begin{enumerate}
    \item Test that \alert{initial state} satisfies $\varphi$.
    \item Test that if $\varphi$ is true in the current state,
      it remains true after applying a single operator.
    \end{enumerate}
  \item \hilite{Repair:} If invariant test fails, replace candidate
    $\varphi$ by a \alert{weaker} formula, ideally exploiting
    \alert{why} the proof failed.
  \end{itemize}
\end{frame}

\begin{frame}{Invariant synthesis: references}
  We discussed invariant synthesis in detail in previous courses on AI
  planning, but this year we will focus on other aspects of planning.% do not have enough time.

  \bigskip

  \hilite{Literature on invariant synthesis:}
  \begin{itemize}
  \item DISCOPLAN (Gerevini \& Schubert, 1998)
  \item TIM (Fox \& Long, 1998)
  \item Edelkamp \& Helmert's algorithm (1999)
  \item Rintanen's algorithm (2000)
  \item Bonet \& Geffner's algorithm (2001)
  \item Helmert's algorithm (2009)
  \end{itemize}
\end{frame}

\subsection{Exploiting variants}

\begin{frame}{Exploiting invariants}
  Invariants have many uses in planning:
  \begin{itemize}
  \item \hilite{Regression search:} \\
    \alert{Prune states} that violate (are inconsistent with)
    invariants.
  \item \hilite{Planning as satisfiability:} \\
    \alert{Add invariants} to a SAT encoding of a planning task to get
    tighter constraints.
  \item \hilite{Reformulation:} \\
    Derive a \alert{more compact} state space representation \\
    (\ie, with lower percentage of unreachable states).
  \end{itemize}

  \bigskip

  We now briefly discuss the last point, since it leads to
  \alert{planning tasks in finite-domain representation}, which are
  very important for the next chapters.
\end{frame}

\section[FDR planning tasks]{Planning tasks in finite-domain representation}

\subsection{Mutexes}

\begin{frame}{Mutexes}
  Invariants that take the form of \alert{binary clauses} are called
  \alert{mutexes} because they state that certain variable assignments
  cannot be simultaneously true and are hence \alert{mutually
    exclusive}.
  
  \begin{example}[Blocksworld]
    The invariant $\neg \AONB \lor \neg \AONC$ states that $\AONB$ and
    $\AONC$ are mutex.
  \end{example}

  Often, a larger \alert{set of literals} is mutually exclusive
  because every pair of them forms a mutex.

  \begin{example}[Blocksworld]
    Every pair in $\{\BONA, \CONA, \DONA, \CLEARA\}$ is mutex.
  \end{example}
\end{frame}

\begin{frame}{Encoding mutex groups as finite-domain variables}
  Let $L = \{l_1, \dots, l_n\}$ be mutually exclusive literals over
  $n$ different variables $A_L = \{a_1, \dots, a_n\}$.

  \smallskip
    
  Then the planning task can be rephrased using a single
  \alert{finite-domain} (i.e., non-binary) state variable $v_L$ with
  $n + 1$ possible values in place of the $n$ variables in $A_L$:
  \begin{itemize}
  \item $n$ of the possible values represent situations in which
    \alert{exactly one} of the literals in $L$ is true.
  \item The remaining value represents situations in which
    \alert{none of the literals} in $L$ is true.
    \begin{itemize}
    \item \hilite{Note:} If we can prove that one of the literals in
      $L$ has to be true in each state, this additional value can be
      omitted.
    \end{itemize}
  \end{itemize}

  In many cases, the reduction in the number of variables can
  dramatically improve performance of a planning algorithm.
\end{frame}

\subsection{FDR planning tasks}

\begin{frame}{Finite-domain state variables}
  \begin{definition}[finite-domain state variable]
    A \alert{finite-domain state variable} is a symbol $v$ with an
    associated \alert{finite domain}, \ie, a non-empty finite set.

    \smallskip

    We write $\mathcal D_v$ for the domain of $v$.
  \end{definition}

  \begin{example}
    $v = \var{above-a}$, $\mathcal D_{\var{above-a}} =
    \{\val{b}, \val{c}, \val{d}, \val{nothing}\}$

    \smallskip

    This state variable encodes the same information as the
    propositional variables \BONA, \CONA, \DONA\ and \CLEARA.
  \end{example}
\end{frame}

\begin{frame}{Finite-domain states}
  \begin{definition}[finite-domain state]
    Let $V$ be a finite set of finite-domain state variables.

    \smallskip
    
    A \alert{state} over $V$ is an assignment $s: V \rightarrow
    \bigcup_{v \in V} \mathcal D_v$ such that $s(v) \in \mathcal D_v$
    for all $v \in V$.
  \end{definition}

  \begin{example}
    \begin{tightalign}
      s = \{
        &
        \var{above-a} \mapsto \val{nothing},
        \var{above-b} \mapsto \val{a},
        \var{above-c} \mapsto \val{b}, \\
        &
        \var{below-a} \mapsto \val{b},
        \var{below-b} \mapsto \val{c},
        \var{below-c} \mapsto \val{table}
        \}
    \end{tightalign}
  \end{example}
\end{frame}

\begin{frame}{Finite-domain formulae}
  \begin{definition}[finite-domain formulae]
    \alert{Logical formulae over finite-domain state variables} $V$
    are defined as in the propositional case, except that
    instead of atomic formulae of the form $a \in A$,
    there are atomic formulae of the form $v = d$, where $v \in V$
    and $d \in \mathcal D_v$.
  \end{definition}

  \begin{example}
    The formula $(\var{above-a} = \val{nothing}) \lor
    \neg (\var{below-b} = \val{c})$ corresponds to the formula
    $\CLEARA \lor \neg \BONC$.
  \end{example}
\end{frame}

\begin{frame}{Finite-domain effects}
  \begin{definition}[finite-domain effects]
    \alert{Effects over finite-domain state variables} $V$ are defined
    as in the propositional case, except that instead of atomic
    effects of the form $a$ and $\neg a$ with $a \in A$, there are
    atomic effects of the form $v := d$, where $v \in V$ and $d \in
    \mathcal D_v$.
  \end{definition}

  \begin{example}
    The effect $(\var{below-a} := \val{table}) \land
    ((\var{above-b} = \val{a}) \CEF
    (\var{above-b} := \val{nothing}))$
    corresponds to the effect
    $\AONTABLE \land \neg \AONB \land \neg \AONC \land \neg \AOND
    \land (\AONB \CEF \CLEARB)$.
  \end{example}

  $\leadsto$ definition of \alert{finite-domain operators} follows
  from this
\end{frame}

\begin{frame}{Planning tasks in finite-domain representation}
  \begin{definition}[planning task in finite-domain representation]
    A \alert{deterministic planning task in finite-domain
      representation} or \alert{FDR planning task} is a 4-tuple
    $\Pi = \langle V,I,O,\gamma\rangle$ where
    \begin{itemize}
    \item $V$ is a finite set of \alert{finite-domain state
      variables},
    \item $I$ is an \alert{initial state} over $V$,
    \item $O$ is a finite set of \alert{finite-domain operators} over
      $V$, and
    \item $\gamma$ is a formula over $V$ describing the \alert{goal
      states}.
    \end{itemize}
  \end{definition}
\end{frame}

\subsection{Relationship to propositional planning tasks}

\begin{frame}{Relationship to propositional planning tasks}
  \begin{definition}[induced propositional planning task]
    Let $\Pi = \langle V,I,O,\gamma\rangle$ be an FDR planning task.

    The \alert{induced propositional planning task} $\Pi'$ is
    the (regular) planning task $\Pi' = \langle A', I', O',
    \gamma'\rangle$, where
    \begin{itemize}
    \item $A' = \{ (v,d) \mid v \in V, d \in \mathcal D_v \}$
    \item $I'((v,d)) = 1$ iff $I(v) = d$
    \item $O'$ and $\gamma'$ are obtained from $O$ and $\gamma$ by replacing
      \begin{itemize}
      \item each atomic formula $v = d$ with the proposition $(v,d)$,
        and
      \item each atomic effect $v := d$ with the effect $(v,d) \land 
        \bigwedge_{d' \in \mathcal D_v \setminus \{d\}} \neg (v, d')$.
      \end{itemize}
    \end{itemize}
  \end{definition}
  \begin{itemize}
  \item $\leadsto$ can define operator semantics, plans, relaxed
    planning graphs, \dots\ for $\Pi$ in terms of its induced
    propositional planning task
  \end{itemize}
\end{frame}

\subsection{\sasplus\ planning tasks}

\begin{frame}{\sasplus\ planning tasks}
  \begin{definition}[\sasplus\ planning task]
    An FDR planning task $\Pi = \langle V, I, O, \gamma\rangle$ is called
    an \alert{\sasplus\ planning task} iff there are no conditional
    effects in $O$ and all operator preconditions in $O$ and the goal
    formula $\gamma$ are conjunctions of atoms.
  \end{definition}
  \begin{itemize}
  \item analogue of STRIPS planning tasks for finite-domain
    representations
  \item induced propositional planning task of a \sasplus\ planning
    task is STRIPS
  \item FDR tasks obtained by invariant-based reformulation of
    STRIPS planning task are \sasplus
  \end{itemize}
\end{frame}

\section*{Summary}

\begin{frame}{Summary}
  \begin{itemize}
  \item \alert{Invariants} are common properties of all reachable
    states, expressed as logical formulas.
  \item A number of algorithms for \alert{computing invariants} exist.
  \item These algorithms will not find \alert{all useful invariants}
    (which is too hard), but try to find some useful subset within
    reasonable (polynomial) time.
  \item \alert{Mutexes} are invariants that express that certain pairs
    of state variable assignments are mutually exclusive.
  \item Groups of mutexes can be used for \alert{problem
    reformulation}, transforming a planning task into
    \alert{finite-domain representation (FDR)}.
  \item Many planning algorithms are more efficient when working on
    these FDR tasks (rather than the original tasks) because they
    contain \alert{fewer unreachable states}.
  \end{itemize}
\end{frame}

\end{document}
