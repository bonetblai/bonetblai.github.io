\documentclass{gkibeamer}

\usepackage{tikz}[2008/02/13]

\title[AI Planning]{Principles of AI Planning}
\author[B.~Nebel,\\R.~Mattm\"uller]{Bernhard Nebel and Robert Mattm\"uller}
\date[]{Winter Semester 2011/2012}

%% Use \alert instead of \textred where it makes sense.
%% (\textred makes sense when you *really* want the color to be red,
%% independent of the beamer style settings.)
\newcommand{\hilite}[1]{{\color{structure} #1}}
\newcommand{\textred}[1]{{\color{red} #1}}
\newcommand{\textblue}[1]{{\color{blue} #1}}
\newcommand{\textgreen}[1]{{\color{green} #1}}
\newcommand{\textbrown}[1]{{\color{brown} #1}}
\newcommand{\textviolet}[1]{{\color{violet} #1}}
\newcommand{\textbg}[1]{{\color{bg} #1}}

\newcommand{\eg}{e.\,g.}
\newcommand{\ie}{i.\,e.}

%% State variables and values for state variables
\newcommand{\var}[1]{\textit{#1}}
\newcommand{\val}[1]{\text{#1}}

%% Various stuff
\newcommand{\astar}{A${}^*$}
\newcommand{\idastar}{IDA${}^*$}

%% Blocksworld stuff
\newcommand{\AONB}{\var{A-on-B}}
\newcommand{\AONC}{\var{A-on-C}}
\newcommand{\AOND}{\var{A-on-D}}
\newcommand{\BONA}{\var{B-on-A}}
\newcommand{\BONC}{\var{B-on-C}}
\newcommand{\BOND}{\var{B-on-D}}
\newcommand{\CONA}{\var{C-on-A}}
\newcommand{\CONB}{\var{C-on-B}}
\newcommand{\COND}{\var{C-on-D}}
\newcommand{\DONA}{\var{D-on-A}}
\newcommand{\DONB}{\var{D-on-B}}
\newcommand{\DONC}{\var{D-on-C}}
\newcommand{\AONTABLE}{\var{A-on-T}}
\newcommand{\BONTABLE}{\var{B-on-T}}
\newcommand{\CONTABLE}{\var{C-on-T}}
\newcommand{\CLEARA}{\var{A-clear}}
\newcommand{\CLEARB}{\var{B-clear}}
\newcommand{\CLEARC}{\var{C-clear}}

\newcommand{\applyop}[2]{\textit{app}_{#1}(#2)}
\newcommand{\applyops}[2]{\textit{app}_{#1}(#2)}
\newcommand{\applyplan}[2]{\textit{app}_{#1}(#2)}
\newcommand{\regr}[2]{\textit{regr}_{#1}(#2)}
\newcommand{\regrstrips}[2]{\textit{sregr}_{#1}(#2)}
\newcommand{\eprecon}[2]{\textit{EPC}_{#1}(#2)}

\newcommand{\compl}[1]{\overline{#1}}

\newcommand{\changes}[2]{\lbrack #1\rbrack_{#2}}
\newcommand{\CEF}{\vartriangleright}

\newcommand{\relaxation}[1]{{#1}^+}
\newcommand{\onset}[1]{\textit{on}(#1)}
\newcommand{\hplus}{\ensuremath{h^+}}
\newcommand{\hmax}{\ensuremath{h_{\text{max}}}}
\newcommand{\hadd}{\ensuremath{h_{\text{add}}}}
\newcommand{\hlmcut}{\ensuremath{h_{\text{LM-cut}}}}
\newcommand{\hff}{\ensuremath{h_{\text{FF}}}}
\newcommand{\hcs}{\ensuremath{h_{\text{cs}}}}
\newcommand{\hsa}{\ensuremath{h_{\text{sa}}}}
\newcommand{\hhhh}{\ensuremath{h_{\text{HHH}}}}

\newcommand{\sasplus}{SAS${}^+$}

\newcommand{\graphequiv}{\stackrel{\textup{G}}{\sim}}

\newcommand{\cg}{\ensuremath{\textit{CG}}}

%% Turing machines and complexity theory
\newcommand{\accept}{{\textsf{Y}}}

\newcommand{\easier}{\ensuremath{\le_{\text{p}}}}

\newcommand{\decisionclass}[1]{\ensuremath{\textsf{\textup{#1}}}}
\newcommand{\dtime}{\decisionclass{DTIME}}
\newcommand{\ntime}{\decisionclass{NTIME}}
\newcommand{\dspace}{\decisionclass{DSPACE}}
\newcommand{\nspace}{\decisionclass{NSPACE}}

\newcommand{\ptime}{\decisionclass{P}}
\newcommand{\np}{\decisionclass{NP}}
\newcommand{\pspace}{\decisionclass{PSPACE}}
\newcommand{\npspace}{\decisionclass{NPSPACE}}
\newcommand{\exptime}{\decisionclass{EXP}}
\newcommand{\nexptime}{\decisionclass{NEXP}}
\newcommand{\expspace}{\decisionclass{EXPSPACE}}
\newcommand{\dblexptime}{\decisionclass{2EXP}}

\newcommand{\planex}{\textsc{PlanEx}}
\newcommand{\planlen}{\textsc{PlanLen}}

\newcommand{\ms}[1]{\mathchoice{\mbox{\normalsize\it #1}}{\mbox{\normalsize\it #1}}{\mbox{\scriptsize\it #1}}{\mbox{\tiny\it #1}}}
\newcommand{\image}[2]{\ms{img}_{#1}(#2)}
\newcommand{\spreimage}[2]{\ms{spreimg}_{#1}(#2)}
\newcommand{\wpreimage}[2]{\ms{wpreimg}_{#1}(#2)}
\newcommand{\disbwd}[2]{\delta^{\ms{bwd}}_{#1}(#2)}
\newcommand{\ats}[1]{\ms{scope}(#1)}
\newcommand{\ndopcpc}[2]{\tau^{\ms{nd}}_{#1}(#2)}
\newcommand{\choice}{\mathop{|}}
\newcommand{\opcpc}[2]{\tau_{#1}(#2)}

%% Don't use \begin{proof}/\end{proof} for multi-part proofs because
%% this places a QED symbol.
\newenvironment{proofstart}{\begin{block}{Proof.}}{\end{block}}
\newenvironment{proofmid}{\begin{block}{Proof (ctd.)}}{\end{block}}
\newenvironment{proofend}{\begin{proof}[Proof (ctd.)]}{\end{proof}}

\newtheorem{remark}[theorem]{Remark}

%% Macros to align the width of things.
\newlength{\mywidth}
\newcommand{\setmywidth}[1]{\settowidth{\mywidth}{#1}}
\newcommand{\usemywidth}[1]{\makebox[\mywidth][l]{#1}}
\newcommand{\usemywidthmath}[1]{\usemywidth{\ensuremath{#1}}}

%% A tighter "align" environment.
\newenvironment{tightalign}[1][c]{\par\(\begin{array}[#1]{@{}r@{}l}}
  {\end{array}\)\par}
\newenvironment{wrappedmath}[1][t]{\begin{array}[#1]{@{}l}}{\end{array}}

%% Nondeterministic transition systems with connector arcs
\newcommand{\arc}[4]{
  \begin{pgfscope}
    \pgfsetlinewidth{1pt}
    \pgfpathmoveto{\pgfpointanchor{#3}{center}}
    \pgfpathlineto{\pgfpointanchor{#1}{#2}}
    \pgfpathlineto{\pgfpointanchor{#4}{center}}
    \pgfusepath{clip}
    \pgfpathcircle{\pgfpointanchor{#1}{#2}}{1.8mm}
    \pgfusepath{draw}
  \end{pgfscope}
}

\input{blocksworld}

\begin{document}

\lectureno{2}
\subtitle{Transition systems and planning tasks}
\date{October 25th, 2011}
\maketitles

\section{Transition systems}
\subsection{Definition}

\begin{frame}{Transition systems}
  \begin{definition}[transition system]
    A \alert{transition system} is a 5-tuple $\mathcal T = \langle S,
    L, T, s_0, S_\star\rangle$ where
    \begin{itemize}
    \item $S$ is a finite set of \alert{states},
    \item $L$ is a finite set of (transition) \alert{labels},
    \item $T \subseteq S \times L \times S$ is the \alert{transition
      relation},
    \item $s_0 \in S$ is the \alert{initial state}, and
    \item $S_\star \subseteq S$ is the set of \alert{goal states}.
    \end{itemize}
    We say that $\mathcal T$ \alert{has the transition} $\langle s, \ell,
    s'\rangle$ if $\langle s, \ell, s'\rangle \in T$.

    We also write this \alert{$s \xrightarrow{\ell} s'$}, or \alert{$s
      \rightarrow s'$} when not interested in $\ell$.
  \end{definition}

  \hilite{Note:}
  Transition systems are also called \alert{state spaces}.
\end{frame}

\begin{frame}{Transition systems: example}
  Transition systems are often depicted as \alert{directed arc-labeled
    graphs} with marks to indicate the initial state and goal states.
  \begin{center}
    \usetikzlibrary{shapes}

\begin{tikzpicture}
\node (A) at (0:25mm) {A};
\node (B) at (60:25mm) {B};
\node (C) at (120:25mm) {C};
\node (D) at (180:25mm) {D};
\node (E) at (240:25mm) {E};
\node (F) at (300:25mm) {F};

\tikzstyle action edge=[->, thick, >=latex]

\begin{scope}[style=action edge, red]
\draw[bend right=10] (A) to (B);
\draw (B) to (F);
\draw[bend right=10] (C) to (D);
\draw (E) to (B);
\draw[bend left=10] (F) to (E);
\end{scope}

\begin{scope}[style=action edge, blue]
\draw[bend left=10] (C) to (B);
\draw[bend right=10] (D) to (E);
\draw[bend right=10] (F) to (A);
\end{scope}

\begin{scope}[style=action edge, green]
\draw (C) to (F);
\draw (D) to (A);
\draw[bend right=10] (E) to (D);
\draw[bend left=10] (F) to (A);
\end{scope}

\node[draw, circle, dashed, minimum size = 7mm] (init) at (F) [label=below:initial state] {};
\node[draw, ellipse, dashed, minimum height=8mm, minimum width=35mm] (goal) at (0,2.18) [label=above:goal states] {};

\end{tikzpicture}

  \end{center}
\end{frame}

\begin{frame}{Transition system terminology}
  We use common graph theory terms for transition systems:
  \begin{itemize}
  \item $s'$ \alert{successor} of $s$ if $s \rightarrow s'$
  \item $s$ \alert{predecessor} of $s'$ if $s \rightarrow s'$
  \item $s'$ \alert{reachable} from $s$ if there exists a sequence of
    transitions \\
    $s^{0} \xrightarrow{\ell_1} s^{1}$, \dots,
    $s^{n-1} \xrightarrow{\ell_n} s^{n}$
    s.t.\ $s^{0} = s$ and $s^{n} = s'$%
    \begin{itemize}
    \item \hilite{Note:} $n = 0$ possible; then $s = s'$
    \item $s^{0} \xrightarrow{\ell_1} s^{1}$, \dots, $s^{n-1}
      \xrightarrow{\ell_n} s^{n}$ is called \alert{path} from $s$ to
      $s'$
    %\item $\ell_1, \dots, \ell_n$ is called \alert{path} from
    %  $s$ to $s'$
    \item $s^{0}, \dots, s^{n}$ is also called \alert{path} from
      $s$ to $s'$
    \item \alert{length} of that path is $n$
    \end{itemize}
  \item additional terms: \alert{strongly connected}, \alert{weakly
    connected}, \alert{strong/weak connected components}, \dots
  \end{itemize}
\end{frame}

\begin{frame}{Transition system terminology (ctd.)}
  Some additional terminology:
  \begin{itemize}
  \item $s'$ \alert{reachable} (without reference state) means \\
    reachable from initial state $s_0$
  \item \alert{solution} or \alert{goal path} from $s$:
    path from $s$ to some $s' \in S_\star$
    \begin{itemize}
    \item if $s$ is omitted, $s = s_0$ is implied
    \end{itemize}
  \item transition system \alert{solvable} if a goal path from $s_0$
    exists
  \end{itemize}
\end{frame}

\begin{frame}{Deterministic transition systems}
  \begin{definition}[deterministic transition system]
    A transition system with transitions $T$ is called
    \alert{deterministic} \\
    if for all states $s$ and labels $\ell$, there is \alert{at most one}
    state $s'$ \\
    with $s \xrightarrow{\ell} s'$.
  \end{definition}

  \hilite{Example:} previously shown transition system
\end{frame}

\subsection{Blocks world}

\begin{frame}{Running example: blocks world}
  \begin{itemize}
  \item Throughout the course, we will often use the \alert{blocks
    world} domain as an example.
  \item In the blocks world, a number of differently coloured blocks
    are arranged on our table.
  \item Our job is to rearrange them according to a given goal.
  \end{itemize}
\end{frame}

\begin{frame}{Blocks world rules}
  \begin{block}{Location on the table does not matter.}
    \begin{center}
      \begin{pgfpicture}{-5mm}{0mm}{55mm}{12mm}
        \pgfputat{\pgfxy(2,0.5)}{\pgfbox[center,center]{\Large $\equiv$}}

        \pgfsetxvec{\pgfpoint{0.5cm}{0cm}}
        \pgfsetyvec{\pgfpoint{0cm}{0.5cm}}
        \pgfsetzvec{\pgfpoint{0.15cm}{0.15cm}}
        
        \drawblock{red}{0}{0}{0}{1}
        \drawblock{green}{0}{1}{0}{1}
        \drawblock{blue}{1}{0}{0}{1}
        
        \drawblock{red}{6}{0}{0}{1}
        \drawblock{green}{6}{1}{0}{1}
        \drawblock{blue}{8}{0}{0}{1}
      \end{pgfpicture}
    \end{center}
  \end{block}

  \begin{block}{Location on a block does not matter.}
    \begin{center}
      \begin{pgfpicture}{-5mm}{0mm}{55mm}{17mm}
        \pgfputat{\pgfxy(2,0.5)}{\pgfbox[center,center]{\Large $\equiv$}}
        
        \pgfsetxvec{\pgfpoint{0.5cm}{0cm}}
        \pgfsetyvec{\pgfpoint{0cm}{0.5cm}}
        \pgfsetzvec{\pgfpoint{0.15cm}{0.15cm}}
        
        \drawblock{red}{1}{0}{0}{1}
        \drawblock{green}{1}{1}{0}{1}
        \drawblock{blue}{1}{2}{0}{1}

        \drawblock{red}{6}{0}{0}{1}
        \drawblock{green}{6.4}{1}{0}{1}
        \drawblock{blue}{6}{2}{0}{1}
      \end{pgfpicture}
    \end{center}
  \end{block}
\end{frame}

\begin{frame}{Blocks world rules (ctd.)}
  \begin{block}{At most one block may be below a block.}
    \begin{center}
      \begin{pgfpicture}{-2mm}{-3mm}{17mm}{16mm}
        \pgfsetxvec{\pgfpoint{0.5cm}{0cm}}
        \pgfsetyvec{\pgfpoint{0cm}{0.5cm}}
        \pgfsetzvec{\pgfpoint{0.15cm}{0.15cm}}
        
        % only one block below another block
        \drawblock{green}{0.5}{0}{0}{1}
        \drawblock{blue}{1.5}{0}{0}{1}
        \drawblock{red}{1}{1}{0}{1}
        
        \pgfsetxvec{\pgfpoint{1mm}{0cm}}
        \pgfsetyvec{\pgfpoint{0cm}{1mm}}
        
        \pgfsetlinewidth{2pt}
        \color{orange}
        \pgfline{\pgfxy(-1,-2)}{\pgfxy(16,15)}
        \pgfline{\pgfxy(-1,15)}{\pgfxy(16,-2)}
      \end{pgfpicture}
    \end{center}
  \end{block}

  \begin{block}{At most one block may be on top of a block.}
    \begin{center}
      \begin{pgfpicture}{-2mm}{-3mm}{17mm}{16mm}
        \pgfsetxvec{\pgfpoint{0.5cm}{0cm}}
        \pgfsetyvec{\pgfpoint{0cm}{0.5cm}}
        \pgfsetzvec{\pgfpoint{0.15cm}{0.15cm}}
        
        % only one block on top of another block
        \drawblock{red}{1}{0}{0}{1}
        \drawblock{green}{0.5}{1}{0}{1}
        \drawblock{blue}{1.5}{1}{0}{1}
        
        \pgfsetxvec{\pgfpoint{1mm}{0cm}}
        \pgfsetyvec{\pgfpoint{0cm}{1mm}}
        
        \pgfsetlinewidth{2pt}
        \color{orange}
        \pgfline{\pgfxy(-1,-2)}{\pgfxy(16,15)}
        \pgfline{\pgfxy(-1,15)}{\pgfxy(16,-2)}
      \end{pgfpicture}
    \end{center}
  \end{block}
\end{frame}

\begin{frame}{Blocks world transition system for three blocks}
  (Transition labels omitted for clarity.)
  \begin{center}
    \begin{pgfpicture}{-39mm}{-34.4mm}{39mm}{34.4mm}
      \pgfnodebox{RsGsB}[virtual]{\pgfpolar{0}{0cm}}{\RsGsB}{2pt}{2pt}
      
      \pgfnodebox{RGsB}[virtual]{\pgfpolar{0}{17mm}}{\RGsB}{2pt}{2pt}
      \pgfnodebox{RBsG}[virtual]{\pgfpolar{60}{17mm}}{\RBsG}{2pt}{2pt}
      \pgfnodebox{BRsG}[virtual]{\pgfpolar{120}{17mm}}{\BRsG}{2pt}{2pt}
      \pgfnodebox{RsBG}[virtual]{\pgfpolar{180}{17mm}}{\RsBG}{2pt}{2pt}
      \pgfnodebox{RsGB}[virtual]{\pgfpolar{240}{17mm}}{\RsGB}{2pt}{2pt}
      \pgfnodebox{GRsB}[virtual]{\pgfpolar{300}{17mm}}{\GRsB}{2pt}{2pt}
      
      \pgfnodebox{BRG}[virtual]{\pgfpolar{0}{34mm}}{\BRG}{2pt}{2pt}
      \pgfnodebox{GRB}[virtual]{\pgfpolar{50}{34mm}}{\GRB}{2pt}{2pt}
      \pgfnodebox{GBR}[virtual]{\pgfpolar{130}{34mm}}{\GBR}{2pt}{2pt}
      \pgfnodebox{RBG}[virtual]{\pgfpolar{180}{34mm}}{\RBG}{2pt}{2pt}
      \pgfnodebox{RGB}[virtual]{\pgfpolar{230}{34mm}}{\RGB}{2pt}{2pt}
      \pgfnodebox{BGR}[virtual]{\pgfpolar{310}{34mm}}{\BGR}{2pt}{2pt}

      \pgfsetendarrow{\pgfarrowtriangle{5pt}}
      \pgfsetstartarrow{\pgfarrowtriangle{5pt}}
      
      \pgfnodeconnline{RsGsB}{RGsB}
      \pgfnodeconnline{RsGsB}{RBsG}
      \pgfnodeconnline{RsGsB}{BRsG}
      \pgfnodeconnline{RsGsB}{RsBG}
      \pgfnodeconnline{RsGsB}{RsGB}
      \pgfnodeconnline{RsGsB}{GRsB}

      \pgfnodeconnline{RGsB}{BRG}
      \pgfnodeconnline{RBsG}{GRB}
      \pgfnodeconnline{BRsG}{GBR}
      \pgfnodeconnline{RsBG}{RBG}
      \pgfnodeconnline{RsGB}{RGB}
      \pgfnodeconnline{GRsB}{BGR}

      \pgfnodeconnline{RGsB}{RBsG}
      \pgfnodeconnline{BRsG}{RsBG}
      \pgfnodeconnline{RsGB}{GRsB}
    \end{pgfpicture}
  \end{center}
\end{frame}

\begin{frame}{Blocks world computational properties}
  \begin{small}
    \begin{center}
      \begin{tabular}{rr}
        blocks & states \\ \hline
        1 & 1 \\
        2 & 3 \\
        3 & 13 \\
        4 & 73 \\
        5 & 501 \\
        6 & 4051 \\
        7 & 37633 \\
        8 & 394353 \\
        9 & 4596553
      \end{tabular}\qquad
      \begin{tabular}{rr}
        blocks & states \\ \hline
        10 & 58941091 \\
        11 & 824073141 \\
        12 & 12470162233 \\
        13 & 202976401213 \\
        14 & 3535017524403 \\
        15 & 65573803186921 \\
        16 & 1290434218669921 \\
        17 & 26846616451246353 \\
        18 & 588633468315403843
      \end{tabular}
    \end{center}
  \end{small}

  \begin{itemize}
  \item \alert{Finding a solution} is \alert{polynomial time} in the
    number of blocks (move everything onto the table and then
    construct the goal configuration).
  \item Finding a \alert{shortest solution} is \alert{NP-complete}
    (for a compact description of the problem).
  \end{itemize}
\end{frame}

\section{Planning tasks}
\subsection{State variables}

\begin{frame}{Compact representations}
  \begin{itemize}
  \item Classical (\ie, deterministic) planning is in essence the
    problem of finding solutions in \alert{huge} transition systems.
  \item The transition systems we are usually interested in are \\
    too large to explicitly enumerate all states or transitions.
  \item Hence, the input to a planning algorithm must be given \\ in a
    more \alert{concise} form.
  \item In the rest of chapter, we discuss how to represent planning
    tasks in a suitable way.
  \end{itemize}
\end{frame}

\begin{frame}{State variables}
  How to represent huge state sets without enumerating them?
  \begin{itemize}
  \item represent different aspects of the world in terms of different
    \alert{state variables}
  \item[$\leadsto$] a state is a \alert{valuation of state variables}
  \item $n$ state variables with $m$ possible values each \\ induce
    $m^n$ different states
  \item[$\leadsto$] \alert{exponentially more compact} than ``flat''
    representations
  \item \hilite{Example:} $n$ variables suffice for blocks world with
    $n$ blocks
  \end{itemize}
\end{frame}

\begin{frame}{Blocks world with finite-domain state variables}
  Describe blocks world state with three state variables:

  \begin{itemize}
  \item \var{location-of-A}: $\{\val{B},\val{C},\val{table}\}$
  \item \var{location-of-B}: $\{\val{A},\val{C},\val{table}\}$
  \item\var{location-of-C}: $\{\val{A},\val{B},\val{table}\}$
  \end{itemize}

  \begin{example}
    \begin{minipage}[c]{6cm}
      \begin{align*}
        s(\var{location-of-A}) & = \val{table} \\
        s(\var{location-of-B}) & = \val{A} \\
        s(\var{location-of-C}) & = \val{table}
      \end{align*}
    \end{minipage}
    \begin{minipage}[c]{2.5cm}
      \begin{pgfpicture}{0mm}{0mm}{20mm}{20mm}
        \pgfsetxvec{\pgfpoint{0.8cm}{0cm}}
        \pgfsetyvec{\pgfpoint{0cm}{0.8cm}}
        \pgfsetzvec{\pgfpoint{0.3cm}{0.3cm}}
        
        \drawblock{red}{0}{0}{0}{1}
        \drawblock{blue}{0}{1}{0}{1}
        \drawblock{green}{1}{0}{0}{1}
        
        \pgfputat{\pgfxyz(0.2,0.2,0)}{\pgfbox[left,left]{\huge A}}
        \pgfputat{\pgfxyz(0.2,1.2,0)}{\pgfbox[left,left]{\huge B}}
        \pgfputat{\pgfxyz(1.2,0.2,0)}{\pgfbox[left,left]{\huge C}}
      \end{pgfpicture}
    \end{minipage}
  \end{example}

  Not all valuations correspond to intended blocks world states.

  \hilite{Example:} $s$ with $s(\var{location-of-A}) = \val{B}$,
  $s(\var{location-of-B}) = \val{A}$.
\end{frame}

\begin{frame}{Boolean state variables}
  \hilite{Problem:}
  \begin{itemize}
  \item How to \alert{succinctly} represent \alert{transitions} and
    \alert{goal states}?
  \end{itemize}
  
  \bigskip

  \hilite{Idea:} Use \alert{propositional logic}
  \begin{itemize}
  \item \hilite{state variables:} propositional variables ($0$ or $1$)
  \item \hilite{goal states:} defined by a propositional formula
  \item \hilite{transitions:} defined by \alert{actions} given by
    \begin{itemize}
    \item \alert{precondition}: when is the action applicable?
    \item \alert{effect}: how does it change the valuation?
    \end{itemize}
  \end{itemize}

  \bigskip

  \hilite{Note:} general finite-domain state variables can be
  compactly encoded as Boolean variables
\end{frame}

\begin{frame}{Blocks world with Boolean state variables}
  \begin{example}
    \begin{minipage}[c]{6cm}
      \begin{align*}
        s(\var{A-on-B}) & = 0 \\
        s(\var{A-on-C}) & = 0 \\
        s(\var{A-on-table}) & = 1 \\
        s(\var{B-on-A}) & = 1 \\
        s(\var{B-on-C}) & = 0 \\
        s(\var{B-on-table}) & = 0 \\
        s(\var{C-on-A}) & = 0 \\
        s(\var{C-on-B}) & = 0 \\
        s(\var{C-on-table}) & = 1
      \end{align*}
    \end{minipage}
    \begin{minipage}[c]{2.5cm}
    \begin{pgfpicture}{0mm}{0mm}{20mm}{20mm}
      \pgfsetxvec{\pgfpoint{0.8cm}{0cm}}
      \pgfsetyvec{\pgfpoint{0cm}{0.8cm}}
      \pgfsetzvec{\pgfpoint{0.3cm}{0.3cm}}
        
      \drawblock{red}{0}{0}{0}{1}
      \drawblock{blue}{0}{1}{0}{1}
      \drawblock{green}{1}{0}{0}{1}
      
      \pgfputat{\pgfxyz(0.2,0.2,0)}{\pgfbox[left,left]{\huge A}}
      \pgfputat{\pgfxyz(0.2,1.2,0)}{\pgfbox[left,left]{\huge B}}
      \pgfputat{\pgfxyz(1.2,0.2,0)}{\pgfbox[left,left]{\huge C}}
    \end{pgfpicture}
    \end{minipage}
  \end{example}
\end{frame}

\subsection[Logic]{Propositional logic}

\begin{frame}{Syntax of propositional logic}
  \begin{definition}[propositional formula]
    Let $A$ be a set of \alert{atomic propositions} (\hilite{here:}
    state variables).

    \smallskip

    The \alert{propositional formulae} over $A$ are constructed by
    finite application of the following rules:
    \begin{itemize}
    \item \alert{$\top$} and \alert{$\bot$} are propositional formulae
      (\alert{truth} and \alert{falsity}).
    \item For all $a \in A$, \alert{$a$} is a propositional formula
      (\alert{atom}).
    \item If $\varphi$ is a propositional formula, then so is
      \alert{$\neg\varphi$}
      (\alert{negation})
    \item If $\varphi$ and $\psi$ are propositional formulas, then so
      are \\
      \alert{$(\varphi \lor \psi)$} (\alert{disjunction}) and
      \alert{$(\varphi \land \psi)$} (\alert{conjunction}).
    \end{itemize}
  \end{definition}
  \hilite{Note:} We often omit the word ``propositional''.
\end{frame}

\begin{frame}{Propositional logic conventions}
  \hilite{Abbreviations:}
  \begin{itemize}
  \item \alert{$(\varphi \rightarrow \psi)$} is short for
    $(\neg \varphi \lor \psi)$ (\alert{implication})
  \item \alert{$(\varphi \leftrightarrow \psi)$} is short for
    $((\varphi \rightarrow \psi) \land (\psi \rightarrow \varphi))$
    (\alert{equivalence})
  \item parentheses omitted when not necessary
  \item $(\neg)$ binds more tightly than binary connectives
  \item $(\land)$ binds more tightly than $(\lor)$ than
    $(\rightarrow)$ than $(\leftrightarrow)$
  \end{itemize}
\end{frame}

\begin{frame}{Semantics of propositional logic}
  \begin{definition}[propositional valuation]
    A \alert{valuation} of propositions $A$ is a function
    $v: A \rightarrow \{0, 1\}$.

    \medskip
    
    Define the notation \alert{$v \models \varphi$} ($v$
    \alert{satisfies} $\varphi$; $v$ is a \alert{model} of $\varphi$;
    \\ $\varphi$ is \alert{true} under $v$) for valuations $v$ and
    formulae $\varphi$ by
    \begin{itemize}
    \item $v \models \top$
    \item $v \not\models \bot$
    \item $v \models a$ iff $v(a) = 1$, for $a \in A$.
    \item $v \models \neg \varphi$ iff $v \not\models \varphi$
    \item $v \models \varphi \lor \psi$ iff $v \models \varphi$ or
      $v \models \psi$ 
    \item $v \models \varphi \land \psi$ iff $v \models \varphi$
      and $v \models \psi$ 
    \end{itemize}
  \end{definition}
\end{frame}

\begin{frame}{Propositional logic terminology}
  \begin{itemize}
  \item A propositional formula $\varphi$ is \alert{satisfiable} \\
    if there is at least one valuation $v$ so that $v \models \varphi$.
  \item Otherwise it is \alert{unsatisfiable}.
  \item A propositional formula $\varphi$ is \alert{valid} or a
    \alert{tautology} \\
    if $v \models \varphi$ for all valuations $v$.
  \item A propositional formula $\psi$ is a \alert{logical
    consequence} of a propositional formula $\varphi$, written
    \alert{$\varphi \models \psi$}, if $v \models \psi$ for all
    valuations $v$ with $v \models \varphi$.
  \item Two propositional formulae $\varphi$ and $\psi$ are
    \alert{logically equivalent}, written \alert{$\varphi \equiv \psi$},
    if $\varphi \models \psi$ and $\psi \models \varphi$.
  \end{itemize}
  \hilite{Question:} How to phrase these in terms of \hilite{models}?
\end{frame}

\begin{frame}{Propositional logic terminology (ctd.)}
  \begin{itemize}
  \item A propositional formula that is a proposition $a$ or \\
    a negated proposition $\neg a$ for some $a \in A$ is a \alert{literal}.
  \item A formula that is a disjunction of literals is a
    \alert{clause}.

    This includes \alert{unit clauses} $l$ consisting of a single
    literal, \\
    and the \alert{empty clause} $\bot$ consisting of zero literals.
  \end{itemize}

  \alert{Normal forms:} NNF, CNF, DNF
\end{frame}

\subsection{Operators}

\begin{frame}{Operators}
  Transitions for state sets described by propositions $A$ can be
  concisely represented as \alert{operators} or \alert{actions}
  $\langle \chi, e\rangle$ where
  \begin{itemize}
  \item the \alert{precondition} $\chi$ is a propositional formula
    over $A$ describing the set of states in which the transition can
    be taken (states in which a transition starts), and
  \item the \alert{effect} $e$ describes how the resulting successor
    states are obtained from the state where the transitions is taken
    (where the transition goes).
  \end{itemize}
\end{frame}

\begin{frame}{Example: blocks world operators}
  \begin{exampleblock}{Blocks world operators}
    To model blocks world operators conveniently, we use auxiliary
    state variables $\CLEARA$, $\CLEARB$, and $\CLEARC$ to denote that
    there is nothing on top of a given block.

    \medskip

    Then blocks world operators can be modeled as:
    \small
    \begin{itemize}
    \item $\langle \CLEARA \land \AONTABLE \land \CLEARB,~ \AONB \land
      \neg \AONTABLE \land \neg \CLEARB \rangle$
    \item $\langle \CLEARA \land \AONTABLE \land\CLEARC,~ \AONC \land
      \neg \AONTABLE \land \neg \CLEARC \rangle$
    \item $\langle \CLEARA \land \AONB,~ \AONTABLE \land \neg \AONB
      \land \CLEARB \rangle$
    \item $\langle \CLEARA \land \AONC,~ \AONTABLE \land \neg \AONC
      \land \CLEARC \rangle$
    \item $\langle \CLEARA \land \AONB \land \CLEARC,~ \AONC \land
      \neg \AONB \land \CLEARB \land \neg \CLEARC\rangle$
    \item $\langle \CLEARA \land \AONC \land \CLEARB,~
      \AONB \land \neg \AONC \land \CLEARC \land \neg \CLEARB\rangle$
    \item \dots
    \end{itemize}
  \end{exampleblock}
\end{frame}

\begin{frame}{Effects (for deterministic operators)}
  \begin{definition}[effects]
    (Deterministic) \alert{effects} are recursively defined as follows:
    \begin{itemize}
    \item If $a \in A$ is a state variable, then \alert{$a$} and
      \alert{$\neg a$} are effects (\alert{atomic effect}).
    \item If $e_1, \dots, e_n$ are effects, then \alert{$e_1 \land
      \dots \land e_n$} is an effect (\alert{conjunctive effect}).

      The special case with $n=0$ is the empty effect $\top$.
    \item If $\chi$ is a propositional formula and $e$ is an effect, then
      \alert{$\chi \CEF e$} is an effect (\alert{conditional effect}).
    \end{itemize}
  \end{definition}
  Atomic effects $a$ and $\neg a$ are best understood
  as assignments $a := 1$ and $a := 0$, respectively.
\end{frame}

\begin{frame}{Effect example}
  $\chi \CEF e$ means that change $e$ takes place if $\chi$ is true in the
  current state.

  \begin{example}
    Increment 4-bit number $b_3b_2b_1b_0$ represented as four state
    variables $b_0$, \dots, $b_3$:
    \[
      \begin{array}{r@{\,}c@{\,}l}
        (\neg b_0 & \CEF
        & b_0) \land {} \\
        ((\neg b_1 \land b_0)
        & \CEF & (b_1 \land \neg b_0)) \land {} \\
        ((\neg b_2 \land b_1 \land b_0)
        & \CEF & (b_2 \land \neg b_1 \land \neg b_0)) \land {} \\
        ((\neg b_3 \land b_2 \land b_1 \land b_0)
        & \CEF & (b_3 \land \neg b_2 \land \neg b_1 \land \neg b_0))
      \end{array}
      \]
  \end{example}
\end{frame}

\begin{frame}{Operator semantics}
  \begin{definition}[changes caused by an operator]
    For each effect $e$ and state $s$, we define the \alert{change
      set} of $e$ in $s$, written \alert{$\changes{e}{s}$}, as the
    following set of literals:
    \begin{itemize}
    \item $\changes{a}{s} = \{a\}$ and $\changes{\neg a}{s} = \{\neg
        a\}$ for atomic effects $a$, $\neg a$
    \item $\changes{e_1 \land \dots \land e_n}{s} =
      \changes{e_1}{s} \cup \dots \cup \changes{e_n}{s}$
    \item $\changes{\chi \CEF e}{s} = \changes{e}{s}$ if $s \models \chi$ and
      $\changes{\chi\CEF e}{s} = \emptyset$ otherwise
    \end{itemize}
  \end{definition}
  %% TODO: Jan findet den Begriff "change set" nicht so gut, weil
  %% einige davon ja keine Änderungen sind, sondern schon gelten
  %% können. Also vielleicht ändern.

  \begin{definition}[applicable operators]
    Operator $\langle \chi,e\rangle$ is \alert{applicable in a state $s$}
    iff $s \models \chi$ and $\changes{e}{s}$ is consistent
    (\ie, does not contain two complementary literals).
  \end{definition}
\end{frame}

\begin{frame}{Operator semantics (ctd.)}
  \begin{definition}[successor state]
    The \alert{successor state} $\applyop{o}{s}$ of $s$ with respect
    to operator $o = \langle \chi,e\rangle$ is the state $s'$ with $s'
    \models \changes{e}{s}$ and $s'(v) = s(v)$ for all state variables
    $v$ not mentioned in $\changes{e}{s}$.
    
    This is defined only if $o$ is applicable in $s$.
  \end{definition}

  \begin{example}
    Consider the operator
    $\langle a, \neg a \land (\neg c \CEF \neg b)\rangle$ 
    and the state $s = \{a \mapsto 1, b \mapsto 1, c \mapsto 1, d
      \mapsto 1\}$.

    The operator is applicable because $s \models a$ and $\changes{\neg
      a \land (\neg c \CEF \neg b)}{s} = \{\neg a\}$ is consistent.
      
    Applying the operator results in the successor state
    $\applyop{\langle a, \neg a \land (\neg c \CEF \neg b)
      \rangle}{s} = \{a \mapsto 0, b \mapsto 1, c \mapsto 1, d \mapsto
      1\}$.
  \end{example}
\end{frame}

\subsection[Tasks]{Deterministic planning tasks}

\begin{frame}{Deterministic planning tasks}
  \begin{definition}[deterministic planning task]
    A \alert{deterministic planning task} is a 4-tuple $\Pi = \langle
    A,I,O, \gamma\rangle$ where
    \begin{itemize}
    \item $A$ is a finite set of \alert{state variables}
      (propositions),
    \item $I$ is a valuation over $A$ called the \alert{initial state},
    \item $O$ is a finite set of \alert{operators} over $A$, and
    \item $\gamma$ is a formula over $A$ called the \alert{goal}.
    \end{itemize}
  \end{definition}

  \hilite{Note:}
  \begin{itemize}
  \item In the major part of this course, in which we talk about
    deterministic planning tasks, we usually omit the word
    ``deterministic''.
  \item When we will talk about nondeterministic planning tasks later,
    we will explicitly qualify them as ``nondeterministic''.
%  \item In this course, we usually omit the word ``deterministic''
%    since all our tasks are deterministic.
%  \item In the general literature ``planning task'' often refers to
%    broader problem classes, \eg\ including nondeterminism.
  \end{itemize}
\end{frame}

\begin{frame}{Mapping planning tasks to transition systems}
  \begin{definition}[induced transition system of a planning task]
    Every planning task $\Pi = \langle A,I,O,\gamma\rangle$ induces a
    corresponding deterministic transition system $\alert{\mathcal T(\Pi)} =
    \langle S, L, T, s_0, S_\star\rangle$:
    \begin{itemize}
    \item $S$ is the set of all valuations of $A$,
    \item $L$ is the set of operators $O$,
    \item $T = \{ \langle s, o, s'\rangle \mid
      s \in S,~ o \text{~applicable in~}s,~ s' = \applyop{o}{s}\}$,
    \item $s_0 = I$, and
    \item $S_\star = \{ s \in S \mid s \models \gamma\}$
    \end{itemize}
  \end{definition}
\end{frame}

\begin{frame}{Planning tasks: terminology}
  \begin{itemize}
  \item Terminology for transitions systems is also applied to the
    planning tasks that induce them.
  \item For example, when we speak of the \alert{states of $\Pi$}, we
    mean the states of $\mathcal T(\Pi)$.
  \item A sequence of operators that forms a goal path of $\mathcal
    T(\Pi)$ is called a \alert{plan} of $\Pi$.
  \end{itemize}
\end{frame}

\begin{frame}{Planning}
  By \alert{planning}, we mean the following two algorithmic problems:

  \medskip

  \begin{definition}[satisficing planning]
    \begin{tabular}{@{}ll@{}}
      \hilite{Given:} & a planning task $\Pi$ \\
      \hilite{Output:} & a plan for $\Pi$, or \textbf{unsolvable} if no
      plan for $\Pi$ exists
    \end{tabular}
  \end{definition}

  \medskip

  \begin{definition}[optimal planning]
    \begin{tabular}{@{}ll@{}}
      \hilite{Given:} & a planning task $\Pi$ \\
      \hilite{Output:} & a plan for $\Pi$ with minimal length among
      all plans \\
      & for $\Pi$, or \textbf{unsolvable} if no
      plan for $\Pi$ exists
    \end{tabular}
  \end{definition}
\end{frame}

\section*{Summary}

\begin{frame}{Summary}
  \begin{itemize}
  \item \alert{Transition systems} are a kind of directed graph
    (typically huge) that encode how the state of the world can
    change.
  \item \alert{Planning tasks} are compact representations for
    transition systems, suitable as input for planning algorithms.
  \item Planning tasks are based on concepts from \alert{propositional
    logic}, suitably enhanced to model state change.
  \item \alert{States} of planning tasks are propositional valuations.
  \item \alert{Operators} of planning tasks describe \alert{when}
    (precondition) and \alert{how} (effect) to change the current
    state of the world.
  \item In \alert{satisficing planning}, we must find a solution to
    planning tasks (or show that no solution exists).
  \item In \alert{optimal planning}, we must additionally guarantee
    that generated solutions are of the shortest possible length.
  \end{itemize}
\end{frame}

\end{document}
