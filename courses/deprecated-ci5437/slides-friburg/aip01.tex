\documentclass{gkibeamer}
\usepackage{url}

\usepackage{tikz}[2008/02/13]

\title[AI Planning]{Principles of AI Planning}
\author[B.~Nebel,\\R.~Mattm\"uller]{Bernhard Nebel and Robert Mattm\"uller}
\date[]{Winter Semester 2011/2012}

%% Use \alert instead of \textred where it makes sense.
%% (\textred makes sense when you *really* want the color to be red,
%% independent of the beamer style settings.)
\newcommand{\hilite}[1]{{\color{structure} #1}}
\newcommand{\textred}[1]{{\color{red} #1}}
\newcommand{\textblue}[1]{{\color{blue} #1}}
\newcommand{\textgreen}[1]{{\color{green} #1}}
\newcommand{\textbrown}[1]{{\color{brown} #1}}
\newcommand{\textviolet}[1]{{\color{violet} #1}}
\newcommand{\textbg}[1]{{\color{bg} #1}}

\newcommand{\eg}{e.\,g.}
\newcommand{\ie}{i.\,e.}

%% State variables and values for state variables
\newcommand{\var}[1]{\textit{#1}}
\newcommand{\val}[1]{\text{#1}}

%% Various stuff
\newcommand{\astar}{A${}^*$}
\newcommand{\idastar}{IDA${}^*$}

%% Blocksworld stuff
\newcommand{\AONB}{\var{A-on-B}}
\newcommand{\AONC}{\var{A-on-C}}
\newcommand{\AOND}{\var{A-on-D}}
\newcommand{\BONA}{\var{B-on-A}}
\newcommand{\BONC}{\var{B-on-C}}
\newcommand{\BOND}{\var{B-on-D}}
\newcommand{\CONA}{\var{C-on-A}}
\newcommand{\CONB}{\var{C-on-B}}
\newcommand{\COND}{\var{C-on-D}}
\newcommand{\DONA}{\var{D-on-A}}
\newcommand{\DONB}{\var{D-on-B}}
\newcommand{\DONC}{\var{D-on-C}}
\newcommand{\AONTABLE}{\var{A-on-T}}
\newcommand{\BONTABLE}{\var{B-on-T}}
\newcommand{\CONTABLE}{\var{C-on-T}}
\newcommand{\CLEARA}{\var{A-clear}}
\newcommand{\CLEARB}{\var{B-clear}}
\newcommand{\CLEARC}{\var{C-clear}}

\newcommand{\applyop}[2]{\textit{app}_{#1}(#2)}
\newcommand{\applyops}[2]{\textit{app}_{#1}(#2)}
\newcommand{\applyplan}[2]{\textit{app}_{#1}(#2)}
\newcommand{\regr}[2]{\textit{regr}_{#1}(#2)}
\newcommand{\regrstrips}[2]{\textit{sregr}_{#1}(#2)}
\newcommand{\eprecon}[2]{\textit{EPC}_{#1}(#2)}

\newcommand{\compl}[1]{\overline{#1}}

\newcommand{\changes}[2]{\lbrack #1\rbrack_{#2}}
\newcommand{\CEF}{\vartriangleright}

\newcommand{\relaxation}[1]{{#1}^+}
\newcommand{\onset}[1]{\textit{on}(#1)}
\newcommand{\hplus}{\ensuremath{h^+}}
\newcommand{\hmax}{\ensuremath{h_{\text{max}}}}
\newcommand{\hadd}{\ensuremath{h_{\text{add}}}}
\newcommand{\hlmcut}{\ensuremath{h_{\text{LM-cut}}}}
\newcommand{\hff}{\ensuremath{h_{\text{FF}}}}
\newcommand{\hcs}{\ensuremath{h_{\text{cs}}}}
\newcommand{\hsa}{\ensuremath{h_{\text{sa}}}}
\newcommand{\hhhh}{\ensuremath{h_{\text{HHH}}}}

\newcommand{\sasplus}{SAS${}^+$}

\newcommand{\graphequiv}{\stackrel{\textup{G}}{\sim}}

\newcommand{\cg}{\ensuremath{\textit{CG}}}

%% Turing machines and complexity theory
\newcommand{\accept}{{\textsf{Y}}}

\newcommand{\easier}{\ensuremath{\le_{\text{p}}}}

\newcommand{\decisionclass}[1]{\ensuremath{\textsf{\textup{#1}}}}
\newcommand{\dtime}{\decisionclass{DTIME}}
\newcommand{\ntime}{\decisionclass{NTIME}}
\newcommand{\dspace}{\decisionclass{DSPACE}}
\newcommand{\nspace}{\decisionclass{NSPACE}}

\newcommand{\ptime}{\decisionclass{P}}
\newcommand{\np}{\decisionclass{NP}}
\newcommand{\pspace}{\decisionclass{PSPACE}}
\newcommand{\npspace}{\decisionclass{NPSPACE}}
\newcommand{\exptime}{\decisionclass{EXP}}
\newcommand{\nexptime}{\decisionclass{NEXP}}
\newcommand{\expspace}{\decisionclass{EXPSPACE}}
\newcommand{\dblexptime}{\decisionclass{2EXP}}

\newcommand{\planex}{\textsc{PlanEx}}
\newcommand{\planlen}{\textsc{PlanLen}}

\newcommand{\ms}[1]{\mathchoice{\mbox{\normalsize\it #1}}{\mbox{\normalsize\it #1}}{\mbox{\scriptsize\it #1}}{\mbox{\tiny\it #1}}}
\newcommand{\image}[2]{\ms{img}_{#1}(#2)}
\newcommand{\spreimage}[2]{\ms{spreimg}_{#1}(#2)}
\newcommand{\wpreimage}[2]{\ms{wpreimg}_{#1}(#2)}
\newcommand{\disbwd}[2]{\delta^{\ms{bwd}}_{#1}(#2)}
\newcommand{\ats}[1]{\ms{scope}(#1)}
\newcommand{\ndopcpc}[2]{\tau^{\ms{nd}}_{#1}(#2)}
\newcommand{\choice}{\mathop{|}}
\newcommand{\opcpc}[2]{\tau_{#1}(#2)}

%% Don't use \begin{proof}/\end{proof} for multi-part proofs because
%% this places a QED symbol.
\newenvironment{proofstart}{\begin{block}{Proof.}}{\end{block}}
\newenvironment{proofmid}{\begin{block}{Proof (ctd.)}}{\end{block}}
\newenvironment{proofend}{\begin{proof}[Proof (ctd.)]}{\end{proof}}

\newtheorem{remark}[theorem]{Remark}

%% Macros to align the width of things.
\newlength{\mywidth}
\newcommand{\setmywidth}[1]{\settowidth{\mywidth}{#1}}
\newcommand{\usemywidth}[1]{\makebox[\mywidth][l]{#1}}
\newcommand{\usemywidthmath}[1]{\usemywidth{\ensuremath{#1}}}

%% A tighter "align" environment.
\newenvironment{tightalign}[1][c]{\par\(\begin{array}[#1]{@{}r@{}l}}
  {\end{array}\)\par}
\newenvironment{wrappedmath}[1][t]{\begin{array}[#1]{@{}l}}{\end{array}}

%% Nondeterministic transition systems with connector arcs
\newcommand{\arc}[4]{
  \begin{pgfscope}
    \pgfsetlinewidth{1pt}
    \pgfpathmoveto{\pgfpointanchor{#3}{center}}
    \pgfpathlineto{\pgfpointanchor{#1}{#2}}
    \pgfpathlineto{\pgfpointanchor{#4}{center}}
    \pgfusepath{clip}
    \pgfpathcircle{\pgfpointanchor{#1}{#2}}{1.8mm}
    \pgfusepath{draw}
  \end{pgfscope}
}


\begin{document}

\lectureno{1}
\subtitle{Introduction}
\date{October 25th, 2011}
\maketitles

\newcommand{\drawblock}[5]{% PARAMETERS: COLOR, CORNERCOORDS, SIZE
% TOP SIDE
\color{#1!55}
\pgfmoveto{\pgfrelative{\pgfxyz(#2,#3,#4)}{\pgfxyz(0,#5,0)}}
\pgflineto{\pgfrelative{\pgfxyz(#2,#3,#4)}{\pgfxyz(#5,#5,0)}}
\pgflineto{\pgfrelative{\pgfxyz(#2,#3,#4)}{\pgfxyz(#5,#5,#5)}}
\pgflineto{\pgfrelative{\pgfxyz(#2,#3,#4)}{\pgfxyz(0,#5,#5)}}
\pgflineto{\pgfrelative{\pgfxyz(#2,#3,#4)}{\pgfxyz(0,#5,0)}}
\pgffill
\color{black}
\pgfmoveto{\pgfrelative{\pgfxyz(#2,#3,#4)}{\pgfxyz(0,#5,0)}}
\pgflineto{\pgfrelative{\pgfxyz(#2,#3,#4)}{\pgfxyz(#5,#5,0)}}
\pgflineto{\pgfrelative{\pgfxyz(#2,#3,#4)}{\pgfxyz(#5,#5,#5)}}
\pgflineto{\pgfrelative{\pgfxyz(#2,#3,#4)}{\pgfxyz(0,#5,#5)}}
\pgflineto{\pgfrelative{\pgfxyz(#2,#3,#4)}{\pgfxyz(0,#5,0)}}
\pgfstroke
% RIGHT SIDE
\color{#1!65}
\pgfmoveto{\pgfrelative{\pgfxyz(#2,#3,#4)}{\pgfxyz(#5,0,0)}}
\pgflineto{\pgfrelative{\pgfxyz(#2,#3,#4)}{\pgfxyz(#5,#5,0)}}
\pgflineto{\pgfrelative{\pgfxyz(#2,#3,#4)}{\pgfxyz(#5,#5,#5)}}
\pgflineto{\pgfrelative{\pgfxyz(#2,#3,#4)}{\pgfxyz(#5,0,#5)}}
\pgflineto{\pgfrelative{\pgfxyz(#2,#3,#4)}{\pgfxyz(#5,0,0)}}
\pgffill
\color{black}
\pgfmoveto{\pgfrelative{\pgfxyz(#2,#3,#4)}{\pgfxyz(#5,0,0)}}
\pgflineto{\pgfrelative{\pgfxyz(#2,#3,#4)}{\pgfxyz(#5,#5,0)}}
\pgflineto{\pgfrelative{\pgfxyz(#2,#3,#4)}{\pgfxyz(#5,#5,#5)}}
\pgflineto{\pgfrelative{\pgfxyz(#2,#3,#4)}{\pgfxyz(#5,0,#5)}}
\pgflineto{\pgfrelative{\pgfxyz(#2,#3,#4)}{\pgfxyz(#5,0,0)}}
\pgfstroke
%FRONT
\color{#1!99}
\pgfmoveto{\pgfxyz(#2,#3,#4)}
\pgflineto{\pgfrelative{\pgfxyz(#2,#3,#4)}{\pgfxyz(#5,0,0)}}
\pgflineto{\pgfrelative{\pgfxyz(#2,#3,#4)}{\pgfxyz(#5,#5,0)}}
\pgflineto{\pgfrelative{\pgfxyz(#2,#3,#4)}{\pgfxyz(0,#5,0)}}
\pgflineto{\pgfxyz(#2,#3,#4)}
\pgffill
\color{black}
\pgfmoveto{\pgfxyz(#2,#3,#4)}
\pgflineto{\pgfrelative{\pgfxyz(#2,#3,#4)}{\pgfxyz(#5,0,0)}}
\pgflineto{\pgfrelative{\pgfxyz(#2,#3,#4)}{\pgfxyz(#5,#5,0)}}
\pgflineto{\pgfrelative{\pgfxyz(#2,#3,#4)}{\pgfxyz(0,#5,0)}}
\pgflineto{\pgfxyz(#2,#3,#4)}
\pgfstroke
}

\newcommand{\RsGsB}{\begin{pgfpicture}{0.5mm}{0.5mm}{12mm}{5mm}
\pgfsetxvec{\pgfpoint{0.4cm}{0cm}}
\pgfsetyvec{\pgfpoint{0cm}{0.4cm}}
\pgfsetzvec{\pgfpoint{0.15cm}{0.15cm}}
\drawblock{red}{0}{0}{0}{1}
\drawblock{green}{1}{0}{0}{1}
\drawblock{blue}{2}{0}{0}{1}
\end{pgfpicture}
}

\newcommand{\twostacksA}[3]{\begin{pgfpicture}{0.5mm}{0.5mm}{0.8cm}{0.9cm}
\pgfsetxvec{\pgfpoint{0.4cm}{0cm}}
\pgfsetyvec{\pgfpoint{0cm}{0.4cm}}
\pgfsetzvec{\pgfpoint{0.15cm}{0.15cm}}
\drawblock{#2}{0}{0}{0}{1}
\drawblock{#1}{0}{1}{0}{1}
\drawblock{#3}{1}{0}{0}{1}
\end{pgfpicture}
}

\newcommand{\twostacksB}[3]{\begin{pgfpicture}{0.5mm}{0.5mm}{0.8cm}{0.9cm}
\pgfsetxvec{\pgfpoint{0.4cm}{0cm}}
\pgfsetyvec{\pgfpoint{0cm}{0.4cm}}
\pgfsetzvec{\pgfpoint{0.15cm}{0.15cm}}
\drawblock{#1}{0}{0}{0}{1}
\drawblock{#3}{1}{0}{0}{1}
\drawblock{#2}{1}{1}{0}{1}
\end{pgfpicture}
}

\newcommand{\onestack}[3]{\begin{pgfpicture}{0.5mm}{0.5mm}{4mm}{13mm}
\pgfsetxvec{\pgfpoint{0.4cm}{0cm}}
\pgfsetyvec{\pgfpoint{0cm}{0.4cm}}
\pgfsetzvec{\pgfpoint{0.15cm}{0.15cm}}
\drawblock{#3}{0}{0}{0}{1}
\drawblock{#2}{0}{1}{0}{1}
\drawblock{#1}{0}{2}{0}{1}
\end{pgfpicture}
}

\newcommand{\RGsB}{\twostacksA{red}{green}{blue}}

\newcommand{\RBsG}{\twostacksB{green}{red}{blue}}
\newcommand{\RBsGr}{\twostacksA{red}{blue}{green}}

\newcommand{\BRsG}{\twostacksA{blue}{red}{green}}
\newcommand{\BRsGr}{\twostacksB{green}{blue}{red}}

\newcommand{\RsBG}{\twostacksB{red}{blue}{green}}

\newcommand{\GRsB}{\twostacksA{green}{red}{blue}}
\newcommand{\RsGB}{\twostacksB{red}{green}{blue}}

\newcommand{\RGB}{\onestack{red}{green}{blue}}
\newcommand{\RBG}{\onestack{red}{blue}{green}}
\newcommand{\GRB}{\onestack{green}{red}{blue}}
\newcommand{\GBR}{\onestack{green}{blue}{red}}
\newcommand{\BRG}{\onestack{blue}{red}{green}}
\newcommand{\BGR}{\onestack{blue}{green}{red}}


\section[About\dots]{About the course}

\subsection{Coordinates}

\begin{frame}{People}
  \begin{block}{Lecturers}    
    Prof.~Dr.~Bernhard Nebel
    \begin{itemize}
    \item \hilite{email:} \texttt{nebel@informatik.uni-freiburg.de}
    \item \hilite{office:} room 052-00-029
    \item \hilite{consultation:} by appointment (email)
    \end{itemize}

    \medskip

    Robert Mattm\"uller
    \begin{itemize}
    \item \hilite{email:} \texttt{mattmuel@informatik.uni-freiburg.de}
    \item \hilite{office:} room 052-00-045
    \item \hilite{consultation:} by appointment (email)
      or just drop by in the office
    \end{itemize}
  \end{block}
\end{frame}

\begin{frame}{People}
  \begin{block}{Assistant}
    Thomas Keller
    \begin{itemize}
    \item \hilite{email:} \texttt{tkeller@informatik.uni-freiburg.de}
    \item \hilite{office:} room 052-00-030
    \item \hilite{consultation:} by appointment (email)
      or just drop by in the office
    \end{itemize}
  \end{block}

  \medskip

  \begin{block}{Tutor}
    Yusra Alkhazraji
    % Yusra Wahby
    \begin{itemize}
    % \item \hilite{email:} \texttt{yousra\_wahby@yahoo.com}
    % \item \hilite{email:} \texttt{alkhazry@informatik.uni-freiburg.de}
    \item \hilite{email:} \texttt{yusra.alkhazraji@uranus.uni-freiburg.de}
    \end{itemize}
  \end{block}
\end{frame}

\begin{frame}{Time \& place}
  \begin{block}{Lectures}
    \begin{itemize}
    \item \hilite{time:} Tuesday 16:15-18:00, Friday 14:15-15:00
    \item \hilite{place:} SR 101-01-018
    \end{itemize}
  \end{block}

  \begin{block}{Exercises}
    \begin{itemize}
    \item \hilite{time:} Friday 15:15-16:00
    \item \hilite{place:} SR 101-01-018
    \end{itemize}
  \end{block}
\end{frame}

\begin{frame}{Web site}
  \begin{block}{Course web site}
    {\footnotesize\url{http://gki.informatik.uni-freiburg.de/teaching/ws1112/aip/}}
    \begin{itemize}
    \item \hilite{main page:} course description
    \item \hilite{lecture page:} slides
    \item \hilite{exercise page:} assignments, model solutions,
      software
    \item \hilite{bibliography page:} literature references and
      papers
    \end{itemize}
  \end{block}
\end{frame}

\begin{frame}{Teaching materials}
  \begin{itemize}
  \item no textbook, no script
  \item slides handed out during lectures and available on the web
  \item additional resources: bibliography page on web +
    \alert{ask us!}
  \end{itemize}

  \bigskip

  \hilite{Acknowledgments:}
  \begin{itemize}
  \item slides based on earlier courses by Jussi Rintanen, \\
    Bernhard Nebel and Malte Helmert
  \item many figures by Gabi R\"oger
  \end{itemize}
\end{frame}

\subsection{Rules}

\begin{frame}{Target audience}
  \hilite{Students of Computer Science:}
  \begin{itemize}
  \item Master of Science, any year
  \item Bachelor of Science, $\sim$3rd year
  \end{itemize}

  \bigskip

  \hilite{Students of Applied Computer Science:}
  \begin{itemize}
  \item Master of Science, $\sim$2nd year
  \end{itemize}

  \bigskip

  \hilite{Other students:}
  \begin{itemize}
  \item advanced study period ($\sim$4th year)
  \end{itemize}
\end{frame}

\begin{frame}{Prerequisites}
  Course prerequisites:
  \begin{itemize}
  \item \hilite{propositional logic:} syntax and semantics
  \item \hilite{foundations of AI:} search, heuristic search
  \item \hilite{computational complexity theory:} decision problems,
    reductions, NP-completeness
  \end{itemize}
\end{frame}

\begin{frame}{Credit points \& exam}
  \begin{itemize}
  \item 6 ECTS points
  \item special lecture in concentration subject \\
    \hilite{Artificial Intelligence and Robotics}
  \item \hilite{oral exam} of about 30 minutes B.Sc.\ students
  \item \hilite{written or oral exam} for M.Sc.\ students\\
    (depending on their number)
    %% NOTE: According to an email from Friederike Schneider received
    %% March 15th, 2010, the duration for oral exams should be at most
    %% 5 mins/ECTS point with the MSc regulations and the "old" BSc
    %% regulations, and at most 10 mins/ECTS point with the "new" BSc
    %% regulations.
  \end{itemize}
\end{frame}

\begin{frame}{Exercises}
  \hilite{Exercises} (written assignments):
  \begin{itemize}
  \item handed out on Tuesdays (exception: sheet 1 handed out this
    Friday instead of Tuesday next week because of All Saints' Day)
  \item due Tuesday following week, before the lecture
  \item discussed Friday that week
  \item may be solved in groups of two students ($2 \neq 3$)
  \item successful participation prerequisite for exam
    admission
  \end{itemize}
\end{frame}

\begin{frame}{Projects}
  \hilite{Projects} (programming assignments):
  \begin{itemize}
  \item handed out every now and then\\
    (probably three times over the course of the semester)
  \item more time to work on than for exercises
  \item may be solved in groups of two students ($2 = 2$)
  \item language: Python
  \item codebase: \url{https://bitbucket.org/malte/pyperplan}
  \item solutions that obviously do not work: 0 marks
    \begin{itemize}
    \item may fix bugs uncovered by our testing \\
      \alert{if still within submission deadline}
    \end{itemize}
  \item successful participation prerequisite for exam
    admission
  \end{itemize}
\end{frame}

\begin{frame}{Admission to exam}
  \begin{itemize}
  \item \hilite{points can be earned for ``reasonable'' solutions} to
    exercises and projects (one project counts like two exercise
    sheets).
  \item at least \hilite{50\% of points prerequisite for admission to
      final exam}.
  \end{itemize}
\end{frame}

\begin{frame}{Plagiarism}
  \hilite{What is plagiarism?}
  \begin{itemize}
  \item passing off solutions as your own that are not based on your
    ideas (work of other students, Internet, books, \dots)
  \item \url{http://en.wikipedia.org/wiki/Plagiarism} \\
    is a good intro
  \end{itemize}

  \medskip

  \hilite{Consequence:} no admission to the final exam.
  \begin{itemize}
  \item We may (!) be generous on first offense.
  \item Don't tell us ``We did the work together.''
  \item Don't tell us ``I did not know this was not allowed.''
  \end{itemize}
\end{frame}

\section{Introduction}
\subsection{What is planning?}

\begin{frame}{What is planning?}
  \begin{block}{Planning}
    \alert{``Planning is the art and practice of thinking before
      acting.''}
    
    \hspace*{\fill}--- Patrik Haslum
  \end{block}

  \begin{itemize}
  \item intelligent decision making: What actions to take?
  \item general-purpose problem representation
  \item algorithms for solving any problem expressible in the
    representation
  \item application areas:
    \begin{itemize}
    \item high-level planning for intelligent robots
    \item autonomous systems: NASA Deep Space One, \dots
    \item problem solving (single-agent games like Rubik's cube)
    \end{itemize}
  \end{itemize}
\end{frame}

\begin{frame}{Why is planning difficult?}
  \begin{itemize}
  \item solutions to classical planning problems are \alert{paths from
    an initial state to a goal state} in the \alert{transition graph}
    \begin{itemize}
    \item efficiently solvable by Dijkstra's algorithm in
      $O(|V|\log |V| + |E|)$ time
    \item Why don't we solve all planning problems this way?
    \end{itemize}
  \item state spaces may be \alert{huge}: $10^{10}, 10^{100}, 10^{1000},
    \dots$ states
    \begin{itemize}
    \item constructing the transition graph is infeasible!
    \item planning algorithms try to \alert{avoid constructing whole
      graph}
    \end{itemize}
  \item planning algorithms are often much more efficient than obvious
    solution methods constructing the transition graph and using
    \eg\ Dijkstra's algorithm
  \end{itemize}
\end{frame}

\subsection{Problem classes}

\begin{frame}{Different classes of problems}
  \begin{itemize}
  \item \hilite{dynamics:} \alert<all:2,5>{deterministic},
    \alert<all:3,4,5>{nondeterministic} or
    \alert<all:6,7>{probabilistic}
  \item \hilite{observability:} \alert<all:2,3,6>{full},
    \alert<all:4,7>{partial} or \alert<all:5>{none}
  \item \hilite{horizon:} \alert<all:2,3,4,5,6,7>{finite} or
    \alert<all:3,4,6,7>{infinite}
  \item \dots
  \end{itemize}

  \begin{enumerate}
  \item \alert<all:2>{classical planning}
  \item \alert<all:3>{conditional planning with full observability}
  \item \alert<all:4>{conditional planning with partial observability}
  \item \alert<all:5>{conformant planning}
  \item \alert<all:6>{Markov decision processes (MDP)}
  \item \alert<all:7>{partially observable MDPs (POMDP)}
  \end{enumerate}
\end{frame}

\subsection{Dynamics}

\begin{frame}{Properties of the world: dynamics}

  \begin{block}{Deterministic dynamics}
    Action + current state \alert{uniquely} determine successor
    state.
  \end{block}

  \begin{block}{Nondeterministic dynamics}
    For each action and current state there may be \alert{several
      possible} successor states.
  \end{block}

  \begin{block}{Probabilistic dynamics}
    For each action and current state there is a \alert{probability
      distribution} over possible successor states.
  \end{block}
  
  \medskip

  \hilite{Analogy:} deterministic versus nondeterministic automata
\end{frame}

\begin{frame}
  \frametitle<all:1>{Deterministic dynamics example}
  \frametitle<all:2>{Nondeterministic dynamics example}
  \frametitle<all:3>{Probabilistic dynamics example}
  Moving objects with
  \only<all:1>{a}\only<all:2,3>{an \alert{unreliable}}
  robotic hand: \\
  move the green block onto the blue block.

  \begin{center}
    \begin{pgfpicture}{0mm}{0mm}{60mm}{55mm}
      \pgfsetxvec{\pgfpoint{0.7cm}{0cm}}
      \pgfsetyvec{\pgfpoint{0cm}{0.7cm}}
      \pgfsetzvec{\pgfpoint{0.18cm}{0.18cm}}

      \drawblock{red}{0}{3}{0}{1}
      \drawblock{blue}{1}{3}{0}{1}
      \drawblock{green}{0}{4}{0}{1}

      \drawblock{red}{6}{0}{0}{1}
      \drawblock{blue}{7}{0}{0}{1}
      \drawblock{green}{7}{1}{0}{1}
      
      \only<all:2,3>{
        \drawblock{red}{6}{6}{0}{1}
        \drawblock{blue}{7}{6}{0}{1}
        \drawblock{green}{6.5}{5.5}{0}{1}
      }

      \pgfsetendarrow{\pgfarrowtriangle{8pt}}
      
      \pgfline{\pgfxy(2.5,3.6)}{\pgfxy(5.5,1)}
      \pgfputat{\pgfxy(4,2.7)}{\pgfbox[left,center]{\only<all:3>{$p=0.9$}}}
      \only<all:2,3>{
        \pgfline{\pgfxy(2.5,3.6)}{\pgfxy(5.5,6.0)}
        \pgfputat{\pgfxy(4,4.5)}{\pgfbox[left,center]{\only<all:3>{$p=0.1$}}}
      }
    \end{pgfpicture}
  \end{center}
\end{frame}

\subsection{Observability}

\begin{frame}{Properties of the world: observability}
  \begin{block}{Full observability}
    Observations/sensing determine current world state
    \alert{uniquely}.
  \end{block}

  \begin{block}{Partial observability}
    Observations determine current world state \alert{only partially}:
    \\
    we only know that current state is one of several possible
    ones.
  \end{block}

  \begin{block}{No observability}
    There are \alert{no observations} to narrow down possible current
    states. However, can use knowledge of \alert{action dynamics} to
    deduce which states we might be in.
  \end{block}

  \hilite{Consequence}: If observability is not full, must
  represent the \alert{knowledge} an agent has.
\end{frame}



\begin{frame}{What difference does observability make?}
  \begin{center}
    \begin{pgfpicture}{1.5cm}{-2cm}{8.5cm}{5cm}

      \pgfputat{\pgfxy(1,4.5)}{\pgfbox[left,center]{Camera A}}
      \pgfputat{\pgfxy(7,4.5)}{\pgfbox[left,center]{Camera B}}
      \pgfputat{\pgfxy(4.5,0.7)}{\pgfbox[left,center]{Goal}}
      
      \pgfsetxvec{\pgfpoint{0.8cm}{0cm}}
      \pgfsetyvec{\pgfpoint{0cm}{0.8cm}}
      \pgfsetzvec{\pgfpoint{0.3cm}{0.3cm}}
      
      \drawblock{red}{1}{2}{2}{1}
      \drawblock{blue}{2}{2}{2}{1}
      \drawblock{green}{2}{3}{2}{1}
      \drawblock{black}{2}{2}{0}{1}
      
      \pgfsetxvec{\pgfpoint{8mm}{0cm}}
      \pgfsetyvec{\pgfpoint{0cm}{0cm}}
      \pgfsetzvec{\pgfpoint{0cm}{8mm}}

      \drawblock{red}{9}{2}{4}{1}
      \drawblock{blue}{10}{2}{4}{1}
      \drawblock{green}{10}{3}{4}{1}
      \drawblock{black}{10}{2}{2}{1}
      
      \pgfsetxvec{\pgfpoint{0.8cm}{0cm}}
      \pgfsetyvec{\pgfpoint{0cm}{0.8cm}}
      \pgfsetzvec{\pgfpoint{0.3cm}{0.3cm}}
      
      \drawblock{red}{9}{2}{-11}{1}
      \drawblock{blue}{10}{2}{-11}{1}
      \drawblock{black}{9}{3}{-11}{1}
      \drawblock{green}{10}{3}{-11}{1}
    \end{pgfpicture}
  \end{center}
\end{frame}

\subsection{Objectives}

\begin{frame}{Different objectives}
  \begin{enumerate}
  \item Reach a goal state.
    \begin{itemize}
    \item \hilite{Example}: Earn 500 Euro.
    \end{itemize}
  \item Stay in goal states indefinitely (infinite horizon).
    \begin{itemize}
    \item \hilite{Example:} Never allow bank account balance to be
      negative.
    \end{itemize}
  \item Maximize the probability of reaching a goal state.
    \begin{itemize}
    \item \hilite{Example:} To be able to finance buying a house by 2022
      study hard and save money.
    \end{itemize}
  \item Collect the maximal \hilite{expected} rewards/minimal expected costs
    (infinite horizon).
    \begin{itemize}
    \item \hilite{Example:} Maximize your future income.
    \end{itemize}
  \item ...
  \end{enumerate}
\end{frame}

\subsection{Planning vs.\ game theory}

\begin{frame}{Relation to games and game theory}
  \begin{itemize}
  \item Game theory addresses decision making in multi-agent setting:
    ``Assuming that the other agents are rational, what do I have to
    do to achieve my goals?''
  \item Game theory is related to \alert{multi-agent planning}.
  \item In this course we concentrate on \alert{single-agent
    planning}.
  \item Some of the techniques are also applicable to special cases of
    multi-agent planning.
    \begin{itemize}
    \item \hilite{Example:} Finding a \alert{winning strategy} of a
      game like chess. \\ In this case it is not necessary to
      distinguish between \alert{an intelligent opponent} and \alert{a
        randomly behaving opponent}.
    \end{itemize}
  \item Game theory in general is about \alert{optimal strategies}
    which do not necessarily guarantee winning. For example card games
    like poker do not have a winning strategy.
  \end{itemize}
\end{frame}

\subsection{Summary}

\begin{frame}{What do you learn in this course?}
  \begin{itemize}
    %% NOTE: Changed that since we mostly got rid of the
    %% nondeterministic stuff.
    %% \item ``big picture'' of different kinds of planning problems
    %% \begin{itemize}
    %% \item \alert{classification} according to dynamics, observability,
    %%   objectives, \dots
    %% \item \alert{computational complexity} for different problem classes
    %% \end{itemize}
  \item emphasis on \alert{classical} planning (``simplest'' case)
  \item \alert{theoretical background} for planning
    \begin{itemize}
    \item formal \alert{problem definition}
    \item basic \alert{theoretical notions} \\
      (\eg, normal forms, progression, regression)
    \item \alert{computational complexity} of planning
    \end{itemize}
  \item \alert{algorithms} for planning:
    \begin{itemize}
    \item based on \alert{heuristic search}
    \item based on satisfiability testing (\alert{SAT})\\
      (time permitting)
    % \item based on exhaustive search with logic-based data structures
    %   (\alert{BDDs})
    \end{itemize}
    Many of these techniques are applicable to problems outside AI as
    well.
  \item \alert{hands-on experience} with a classical planner
    %(optional)
  \end{itemize}
\end{frame}
\end{document}
