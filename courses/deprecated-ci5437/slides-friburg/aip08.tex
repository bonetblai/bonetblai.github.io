\documentclass{gkibeamer}

\usepackage{tikz}[2008/02/13]

\title[AI Planning]{Principles of AI Planning}
\author[B.~Nebel,\\R.~Mattm\"uller]{Bernhard Nebel and Robert Mattm\"uller}
\date[]{Winter Semester 2011/2012}

%% Use \alert instead of \textred where it makes sense.
%% (\textred makes sense when you *really* want the color to be red,
%% independent of the beamer style settings.)
\newcommand{\hilite}[1]{{\color{structure} #1}}
\newcommand{\textred}[1]{{\color{red} #1}}
\newcommand{\textblue}[1]{{\color{blue} #1}}
\newcommand{\textgreen}[1]{{\color{green} #1}}
\newcommand{\textbrown}[1]{{\color{brown} #1}}
\newcommand{\textviolet}[1]{{\color{violet} #1}}
\newcommand{\textbg}[1]{{\color{bg} #1}}

\newcommand{\eg}{e.\,g.}
\newcommand{\ie}{i.\,e.}

%% State variables and values for state variables
\newcommand{\var}[1]{\textit{#1}}
\newcommand{\val}[1]{\text{#1}}

%% Various stuff
\newcommand{\astar}{A${}^*$}
\newcommand{\idastar}{IDA${}^*$}

%% Blocksworld stuff
\newcommand{\AONB}{\var{A-on-B}}
\newcommand{\AONC}{\var{A-on-C}}
\newcommand{\AOND}{\var{A-on-D}}
\newcommand{\BONA}{\var{B-on-A}}
\newcommand{\BONC}{\var{B-on-C}}
\newcommand{\BOND}{\var{B-on-D}}
\newcommand{\CONA}{\var{C-on-A}}
\newcommand{\CONB}{\var{C-on-B}}
\newcommand{\COND}{\var{C-on-D}}
\newcommand{\DONA}{\var{D-on-A}}
\newcommand{\DONB}{\var{D-on-B}}
\newcommand{\DONC}{\var{D-on-C}}
\newcommand{\AONTABLE}{\var{A-on-T}}
\newcommand{\BONTABLE}{\var{B-on-T}}
\newcommand{\CONTABLE}{\var{C-on-T}}
\newcommand{\CLEARA}{\var{A-clear}}
\newcommand{\CLEARB}{\var{B-clear}}
\newcommand{\CLEARC}{\var{C-clear}}

\newcommand{\applyop}[2]{\textit{app}_{#1}(#2)}
\newcommand{\applyops}[2]{\textit{app}_{#1}(#2)}
\newcommand{\applyplan}[2]{\textit{app}_{#1}(#2)}
\newcommand{\regr}[2]{\textit{regr}_{#1}(#2)}
\newcommand{\regrstrips}[2]{\textit{sregr}_{#1}(#2)}
\newcommand{\eprecon}[2]{\textit{EPC}_{#1}(#2)}

\newcommand{\compl}[1]{\overline{#1}}

\newcommand{\changes}[2]{\lbrack #1\rbrack_{#2}}
\newcommand{\CEF}{\vartriangleright}

\newcommand{\relaxation}[1]{{#1}^+}
\newcommand{\onset}[1]{\textit{on}(#1)}
\newcommand{\hplus}{\ensuremath{h^+}}
\newcommand{\hmax}{\ensuremath{h_{\text{max}}}}
\newcommand{\hadd}{\ensuremath{h_{\text{add}}}}
\newcommand{\hlmcut}{\ensuremath{h_{\text{LM-cut}}}}
\newcommand{\hff}{\ensuremath{h_{\text{FF}}}}
\newcommand{\hcs}{\ensuremath{h_{\text{cs}}}}
\newcommand{\hsa}{\ensuremath{h_{\text{sa}}}}
\newcommand{\hhhh}{\ensuremath{h_{\text{HHH}}}}

\newcommand{\sasplus}{SAS${}^+$}

\newcommand{\graphequiv}{\stackrel{\textup{G}}{\sim}}

\newcommand{\cg}{\ensuremath{\textit{CG}}}

%% Turing machines and complexity theory
\newcommand{\accept}{{\textsf{Y}}}

\newcommand{\easier}{\ensuremath{\le_{\text{p}}}}

\newcommand{\decisionclass}[1]{\ensuremath{\textsf{\textup{#1}}}}
\newcommand{\dtime}{\decisionclass{DTIME}}
\newcommand{\ntime}{\decisionclass{NTIME}}
\newcommand{\dspace}{\decisionclass{DSPACE}}
\newcommand{\nspace}{\decisionclass{NSPACE}}

\newcommand{\ptime}{\decisionclass{P}}
\newcommand{\np}{\decisionclass{NP}}
\newcommand{\pspace}{\decisionclass{PSPACE}}
\newcommand{\npspace}{\decisionclass{NPSPACE}}
\newcommand{\exptime}{\decisionclass{EXP}}
\newcommand{\nexptime}{\decisionclass{NEXP}}
\newcommand{\expspace}{\decisionclass{EXPSPACE}}
\newcommand{\dblexptime}{\decisionclass{2EXP}}

\newcommand{\planex}{\textsc{PlanEx}}
\newcommand{\planlen}{\textsc{PlanLen}}

\newcommand{\ms}[1]{\mathchoice{\mbox{\normalsize\it #1}}{\mbox{\normalsize\it #1}}{\mbox{\scriptsize\it #1}}{\mbox{\tiny\it #1}}}
\newcommand{\image}[2]{\ms{img}_{#1}(#2)}
\newcommand{\spreimage}[2]{\ms{spreimg}_{#1}(#2)}
\newcommand{\wpreimage}[2]{\ms{wpreimg}_{#1}(#2)}
\newcommand{\disbwd}[2]{\delta^{\ms{bwd}}_{#1}(#2)}
\newcommand{\ats}[1]{\ms{scope}(#1)}
\newcommand{\ndopcpc}[2]{\tau^{\ms{nd}}_{#1}(#2)}
\newcommand{\choice}{\mathop{|}}
\newcommand{\opcpc}[2]{\tau_{#1}(#2)}

%% Don't use \begin{proof}/\end{proof} for multi-part proofs because
%% this places a QED symbol.
\newenvironment{proofstart}{\begin{block}{Proof.}}{\end{block}}
\newenvironment{proofmid}{\begin{block}{Proof (ctd.)}}{\end{block}}
\newenvironment{proofend}{\begin{proof}[Proof (ctd.)]}{\end{proof}}

\newtheorem{remark}[theorem]{Remark}

%% Macros to align the width of things.
\newlength{\mywidth}
\newcommand{\setmywidth}[1]{\settowidth{\mywidth}{#1}}
\newcommand{\usemywidth}[1]{\makebox[\mywidth][l]{#1}}
\newcommand{\usemywidthmath}[1]{\usemywidth{\ensuremath{#1}}}

%% A tighter "align" environment.
\newenvironment{tightalign}[1][c]{\par\(\begin{array}[#1]{@{}r@{}l}}
  {\end{array}\)\par}
\newenvironment{wrappedmath}[1][t]{\begin{array}[#1]{@{}l}}{\end{array}}

%% Nondeterministic transition systems with connector arcs
\newcommand{\arc}[4]{
  \begin{pgfscope}
    \pgfsetlinewidth{1pt}
    \pgfpathmoveto{\pgfpointanchor{#3}{center}}
    \pgfpathlineto{\pgfpointanchor{#1}{#2}}
    \pgfpathlineto{\pgfpointanchor{#4}{center}}
    \pgfusepath{clip}
    \pgfpathcircle{\pgfpointanchor{#1}{#2}}{1.8mm}
    \pgfusepath{draw}
  \end{pgfscope}
}


\usetikzlibrary{shapes}

\tikzstyle{relaxed planning graph}=[draw,inner sep=0pt,font=\small]
\tikzstyle{rpg square}=[relaxed planning graph,minimum size=0.52cm,rectangle]
\tikzstyle{rpg circle}=[relaxed planning graph,minimum size=0.52cm,circle]

\tikzstyle{prop}=        [rpg circle,fill=yellow]
\tikzstyle{trueinitprop}=[rpg square,fill=yellow]
\tikzstyle{logic-and}=   [rpg square,fill=orange]
\tikzstyle{logic-or}=    [rpg circle,fill=orange]
\tikzstyle{logic-cond}=  [rpg square,fill=blue!40]
\tikzstyle{operator}=    [rpg square,fill=blue!40]
\tikzstyle{goal-and}=    [rpg square,fill=white]
\tikzstyle{goal-or}=     [rpg circle,fill=white]

\tikzstyle{true}=[fill=green]
\tikzstyle{false}=[fill=red]
\tikzstyle{selected}=[fill=cyan]
\tikzstyle{dirty}=[fill=lightgray]

\tikzstyle{idle}=[thin]
\tikzstyle{nonidle}=[thin]

\newcommand{\markplusone}[1]{\path (#1) +(-0.04cm,0.28cm) node {\tiny $+1$};}
\newcommand{\markopnode}[3][0.35cm]{\path (#2) +(0cm,#1) node {\tiny
    \ensuremath{#3}};}
\newcommand{\markopnodeff}[3][0.30cm]{\path (#2) +(0cm,#1) node {\tiny
    \ensuremath{#3}};}

\begin{document}
\lectureno{8}
\subtitle{Planning as search: relaxation heuristics}
\date{November 22nd, 2011}
\maketitles

\section{Parallel plans}
\subsection[Plan steps]{Plan steps, serializations and parallel plans}

\begin{frame}{Towards better relaxed plans}
  Why does the greedy algorithm compute low-quality plans?
  \begin{itemize}
  \item It may apply many operators which are not
    \alert{goal-directed}.
  \end{itemize}

  \bigskip

  How can this problem be fixed?
  \begin{itemize}
  \item \alert{Reaching the goal} of a relaxed planning task is most
    easily achieved with \alert{forward search}.
  \item Analyzing \alert{relevance} of an operator for achieving a
    goal (or subgoal) is most easily achieved with \alert{backward
      search}.
  \end{itemize}
  
  \bigskip
  
  \hilite{Idea:} Use a \alert{forward-backward} algorithm that first
  finds a path to the goal greedily, then prunes it to a relevant
  subplan.
\end{frame}

\begin{frame}{Relaxed plan steps}
  How to decide which operators to apply in forward direction?
  \begin{itemize}
  \item We \alert{avoid} such a decision by applying all applicable
    operators \alert{simultaneously}.
  \end{itemize}
  
  \begin{definition}[plan step]
    A \alert{plan step} is a set of operators $\omega =
    \{\langle \chi_1, e_1\rangle, \dots, \langle \chi_n,
      e_n\rangle\}$.
      
    \smallskip
   
    In the \alert{special case of all operators of $\omega$ being
      relaxed}, \\
    we further define:
    \begin{itemize}
    \item Plan step $\omega$ is \alert{applicable} in state $s$ iff
      $s \models \chi_i$ for all $i \in \{1, \dots, n\}$.
    \item The \alert{result} of applying $\omega$ to $s$, in
      symbols $\applyops{\omega}{s}$, is defined as the state $s'$ with
      $\onset{s'} = \onset{s} \cup \bigcup_{i=1}^n \changes{e_i}{s}$.
    \end{itemize}
  \end{definition}
  \alert{general} semantics for plan steps $\leadsto$ much later
\end{frame}

\begin{frame}{Applying relaxed plan steps: examples}
  In all cases, $s = \{a \mapsto 0, b \mapsto 0, c \mapsto 1,
    d \mapsto 0\}$.
  \begin{itemize}
  \item $\omega = \{
    \langle c, a\rangle,
    \langle \top, b \rangle
    \}$
  \item $\omega = \{
    \langle c, a\rangle,
    \langle c, a \CEF b \rangle
    \}$
  \item $\omega = \{
    \langle c, a \land b\rangle,
    \langle a, b \CEF d\rangle
    \}$
  \item $\omega = \{
    \langle c, a \land (b \CEF d)\rangle,
    \langle c, b \land (a \CEF d)\rangle
    \}$
  \end{itemize}
\end{frame}

\begin{frame}{Serializations}
  Applying a relaxed plan step to a state is related to applying
  the operators in the step to a state in sequence.
  \begin{definition}[serialization]
    A \alert{serialization} of plan step 
    $\omega = \{\relaxation{o}_1, \dots, \relaxation{o}_n\}$ is a sequence
    $\relaxation{o}_{\pi(1)}, \dots, \relaxation{o}_{\pi(n)}$ where
    $\pi$ is a permutation of $\{1, \dots, n\}$.
  \end{definition}

  \begin{lemma}[conservativeness of plan step semantics]
    If $\omega$ is a plan step applicable in a state $s$ of a
    relaxed planning task, then each serialization $o_1, \dots, o_n$
    of $\omega$ is applicable in $s$ and
    $\applyplan{o_1, \dots, o_n}{s}$ dominates
    $\applyplan{\omega}{s}$.
  \end{lemma}

  \begin{itemize}
  \item Does equality hold for all serializations/some serialization?
  \item What if there are no conditional effects?
  \item What if we allowed general (unrelaxed) planning tasks?
  \end{itemize}
\end{frame}

\begin{frame}{Parallel plans}
  \begin{definition}[parallel plan]
    A \alert{parallel plan} for a relaxed planning task
    $\langle A, I, \relaxation{O}, \gamma\rangle$
    is a sequence of plan steps $\omega_1, \dots, \omega_n$
    of operators in $\relaxation{O}$ with:
    \begin{itemize}
    \item $s_0 := I$
    \item For $i = 1, \dots, n$, step $\omega_i$ is applicable in
      $s_{i-1}$ \\
      and $s_i := \applyops{\omega_i}{s_{i-1}}$.
    \item $s_n \models \gamma$
    \end{itemize}
  \end{definition}
  
  \hilite{Remark:}
  By ordering the operators within each single step arbitrarily, we
  obtain a (regular, non-parallel) plan.
\end{frame}

\subsection[Forward distances]{Forward states and parallel forward distances}

\begin{frame}{Forward states, plan steps and sets}
  \hilite{Idea:} In the forward phase of the heuristic computation,
  \begin{itemize}
  \item first apply plan step with \alert{all operators applicable
    initially},
  \item then apply plan step with \alert{all operators applicable
    then},
  \item and so on.
  \end{itemize}

  \begin{definition}[forward state, forward plan step, forward set]
    Let $\relaxation{\Pi} = \langle A, I, \relaxation{O}, \gamma\rangle$ be
    a relaxed planning task.

    \smallskip

    The \alert{$n$-th forward state}, in symbols \alert{$s^{\text
      F}_n$} ($n \in \mathbb N_0$), \\
    the \alert{$n$-th forward plan step}, in symbols
    \alert{$\omega^{\text F}_n$} ($n \in \mathbb N_1$), and \\
    the \alert{$n$-th forward set}, in symbols
    \alert{$S^{\text F}_n$} ($n \in \mathbb N_0$),
    are defined as:
    \begin{itemize}
    \item $s^{\text F}_0 := I$
    \item $\omega^{\text F}_n := \{ o \in \relaxation{O} \mid o
      \text{~applicable in~} s^{\text F}_{n-1}\}$ 
      for all $n \in \mathbb N_1$
    \item $s^{\text F}_n := \applyops{\omega^{\text F}_n}{s^{\text
        F}_{n-1}}$ for all $n \in \mathbb N_1$
    \item $S^{\text F}_n := \onset{s^{\text F}_n}$ for all $n \in
      \mathbb N_0$
    \end{itemize}
  \end{definition}
\end{frame}

\begin{frame}{The max heuristic \hmax}
  \begin{definition}[parallel forward distance]
    The \alert{parallel forward distance} of a relaxed planning
    task $\langle A, I, \relaxation{O}, \gamma\rangle$ is the lowest number
    $n \in \mathbb N_0$ such that $s^{\text F}_n \models \gamma$, or
    $\infty$ if no forward state satisfies $\gamma$.
  \end{definition}

  \hilite{Remark:} The parallel forward distance can be computed in
  polynomial time. (How?)

  \medskip

  \begin{definition}[max heuristic \hmax]
    Let $\Pi = \langle A, I, O, \gamma\rangle$ be a planning task in
    positive normal form, and let $s$ be a state of $\Pi$.

    The \alert{max heuristic} estimate for $s$,
    \alert{$\hmax(s)$}, is the parallel forward distance of the
    relaxed planning task $\langle A, s, \relaxation{O}, \gamma\rangle$.
  \end{definition}
  
  \hilite{Remark:} {\hmax} is safe, goal-aware, admissible and
  consistent. (Why?)
\end{frame}

\begin{frame}{So far, so good\dots}
  \begin{itemize}
  \item We have seen how systematic computation of forward states
    leads to an admissible heuristic estimate.
  \item However, this estimate is \alert{very coarse}.
  \item To improve it, we need to include \alert{backward propagation}
    of information.
  \end{itemize}
  
  For this purpose, we use so-called \alert{relaxed planning graphs}.
\end{frame}

\section{Relaxed planning graphs}
\subsection{Introduction}

\begin{frame}{AND/OR dags}
  \begin{definition}[AND/OR dag]
    An \alert{AND/OR dag}
    $\langle V, A, \textit{type}\rangle$
    is a directed acyclic graph $\langle V, A\rangle$ with a
    label function $\textit{type}: V \to \{ \land, \lor\}$
    partitioning nodes into \alert{AND nodes} ($\textit{type}(v) = \land)$
    and \alert{OR nodes} ($\textit{type}(v) = \lor$).
  \end{definition}
  
  \hilite{Note:} We draw AND nodes as squares and OR
  nodes as circles.

  \begin{definition}[truth values in AND/OR dags]
    Let $G = \langle V, A, \textit{type}\rangle$ be an AND/OR dag, and
    let $u \in V$ be a node with successor set $\{v_1, \dots, v_k\}
      \subseteq V$.

    \medskip

    The (truth) \alert{value} of $u$, $\alert{\textit{val}(u)}$, is
    inductively defined as:
    \begin{itemize}
    \item If $\textit{type}(u) = \land$, then
      $\textit{val}(u) = \textit{val}(v_1) \land \dots \land
      \textit{val}(v_k)$.
    \item If $\textit{type}(u) = \lor$, then
      $\textit{val}(u) = \textit{val}(v_1) \lor \dots \lor
      \textit{val}(v_k)$.
    \end{itemize}
  \end{definition}
  \hilite{Note:} No separate base case is needed. (Why not?)
\end{frame}

\begin{frame}{Relaxed planning graphs}
  Let $\relaxation{\Pi}$ be a relaxed planning task, and let $k \in
  \mathbb N_0$.

  \medskip

  The \alert{relaxed planning graph} of $\relaxation{\Pi}$ for depth
  $k$, in symbols \alert{$\textit{RPG}_k(\relaxation{\Pi})$}, is an
  AND/OR dag that encodes
  \begin{itemize}
  \item \alert{which propositions} can be made true in $k$ plan
    steps, and
  \item \alert{how} they can be made true.
  \end{itemize}
  
  \smallskip

  Its construction is a bit involved, so we present it in stages.
\end{frame}

\begin{frame}{Running example}
  As a running example, consider the relaxed planning task
  $\langle A, I, \{o_1, o_2, o_3, o_4\}, \gamma\rangle$ with
  \begin{align*}
    A ={} & \{a, b, c, d, e, f, g, h\} \\
    I ={} & \{a \mapsto 1, b \mapsto 0, c \mapsto 1,
      d \mapsto 1, \\
      & \phantom{\{}e \mapsto 0, f \mapsto 0, g \mapsto 0, h \mapsto
      0\} \\
    o_1 ={} &
    \langle b \lor (c \land d),
    b \land ((a \land b) \CEF e)\rangle \\
    o_2 ={} & \langle \top, f\rangle \\
    o_3 ={} & \langle f, g\rangle \\
    o_4 ={} & \langle f, h\rangle \\
    \gamma ={} & e \land (g \land h)
  \end{align*}
\end{frame}

\begin{frame}{Running example: forward sets and plan steps}
  \begin{small}
    \begin{align*}
      I & = \{a \mapsto 1, b \mapsto 0, c \mapsto 1, d \mapsto 1,
      e \mapsto 0, f \mapsto 0, g \mapsto 0, h \mapsto 0\} \\
      o_1 & = \langle b \lor (c \land d),
      b \land ((a \land b) \CEF e)\rangle \\
      o_2 & = \langle \top, f\rangle, \quad
      o_3 = \langle f, g\rangle, \quad
      o_4 = \langle f, h\rangle
    \end{align*}
  \end{small}
  \vspace{-10pt}
  \begin{align*}
    S^{\text F}_0 & = \{a, c, d\} \\
    \omega^{\text F}_1 & = \{o_1, o_2\} \\
    S^{\text F}_1 & = \{a, b, c, d, f\} \\
    \omega^{\text F}_2 & = \{o_1, o_2,
    o_3, o_4\} \\
    S^{\text F}_2 & = \{a, b, c, d, e, f, g, h\} \\
    \omega^{\text F}_3 & = \omega^{\text F}_2 \\
    S^{\text F}_3 & = S^{\text F}_2 \text{~etc.}
  \end{align*}
\end{frame}

\subsection{Construction}

\begin{frame}{Components of relaxed planning graphs}
  A relaxed planning graph consists of four kinds of components:
  \begin{itemize}
  \item \alert{Proposition nodes} represent the truth value of
    propositions after applying a certain number of plan steps.
  \item \alert{Idle arcs} represent the fact that state variables,
    once true, remain true.
  \item \alert{Operator subgraphs} represent the possibility and effect
    of applying a given operator in a given plan step.
  \item The \alert{goal subgraph} represents the truth value of the
    goal condition after $k$ plan steps.
  \end{itemize}
\end{frame}

\begin{frame}{Relaxed planning graph: proposition layers}
  Let $\relaxation{\Pi} = \langle A, I, \relaxation{O}, \gamma\rangle$
  be a relaxed planning task, let $k \in \mathbb N_0$.

  \bigskip

  For each $i \in \{0, \dots, k\}$, $\textit{RPG}_k(\relaxation{\Pi})$
  contains one \alert{proposition layer} which consists of:
  \begin{itemize}
  \item a \alert{proposition node $a^{i}$} for each state variable $a
    \in A$.
  \end{itemize}

  Node $a^i$ is an AND node if $i = 0$ and $I \models a$. \\
  Otherwise, it is an OR node.
\end{frame}

\begin{frame}{Relaxed planning graph: proposition layers}
  \begin{center}
    \begin{tikzpicture}
      \pgfsetxvec{\pgfpoint{1.2cm}{0.0cm}}
      \pgfsetyvec{\pgfpoint{0.0cm}{0.96cm}}
      %% layer 0
      \node[trueinitprop] at (0,7) (a0) {$a^0$};
      \node[prop] at (0,6) (b0) {$b^0$};
      \node[trueinitprop] at (0,5) (c0) {$c^0$};
      \node[trueinitprop] at (0,4) (d0) {$d^0$};
      \node[prop] at (0,3) (e0) {$e^0$};
      \node[prop] at (0,2) (f0) {$f^0$};
      \node[prop] at (0,1) (g0) {$g^0$};
      \node[prop] at (0,0) (h0) {$h^0$};
      %% layer 1
      \pause
      \node[prop] at (1,7) (a1) {$a^1$};
      \node[prop] at (1,6) (b1) {$b^1$};
      \node[prop] at (1,5) (c1) {$c^1$};
      \node[prop] at (1,4) (d1) {$d^1$};
      \node[prop] at (1,3) (e1) {$e^1$};
      \node[prop] at (1,2) (f1) {$f^1$};
      \node[prop] at (1,1) (g1) {$g^1$};
      \node[prop] at (1,0) (h1) {$h^1$};
      %% layer 2
      \pause
      \node[prop] at (2,7) (a2) {$a^2$};
      \node[prop] at (2,6) (b2) {$b^2$};
      \node[prop] at (2,5) (c2) {$c^2$};
      \node[prop] at (2,4) (d2) {$d^2$};
      \node[prop] at (2,3) (e2) {$e^2$};
      \node[prop] at (2,2) (f2) {$f^2$};
      \node[prop] at (2,1) (g2) {$g^2$};
      \node[prop] at (2,0) (h2) {$h^2$};
      %% layer 3
      \node[prop] at (3,7) (a3) {$a^3$};
      \node[prop] at (3,6) (b3) {$b^3$};
      \node[prop] at (3,5) (c3) {$c^3$};
      \node[prop] at (3,4) (d3) {$d^3$};
      \node[prop] at (3,3) (e3) {$e^3$};
      \node[prop] at (3,2) (f3) {$f^3$};
      \node[prop] at (3,1) (g3) {$g^3$};
      \node[prop] at (3,0) (h3) {$h^3$};
      %% layer 4
      \node[prop] at (4,7) (a4) {$a^4$};
      \node[prop] at (4,6) (b4) {$b^4$};
      \node[prop] at (4,5) (c4) {$c^4$};
      \node[prop] at (4,4) (d4) {$d^4$};
      \node[prop] at (4,3) (e4) {$e^4$};
      \node[prop] at (4,2) (f4) {$f^4$};
      \node[prop] at (4,1) (g4) {$g^4$};
      \node[prop] at (4,0) (h4) {$h^4$};
      %% layer 5
      \node[prop] at (5,7) (a5) {$a^5$};
      \node[prop] at (5,6) (b5) {$b^5$};
      \node[prop] at (5,5) (c5) {$c^5$};
      \node[prop] at (5,4) (d5) {$d^5$};
      \node[prop] at (5,3) (e5) {$e^5$};
      \node[prop] at (5,2) (f5) {$f^5$};
      \node[prop] at (5,1) (g5) {$g^5$};
      \node[prop] at (5,0) (h5) {$h^5$};
      %% layer 6
      \node[prop] at (6,7) (a6) {$a^6$};
      \node[prop] at (6,6) (b6) {$b^6$};
      \node[prop] at (6,5) (c6) {$c^6$};
      \node[prop] at (6,4) (d6) {$d^6$};
      \node[prop] at (6,3) (e6) {$e^6$};
      \node[prop] at (6,2) (f6) {$f^6$};
      \node[prop] at (6,1) (g6) {$g^6$};
      \node[prop] at (6,0) (h6) {$h^6$};
      %% layer 7
      \node[prop] at (7,7) (a7) {$a^7$};
      \node[prop] at (7,6) (b7) {$b^7$};
      \node[prop] at (7,5) (c7) {$c^7$};
      \node[prop] at (7,4) (d7) {$d^7$};
      \node[prop] at (7,3) (e7) {$e^7$};
      \node[prop] at (7,2) (f7) {$f^7$};
      \node[prop] at (7,1) (g7) {$g^7$};
      \node[prop] at (7,0) (h7) {$h^7$};
      %% layer 8
      \node[prop] at (8,7) (a8) {$a^8$};
      \node[prop] at (8,6) (b8) {$b^8$};
      \node[prop] at (8,5) (c8) {$c^8$};
      \node[prop] at (8,4) (d8) {$d^8$};
      \node[prop] at (8,3) (e8) {$e^8$};
      \node[prop] at (8,2) (f8) {$f^8$};
      \node[prop] at (8,1) (g8) {$g^8$};
      \node[prop] at (8,0) (h8) {$h^8$};
    \end{tikzpicture}
  \end{center}
\end{frame}

\begin{frame}{Relaxed planning graph: idle arcs}
  For each proposition node $a^i$ with $i \in \{1, \dots, k\}$,
  $\textit{RPG}_k(\relaxation{\Pi})$ contains an arc from $a^i$ to
  $a^{i-1}$ (\alert{idle arcs}).
 
  \bigskip

  \hilite{Intuition:} If a state variable is true in step $i$, one of
  the possible reasons is that it \alert{was already previously true}.
\end{frame}

\begin{frame}{Relaxed planning graph: idle arcs}
  \begin{center}
    \begin{tikzpicture}
      \pgfsetxvec{\pgfpoint{1.2cm}{0.0cm}}
      \pgfsetyvec{\pgfpoint{0.0cm}{0.96cm}}
      %% layer 0
      \node[trueinitprop] at (0,7) (a0) {$a^0$};
      \node[prop] at (0,6) (b0) {$b^0$};
      \node[trueinitprop] at (0,5) (c0) {$c^0$};
      \node[trueinitprop] at (0,4) (d0) {$d^0$};
      \node[prop] at (0,3) (e0) {$e^0$};
      \node[prop] at (0,2) (f0) {$f^0$};
      \node[prop] at (0,1) (g0) {$g^0$};
      \node[prop] at (0,0) (h0) {$h^0$};
      %% layer 1
      \node[prop] at (1,7) (a1) {$a^1$};
      \node[prop] at (1,6) (b1) {$b^1$};
      \node[prop] at (1,5) (c1) {$c^1$};
      \node[prop] at (1,4) (d1) {$d^1$};
      \node[prop] at (1,3) (e1) {$e^1$};
      \node[prop] at (1,2) (f1) {$f^1$};
      \node[prop] at (1,1) (g1) {$g^1$};
      \node[prop] at (1,0) (h1) {$h^1$};
      %% layer 2
      \node[prop] at (2,7) (a2) {$a^2$};
      \node[prop] at (2,6) (b2) {$b^2$};
      \node[prop] at (2,5) (c2) {$c^2$};
      \node[prop] at (2,4) (d2) {$d^2$};
      \node[prop] at (2,3) (e2) {$e^2$};
      \node[prop] at (2,2) (f2) {$f^2$};
      \node[prop] at (2,1) (g2) {$g^2$};
      \node[prop] at (2,0) (h2) {$h^2$};
      %% layer 3
      \node[prop] at (3,7) (a3) {$a^3$};
      \node[prop] at (3,6) (b3) {$b^3$};
      \node[prop] at (3,5) (c3) {$c^3$};
      \node[prop] at (3,4) (d3) {$d^3$};
      \node[prop] at (3,3) (e3) {$e^3$};
      \node[prop] at (3,2) (f3) {$f^3$};
      \node[prop] at (3,1) (g3) {$g^3$};
      \node[prop] at (3,0) (h3) {$h^3$};
      %% layer 4
      \node[prop] at (4,7) (a4) {$a^4$};
      \node[prop] at (4,6) (b4) {$b^4$};
      \node[prop] at (4,5) (c4) {$c^4$};
      \node[prop] at (4,4) (d4) {$d^4$};
      \node[prop] at (4,3) (e4) {$e^4$};
      \node[prop] at (4,2) (f4) {$f^4$};
      \node[prop] at (4,1) (g4) {$g^4$};
      \node[prop] at (4,0) (h4) {$h^4$};
      %% layer 5
      \node[prop] at (5,7) (a5) {$a^5$};
      \node[prop] at (5,6) (b5) {$b^5$};
      \node[prop] at (5,5) (c5) {$c^5$};
      \node[prop] at (5,4) (d5) {$d^5$};
      \node[prop] at (5,3) (e5) {$e^5$};
      \node[prop] at (5,2) (f5) {$f^5$};
      \node[prop] at (5,1) (g5) {$g^5$};
      \node[prop] at (5,0) (h5) {$h^5$};
      %% layer 6
      \node[prop] at (6,7) (a6) {$a^6$};
      \node[prop] at (6,6) (b6) {$b^6$};
      \node[prop] at (6,5) (c6) {$c^6$};
      \node[prop] at (6,4) (d6) {$d^6$};
      \node[prop] at (6,3) (e6) {$e^6$};
      \node[prop] at (6,2) (f6) {$f^6$};
      \node[prop] at (6,1) (g6) {$g^6$};
      \node[prop] at (6,0) (h6) {$h^6$};
      %% layer 7
      \node[prop] at (7,7) (a7) {$a^7$};
      \node[prop] at (7,6) (b7) {$b^7$};
      \node[prop] at (7,5) (c7) {$c^7$};
      \node[prop] at (7,4) (d7) {$d^7$};
      \node[prop] at (7,3) (e7) {$e^7$};
      \node[prop] at (7,2) (f7) {$f^7$};
      \node[prop] at (7,1) (g7) {$g^7$};
      \node[prop] at (7,0) (h7) {$h^7$};
      %% layer 8
      \node[prop] at (8,7) (a8) {$a^8$};
      \node[prop] at (8,6) (b8) {$b^8$};
      \node[prop] at (8,5) (c8) {$c^8$};
      \node[prop] at (8,4) (d8) {$d^8$};
      \node[prop] at (8,3) (e8) {$e^8$};
      \node[prop] at (8,2) (f8) {$f^8$};
      \node[prop] at (8,1) (g8) {$g^8$};
      \node[prop] at (8,0) (h8) {$h^8$};
      %% idle arcs
      \pause
      \draw[->,idle] (a8)--(a7);
      \draw[->,idle] (a7)--(a6);
      \draw[->,idle] (a6)--(a5);
      \draw[->,idle] (a5)--(a4);
      \draw[->,idle] (a4)--(a3);
      \draw[->,idle] (a3)--(a2);
      \draw[->,idle] (a2)--(a1);
      \draw[->,idle] (a1)--(a0);
      \draw[->,idle] (b8)--(b7);
      \draw[->,idle] (b7)--(b6);
      \draw[->,idle] (b6)--(b5);
      \draw[->,idle] (b5)--(b4);
      \draw[->,idle] (b4)--(b3);
      \draw[->,idle] (b3)--(b2);
      \draw[->,idle] (b2)--(b1);
      \draw[->,idle] (b1)--(b0);
      \draw[->,idle] (c8)--(c7);
      \draw[->,idle] (c7)--(c6);
      \draw[->,idle] (c6)--(c5);
      \draw[->,idle] (c5)--(c4);
      \draw[->,idle] (c4)--(c3);
      \draw[->,idle] (c3)--(c2);
      \draw[->,idle] (c2)--(c1);
      \draw[->,idle] (c1)--(c0);
      \draw[->,idle] (d8)--(d7);
      \draw[->,idle] (d7)--(d6);
      \draw[->,idle] (d6)--(d5);
      \draw[->,idle] (d5)--(d4);
      \draw[->,idle] (d4)--(d3);
      \draw[->,idle] (d3)--(d2);
      \draw[->,idle] (d2)--(d1);
      \draw[->,idle] (d1)--(d0);
      \draw[->,idle] (e8)--(e7);
      \draw[->,idle] (e7)--(e6);
      \draw[->,idle] (e6)--(e5);
      \draw[->,idle] (e5)--(e4);
      \draw[->,idle] (e4)--(e3);
      \draw[->,idle] (e3)--(e2);
      \draw[->,idle] (e2)--(e1);
      \draw[->,idle] (e1)--(e0);
      \draw[->,idle] (f8)--(f7);
      \draw[->,idle] (f7)--(f6);
      \draw[->,idle] (f6)--(f5);
      \draw[->,idle] (f5)--(f4);
      \draw[->,idle] (f4)--(f3);
      \draw[->,idle] (f3)--(f2);
      \draw[->,idle] (f2)--(f1);
      \draw[->,idle] (f1)--(f0);
      \draw[->,idle] (g8)--(g7);
      \draw[->,idle] (g7)--(g6);
      \draw[->,idle] (g6)--(g5);
      \draw[->,idle] (g5)--(g4);
      \draw[->,idle] (g4)--(g3);
      \draw[->,idle] (g3)--(g2);
      \draw[->,idle] (g2)--(g1);
      \draw[->,idle] (g1)--(g0);
      \draw[->,idle] (h8)--(h7);
      \draw[->,idle] (h7)--(h6);
      \draw[->,idle] (h6)--(h5);
      \draw[->,idle] (h5)--(h4);
      \draw[->,idle] (h4)--(h3);
      \draw[->,idle] (h3)--(h2);
      \draw[->,idle] (h2)--(h1);
      \draw[->,idle] (h1)--(h0);
    \end{tikzpicture}
  \end{center}
\end{frame}

\begin{frame}{Relaxed planning graph: operator subgraphs}
  For each $i \in \{1, \dots, k\}$ and each operator
  $\relaxation{o} = \langle \chi, \relaxation{e}\rangle \in
    \relaxation{O}$,
  $\textit{RPG}_k(\relaxation{\Pi})$
  contains a subgraph called an \alert{operator subgraph} with the
  following parts:
  \begin{itemize}
    \only<all:1>{
    \item one \alert{formula node $n^i_\varphi$} for each
      formula $\varphi$ which is a subformula of $\chi$ or of some effect
      condition in $\relaxation{e}$:
      \begin{itemize}
      \item If $\varphi = a$ for some atom $a$,
        $n^i_\varphi$ is the proposition node $a^{i-1}$.
      \item If $\varphi = \top$, $n^i_\varphi$ is a new AND node
        without outgoing arcs.
      \item If $\varphi = \bot$, $n^i_\varphi$ is a new OR node
        without outgoing arcs.
      \item If $\varphi = (\varphi' \land \varphi'')$,
        $n^i_\varphi$ is a new AND node \\
        with outgoing arcs to $n^i_{\varphi'}$ and $n^i_{\varphi''}$.
      \item If $\varphi = (\varphi' \lor \varphi'')$,
        $n^i_\varphi$ is a new OR node \\
        with outgoing arcs to $n^i_{\varphi'}$ and $n^i_{\varphi''}$.
      \end{itemize}
    }
    \only<all:2>{
    \item for each conditional effect $(\chi' \CEF a)$ in
      $\relaxation{e}$, an \alert{effect node $o^i_{\chi'}$} (an AND
      node) with outgoing arcs to the precondition formula node
      $n^i_\chi$ and effect condition formula node $n^i_{\chi'}$, and
      incoming arc from proposition node $a^i$
      \begin{itemize}
      \item unconditional effects $a$ (effects which are not part of a
        conditional effect) are treated the same, except that there is
        no arc to an effect condition formula node
      \item effects with identical condition (including groups of
        unconditional effects) share the same effect node
      \item the effect node for unconditional effects is denoted by
        $o^i$
      \end{itemize}
    }
  \end{itemize}
\end{frame}

\begin{frame}{Relaxed planning graph: operator subgraphs}
  Operator subgraph for
  \alert{$o_1 = \langle b \lor (c \land d),
    b \land ((a \land b) \CEF e)\rangle$} \\
  for layer \alert{$i = 1$}.

  \begin{center}
    \begin{tikzpicture}
      \pgfsetxvec{\pgfpoint{7.0cm}{0.0cm}}
      \pgfsetyvec{\pgfpoint{0.0cm}{0.96cm}}
      %% layer 0
      \node[trueinitprop] at (0,7) (a0) {$a^0$};
      \node[prop] at (0,6) (b0) {$b^0$};
      \node[trueinitprop] at (0,5) (c0) {$c^0$};
      \node[trueinitprop] at (0,4) (d0) {$d^0$};
      %% layer 1
      \node[prop] at (1,6) (b1) {$b^1$};
      \node[prop] at (1,3) (e1) {$e^1$};
      %% operator
      \node[logic-and] at (0.25,4.5) (c0-and-d0) {};
      \draw[->,idle] (c0-and-d0)--(c0);
      \draw[->,idle] (c0-and-d0)--(d0);
      \node[logic-or] at (0.5,5.35) (precond-o1-0) {};
      \draw[->,idle] (precond-o1-0)--(c0-and-d0);
      \draw[->,idle] (precond-o1-0)--(b0);
      \node[logic-and] at (0.5,6.5) (effcond-o1-0) {};
      \draw[->,idle] (effcond-o1-0)--(a0);
      \draw[->,idle] (effcond-o1-0)--(b0);
      \node[operator] at (0.75,5.35) (o1-0) {$o_1^1$};
      \draw[->,idle] (o1-0)--(precond-o1-0);
      \node[logic-cond, inner sep=1pt] at (0.75,6.5) (efftrigger-o1-0) {\tiny $o_{1,a
          \land b}^1$};
      \draw[->,idle] (efftrigger-o1-0)--(effcond-o1-0);
      \draw[->,idle] (efftrigger-o1-0)--(precond-o1-0);
      \draw[->,idle] (b1)--(o1-0);
      \draw[->,idle] (e1)--(efftrigger-o1-0);
    \end{tikzpicture}
  \end{center}
\end{frame}

\begin{frame}{Relaxed planning graph: goal subgraph}
  $\textit{RPG}_k(\relaxation{\Pi})$ contains a subgraph called a
  \alert{goal subgraph} with the following parts:
  \begin{itemize}
  \item one \alert{formula node $n^k_\varphi$} for each
    formula $\varphi$ which is a subformula of $\gamma$:
    \begin{itemize}
    \item If $\varphi = a$ for some atom $a$,
      $n^k_\varphi$ is the proposition node $a^i$.
    \item If $\varphi = \top$, $n^k_\varphi$ is a new AND node
      without outgoing arcs.
    \item If $\varphi = \bot$, $n^k_\varphi$ is a new OR node
      without outgoing arcs.
    \item If $\varphi = (\varphi' \land \varphi'')$,
      $n^k_\varphi$ is a new AND node \\
      with outgoing arcs to $n^k_{\varphi'}$ and $n^k_{\varphi''}$.
    \item If $\varphi = (\varphi' \lor \varphi'')$,
      $n^k_\varphi$ is a new OR node \\
      with outgoing arcs to $n^k_{\varphi'}$ and $n^k_{\varphi''}$.
    \end{itemize}
  \end{itemize}
  The node $n^k_\gamma$ is called the \alert{goal node}.
\end{frame}

\begin{frame}{Relaxed planning graph: goal subgraphs}
  Goal subgraph for \alert{$\gamma = e \land (g \land h)$}
  and depth \alert{$k = 2$}:

  \begin{center}
    \begin{tikzpicture}
      \pgfsetxvec{\pgfpoint{7.0cm}{0.0cm}}
      \pgfsetyvec{\pgfpoint{0.0cm}{0.96cm}}
      %% layer 2
      \node[prop] at (2,3) (e2) {$e^2$};
      \node[prop] at (2,1) (g2) {$g^2$};
      \node[prop] at (2,0) (h2) {$h^2$};
      %% goal
      \node[logic-and] at (2.25,0.5) (g2-and-h2) {};
      \draw[->,idle] (g2-and-h2)--(g2);
      \draw[->,idle] (g2-and-h2)--(h2);
      \node[goal-and] at (2.5,1.75) (goal2) {$\gamma$};
      \draw[->,idle] (goal2)--(g2-and-h2);
      \draw[->,idle] (goal2)--(e2);
    \end{tikzpicture}
  \end{center}
\end{frame}

\begin{frame}{Relaxed planning graph: complete (depth 2)}
  \begin{center}
    \only<beamer>{
      \alert{
        \hspace*{\fill}
        \only<3-4>{$o_1 = \langle b \lor (c \land d),
          b \land ((a \land b) \CEF e)\rangle$}
        \only<5-6>{$o_2 = \langle \top, f\rangle$}
        \only<7-8>{$o_3 = \langle f, g\rangle$}
        \only<9-10>{$o_4 = \langle f, h\rangle$}
        \only<11>{$\gamma = e \land (g \land h)$}
        \hspace*{\fill}
      }
      \medskip

    }
    \begin{tikzpicture}
      \tikzstyle{tiny rpg}=[draw,inner sep=1pt,font=\tiny]
      \tikzstyle{rpg square}=[tiny rpg,minimum size=0.40cm,rectangle]
      \tikzstyle{rpg circle}=[tiny rpg,minimum size=0.40cm,circle]
      \pgfsetxvec{\pgfpoint{3.77cm}{0.0cm}}
      \pgfsetyvec{\pgfpoint{0.0cm}{0.87cm}}
      %% layer 0
      \node[trueinitprop] at (0.0,7.0) (a0) {$a^0$};
      \node[prop] at (0.0,6.0) (b0) {$b^0$};
      \node[trueinitprop] at (0.0,5.0) (c0) {$c^0$};
      \node[trueinitprop] at (0.0,4.0) (d0) {$d^0$};
      \node[prop] at (0.0,3.0) (e0) {$e^0$};
      \node[prop] at (0.0,2.0) (f0) {$f^0$};
      \node[prop] at (0.0,1.0) (g0) {$g^0$};
      \node[prop] at (0.0,0.0) (h0) {$h^0$};
      %% layer 1
      \node[prop] at (1.0,7.0) (a1) {$a^1$};
      \node[prop] at (1.0,6.0) (b1) {$b^1$};
      \node[prop] at (1.0,5.0) (c1) {$c^1$};
      \node[prop] at (1.0,4.0) (d1) {$d^1$};
      \node[prop] at (1.0,3.0) (e1) {$e^1$};
      \node[prop] at (1.0,2.0) (f1) {$f^1$};
      \node[prop] at (1.0,1.0) (g1) {$g^1$};
      \node[prop] at (1.0,0.0) (h1) {$h^1$};
      %% layer 2
      \node[prop] at (2.0,7.0) (a2) {$a^2$};
      \node[prop] at (2.0,6.0) (b2) {$b^2$};
      \node[prop] at (2.0,5.0) (c2) {$c^2$};
      \node[prop] at (2.0,4.0) (d2) {$d^2$};
      \node[prop] at (2.0,3.0) (e2) {$e^2$};
      \node[prop] at (2.0,2.0) (f2) {$f^2$};
      \node[prop] at (2.0,1.0) (g2) {$g^2$};
      \node[prop] at (2.0,0.0) (h2) {$h^2$};
      %% idle arcs
      \pause
      \draw[->,idle] (a2)--(a1);
      \draw[->,idle] (a1)--(a0);
      \draw[->,idle] (b2)--(b1);
      \draw[->,idle] (b1)--(b0);
      \draw[->,idle] (c2)--(c1);
      \draw[->,idle] (c1)--(c0);
      \draw[->,idle] (d2)--(d1);
      \draw[->,idle] (d1)--(d0);
      \draw[->,idle] (e2)--(e1);
      \draw[->,idle] (e1)--(e0);
      \draw[->,idle] (f2)--(f1);
      \draw[->,idle] (f1)--(f0);
      \draw[->,idle] (g2)--(g1);
      \draw[->,idle] (g1)--(g0);
      \draw[->,idle] (h2)--(h1);
      \draw[->,idle] (h1)--(h0);
      %% operator subgraph for o_1 at layer 1
      \pause
      \node[logic-and] at (0.25,4.5) (c0-and-d0) {};
      \node[logic-or] at (0.5,5.35) (precond-o1-1) {};
      \node[logic-and] at (0.5,6.5) (effcond-o1-1) {};
      \node[operator] at (0.75,5.35) (o1-1) {$o_1^1$};
      \node[logic-cond] at (0.75,6.5) (efftrigger-o1-1) {$o_{1,a \land b}^1$};
      \draw[->,nonidle] (c0-and-d0)--(c0);
      \draw[->,nonidle] (c0-and-d0)--(d0);
      \draw[->,nonidle] (precond-o1-1)--(c0-and-d0);
      \draw[->,nonidle] (precond-o1-1)--(b0);
      \draw[->,nonidle] (effcond-o1-1)--(a0);
      \draw[->,nonidle] (effcond-o1-1)--(b0);
      \draw[->,nonidle] (o1-1)--(precond-o1-1);
      \draw[->,nonidle] (efftrigger-o1-1)--(effcond-o1-1);
      \draw[->,nonidle] (efftrigger-o1-1)--(precond-o1-1);
      \draw[->,nonidle] (b1)--(o1-1);
      \draw[->,nonidle] (e1)--(efftrigger-o1-1);
     %% operator subgraph for o_1 at layer 2
     \pause
     \node[logic-and] at (1.25,4.5) (c1-and-d1) {};
     \node[logic-or] at (1.5,5.35) (precond-o1-2) {};
     \node[logic-and] at (1.5,6.5) (effcond-o1-2) {};
     \node[operator] at (1.75,5.35) (o1-2) {$o_1^2$};
     \node[logic-cond] at (1.75,6.5) (efftrigger-o1-2) {$o_{1,a \land b}^2$};
     \draw[->,nonidle] (c1-and-d1)--(c1);
     \draw[->,nonidle] (c1-and-d1)--(d1);
     \draw[->,nonidle] (precond-o1-2)--(c1-and-d1);
     \draw[->,nonidle] (precond-o1-2)--(b1);
     \draw[->,nonidle] (effcond-o1-2)--(a1);
     \draw[->,nonidle] (effcond-o1-2)--(b1);
     \draw[->,nonidle] (o1-2)--(precond-o1-2);
     \draw[->,nonidle] (efftrigger-o1-2)--(effcond-o1-2);
     \draw[->,nonidle] (efftrigger-o1-2)--(precond-o1-2);
     \draw[->,nonidle] (b2)--(o1-2);
     \draw[->,nonidle] (e2)--(efftrigger-o1-2);
     %% operator subgraph for o_2 at layer 1
     \pause
     \node[logic-and] at (0.5,2.35) (precond-o2-1) {};
     \node[operator] at (0.75,2.35) (o2-1) {$o_2^1$};
     \draw[->,nonidle] (o2-1)--(precond-o2-1);
     \draw[->,nonidle] (f1)--(o2-1);
     %% operator subgraph for o_2 at layer 2
     \pause
     \node[logic-and] at (1.5,2.35) (precond-o2-2) {};
     \node[operator] at (1.75,2.35) (o2-2) {$o_2^2$};
     \draw[->,nonidle] (o2-2)--(precond-o2-2);
     \draw[->,nonidle] (f2)--(o2-2);
     %% operator subgraph for o_3 at layer 1
     \pause
     \node[operator] at (0.75,1.35) (o3-1) {$o_3^1$};
     \draw[->,nonidle] (o3-1)--(f0);
     \draw[->,nonidle] (g1)--(o3-1);
     %% operator subgraph for o_3 at layer 2
     \pause
     \node[operator] at (1.75,1.35) (o3-2) {$o_3^2$};
     \draw[->,nonidle] (o3-2)--(f1);
     \draw[->,nonidle] (g2)--(o3-2);
     %% operator subgraph for o_4 at layer 1
     \pause
     \node[operator] at (0.75,0.35) (o4-1) {$o_4^1$};
     \draw[->,nonidle] (o4-1)--(f0);
     \draw[->,nonidle] (h1)--(o4-1);
     %% operator subgraph for o_4 at layer 2
     \pause
     \node[operator] at (1.75,0.35) (o4-2) {$o_4^2$};
     \draw[->,nonidle] (o4-2)--(f1);
     \draw[->,nonidle] (h2)--(o4-2);
     %% goal subgraph at layer 2
     \pause
     \node[logic-and] at (2.25,0.5) (g2-and-h2) {};
     \node[goal-and] at (2.5,1.75) (goal2) {$\gamma$};
     \draw[->,nonidle] (g2-and-h2)--(g2);
     \draw[->,nonidle] (g2-and-h2)--(h2);
     \draw[->,nonidle] (goal2)--(g2-and-h2);
     \draw[->,nonidle] (goal2)--(e2);
     %% complete picture
     \pause
   \end{tikzpicture}
 \end{center}
\end{frame}

\subsection{Truth values}

\begin{frame}{Connection to forward sets and plan steps}
  \begin{theorem}[relaxed planning graph truth values]
    Let $\relaxation{\Pi} = \langle A, I, \relaxation{O}, \gamma\rangle$ be
    a relaxed planning task.

    Then the truth values of the nodes of its depth-$k$ relaxed
    planning graph $\textit{RPG}_k(\relaxation{\Pi})$ relate to
    the forward sets and forward plan steps of $\relaxation{\Pi}$ as
    follows:
    \begin{itemize}
    \item \alert{Proposition nodes:} \\
      For all $a \in A$ and $i \in \{0, \dots, k\}$,
      $\textit{val}(a^i) = 1$ iff $a \in S^{\text F}_i$.
    \item \alert{(Unconditional) effect nodes:} \\
      For all $o \in \relaxation{O}$ and $i \in \{1, \dots, k\}$, 
      $\textit{val}(o^i) = 1$ iff $o \in \omega^{\text F}_i$.
    \item \alert{Goal nodes:} \\
      $\textit{val}(n^k_\gamma) = 1$ iff the parallel forward distance of
      $\relaxation{\Pi}$ is at most $k$.
    \end{itemize}
  \end{theorem}
  (We omit the straight-forward proof.)
\end{frame}

\begin{frame}<handout:12>{Computing the node truth values}
  %% TODO: Fix overly thick borders of some nodes.
  \begin{center}
    \begin{tikzpicture}
      \tikzstyle{tiny rpg}=[draw,inner sep=1pt,font=\tiny]
      \tikzstyle{rpg square}=[tiny rpg,minimum size=0.40cm,rectangle]
      \tikzstyle{rpg circle}=[tiny rpg,minimum size=0.40cm,circle]
      \pgfsetxvec{\pgfpoint{3.77cm}{0.0cm}}
      \pgfsetyvec{\pgfpoint{0.0cm}{0.87cm}}
      \only<all:1-1>{
        \node[trueinitprop] at (0.0,7.0) (a0) {$a^0$};
        \node[prop] at (0.0,6.0) (b0) {$b^0$};
        \node[trueinitprop] at (0.0,5.0) (c0) {$c^0$};
        \node[trueinitprop] at (0.0,4.0) (d0) {$d^0$};
        \node[prop] at (0.0,3.0) (e0) {$e^0$};
        \node[prop] at (0.0,2.0) (f0) {$f^0$};
        \node[prop] at (0.0,1.0) (g0) {$g^0$};
        \node[prop] at (0.0,0.0) (h0) {$h^0$};
      }
      \only<all:1-2>{
        \node[logic-and] at (0.25,4.5) (c0-and-d0) {};
      }
      \only<all:1-3>{
        \node[logic-or] at (0.5,5.35) (precond-o1-1) {};
        \node[logic-and] at (0.5,6.5) (effcond-o1-1) {};
        \node[logic-and] at (0.5,2.35) (precond-o2-1) {};
      }
      \only<all:1-4>{
        \node[logic-cond] at (0.75,6.5) (efftrigger-o1-1) {$o_{1,a \land b}^1$};
        \node[operator] at (0.75,5.35) (o1-1) {$o_1^1$};
        \node[operator] at (0.75,2.35) (o2-1) {$o_2^1$};
        \node[operator] at (0.75,1.35) (o3-1) {$o_3^1$};
        \node[operator] at (0.75,0.35) (o4-1) {$o_4^1$};
      }
      \only<all:1-5> {
        \node[prop] at (1.0,7.0) (a1) {$a^1$};
        \node[prop] at (1.0,6.0) (b1) {$b^1$};
        \node[prop] at (1.0,5.0) (c1) {$c^1$};
        \node[prop] at (1.0,4.0) (d1) {$d^1$};
        \node[prop] at (1.0,3.0) (e1) {$e^1$};
        \node[prop] at (1.0,2.0) (f1) {$f^1$};
        \node[prop] at (1.0,1.0) (g1) {$g^1$};
        \node[prop] at (1.0,0.0) (h1) {$h^1$};
      }         
      \only<all:1-6> {
        \node[logic-and] at (1.25,4.5) (c1-and-d1) {};
      }
      \only<all:1-7> {
        \node[logic-or] at (1.5,5.35) (precond-o1-2) {};
        \node[logic-and] at (1.5,6.5) (effcond-o1-2) {};
        \node[logic-and] at (1.5,2.35) (precond-o2-2) {};
      }
      \only<all:1-8> {
        \node[logic-cond] at (1.75,6.5) (efftrigger-o1-2) {$o_{1,a \land b}^2$};
        \node[operator] at (1.75,5.35) (o1-2) {$o_1^2$};
        \node[operator] at (1.75,2.35) (o2-2) {$o_2^2$};
        \node[operator] at (1.75,1.35) (o3-2) {$o_3^2$};
        \node[operator] at (1.75,0.35) (o4-2) {$o_4^2$};
      }
      \only<all:1-9> {
        \node[prop] at (2.0,7.0) (a2) {$a^2$};
        \node[prop] at (2.0,6.0) (b2) {$b^2$};
        \node[prop] at (2.0,5.0) (c2) {$c^2$};
        \node[prop] at (2.0,4.0) (d2) {$d^2$};
        \node[prop] at (2.0,3.0) (e2) {$e^2$};
        \node[prop] at (2.0,2.0) (f2) {$f^2$};
        \node[prop] at (2.0,1.0) (g2) {$g^2$};
        \node[prop] at (2.0,0.0) (h2) {$h^2$};
      }
      \only<all:1-10> {
        \node[logic-and] at (2.25,0.5) (g2-and-h2) {};
      }
      \only<all:1-11> {
        \node[goal-and] at (2.5,1.75) (goal2) {$\gamma$};
      }
      \only<all:2->{
        \node[trueinitprop,true] at (0.0,7.0) (a0) {$1$};
        \node[prop,false] at (0.0,6.0) (b0) {$0$};
        \node[trueinitprop,true] at (0.0,5.0) (c0) {$1$};
        \node[trueinitprop,true] at (0.0,4.0) (d0) {$1$};
        \node[prop,false] at (0.0,3.0) (e0) {$0$};
        \node[prop,false] at (0.0,2.0) (f0) {$0$};
        \node[prop,false] at (0.0,1.0) (g0) {$0$};
        \node[prop,false] at (0.0,0.0) (h0) {$0$};
      }
      \only<all:3->{
        \node[logic-and,true] at (0.25,4.5) (c0-and-d0) {$1$};
      }
      \only<all:4->{
        \node[logic-or,true] at (0.5,5.35) (precond-o1-1) {$1$};
        \node[logic-and,false] at (0.5,6.5) (effcond-o1-1) {$0$};
        \node[logic-and,true] at (0.5,2.35) (precond-o2-1) {$1$};
      }
      \only<all:5->{
        \node[logic-cond,false] at (0.75,6.5) (efftrigger-o1-1) {$0$};
        \node[operator,true] at (0.75,5.35) (o1-1) {$1$};
        \node[operator,true] at (0.75,2.35) (o2-1) {$1$};
        \node[operator,false] at (0.75,1.35) (o3-1) {$0$};
        \node[operator,false] at (0.75,0.35) (o4-1) {$0$};
      }
      \only<all:6-> {
        \node[prop,true] at (1.0,7.0) (a1) {$1$};
        \node[prop,true] at (1.0,6.0) (b1) {$1$};
        \node[prop,true] at (1.0,5.0) (c1) {$1$};
        \node[prop,true] at (1.0,4.0) (d1) {$1$};
        \node[prop,false] at (1.0,3.0) (e1) {$0$};
        \node[prop,true] at (1.0,2.0) (f1) {$1$};
        \node[prop,false] at (1.0,1.0) (g1) {$0$};
        \node[prop,false] at (1.0,0.0) (h1) {$0$};
      }   
      \only<all:7-> {
        \node[logic-and,true] at (1.25,4.5) (c1-and-d1) {$1$};
      }
      \only<all:8-> {
        \node[logic-or,true] at (1.5,5.35) (precond-o1-2) {$1$};
        \node[logic-and,true] at (1.5,6.5) (effcond-o1-2) {$1$};
        \node[logic-and,true] at (1.5,2.35) (precond-o2-2) {$1$};
      }
      \only<all:9-> {
        \node[logic-cond,true] at (1.75,6.5) (efftrigger-o1-2) {$1$};
        \node[operator,true] at (1.75,5.35) (o1-2) {$1$};
        \node[operator,true] at (1.75,2.35) (o2-2) {$1$};
        \node[operator,true] at (1.75,1.35) (o3-2) {$1$};
        \node[operator,true] at (1.75,0.35) (o4-2) {$1$};
      }
      \only<all:10-> {
        \node[prop,true] at (2.0,7.0) (a2) {$1$};
        \node[prop,true] at (2.0,6.0) (b2) {$1$};
        \node[prop,true] at (2.0,5.0) (c2) {$1$};
        \node[prop,true] at (2.0,4.0) (d2) {$1$};
        \node[prop,true] at (2.0,3.0) (e2) {$1$};
        \node[prop,true] at (2.0,2.0) (f2) {$1$};
        \node[prop,true] at (2.0,1.0) (g2) {$1$};
        \node[prop,true] at (2.0,0.0) (h2) {$1$};
      }
      \only<all:11-> {
        \node[logic-and,true] at (2.25,0.5) (g2-and-h2) {$1$};
      }
      \only<all:12-> {
        \node[goal-and,true] at (2.5,1.75) (goal2) {$1$};
      }
      
      \draw[->,idle] (a2)--(a1);
      \draw[->,idle] (a1)--(a0);
      \draw[->,idle] (b2)--(b1);
      \draw[->,idle] (b1)--(b0);
      \draw[->,idle] (c2)--(c1);
      \draw[->,idle] (c1)--(c0);
      \draw[->,idle] (d2)--(d1);
      \draw[->,idle] (d1)--(d0);
      \draw[->,idle] (e2)--(e1);
      \draw[->,idle] (e1)--(e0);
      \draw[->,idle] (f2)--(f1);
      \draw[->,idle] (f1)--(f0);
      \draw[->,idle] (g2)--(g1);
      \draw[->,idle] (g1)--(g0);
      \draw[->,idle] (h2)--(h1);
      \draw[->,idle] (h1)--(h0);
      \draw[->,nonidle] (c0-and-d0)--(c0);
      \draw[->,nonidle] (c0-and-d0)--(d0);
      \draw[->,nonidle] (precond-o1-1)--(c0-and-d0);
      \draw[->,nonidle] (precond-o1-1)--(b0);
      \draw[->,nonidle] (effcond-o1-1)--(a0);
      \draw[->,nonidle] (effcond-o1-1)--(b0);
      \draw[->,nonidle] (o1-1)--(precond-o1-1);
      \draw[->,nonidle] (efftrigger-o1-1)--(effcond-o1-1);
      \draw[->,nonidle] (efftrigger-o1-1)--(precond-o1-1);
      \draw[->,nonidle] (b1)--(o1-1);
      \draw[->,nonidle] (e1)--(efftrigger-o1-1);
      \draw[->,nonidle] (c1-and-d1)--(c1);
      \draw[->,nonidle] (c1-and-d1)--(d1);
      \draw[->,nonidle] (precond-o1-2)--(c1-and-d1);
      \draw[->,nonidle] (precond-o1-2)--(b1);
      \draw[->,nonidle] (effcond-o1-2)--(a1);
      \draw[->,nonidle] (effcond-o1-2)--(b1);
      \draw[->,nonidle] (o1-2)--(precond-o1-2);
      \draw[->,nonidle] (efftrigger-o1-2)--(effcond-o1-2);
      \draw[->,nonidle] (efftrigger-o1-2)--(precond-o1-2);
      \draw[->,nonidle] (b2)--(o1-2);
      \draw[->,nonidle] (e2)--(efftrigger-o1-2);

      \draw[->,nonidle] (o2-1)--(precond-o2-1);
      \draw[->,nonidle] (f1)--(o2-1);

      \draw[->,nonidle] (o2-2)--(precond-o2-2);
      \draw[->,nonidle] (f2)--(o2-2);
      
      \draw[->,nonidle] (o3-1)--(f0);
      \draw[->,nonidle] (g1)--(o3-1);
      
      \draw[->,nonidle] (o3-2)--(f1);
      \draw[->,nonidle] (g2)--(o3-2);
      
      \draw[->,nonidle] (o4-1)--(f0);
      \draw[->,nonidle] (h1)--(o4-1);
      
      \draw[->,nonidle] (o4-2)--(f1);
      \draw[->,nonidle] (h2)--(o4-2);
      
      \draw[->,nonidle] (g2-and-h2)--(g2);
      \draw[->,nonidle] (g2-and-h2)--(h2);
      \draw[->,nonidle] (goal2)--(g2-and-h2);
      \draw[->,nonidle] (goal2)--(e2);
    \end{tikzpicture}
  \end{center}
\end{frame}

\begin{frame}{Relaxed planning graphs for STRIPS}
  \hilite{Remark:} Relaxed planning graphs have historically been
  defined for STRIPS tasks only. In this case, we can simplify:
  \begin{itemize}
  \item \alert{Only one effect node per operator:} STRIPS does not
    have conditional effects.
    \begin{itemize}
    \item Because each operator has only one effect node, effect nodes
      are called \alert{operator nodes} in relaxed planning graphs for
      STRIPS.
    \end{itemize}
  \item \alert{No goal nodes:}
    The test whether all goals are reached is done by the algorithm
    that evaluates the AND/OR dag.
  \item \alert{No formula nodes:} Operator nodes are directly connected
    to their preconditions.
  \end{itemize}
  $\leadsto$ Relaxed planning graphs for STRIPS are \alert{layered}
    digraphs and only have \alert{proposition and operator nodes}.
\end{frame}

\section{Relaxation heuristics}
\subsection[Generic template]{Generic template for relaxation heuristics}

\begin{frame}{Computing parallel forward distances from RPGs}
  So far, relaxed planning graphs offer us a way to compute parallel
  forward distances:
  \begin{block}{Parallel forward distances from relaxed planning
      graphs}
    \textbf{def}
    $\textit{parallel-forward-distance}(\relaxation{\Pi})$: \\
    {}\qquad Let $A$ be the set of state variables of
    $\relaxation{\Pi}$. \\
    {}\qquad\textbf{for} $k \in \{0, 1, 2, \dots\}$: \\
    {}\qquad\qquad $\textit{rpg} := \textit{RPG}_k(\relaxation{\Pi})$ \\
    {}\qquad\qquad Evaluate truth values for \textit{rpg}. \\
    {}\qquad\qquad \textbf{if} goal node of \textit{rpg}
    has value $1$: \\
    {}\qquad\qquad\qquad \textbf{return} $k$ \\
    {}\qquad\qquad \textbf{else if} $k = |A|$: \\
    {}\qquad\qquad\qquad \textbf{return} $\infty$
  \end{block}
\end{frame}

\begin{frame}{Remarks on the algorithm}
  \begin{itemize}
  \item The relaxed planning graph for depth $k \ge 1$ can be built
    \alert{incrementally} from the one for depth $k-1$:
    \begin{itemize}
    \item Add new layer $k$.
    \item Move goal subgraph from layer $k-1$ to layer $k$.
    \end{itemize}
  \item Similarly, all truth values up to layer $k-1$ can be reused.
  \item Thus, overall computation with maximal depth $m$ requires time
    $O(\|\textit{RPG}_m(\relaxation{\Pi})\|) = O((m+1) \cdot
    \|\relaxation{\Pi}\|)$.
  \item This is not a very efficient way of computing parallel forward
    distances (and wouldn't be used in practice).
  \item However, it allows computing \alert{additional information}
    for the relaxed planning graph nodes along the way, which can be
    used for heuristic estimates.
  \end{itemize}
\end{frame}

\begin{frame}{Generic relaxed planning graph heuristics}
  \begin{block}{Computing heuristics from relaxed planning graphs}
    \textbf{def}
    $\textit{generic-rpg-heuristic}(\langle A, I, O,
    \gamma\rangle, s)$: \\
    {}\qquad $\relaxation{\Pi} := \langle A, s, \relaxation{O},
    \gamma\rangle$ \\
    {}\qquad\textbf{for} $k \in \{0, 1, 2, \dots\}$: \\
    {}\qquad\qquad $\textit{rpg} := \textit{RPG}_k(\relaxation{\Pi})$
    \\
    {}\qquad\qquad Evaluate truth values for \textit{rpg}. \\
    {}\qquad\qquad \textbf{if} goal node of \textit{rpg}
    has value $1$: \\
    {}\qquad\qquad\qquad \hilite{Annotate true nodes of \textit{rpg}.}
    \\
    \qquad\qquad\qquad \textbf{if} \hilite{termination criterion is
    true}: \\
    \qquad\qquad\qquad\qquad \textbf{return} \hilite{heuristic value
    from annotations} \\
    \qquad\qquad \textbf{else if} $k = |A|$: \\
    {}\qquad\qquad\qquad \textbf{return} $\infty$
  \end{block}
  \begin{itemize}
  \item[$\leadsto$] \alert{generic template} for heuristic functions
  \item[$\leadsto$] to get \alert{concrete heuristic}: fill in
    \hilite{highlighted parts}
  \end{itemize}
\end{frame}

\begin{frame}{Concrete examples for the generic heuristic}
  Many planning heuristics fit the generic template:
  \begin{itemize}
  \item \alert{additive heuristic \hadd}
    (Bonet, Loerincs \& Geffner, 1997)
  \item \alert{max heuristic \hmax}
    (Bonet \& Geffner, 1999)
  \item \alert{FF heuristic \hff}
    (Hoffmann \& Nebel, 2001)
  \item \alert{cost-sharing heuristic \hcs}
    (Mirkis \& Domshlak, 2007)
    \begin{itemize}
    \item not covered in this course
    \end{itemize}
  \item \alert{set-additive heuristic \hsa}
    (Keyder \& Geffner, 2008)
  \end{itemize}
  \hilite{Remarks:}
  \begin{itemize}
  \item For all these heuristics, equivalent definitions that don't
    refer to relaxed planning graphs are possible.
  \item Historically, such equivalent definitions have mostly been
    used for $\hmax$, $\hadd$ and $\hsa$.
  \item For those heuristics, the most efficient implementations do
    not use relaxed planning graphs explicitly.
  \end{itemize}
\end{frame}

\begin{frame}{Forward cost heuristics}
  \begin{itemize}
  \item The simplest relaxed planning graph heuristics are
    \alert{forward cost heuristics}.
  \item Examples: \alert{$\hmax$}, \alert{$\hadd$}
  \item Here, node annotations are \alert{cost values} (natural
    numbers).
  \item The cost of a node estimates how expensive (in terms
    of required operators) it is to make this node true.
  \end{itemize}
\end{frame}

\begin{frame}{Forward cost heuristics: fitting the template}
  \begin{block}{Forward cost heuristics}
    \hilite{Computing annotations:}
    \begin{itemize}
    \item Propagate cost values bottom-up using a \hilite{combination
      rule for OR nodes} and a \hilite{combination rule for AND nodes}.
    \item At \alert{effect nodes}, \alert{add 1} after applying
      combination rule.
    \end{itemize}

    \hilite{Termination criterion:}
    \begin{itemize}
    \item \alert{stability:} terminate if cost for proposition node
      $a^k$ equals cost for $a^{k-1}$ for all true propositions $a$ in
      layer $k$
    \end{itemize}
    
    \hilite{Heuristic value:}
    \begin{itemize}
    \item The heuristic value is the cost of the goal node.
    \end{itemize}
  \end{block}
  \begin{itemize}
  \item Different forward cost heuristics only differ in their choice
    of combination rules.
  \end{itemize}
\end{frame}

\subsection[\hmax]{The max heuristic \hmax}

\begin{frame}{The max heuristic {\hmax} (again)}
  \begin{block}{Forward cost heuristics: max heuristic \hmax}
    \hilite{Combination rule for AND nodes:}
    \begin{itemize}
    \item $\textit{cost}(u) = \alert{\max} (\{\textit{cost}(v_1), \dots,
      \textit{cost}(v_k)\})$ \\
      (with $\max(\emptyset) := 0$)
    \end{itemize}

    \hilite{Combination rule for OR nodes:}
    \begin{itemize}
    \item $\textit{cost}(u) = \min(\{\textit{cost}(v_1), \dots,
      \textit{cost}(v_k)\})$
    \end{itemize}

    In both cases, $\{v_1, \dots, v_k\}$ is the set of true
    successors of $u$.
  \end{block}

  \hilite{Intuition:}
  \begin{itemize}
  \item \hilite{AND rule:} If we have to achieve several conditions,
    estimate this by the \alert{most expensive} cost.
  \item \hilite{OR rule:} If we have a choice how to achieve a
    condition, pick the \alert{cheapest} possibility.
  \end{itemize}
\end{frame}

\begin{frame}{Running example: \hmax}
  \begin{center}
    \begin{tikzpicture}
      \tikzstyle{tiny rpg}=[draw,inner sep=1pt,font=\tiny]
      \tikzstyle{rpg square}=[tiny rpg,minimum size=0.40cm,rectangle]
      \tikzstyle{rpg circle}=[tiny rpg,minimum size=0.40cm,circle]
      \pgfsetxvec{\pgfpoint{3.77cm}{0.0cm}}
      \pgfsetyvec{\pgfpoint{0.0cm}{0.87cm}}

      \node[trueinitprop,true] at (0.0,7.0) (a0) {\only<2->{$0$}};
      \node[prop,false] at (0.0,6.0) (b0) {};
      \node[trueinitprop,true] at (0.0,5.0) (c0) {\only<2->{$0$}};
      \node[trueinitprop,true] at (0.0,4.0) (d0) {\only<2->{$0$}};
      \node[prop,false] at (0.0,3.0) (e0) {};
      \node[prop,false] at (0.0,2.0) (f0) {};
      \node[prop,false] at (0.0,1.0) (g0) {};
      \node[prop,false] at (0.0,0.0) (h0) {};

      \node[logic-and,true] at (0.25,4.5) (c0-and-d0) {\only<3->{$0$}};

      \node[logic-or,true] at (0.5,5.35) (precond-o1-1) {\only<4->{$0$}};
      \node[logic-and,false] at (0.5,6.5) (effcond-o1-1) {};
      \node[logic-and,true] at (0.5,2.35) (precond-o2-1) {\only<4->{$0$}};

      \node[logic-cond,false] at (0.75,6.5) (efftrigger-o1-1) {};
      \node[operator,true] at (0.75,5.35) (o1-1) {\only<5->{$1$}};
      \markplusone{o1-1}
      \node[operator,true] at (0.75,2.35) (o2-1) {\only<5->{$1$}};
      \markplusone{o2-1}
      \node[operator,false] at (0.75,1.35) (o3-1) {};
      \node[operator,false] at (0.75,0.35) (o4-1) {};

      \node[prop,true] at (1.0,7.0) (a1) {\only<6->{$0$}};
      \node[prop,true] at (1.0,6.0) (b1) {\only<6->{$1$}};
      \node[prop,true] at (1.0,5.0) (c1) {\only<6->{$0$}};
      \node[prop,true] at (1.0,4.0) (d1) {\only<6->{$0$}};
      \node[prop,false] at (1.0,3.0) (e1) {};
      \node[prop,true] at (1.0,2.0) (f1) {\only<6->{$1$}};
      \node[prop,false] at (1.0,1.0) (g1) {};
      \node[prop,false] at (1.0,0.0) (h1) {};

      \node[logic-and,true] at (1.25,4.5) (c1-and-d1) {\only<7->{$0$}};

      \node[logic-or,true] at (1.5,5.35) (precond-o1-2) {\only<8->{$0$}};
      \node[logic-and,true] at (1.5,6.5) (effcond-o1-2) {\only<8->{$1$}};
      \node[logic-and,true] at (1.5,2.35) (precond-o2-2) {\only<8->{$0$}};

      \node[logic-cond,true] at (1.75,6.5) (efftrigger-o1-2) {\only<9->{$2$}};
      \markplusone{efftrigger-o1-2}
      \node[operator,true] at (1.75,5.35) (o1-2) {\only<9->{$1$}};
      \markplusone{o1-2}
      \node[operator,true] at (1.75,2.35) (o2-2) {\only<9->{$1$}};
      \markplusone{o2-2}
      \node[operator,true] at (1.75,1.35) (o3-2) {\only<9->{$2$}};
      \markplusone{o3-2}
      \node[operator,true] at (1.75,0.35) (o4-2) {\only<9->{$2$}};
      \markplusone{o4-2}

      \node[prop,true] at (2.0,7.0) (a2) {\only<10->{$0$}};
      \node[prop,true] at (2.0,6.0) (b2) {\only<10->{$1$}};
      \node[prop,true] at (2.0,5.0) (c2) {\only<10->{$0$}};
      \node[prop,true] at (2.0,4.0) (d2) {\only<10->{$0$}};
      \node[prop,true] at (2.0,3.0) (e2) {\only<10->{$2$}};
      \node[prop,true] at (2.0,2.0) (f2) {\only<10->{$1$}};
      \node[prop,true] at (2.0,1.0) (g2) {\only<10->{$2$}};
      \node[prop,true] at (2.0,0.0) (h2) {\only<10->{$2$}};

      \node[logic-and,true] at (2.25,0.5) (g2-and-h2) {\only<11->{$2$}};

      \node[goal-and,true] at (2.5,1.75) (goal2) {\only<12->{$2$}};
      
      \draw[->,idle] (a2)--(a1);
      \draw[->,idle] (a1)--(a0);
      \draw[->,idle] (b2)--(b1);
      \draw[->,idle] (b1)--(b0);
      \draw[->,idle] (c2)--(c1);
      \draw[->,idle] (c1)--(c0);
      \draw[->,idle] (d2)--(d1);
      \draw[->,idle] (d1)--(d0);
      \draw[->,idle] (e2)--(e1);
      \draw[->,idle] (e1)--(e0);
      \draw[->,idle] (f2)--(f1);
      \draw[->,idle] (f1)--(f0);
      \draw[->,idle] (g2)--(g1);
      \draw[->,idle] (g1)--(g0);
      \draw[->,idle] (h2)--(h1);
      \draw[->,idle] (h1)--(h0);
      \draw[->,nonidle] (c0-and-d0)--(c0);
      \draw[->,nonidle] (c0-and-d0)--(d0);
      \draw[->,nonidle] (precond-o1-1)--(c0-and-d0);
      \draw[->,nonidle] (precond-o1-1)--(b0);
      \draw[->,nonidle] (effcond-o1-1)--(a0);
      \draw[->,nonidle] (effcond-o1-1)--(b0);
      \draw[->,nonidle] (o1-1)--(precond-o1-1);
      \draw[->,nonidle] (efftrigger-o1-1)--(effcond-o1-1);
      \draw[->,nonidle] (efftrigger-o1-1)--(precond-o1-1);
      \draw[->,nonidle] (b1)--(o1-1);
      \draw[->,nonidle] (e1)--(efftrigger-o1-1);
      \draw[->,nonidle] (c1-and-d1)--(c1);
      \draw[->,nonidle] (c1-and-d1)--(d1);
      \draw[->,nonidle] (precond-o1-2)--(c1-and-d1);
      \draw[->,nonidle] (precond-o1-2)--(b1);
      \draw[->,nonidle] (effcond-o1-2)--(a1);
      \draw[->,nonidle] (effcond-o1-2)--(b1);
      \draw[->,nonidle] (o1-2)--(precond-o1-2);
      \draw[->,nonidle] (efftrigger-o1-2)--(effcond-o1-2);
      \draw[->,nonidle] (efftrigger-o1-2)--(precond-o1-2);
      \draw[->,nonidle] (b2)--(o1-2);
      \draw[->,nonidle] (e2)--(efftrigger-o1-2);

      \draw[->,nonidle] (o2-1)--(precond-o2-1);
      \draw[->,nonidle] (f1)--(o2-1);

      \draw[->,nonidle] (o2-2)--(precond-o2-2);
      \draw[->,nonidle] (f2)--(o2-2);
      
      \draw[->,nonidle] (o3-1)--(f0);
      \draw[->,nonidle] (g1)--(o3-1);
      
      \draw[->,nonidle] (o3-2)--(f1);
      \draw[->,nonidle] (g2)--(o3-2);
				   
      \draw[->,nonidle] (o4-1)--(f0);
      \draw[->,nonidle] (h1)--(o4-1);
      
      \draw[->,nonidle] (o4-2)--(f1);
      \draw[->,nonidle] (h2)--(o4-2);
      
      \draw[->,nonidle] (g2-and-h2)--(g2);
      \draw[->,nonidle] (g2-and-h2)--(h2);
      \draw[->,nonidle] (goal2)--(g2-and-h2);
      \draw[->,nonidle] (goal2)--(e2);
    \end{tikzpicture}
  \end{center}
\end{frame}

\begin{frame}{Remarks on {\hmax}}
  \begin{itemize}
  \item The definition of {\hmax} as a forward cost heuristic is
    equivalent to our earlier definition in this chapter.
  \item Unlike the earlier definition, it generalizes to an extension
    where every operator has an associated non-negative \alert{cost}
    (rather than all operators having cost $1$).
  \item In the case without costs (and only then), it is easy to prove
    that the goal node has the same cost in all graphs
    $\textit{RPG}_k(\relaxation{\Pi})$ where it is true. (Namely, the
    cost is equal to the lowest value of $k$ for which the goal
    node is true.)
  \item We can thus terminate the computation as soon as the goal
    becomes true, without waiting for stability.
  \item The same is \alert{not true} for other forward-propagating
    heuristics ($\hadd$, $\hcs$, $\hsa$).
  \end{itemize}
\end{frame}

\subsection[\hadd]{The additive heuristic \hadd}

\begin{frame}{The additive heuristic}
  \begin{block}{Forward cost heuristics: additive heuristic \hadd}
    \hilite{Combination rule for AND nodes:}
    \begin{itemize}
    \item $\textit{cost}(u) = \textit{cost}(v_1) \alert{+} \dots
      \alert{+} \textit{cost}(v_k)$ \\
      (with $\sum(\emptyset) := 0$)
    \end{itemize}

    \hilite{Combination rule for OR nodes:}
    \begin{itemize}
    \item $\textit{cost}(u) = \min(\{\textit{cost}(v_1), \dots,
      \textit{cost}(v_k)\})$
    \end{itemize}

    In both cases, $\{v_1, \dots, v_k\}$ is the set of true
    successors of $u$.
  \end{block}

  \hilite{Intuition:}
  \begin{itemize}
  \item \hilite{AND rule:} If we have to achieve several conditions,
    estimate this by the cost of achieving \alert{each in isolation}.
  \item \hilite{OR rule:} If we have a choice how to achieve a
    condition, pick the \alert{cheapest} possibility.
  \end{itemize}
\end{frame}

\begin{frame}{Running example: \hadd}
  \begin{center}
    \begin{tikzpicture}
      \tikzstyle{tiny rpg}=[draw,inner sep=1pt,font=\tiny]
      \tikzstyle{rpg square}=[tiny rpg,minimum size=0.40cm,rectangle]
      \tikzstyle{rpg circle}=[tiny rpg,minimum size=0.40cm,circle]
      \pgfsetxvec{\pgfpoint{3.77cm}{0.0cm}}
      \pgfsetyvec{\pgfpoint{0.0cm}{0.87cm}}

      \node[trueinitprop,true] at (0.0,7.0) (a0) {\only<2->{$0$}};
      \node[prop,false] at (0.0,6.0) (b0) {};
      \node[trueinitprop,true] at (0.0,5.0) (c0) {\only<2->{$0$}};
      \node[trueinitprop,true] at (0.0,4.0) (d0) {\only<2->{$0$}};
      \node[prop,false] at (0.0,3.0) (e0) {};
      \node[prop,false] at (0.0,2.0) (f0) {};
      \node[prop,false] at (0.0,1.0) (g0) {};
      \node[prop,false] at (0.0,0.0) (h0) {};

      \node[logic-and,true] at (0.25,4.5) (c0-and-d0) {\only<3->{$0$}};

      \node[logic-or,true] at (0.5,5.35) (precond-o1-1) {\only<4->{$0$}};
      \node[logic-and,false] at (0.5,6.5) (effcond-o1-1) {};
      \node[logic-and,true] at (0.5,2.35) (precond-o2-1) {\only<4->{$0$}};

      \node[logic-cond,false] at (0.75,6.5) (efftrigger-o1-1) {};
      \node[operator,true] at (0.75,5.35) (o1-1) {\only<5->{$1$}};
      \markplusone{o1-1}
      \node[operator,true] at (0.75,2.35) (o2-1) {\only<5->{$1$}};
      \markplusone{o2-1}
      \node[operator,false] at (0.75,1.35) (o3-1) {};
      \node[operator,false] at (0.75,0.35) (o4-1) {};

      \node[prop,true] at (1.0,7.0) (a1) {\only<6->{$0$}};
      \node[prop,true] at (1.0,6.0) (b1) {\only<6->{$1$}};
      \node[prop,true] at (1.0,5.0) (c1) {\only<6->{$0$}};
      \node[prop,true] at (1.0,4.0) (d1) {\only<6->{$0$}};
      \node[prop,false] at (1.0,3.0) (e1) {};
      \node[prop,true] at (1.0,2.0) (f1) {\only<6->{$1$}};
      \node[prop,false] at (1.0,1.0) (g1) {};
      \node[prop,false] at (1.0,0.0) (h1) {};

      \node[logic-and,true] at (1.25,4.5) (c1-and-d1) {\only<7->{$0$}};

      \node[logic-or,true] at (1.5,5.35) (precond-o1-2) {\only<8->{$0$}};
      \node[logic-and,true] at (1.5,6.5) (effcond-o1-2) {\only<8->{$1$}};
      \node[logic-and,true] at (1.5,2.35) (precond-o2-2) {\only<8->{$0$}};

      \node[logic-cond,true] at (1.75,6.5) (efftrigger-o1-2) {\only<9->{$2$}};
      \markplusone{efftrigger-o1-2}
      \node[operator,true] at (1.75,5.35) (o1-2) {\only<9->{$1$}};
      \markplusone{o1-2}
      \node[operator,true] at (1.75,2.35) (o2-2) {\only<9->{$1$}};
      \markplusone{o2-2}
      \node[operator,true] at (1.75,1.35) (o3-2) {\only<9->{$2$}};
      \markplusone{o3-2}
      \node[operator,true] at (1.75,0.35) (o4-2) {\only<9->{$2$}};
      \markplusone{o4-2}

      \node[prop,true] at (2.0,7.0) (a2) {\only<10->{$0$}};
      \node[prop,true] at (2.0,6.0) (b2) {\only<10->{$1$}};
      \node[prop,true] at (2.0,5.0) (c2) {\only<10->{$0$}};
      \node[prop,true] at (2.0,4.0) (d2) {\only<10->{$0$}};
      \node[prop,true] at (2.0,3.0) (e2) {\only<10->{$2$}};
      \node[prop,true] at (2.0,2.0) (f2) {\only<10->{$1$}};
      \node[prop,true] at (2.0,1.0) (g2) {\only<10->{$2$}};
      \node[prop,true] at (2.0,0.0) (h2) {\only<10->{$2$}};

      \node[logic-and,true] at (2.25,0.5) (g2-and-h2) {\only<11->{$4$}};

      \node[goal-and,true] at (2.5,1.75) (goal2) {\only<12->{$6$}};
      
      \draw[->,idle] (a2)--(a1);
      \draw[->,idle] (a1)--(a0);
      \draw[->,idle] (b2)--(b1);
      \draw[->,idle] (b1)--(b0);
      \draw[->,idle] (c2)--(c1);
      \draw[->,idle] (c1)--(c0);
      \draw[->,idle] (d2)--(d1);
      \draw[->,idle] (d1)--(d0);
      \draw[->,idle] (e2)--(e1);
      \draw[->,idle] (e1)--(e0);
      \draw[->,idle] (f2)--(f1);
      \draw[->,idle] (f1)--(f0);
      \draw[->,idle] (g2)--(g1);
      \draw[->,idle] (g1)--(g0);
      \draw[->,idle] (h2)--(h1);
      \draw[->,idle] (h1)--(h0);
      \draw[->,nonidle] (c0-and-d0)--(c0);
      \draw[->,nonidle] (c0-and-d0)--(d0);
      \draw[->,nonidle] (precond-o1-1)--(c0-and-d0);
      \draw[->,nonidle] (precond-o1-1)--(b0);
      \draw[->,nonidle] (effcond-o1-1)--(a0);
      \draw[->,nonidle] (effcond-o1-1)--(b0);
      \draw[->,nonidle] (o1-1)--(precond-o1-1);
      \draw[->,nonidle] (efftrigger-o1-1)--(effcond-o1-1);
      \draw[->,nonidle] (efftrigger-o1-1)--(precond-o1-1);
      \draw[->,nonidle] (b1)--(o1-1);
      \draw[->,nonidle] (e1)--(efftrigger-o1-1);
      \draw[->,nonidle] (c1-and-d1)--(c1);
      \draw[->,nonidle] (c1-and-d1)--(d1);
      \draw[->,nonidle] (precond-o1-2)--(c1-and-d1);
      \draw[->,nonidle] (precond-o1-2)--(b1);
      \draw[->,nonidle] (effcond-o1-2)--(a1);
      \draw[->,nonidle] (effcond-o1-2)--(b1);
      \draw[->,nonidle] (o1-2)--(precond-o1-2);
      \draw[->,nonidle] (efftrigger-o1-2)--(effcond-o1-2);
      \draw[->,nonidle] (efftrigger-o1-2)--(precond-o1-2);
      \draw[->,nonidle] (b2)--(o1-2);
      \draw[->,nonidle] (e2)--(efftrigger-o1-2);

      \draw[->,nonidle] (o2-1)--(precond-o2-1);
      \draw[->,nonidle] (f1)--(o2-1);

      \draw[->,nonidle] (o2-2)--(precond-o2-2);
      \draw[->,nonidle] (f2)--(o2-2);
      
      \draw[->,nonidle] (o3-1)--(f0);
      \draw[->,nonidle] (g1)--(o3-1);
      
      \draw[->,nonidle] (o3-2)--(f1);
      \draw[->,nonidle] (g2)--(o3-2);
				   
      \draw[->,nonidle] (o4-1)--(f0);
      \draw[->,nonidle] (h1)--(o4-1);
      
      \draw[->,nonidle] (o4-2)--(f1);
      \draw[->,nonidle] (h2)--(o4-2);
      
      \draw[->,nonidle] (g2-and-h2)--(g2);
      \draw[->,nonidle] (g2-and-h2)--(h2);
      \draw[->,nonidle] (goal2)--(g2-and-h2);
      \draw[->,nonidle] (goal2)--(e2);
    \end{tikzpicture}
  \end{center}
\end{frame}

\begin{frame}{Remarks on {\hadd}}
  \begin{itemize}
  \item It is important to test for stability in computing \hadd! \\
    (The reason for this is that, unlike \hmax, cost values of true
    propositions can \alert{decrease} from layer to layer.)
  \item Stability is achieved after layer $|A|$ in the worst case.
  \item {\hadd} is \alert{safe} and \alert{goal-aware}.
  \item Unlike {\hmax}, \alert{\hadd} is a \alert{very informative}
    heuristic in many planning domains.
  \item The price for this is that it is \alert{not admissible} (and
    hence also \alert{not consistent}), so not suitable for optimal
    planning.
  \item In fact, it \alert{almost always} overestimates the {\hplus}
    value because it does not take \alert{positive interactions} into
    account.
  \end{itemize}
\end{frame}

\subsection[\hsa]{The set-additive heuristic \hsa}

\begin{frame}{The set-additive heuristic}
  \begin{itemize}
  \item We now discuss a refinement of the additive heuristic
    called the \alert{set-additive heuristic} \alert{\hsa}.
  \item The set-additive heuristic addresses the problem that
    $\hadd$ does not take positive interactions into account.
  \item Like $\hmax$ and $\hadd$, $\hsa$ is calculated through
    \alert{forward propagation} of node annotations.
  \item However, the node annotations are not cost values,
    but \alert{sets of operators} (kind of).
  \item The idea is that by taking \alert{set unions} instead of
    \alert{adding costs}, operators needed only once are
    \alert{counted only once}.
  \end{itemize}
  \begin{scriptsize}
    \hilite{Disclaimer:} There are some quite subtle
    differences between the {\hsa} heuristic as we describe it here and
    the ``real'' heuristic of Keyder \& Geffner. We do not want to
    discuss this in detail, but please note that such differences
    exist.
  \end{scriptsize}
  %% For example, "our" h_sa can lead to relaxed plans of length
  %% greater than the number of propositions, while the ECAI 2008
  %% paper version cannot. However, Emil's actual *implementation*
  %% appears to look more like what we do here, and there's some
  %% reason for that because it's not quite clear how to efficiently
  %% implement the h_sa heuristic as defined in the ECAI paper. See
  %% the discussion on May 15th 2009 in the email folder for this
  %% lecture for details.
\end{frame}

\begin{frame}{Operators needed several times}
  \begin{itemize}
  \item The original {\hsa} heuristic as described in the literature
    is defined for STRIPS tasks and propagates \alert{sets of
      operators}.
  \item This is fine because in relaxed STRIPS tasks, each operator
    \alert{need only be applied once}.
  \item The same is \alert{not true in general}: in our running
    example, operator $o_1$ must be applied twice in the relaxed plan.
  \item In general, it only makes sense to apply an operator again in
    a relaxed planning task if a \alert{previously unsatisfied effect
      condition} has been made true.
  \item For this reason, we keep track of \alert{operator/effect
    condition pairs} rather than just plain operators.
  \end{itemize}
\end{frame}

\begin{frame}{Set-additive heuristic: fitting the
    template\only<all:2>{ (ctd.)}}
  \begin{block}{The set-additive heuristic {\hsa}\only<all:2>{ (ctd.)}}
    \begin{overprint}
    \onslide<all:1>
    \hilite{Computing annotations:}
    \begin{itemize}
    \item Annotations are \alert{sets of operator/effect condition
      pairs}, computed bottom-up.
      \smallskip

      \hilite{Combination rule for AND nodes:}
      \begin{itemize}
      \item $\textit{ann}(u) = \textit{ann}(v_1) \cup \dots
	\cup \textit{ann}(v_k)$
	(with $\bigcup(\emptyset) := \emptyset$)
      \end{itemize}

      \hilite{Combination rule for OR nodes:}
      \begin{itemize}
      \item $\textit{ann}(u) = \textit{ann}(v_i)$
	for some $v_i$ minimizing $|\textit{ann}(v_i)|$

	In case of several minimizers, use any tie-breaking rule.
      \end{itemize}

      In both cases, $\{v_1, \dots, v_k\}$ is the set of true
      successors of $u$.

      \smallskip

      \dots
    \end{itemize}

    \onslide<all:2>
    \hilite{Computing annotations:}      
    \begin{itemize}
    \item \dots

      \smallskip

      At \alert{effect nodes}, \alert{add the corresponding
	operator/effect condition pair} to the set after applying
      combination rule. (Effect nodes for unconditional effects are
      represented just by the operator, without a condition.)
    \end{itemize}
    
    \hilite{Termination criterion:}
    \begin{itemize}
    \item \alert{stability:} terminate if set for proposition node
      $a^k$ has same cardinality as for $a^{k-1}$ for all true
      propositions $a$ in layer $k$
    \end{itemize}
    
    \hilite{Heuristic value:}
    \begin{itemize}
    \item The heuristic value is the \alert{set cardinality} of the
      goal node annotation.
    \end{itemize}
    \end{overprint}
  \end{block}
\end{frame}

\begin{frame}{Running example: \hsa}
  \only<handout>{\hspace*{\fill}}
  \begin{tikzpicture}
    \tikzstyle{tiny rpg}=[draw,inner sep=1pt,font=\tiny]
    \tikzstyle{rpg square}=[tiny rpg,minimum size=0.5cm,rectangle]
    \tikzstyle{rpg circle}=[tiny rpg,minimum size=0.5cm,circle]
    \pgfsetxvec{\pgfpoint{3.77cm}{0.0cm}}
    \pgfsetyvec{\pgfpoint{0.0cm}{0.87cm}}

    \node[trueinitprop,true] at (0.0,7.0) (a0) {\only<2->{%
	$\emptyset$}};
    \node[prop,false] at (0.0,6.0) (b0) {};
    \node[trueinitprop,true] at (0.0,5.0) (c0) {\only<2->{%
	$\emptyset$}};
    \node[trueinitprop,true] at (0.0,4.0) (d0) {\only<2->{%
	$\emptyset$}};
    \node[prop,false] at (0.0,3.0) (e0) {};
    \node[prop,false] at (0.0,2.0) (f0) {};
    \node[prop,false] at (0.0,1.0) (g0) {};
    \node[prop,false] at (0.0,0.0) (h0) {};
    
    \node[logic-and,true] at (0.25,4.5) (c0-and-d0) {\only<3->{%
	$\emptyset$}};

    \node[logic-or,true] at (0.5,5.35) (precond-o1-1) {\only<4->{%
	$\emptyset$}};
    \node[logic-and,false] at (0.5,6.5) (effcond-o1-1) {};
    \node[logic-and,true] at (0.5,2.35) (precond-o2-1) {\only<4->{%
	$\emptyset$}};

    \node[operator,true] at (0.75,5.35) (o1-1) {\only<5->{$\{o_1\}$}};
    \markopnode{o1-1}{o_1}
    \node[operator,true] at (0.75,2.35) (o2-1) {\only<5->{$\{o_2\}$}};
    \markopnode{o2-1}{o_2}
    \node[operator,false] at (0.75,1.35) (o3-1) {};
    \node[operator,false] at (0.75,0.35) (o4-1) {};

    \node[logic-cond,false] at (0.75,6.5) (efftrigger-o1-1) {};
    
    \node[prop,true] at (1.0,7.0) (a1) {\only<6->{$\emptyset$}};
    \node[prop,true] at (1.0,6.0) (b1) {\only<6->{$\{o_1\}$}};
    \node[prop,true] at (1.0,5.0) (c1) {\only<6->{$\emptyset$}};
    \node[prop,true] at (1.0,4.0) (d1) {\only<6->{$\emptyset$}};
    \node[prop,false] at (1.0,3.0) (e1) {};
    \node[prop,true] at (1.0,2.0) (f1) {\only<6->{$\{o_2\}$}};
    \node[prop,false] at (1.0,1.0) (g1) {};
    \node[prop,false] at (1.0,0.0) (h1) {};

    \node[logic-and,true] at (1.25,4.5) (c1-and-d1) {\only<7->{%
	$\emptyset$}};

    \node[logic-or,true] at (1.5,5.35) (precond-o1-2) {\only<8->{%
	$\emptyset$}};
    \node[logic-and,true] at (1.5,6.5) (effcond-o1-2) {\only<8->{%
	$\{o_1\}$}};
    \node[logic-and,true] at (1.5,2.35) (precond-o2-2) {\only<8->{%
	$\emptyset$}};

    \node[logic-cond,true] at (1.75,6.5) (efftrigger-o1-2){\only<9->{%
	$\begin{array}{@{}r@{}l@{}}\{&
            o_1,\\&o_{1, a \land b}\}\end{array}$}};
    \markopnode{efftrigger-o1-2}{o_{1, a \land b}}
    \node[operator,true] at (1.75,5.35) (o1-2) {\only<9->{$\{o_1\}$}};
    \markopnode{o1-2}{o_1}
    \node[operator,true] at (1.75,2.35) (o2-2) {\only<9->{$\{o_2\}$}};
    \markopnode{o2-2}{o_2}
    \node[operator,true] at (1.75,1.35) (o3-2) {\only<9->{%
	$\begin{array}{@{}r@{}l@{}}\{&
            o_2,\\&o_3\}\end{array}$}};
    \markopnode{o3-2}{o_3}
    \node[operator,true] at (1.75,0.35) (o4-2) {\only<9->{%
	$\begin{array}{@{}r@{}l@{}}\{&
            o_2,\\&o_4\}\end{array}$}};
    \markopnode{o4-2}{o_4}

    \node[prop,true] at (2.0,7.0) (a2) {\only<10->{$\emptyset$}};
    \node[prop,true] at (2.0,6.0) (b2) {\only<10->{$\{o_1\}$}};
    \node[prop,true] at (2.0,5.0) (c2) {\only<10->{$\emptyset$}};
    \node[prop,true] at (2.0,4.0) (d2) {\only<10->{$\emptyset$}};
    \node[prop,true,ellipse] at (2.0,3.0) (e2) {\only<10->{%
        $\begin{array}{@{}r@{}l@{}}\{&
            o_1,\\&o_{1, a \land b}\}\end{array}$}};
    \node[prop,true] at (2.0,2.0) (f2) {\only<10->{$\{o_2\}$}};
    \node[prop,true,ellipse] at (2.0,1.0) (g2) {\only<10->{%
        $\begin{array}{@{}r@{}l@{}}\{&
            o_2,\\&o_3\}\end{array}$}};
    \node[prop,true,ellipse] at (2.0,0.0) (h2) {\only<10->{%
	$\begin{array}{@{}r@{}l@{}}\{&
            o_2,\\&o_4\}\end{array}$}};

    \node[logic-and,true] at (2.25,0.5) (g2-and-h2) {\only<11->{%
	$\begin{array}{@{}r@{}l@{}}\{&o_2,\\&o_3,\\&o_4\}\end{array}$}};

    \node[goal-and,true] at (2.5,1.75) (goal2) {\only<12->{%
	$\begin{array}{@{}r@{}l@{}}\{&
            o_1,\\&o_{1, a \land b},\\&o_2,\\&o_3,\\&o_4\}\end{array}$}};

    \draw[->,idle] (a2)--(a1);
    \draw[->,idle] (a1)--(a0);
    \draw[->,idle] (b2)--(b1);
    \draw[->,idle] (b1)--(b0);
    \draw[->,idle] (c2)--(c1);
    \draw[->,idle] (c1)--(c0);
    \draw[->,idle] (d2)--(d1);
    \draw[->,idle] (d1)--(d0);
    \draw[->,idle] (e2)--(e1);
    \draw[->,idle] (e1)--(e0);
    \draw[->,idle] (f2)--(f1);
    \draw[->,idle] (f1)--(f0);
    \draw[->,idle] (g2)--(g1);
    \draw[->,idle] (g1)--(g0);
    \draw[->,idle] (h2)--(h1);
    \draw[->,idle] (h1)--(h0);
    \draw[->,nonidle] (c0-and-d0)--(c0);
    \draw[->,nonidle] (c0-and-d0)--(d0);
    \draw[->,nonidle] (precond-o1-1)--(c0-and-d0);
    \draw[->,nonidle] (precond-o1-1)--(b0);
    \draw[->,nonidle] (effcond-o1-1)--(a0);
    \draw[->,nonidle] (effcond-o1-1)--(b0);
    \draw[->,nonidle] (o1-1)--(precond-o1-1);
    \draw[->,nonidle] (efftrigger-o1-1)--(effcond-o1-1);
    \draw[->,nonidle] (efftrigger-o1-1)--(precond-o1-1);
    \draw[->,nonidle] (b1)--(o1-1);
    \draw[->,nonidle] (e1)--(efftrigger-o1-1);
    \draw[->,nonidle] (c1-and-d1)--(c1);
    \draw[->,nonidle] (c1-and-d1)--(d1);
    \draw[->,nonidle] (precond-o1-2)--(c1-and-d1);
    \draw[->,nonidle] (precond-o1-2)--(b1);
    \draw[->,nonidle] (effcond-o1-2)--(a1);
    \draw[->,nonidle] (effcond-o1-2)--(b1);
    \draw[->,nonidle] (o1-2)--(precond-o1-2);
    \draw[->,nonidle] (efftrigger-o1-2)--(effcond-o1-2);
    \draw[->,nonidle] (efftrigger-o1-2)--(precond-o1-2);
    \draw[->,nonidle] (b2)--(o1-2);
    \draw[->,nonidle] (e2)--(efftrigger-o1-2);

    \draw[->,nonidle] (o2-1)--(precond-o2-1);
    \draw[->,nonidle] (f1)--(o2-1);

    \draw[->,nonidle] (o2-2)--(precond-o2-2);
    \draw[->,nonidle] (f2)--(o2-2);
      
    \draw[->,nonidle] (o3-1)--(f0);
    \draw[->,nonidle] (g1)--(o3-1);
      
    \draw[->,nonidle] (o3-2)--(f1);
    \draw[->,nonidle] (g2)--(o3-2);
				   
    \draw[->,nonidle] (o4-1)--(f0);
    \draw[->,nonidle] (h1)--(o4-1);
      
    \draw[->,nonidle] (o4-2)--(f1);
    \draw[->,nonidle] (h2)--(o4-2);
      
    \draw[->,nonidle] (g2-and-h2)--(g2);
    \draw[->,nonidle] (g2-and-h2)--(h2);
    \draw[->,nonidle] (goal2)--(g2-and-h2);
    \draw[->,nonidle] (goal2)--(e2);
  \end{tikzpicture}%
  \only<handout>{\hspace*{\fill}}
\end{frame}

\begin{frame}{Remarks on {\hsa}}
  \begin{itemize}
  \item The same remarks for stability as for {\hadd} apply.
  \item Like {\hadd}, {\hsa} is \alert{safe} and \alert{goal-aware},
    but neither \alert{admissible} nor \alert{consistent}.
  \item {\hsa} is generally \alert{better informed} than {\hadd}, but
    significantly more expensive to compute.
  \item The $\hsa$ value depends on the tie-breaking rule used, so
    {\hsa} is \alert{not well-defined} without specifying the
    tie-breaking rule.
  \item The operators contained in the goal node annotation, suitably
    ordered, define a \alert{relaxed plan} for the task.
    \begin{itemize}
    \item Operators mentioned several times in the annotation must be
      added as many times in the relaxed plan.
    \end{itemize}
  \end{itemize}
\end{frame}

\subsection{Incremental computation}

\begin{frame}{Incremental computation of forward heuristics}
  One nice property of forward-propagating heuristics is that they
  allow \alert{incremental computation}:
  \begin{itemize}
  \item when evaluating several states in sequence which only differ
    in a few state variables, can
    \begin{itemize}
    \item \alert{start computation from previous results} and
    \item keep track only of \alert{what needs to be recomputed}
    \end{itemize}
  \item typical use case: \alert{depth-first} style searches
    (\eg, \idastar)
  \item rarely exploited in practice
  \end{itemize}
\end{frame}

\begin{frame}{Incremental computation example: \hadd}
  \only<all:1>{\small Result for $\{a \mapsto 1, b \mapsto 0, c
      \mapsto 1, d \mapsto 1, e \mapsto 0, f \mapsto 0, g \mapsto 0, h
      \mapsto 0\}$}%
  \only<all:2>{\small Change value of \alert{e to $1$}.}%
  \only<all:3->{\small \alert{Recompute} outdated values.}%
  \begin{center}
    \begin{tikzpicture}
      \tikzstyle{tiny rpg}=[draw,inner sep=1pt,font=\tiny]
      \tikzstyle{rpg square}=[tiny rpg,minimum size=0.40cm,rectangle]
      \tikzstyle{rpg circle}=[tiny rpg,minimum size=0.40cm,circle]
      \pgfsetxvec{\pgfpoint{3.77cm}{0.0cm}}
      \pgfsetyvec{\pgfpoint{0.0cm}{0.87cm}}

      \node[trueinitprop,true] at (0.0,7.0) (a0) {$0$};
      \node[prop,false] at (0.0,6.0) (b0) {};
      \node[trueinitprop,true] at (0.0,5.0) (c0) {$0$};
      \node[trueinitprop,true] at (0.0,4.0) (d0) {$0$};
      \only<all:1>{\node[prop,false] at (0.0,3.0) (e0) {};}
      \only<all:2-3>{\node[trueinitprop,dirty] at (0.0,3.0) (e0) {?};}
      \only<all:4->{\node[trueinitprop,true] at (0.0,3.0) (e0) {$0$};}
      \node[prop,false] at (0.0,2.0) (f0) {};
      \node[prop,false] at (0.0,1.0) (g0) {};
      \node[prop,false] at (0.0,0.0) (h0) {};

      \node[logic-and,true] at (0.25,4.5) (c0-and-d0) {$0$};

      \node[logic-or,true] at (0.5,5.35) (precond-o1-1) {$0$};
      \node[logic-and,false] at (0.5,6.5) (effcond-o1-1) {};
      \node[logic-and,true] at (0.5,2.35) (precond-o2-1) {$0$};

      \node[logic-cond,false] at (0.75,6.5) (efftrigger-o1-1) {};
      \node[operator,true] at (0.75,5.35) (o1-1) {$1$};
      \markplusone{o1-1}
      \node[operator,true] at (0.75,2.35) (o2-1) {$1$};
      \markplusone{o2-1}
      \node[operator,false] at (0.75,1.35) (o3-1) {};
      \node[operator,false] at (0.75,0.35) (o4-1) {};

      \node[prop,true] at (1.0,7.0) (a1) {$0$};
      \node[prop,true] at (1.0,6.0) (b1) {$1$};
      \node[prop,true] at (1.0,5.0) (c1) {$0$};
      \node[prop,true] at (1.0,4.0) (d1) {$0$};
      \only<all:1-3>{\node[prop,false] at (1.0,3.0) (e1) {};}
      \only<all:4>{\node[prop,dirty] at (1.0,3.0) (e1) {?};}
      \only<all:5->{\node[prop,true] at (1.0,3.0) (e1) {$0$};}

      \node[prop,true] at (1.0,2.0) (f1) {$1$};
      \node[prop,false] at (1.0,1.0) (g1) {};
      \node[prop,false] at (1.0,0.0) (h1) {};

      \node[logic-and,true] at (1.25,4.5) (c1-and-d1) {$0$};

      \node[logic-or,true] at (1.5,5.35) (precond-o1-2) {$0$};
      \node[logic-and,true] at (1.5,6.5) (effcond-o1-2) {$1$};
      \node[logic-and,true] at (1.5,2.35) (precond-o2-2) {$0$};

      \node[logic-cond,true] at (1.75,6.5) (efftrigger-o1-2) {$2$};
      \markplusone{efftrigger-o1-2}
      \node[operator,true] at (1.75,5.35) (o1-2) {$1$};
      \markplusone{o1-2}
      \node[operator,true] at (1.75,2.35) (o2-2) {$1$};
      \markplusone{o2-2}
      \node[operator,true] at (1.75,1.35) (o3-2) {$2$};
      \markplusone{o3-2}
      \node[operator,true] at (1.75,0.35) (o4-2) {$2$};
      \markplusone{o4-2}

      \node[prop,true] at (2.0,7.0) (a2) {$0$};
      \node[prop,true] at (2.0,6.0) (b2) {$1$};
      \node[prop,true] at (2.0,5.0) (c2) {$0$};
      \node[prop,true] at (2.0,4.0) (d2) {$0$};
      \only<all:1-4>{\node[prop,true] at (2.0,3.0) (e2) {$2$};}
      \only<all:5>{\node[prop,dirty] at (2.0,3.0) (e2) {?};}
      \only<all:6->{\node[prop,true] at (2.0,3.0) (e2) {$0$};}
      \node[prop,true] at (2.0,2.0) (f2) {$1$};
      \node[prop,true] at (2.0,1.0) (g2) {$2$};
      \node[prop,true] at (2.0,0.0) (h2) {$2$};

      \node[logic-and,true] at (2.25,0.5) (g2-and-h2) {$4$};

      \only<all:1-5>{\node[goal-and,true] at (2.5,1.75) (goal2) {$6$};}
      \only<all:6>{\node[goal-and,dirty] at (2.5,1.75) (goal2) {?};}
      \only<all:7->{\node[goal-and,true] at (2.5,1.75) (goal2) {$4$};}
      
      \draw[->,idle] (a2)--(a1);
      \draw[->,idle] (a1)--(a0);
      \draw[->,idle] (b2)--(b1);
      \draw[->,idle] (b1)--(b0);
      \draw[->,idle] (c2)--(c1);
      \draw[->,idle] (c1)--(c0);
      \draw[->,idle] (d2)--(d1);
      \draw[->,idle] (d1)--(d0);
      \draw[->,idle] (e2)--(e1);
      \draw[->,idle] (e1)--(e0);
      \draw[->,idle] (f2)--(f1);
      \draw[->,idle] (f1)--(f0);
      \draw[->,idle] (g2)--(g1);
      \draw[->,idle] (g1)--(g0);
      \draw[->,idle] (h2)--(h1);
      \draw[->,idle] (h1)--(h0);
      \draw[->,nonidle] (c0-and-d0)--(c0);
      \draw[->,nonidle] (c0-and-d0)--(d0);
      \draw[->,nonidle] (precond-o1-1)--(c0-and-d0);
      \draw[->,nonidle] (precond-o1-1)--(b0);
      \draw[->,nonidle] (effcond-o1-1)--(a0);
      \draw[->,nonidle] (effcond-o1-1)--(b0);
      \draw[->,nonidle] (o1-1)--(precond-o1-1);
      \draw[->,nonidle] (efftrigger-o1-1)--(effcond-o1-1);
      \draw[->,nonidle] (efftrigger-o1-1)--(precond-o1-1);
      \draw[->,nonidle] (b1)--(o1-1);
      \draw[->,nonidle] (e1)--(efftrigger-o1-1);
      \draw[->,nonidle] (c1-and-d1)--(c1);
      \draw[->,nonidle] (c1-and-d1)--(d1);
      \draw[->,nonidle] (precond-o1-2)--(c1-and-d1);
      \draw[->,nonidle] (precond-o1-2)--(b1);
      \draw[->,nonidle] (effcond-o1-2)--(a1);
      \draw[->,nonidle] (effcond-o1-2)--(b1);
      \draw[->,nonidle] (o1-2)--(precond-o1-2);
      \draw[->,nonidle] (efftrigger-o1-2)--(effcond-o1-2);
      \draw[->,nonidle] (efftrigger-o1-2)--(precond-o1-2);
      \draw[->,nonidle] (b2)--(o1-2);
      \draw[->,nonidle] (e2)--(efftrigger-o1-2);

      \draw[->,nonidle] (o2-1)--(precond-o2-1);
      \draw[->,nonidle] (f1)--(o2-1);

      \draw[->,nonidle] (o2-2)--(precond-o2-2);
      \draw[->,nonidle] (f2)--(o2-2);
      
      \draw[->,nonidle] (o3-1)--(f0);
      \draw[->,nonidle] (g1)--(o3-1);
      
      \draw[->,nonidle] (o3-2)--(f1);
      \draw[->,nonidle] (g2)--(o3-2);
				   
      \draw[->,nonidle] (o4-1)--(f0);
      \draw[->,nonidle] (h1)--(o4-1);
      
      \draw[->,nonidle] (o4-2)--(f1);
      \draw[->,nonidle] (h2)--(o4-2);
      
      \draw[->,nonidle] (g2-and-h2)--(g2);
      \draw[->,nonidle] (g2-and-h2)--(h2);
      \draw[->,nonidle] (goal2)--(g2-and-h2);
      \draw[->,nonidle] (goal2)--(e2);
    \end{tikzpicture}
  \end{center}
\end{frame}

\subsection[\hff]{The FF heuristic \hff}

\begin{frame}{Heuristic estimate $\hff$}
  \begin{itemize}
  \item $\hsa$ is more expensive to compute than the other forward
    propagating heuristics because we must propagate \alert{sets}.
  \item It is possible to get the same advantage over $\hadd$
    combined with efficient propagation.
  \item Key idea of \alert{\hff}: perform a \alert{backward
    propagation} that selects a sufficient subset of nodes to make the
    goal true (called a \alert{solution graph} in AND/OR dag
    literature).
  \item The resulting heuristic is almost as informative as $\hsa$, \\
    yet computable as quickly as $\hadd$.
  \end{itemize}

  \hilite{Note:} Our presentation inverts the historical order. \\
  The set-additive heuristic was defined \alert{after} the FF
  heuristic \\
  (sacrificing speed for even higher informativeness).
\end{frame}

\begin{frame}{FF heuristic: fitting the template\only<all:2-3>{ (ctd.)}}
  \begin{block}{The FF heuristic {\hff}\only<all:2-3>{ (ctd.)}}
    \begin{overprint}
    \onslide<all:1>
    \hilite{Computing annotations:}
    \begin{itemize}
    \item Annotations are \alert{Boolean values}, computed top-down.

      \smallskip

      A node is \alert{marked} when its annotation is set to $1$ and
      \alert{unmarked} if it is set to $0$. Initially, the goal node
      is marked, and all other nodes are unmarked.

      \smallskip

      We say that a true AND node is \alert{justified} if all its true
      successors are marked, and that a true OR node is
      \alert{justified} if at least one of its true successors is
      marked.

      \dots
    \end{itemize}
    \onslide<all:2>
    \hilite{Computing annotations:}      
    \begin{itemize}
    \item \dots

      \smallskip

      Apply these rules until \alert{all marked nodes are justified}:
      \begin{enumerate}
      \item Mark all true successors of a marked unjustified AND node.
      \item Mark the true successor of a marked unjustified OR node
        with only one true successor.
      \item Mark a true successor of a marked unjustified OR node
        connected via an idle arc.
      \item Mark any true successor of a marked unjustified OR node.
      \end{enumerate}
      
      The rules are given in priority order: earlier rules are
      preferred if applicable.
    \end{itemize}

    \onslide<all:3>
    \hilite{Termination criterion:}
    \begin{itemize}
    \item \alert{Always terminate} at first layer where goal node is
      true.
    \end{itemize}
    
    \hilite{Heuristic value:}
    \begin{itemize}
    \item The heuristic value is the \alert{number of operator/effect
      condition pairs} for which \alert{at least one} effect node is
      marked.
    \end{itemize}
    \end{overprint}
  \end{block}
\end{frame}

\begin{frame}{Running example: $\hff$}
  \begin{center}
    \begin{tikzpicture}
      \tikzstyle{tiny rpg}=[draw,inner sep=1pt,font=\tiny]
      \tikzstyle{rpg square}=[tiny rpg,minimum size=0.40cm,rectangle]
      \tikzstyle{rpg circle}=[tiny rpg,minimum size=0.40cm,circle]
      \pgfsetxvec{\pgfpoint{3.77cm}{0.0cm}}
      \pgfsetyvec{\pgfpoint{0.0cm}{0.87cm}}

      \node[prop,false] at (0.0,6.0) (b0) {};
      \node[prop,false] at (0.0,3.0) (e0) {};
      \node[prop,false] at (0.0,2.0) (f0) {};
      \node[prop,false] at (0.0,1.0) (g0) {};
      \node[prop,false] at (0.0,0.0) (h0) {};

      \node[logic-and,false] at (0.5,6.5) (effcond-o1-1) {};

      \node[operator,false] at (0.75,1.35) (o3-1) {};
      \node[operator,false] at (0.75,0.35) (o4-1) {};
      \node[logic-cond,false] at (0.75,6.5) (efftrigger-o1-1) {};

      \node[prop,true] at (1.0,5.0) (c1) {};
      \node[prop,true] at (1.0,4.0) (d1) {};
      \node[prop,false] at (1.0,3.0) (e1) {};
      \node[prop,false] at (1.0,1.0) (g1) {};
      \node[prop,false] at (1.0,0.0) (h1) {};

      \node[logic-and,true] at (1.25,4.5) (c1-and-d1) {};

      \node[logic-and,true] at (1.5,2.35) (precond-o2-2) {};

      \node[operator,true] at (1.75,5.35) (o1-2) {};
      \markopnodeff{o1-2}{o_1}
      \node[operator,true] at (1.75,2.35) (o2-2) {};
      \markopnodeff{o2-2}{o_2}

      \node[prop,true] at (2.0,7.0) (a2) {};
      \node[prop,true] at (2.0,6.0) (b2) {};
      \node[prop,true] at (2.0,5.0) (c2) {};
      \node[prop,true] at (2.0,4.0) (d2) {};
      \node[prop,true] at (2.0,2.0) (f2) {};

      \only<1-16>{
        \node[logic-and,true] at (0.5,2.35) (precond-o2-1) {};
      }
      \only<17->{
        \node[logic-and,selected] at (0.5,2.35) (precond-o2-1) {\textbf{M}};
      }

      \only<1-15>{
        \node[operator,true] at (0.75,2.35) (o2-1) {};
      }
      \only<16->{
        \node[operator,selected] at (0.75,2.35) (o2-1) {\textbf{M}};
      }
      \markopnodeff{o2-1}{o_2}

      \only<1-14>{
        \node[trueinitprop,true] at (0.0,5.0) (c0) {};
        \node[trueinitprop,true] at (0.0,4.0) (d0) {};
      }
      \only<15->{
        \node[trueinitprop,selected] at (0.0,5.0) (c0) {\textbf{M}};
        \node[trueinitprop,selected] at (0.0,4.0) (d0) {\textbf{M}};
      }

      \only<1-13>{
        \node[logic-and,true] at (0.25,4.5) (c0-and-d0) {};
      }
      \only<14->{
        \node[logic-and,selected] at (0.25,4.5) (c0-and-d0) {\textbf{M}};
      }

      \only<1-12>{
        \node[logic-or,true] at (0.5,5.35) (precond-o1-1) {};
      }
      \only<13->{
        \node[logic-or,selected] at (0.5,5.35) (precond-o1-1) {\textbf{M}};
      }

      \only<1-11>{
        \node[operator,true] at (0.75,5.35) (o1-1) {};
      }
      \only<12->{
        \node[operator,selected] at (0.75,5.35) (o1-1) {\textbf{M}};
      }
      \markopnodeff{o1-1}{o_1}

      \only<1-10>{
        \node[trueinitprop,true] at (0.0,7.0) (a0) {};
      }
      \only<11->{
        \node[trueinitprop,selected] at (0.0,7.0) (a0) {\textbf{M}};
      }

      \only<1-9>{
        \node[operator,true] at (1.75,0.35) (o4-2) {};
      }
      \only<10->{
        \node[operator,selected] at (1.75,0.35) (o4-2) {\textbf{M}};
      }
      \markopnodeff{o4-2}{o_4}

      \only<1-8>{
        \node[prop,true] at (1.0,2.0) (f1) {};
      }
      \only<9->{
        \node[prop,selected] at (1.0,2.0) (f1) {\textbf{M}};
      }

      \only<1-7>{
        \node[operator,true] at (1.75,1.35) (o3-2) {};
      }
      \only<8->{
        \node[operator,selected] at (1.75,1.35) (o3-2) {\textbf{M}};
      }
      \markopnodeff{o3-2}{o_3}

      \only<1-6>{
        \node[prop,true] at (1.0,7.0) (a1) {};
        \node[prop,true] at (1.0,6.0) (b1) {};
      }
      \only<7->{
        \node[prop,selected] at (1.0,7.0) (a1) {\textbf{M}};
        \node[prop,selected] at (1.0,6.0) (b1) {\textbf{M}};
      }

      \only<1-5>{
        \node[logic-or,true] at (1.5,5.35) (precond-o1-2) {};
        \node[logic-and,true] at (1.5,6.5) (effcond-o1-2) {};
      }
      \only<6->{
        \node[logic-and,selected] at (1.5,6.5) (effcond-o1-2) {\textbf{M}};
        \node[logic-or,selected] at (1.5,5.35) (precond-o1-2) {\textbf{M}};
      }

      \only<1-4>{
        \node[logic-cond,true] at (1.75,6.5) (efftrigger-o1-2) {};
      }
      \only<5->{
        \node[logic-cond,selected] at (1.75,6.5) (efftrigger-o1-2) {\textbf{M}};
      }
      \markopnodeff{efftrigger-o1-2}{o_{1, a \land b}}

      \only<1-3>{
        \node[prop,true] at (2.0,1.0) (g2) {};
        \node[prop,true] at (2.0,0.0) (h2) {};
      }
      \only<4->{
        \node[prop,selected] at (2.0,1.0) (g2) {\textbf{M}};
        \node[prop,selected] at (2.0,0.0) (h2) {\textbf{M}};
      }

      \only<1-2>{
        \node[logic-and,true] at (2.25,0.5) (g2-and-h2) {};
        \node[prop,true] at (2.0,3.0) (e2) {};
      }
      \only<3->{
        \node[logic-and,selected] at (2.25,0.5) (g2-and-h2) {\textbf{M}};
        \node[prop,selected] at (2.0,3.0) (e2) {\textbf{M}};
      }

      \only<1-1>{
        \node[goal-and,true] at (2.5,1.75) (goal2) {};
      }
      \only<2->{
        \node[goal-and,selected] at (2.5,1.75) (goal2) {\textbf{M}};
      }
      
      \draw[->,idle] (a2)--(a1);
      \draw[->,idle] (a1)--(a0);
      \draw[->,idle] (b2)--(b1);
      \draw[->,idle] (b1)--(b0);
      \draw[->,idle] (c2)--(c1);
      \draw[->,idle] (c1)--(c0);
      \draw[->,idle] (d2)--(d1);
      \draw[->,idle] (d1)--(d0);
      \draw[->,idle] (e2)--(e1);
      \draw[->,idle] (e1)--(e0);
      \draw[->,idle] (f2)--(f1);
      \draw[->,idle] (f1)--(f0);
      \draw[->,idle] (g2)--(g1);
      \draw[->,idle] (g1)--(g0);
      \draw[->,idle] (h2)--(h1);
      \draw[->,idle] (h1)--(h0);
      \draw[->,nonidle] (c0-and-d0)--(c0);
      \draw[->,nonidle] (c0-and-d0)--(d0);
      \draw[->,nonidle] (precond-o1-1)--(c0-and-d0);
      \draw[->,nonidle] (precond-o1-1)--(b0);
      \draw[->,nonidle] (effcond-o1-1)--(a0);
      \draw[->,nonidle] (effcond-o1-1)--(b0);
      \draw[->,nonidle] (o1-1)--(precond-o1-1);
      \draw[->,nonidle] (efftrigger-o1-1)--(effcond-o1-1);
      \draw[->,nonidle] (efftrigger-o1-1)--(precond-o1-1);
      \draw[->,nonidle] (b1)--(o1-1);
      \draw[->,nonidle] (e1)--(efftrigger-o1-1);
      \draw[->,nonidle] (c1-and-d1)--(c1);
      \draw[->,nonidle] (c1-and-d1)--(d1);
      \draw[->,nonidle] (precond-o1-2)--(c1-and-d1);
      \draw[->,nonidle] (precond-o1-2)--(b1);
      \draw[->,nonidle] (effcond-o1-2)--(a1);
      \draw[->,nonidle] (effcond-o1-2)--(b1);
      \draw[->,nonidle] (o1-2)--(precond-o1-2);
      \draw[->,nonidle] (efftrigger-o1-2)--(effcond-o1-2);
      \draw[->,nonidle] (efftrigger-o1-2)--(precond-o1-2);
      \draw[->,nonidle] (b2)--(o1-2);
      \draw[->,nonidle] (e2)--(efftrigger-o1-2);

      \draw[->,nonidle] (o2-1)--(precond-o2-1);
      \draw[->,nonidle] (f1)--(o2-1);

      \draw[->,nonidle] (o2-2)--(precond-o2-2);
      \draw[->,nonidle] (f2)--(o2-2);
      
      \draw[->,nonidle] (o3-1)--(f0);
      \draw[->,nonidle] (g1)--(o3-1);
      
      \draw[->,nonidle] (o3-2)--(f1);
      \draw[->,nonidle] (g2)--(o3-2);
				   
      \draw[->,nonidle] (o4-1)--(f0);
      \draw[->,nonidle] (h1)--(o4-1);
      
      \draw[->,nonidle] (o4-2)--(f1);
      \draw[->,nonidle] (h2)--(o4-2);
      
      \draw[->,nonidle] (g2-and-h2)--(g2);
      \draw[->,nonidle] (g2-and-h2)--(h2);
      \draw[->,nonidle] (goal2)--(g2-and-h2);
      \draw[->,nonidle] (goal2)--(e2);
    \end{tikzpicture}
  \end{center}
\end{frame}

\begin{frame}{Remarks on {\hff}}
  \begin{itemize}
  \item Like {\hadd} and {\hsa}, {\hff} is \alert{safe} and
    \alert{goal-aware}, but neither \alert{admissible} nor
    \alert{consistent}.
  \item Its informativeness can be expected to be slightly worse than
    for \hsa, but is usually not far off.
  \item Unlike \hsa, {\hff} can be computed in \alert{linear time}.
  \item Similar to $\hsa$, the operators corresponding to the marked
    operator/effect condition pairs define a \alert{relaxed plan}.
  \item Similar to $\hsa$, the $\hff$ value depends on tie-breaking
    when the marking rules allow several possible choices, so {\hff}
    is \alert{not well-defined} without specifying the tie-breaking
    rule.
    \begin{itemize}
    \item The implementation in FF uses additional rules of thumb
      to try to reduce the size of the generated relaxed plan.
    \end{itemize}
  \end{itemize}
\end{frame}

\subsection[Comparison \& practice]{Comparison \& relaxation
  heuristics in practice}

%% NOTE/TODO: Would be nice to also provide absolute bounds for
%% relaxed solvable states, but our information on h_FF and h_sa is a
%% bit sketchy there. We know of couse that h_max \le \hplus |A|, but
%% unfortunately the same is not clear for h_FF and h_sa, where the
%% best obvious bound is by the number of operator/effect condition
%% pairs. Of course, h_add can be exponential in ||T||.

\begin{frame}{Comparison of relaxation heuristics}
  \begin{theorem}[relationship between relaxation heuristics]
    Let $s$ be a state of planning task $\langle A, I, O, \gamma\rangle$.
    Then:
    \begin{itemize}
    \item $\hmax(s) \le \hplus(s) \le h^*(s)$
    \item $\hmax(s) \le \hplus(s) \le \hsa(s) \le \hadd(s)$
    \item $\hmax(s) \le \hplus(s) \le \hff(s) \le \hadd(s)$
    \item $h^*$, $\hff$ and $\hsa$ are pairwise incomparable
    \item $h^*$ and $\hadd$ are incomparable
    \end{itemize}
    Moreover, $\hplus$, $\hmax$, $\hadd$, $\hsa$ and $\hff$ assign
    $\infty$ to the same set of states.
  \end{theorem}
  \hilite{Note:} For \alert{inadmissible} heuristics, dominance is in
  general neither desirable nor undesirable. For relaxation
  heuristics, the objective is usually to get as close to $\hplus$ as
  possible.
\end{frame}

\begin{frame}{Relaxation heuristics in practice: HSP}
  \begin{example}[HSP]
    \alert{HSP} (Bonet \& Geffner) was one of the four top performers
    at the 1st International Planning Competition (IPC-1998).
    
    \smallskip

    Key ideas:
    \begin{itemize}
    \item \alert{hill climbing} search using \alert{$\hadd$}
    \item on \alert{plateaus}, keep going for a number of iterations,
      then restart
    \item use a closed list during exploration of plateaus
    \end{itemize}
  \end{example}
  \hilite{Literature:} Bonet, Loerincs \& Geffner (1997),
  Bonet \& Geffner (2001)
\end{frame}

\begin{frame}{Relaxation heuristics in practice: FF}
  \begin{example}[FF]
    \alert{FF} (Hoffmann \& Nebel) won the 2nd International Planning
    Competition (IPC-2000).

    \smallskip

    Key ideas:
    \begin{itemize}
    \item \alert{enforced hill-climbing} search using \alert{$\hff$}
    \item \alert{helpful action pruning:} in each search node, only
      consider successors from operators that add one of the atoms
      marked in proposition layer $1$
    \item \alert{goal ordering:} in certain cases, FF recognizes and
      exploits that certain subgoals should be solved one after the
      other
    \end{itemize}
    If the main search fails, FF performs a greedy best-first search
    using {\hff} without helpful action pruning or goal ordering.
  \end{example}
  \hilite{Literature:} Hoffmann \& Nebel (2001), Hoffmann (2005)
\end{frame}

\begin{frame}{Relaxation heuristics in practice: Fast Downward}
  \begin{example}[Fast Downward]
    \alert{Fast Downward} (Helmert \& Richter) won the satisficing
    track of the 4th International Planning Competition (IPC-2004).

    \smallskip
    
    Key ideas:
    \begin{itemize}
    \item \alert{greedy best-first search} using \alert{$\hff$} and
      \alert{causal graph heuristic} (not relaxation-based)
    \item search enhancements:
      \begin{itemize}
      \item multi-heuristic best-first search
      \item deferred evaluation of heuristic estimates
      \item preferred operators (similar to FF's helpful actions)
      \end{itemize}
    \end{itemize}
  \end{example}
  \hilite{Literature:} Helmert (2006)
\end{frame}

\begin{frame}{Relaxation heuristics in practice: SGPlan}
  \begin{example}[SGPlan]
    \alert{SGPlan} (Wah, Hsu, Chen \& Huang) won the satisficing
    track of the 5th International Planning Competition (IPC-2006).

    \smallskip
    
    Key ideas:
    \begin{itemize}
    \item \alert{FF}
    \item \alert{problem decomposition} techniques
    \item \alert{domain-specific techniques}
    \end{itemize}
  \end{example}
  \hilite{Literature:} Chen, Wah \& Hsu (2006)
\end{frame}

\begin{frame}{Relaxation heuristics in practice: LAMA}
  \begin{example}[LAMA]
    \alert{LAMA} (Richter \& Westphal) won the satisficing
    track of the 6th International Planning Competition (IPC-2008).

    \smallskip
    
    Key ideas:
    \begin{itemize}
    \item \alert{Fast Downward}
    \item \alert{landmark pseudo-heuristic} instead of causal graph \\
      heuristic (``somewhat'' relaxation-based)
    \item anytime variant of \alert{Weighted \astar} instead of greedy
      best-first search
    \end{itemize}
  \end{example}
  \hilite{Literature:} Richter, Helmert \& Westphal (2008), \\
  Richter \& Westphal (2010)
\end{frame}

\section*{Summary}

\begin{frame}{Summary}
  \begin{itemize}
  \item \alert{Relaxed planning graphs} are \alert{AND/OR dags}. They
    encode which propositions can be made true in $\Pi^+$ and how.
    \begin{itemize}
    \item They are closely related to \alert{forward sets} and
      \alert{forward plan steps}, based on the notion of
      \alert{parallel relaxed plans}.
    \item They can be \alert{constructed and evaluated efficiently},
      in time $O((m+1)\|\Pi^+\|)$ for planning task $\Pi$ and depth
      $m$.
    \end{itemize}
  \item By annotating RPG nodes with appropriate information, we can
    compute many useful heuristics.
  \item Examples: the \alert{max} heuristic \alert{\hmax},
    \alert{additive} heuristic \alert{\hadd}, \\ \alert{set-additive}
    heuristic \alert{\hsa} and \alert{FF} heuristic \alert{\hff}
    \begin{itemize}
    \item Of these, only \hmax\ is admissible (but not very accurate).
    \item The others are much more informative. The set-additive
      heuristic is the most sophisticated one.
    \item The FF heuristic is often similarly informative. It offers a
      good trade-off between accuracy and computation time.
    \end{itemize}
  \end{itemize}
\end{frame}

\end{document}
