\documentclass{gkibeamer}

\usepackage{tikz}[2008/02/13]

\title[AI Planning]{Principles of AI Planning}
\author[B.~Nebel,\\R.~Mattm\"uller]{Bernhard Nebel and Robert Mattm\"uller}
\date[]{Winter Semester 2011/2012}

%% Use \alert instead of \textred where it makes sense.
%% (\textred makes sense when you *really* want the color to be red,
%% independent of the beamer style settings.)
\newcommand{\hilite}[1]{{\color{structure} #1}}
\newcommand{\textred}[1]{{\color{red} #1}}
\newcommand{\textblue}[1]{{\color{blue} #1}}
\newcommand{\textgreen}[1]{{\color{green} #1}}
\newcommand{\textbrown}[1]{{\color{brown} #1}}
\newcommand{\textviolet}[1]{{\color{violet} #1}}
\newcommand{\textbg}[1]{{\color{bg} #1}}

\newcommand{\eg}{e.\,g.}
\newcommand{\ie}{i.\,e.}

%% State variables and values for state variables
\newcommand{\var}[1]{\textit{#1}}
\newcommand{\val}[1]{\text{#1}}

%% Various stuff
\newcommand{\astar}{A${}^*$}
\newcommand{\idastar}{IDA${}^*$}

%% Blocksworld stuff
\newcommand{\AONB}{\var{A-on-B}}
\newcommand{\AONC}{\var{A-on-C}}
\newcommand{\AOND}{\var{A-on-D}}
\newcommand{\BONA}{\var{B-on-A}}
\newcommand{\BONC}{\var{B-on-C}}
\newcommand{\BOND}{\var{B-on-D}}
\newcommand{\CONA}{\var{C-on-A}}
\newcommand{\CONB}{\var{C-on-B}}
\newcommand{\COND}{\var{C-on-D}}
\newcommand{\DONA}{\var{D-on-A}}
\newcommand{\DONB}{\var{D-on-B}}
\newcommand{\DONC}{\var{D-on-C}}
\newcommand{\AONTABLE}{\var{A-on-T}}
\newcommand{\BONTABLE}{\var{B-on-T}}
\newcommand{\CONTABLE}{\var{C-on-T}}
\newcommand{\CLEARA}{\var{A-clear}}
\newcommand{\CLEARB}{\var{B-clear}}
\newcommand{\CLEARC}{\var{C-clear}}

\newcommand{\applyop}[2]{\textit{app}_{#1}(#2)}
\newcommand{\applyops}[2]{\textit{app}_{#1}(#2)}
\newcommand{\applyplan}[2]{\textit{app}_{#1}(#2)}
\newcommand{\regr}[2]{\textit{regr}_{#1}(#2)}
\newcommand{\regrstrips}[2]{\textit{sregr}_{#1}(#2)}
\newcommand{\eprecon}[2]{\textit{EPC}_{#1}(#2)}

\newcommand{\compl}[1]{\overline{#1}}

\newcommand{\changes}[2]{\lbrack #1\rbrack_{#2}}
\newcommand{\CEF}{\vartriangleright}

\newcommand{\relaxation}[1]{{#1}^+}
\newcommand{\onset}[1]{\textit{on}(#1)}
\newcommand{\hplus}{\ensuremath{h^+}}
\newcommand{\hmax}{\ensuremath{h_{\text{max}}}}
\newcommand{\hadd}{\ensuremath{h_{\text{add}}}}
\newcommand{\hlmcut}{\ensuremath{h_{\text{LM-cut}}}}
\newcommand{\hff}{\ensuremath{h_{\text{FF}}}}
\newcommand{\hcs}{\ensuremath{h_{\text{cs}}}}
\newcommand{\hsa}{\ensuremath{h_{\text{sa}}}}
\newcommand{\hhhh}{\ensuremath{h_{\text{HHH}}}}

\newcommand{\sasplus}{SAS${}^+$}

\newcommand{\graphequiv}{\stackrel{\textup{G}}{\sim}}

\newcommand{\cg}{\ensuremath{\textit{CG}}}

%% Turing machines and complexity theory
\newcommand{\accept}{{\textsf{Y}}}

\newcommand{\easier}{\ensuremath{\le_{\text{p}}}}

\newcommand{\decisionclass}[1]{\ensuremath{\textsf{\textup{#1}}}}
\newcommand{\dtime}{\decisionclass{DTIME}}
\newcommand{\ntime}{\decisionclass{NTIME}}
\newcommand{\dspace}{\decisionclass{DSPACE}}
\newcommand{\nspace}{\decisionclass{NSPACE}}

\newcommand{\ptime}{\decisionclass{P}}
\newcommand{\np}{\decisionclass{NP}}
\newcommand{\pspace}{\decisionclass{PSPACE}}
\newcommand{\npspace}{\decisionclass{NPSPACE}}
\newcommand{\exptime}{\decisionclass{EXP}}
\newcommand{\nexptime}{\decisionclass{NEXP}}
\newcommand{\expspace}{\decisionclass{EXPSPACE}}
\newcommand{\dblexptime}{\decisionclass{2EXP}}

\newcommand{\planex}{\textsc{PlanEx}}
\newcommand{\planlen}{\textsc{PlanLen}}

\newcommand{\ms}[1]{\mathchoice{\mbox{\normalsize\it #1}}{\mbox{\normalsize\it #1}}{\mbox{\scriptsize\it #1}}{\mbox{\tiny\it #1}}}
\newcommand{\image}[2]{\ms{img}_{#1}(#2)}
\newcommand{\spreimage}[2]{\ms{spreimg}_{#1}(#2)}
\newcommand{\wpreimage}[2]{\ms{wpreimg}_{#1}(#2)}
\newcommand{\disbwd}[2]{\delta^{\ms{bwd}}_{#1}(#2)}
\newcommand{\ats}[1]{\ms{scope}(#1)}
\newcommand{\ndopcpc}[2]{\tau^{\ms{nd}}_{#1}(#2)}
\newcommand{\choice}{\mathop{|}}
\newcommand{\opcpc}[2]{\tau_{#1}(#2)}

%% Don't use \begin{proof}/\end{proof} for multi-part proofs because
%% this places a QED symbol.
\newenvironment{proofstart}{\begin{block}{Proof.}}{\end{block}}
\newenvironment{proofmid}{\begin{block}{Proof (ctd.)}}{\end{block}}
\newenvironment{proofend}{\begin{proof}[Proof (ctd.)]}{\end{proof}}

\newtheorem{remark}[theorem]{Remark}

%% Macros to align the width of things.
\newlength{\mywidth}
\newcommand{\setmywidth}[1]{\settowidth{\mywidth}{#1}}
\newcommand{\usemywidth}[1]{\makebox[\mywidth][l]{#1}}
\newcommand{\usemywidthmath}[1]{\usemywidth{\ensuremath{#1}}}

%% A tighter "align" environment.
\newenvironment{tightalign}[1][c]{\par\(\begin{array}[#1]{@{}r@{}l}}
  {\end{array}\)\par}
\newenvironment{wrappedmath}[1][t]{\begin{array}[#1]{@{}l}}{\end{array}}

%% Nondeterministic transition systems with connector arcs
\newcommand{\arc}[4]{
  \begin{pgfscope}
    \pgfsetlinewidth{1pt}
    \pgfpathmoveto{\pgfpointanchor{#3}{center}}
    \pgfpathlineto{\pgfpointanchor{#1}{#2}}
    \pgfpathlineto{\pgfpointanchor{#4}{center}}
    \pgfusepath{clip}
    \pgfpathcircle{\pgfpointanchor{#1}{#2}}{1.8mm}
    \pgfusepath{draw}
  \end{pgfscope}
}


\newcommand{\cliques}{\ensuremath{\textrm{cliques}}}

\tikzstyle{vertex interior}=[fill,minimum size=0.80cm,circle,
    inner sep=0pt,outer sep=0.2pt,color=red!20]
\tikzstyle{vertex border}=[draw,thin,minimum size=0.80cm,circle,
    inner sep=0pt,outer sep=0.2pt,color=black]
\tikzstyle{vertex hidden}=[draw,minimum size=0.80cm,circle,
    inner sep=0pt,outer sep=0.2pt, color=gray, text=gray]
\tikzstyle{subvertex}=[color=gray]

\tikzstyle{bigvertex interior}=[fill=red!20]
\tikzstyle{bigvertex border}=[]

\tikzstyle{transition}=[semithick]

\newcommand{\vertex}[3][]{
    \node[vertex interior] at (#2) (#3) {};
    \node[vertex border,#1] at (#3) {\small #3};
}

\newcommand{\numberedvertex}[4][]{
    \node[vertex interior] at (#2) (#3) {};
    \node[vertex border,#1] at (#3) {\small #4};
}

\newcommand{\vertexnolabel}[3][]{
    \node[vertex interior] at (#2) (#3) {};
    \node[vertex border,#1] at (#3) {};
}

\newcommand{\wildcard}{\ensuremath{\star}}

% Usage: \if\blank{#1}...\else...\fi
\catcode`\@=11 % as in plain.tex
\long\def\blank#1{\bl@nk#1@@..\bl@nk}%
\long\def\bl@nk#1#2@#3#4\bl@nk{#3#4}
\catcode`\@=12

\long\def\test#1{\begingroup \toks0{[#1]}%
    \newlinechar`\/\message{/\the\toks0:
        \if\blank{#1}EMPTY\else NOT empty\fi%
    }\endgroup}

\newcommand{\hiddenvertex}[1]{
    \if \blank{#1}
    \else
        \node[vertex hidden] at (#1) {#1};
    \fi
}

\newcommand{\hiddenvertices}[4]{
    \hiddenvertex{#1}
    \hiddenvertex{#2}
    \hiddenvertex{#3}
    \hiddenvertex{#4}
}

\newcommand{\verticalvertexborder}[3][]{
    \draw[bigvertex border, #1] (#2.west) arc(180:0:2.9mm) -- 
        (#3.east) arc(0:-180:2.9mm) -- (#2.west);
}

\newcommand{\verticalvertex}[5][]{
    \draw[bigvertex interior, #1] (#2.west) arc(180:0:2.9mm) -- 
        (#3.east) arc(0:-180:2.9mm) -- (#2.west);
	\hiddenvertices{#2}{#3}{#4}{#5}
	\verticalvertexborder{#2}{#3}
}

\newcommand{\verticalgoalvertex}{\verticalvertex[color=green!40]}

\newcommand{\horizontalvertexborder}[3][]{
    \draw[bigvertex border, #1] (#2.south) arc(-90:-270:2.9mm) -- 
        (#3.north) arc(90:-90:2.9mm) -- (#2.south);
}

\newcommand{\horizontalvertex}[3][]{
    \draw[bigvertex interior, #1] (#2.south) arc(-90:-270:2.9mm) -- 
        (#3.north) arc(90:-90:2.9mm) -- (#2.south);
	\hiddenvertices{#2}{#3}{}{}
	\horizontalvertexborder{#2}{#3}
}

\newcommand{\trianglevertexleftborder}[4][]{
    \draw[bigvertex border,#1] (#2.south west) arc(-135:-225:2.9mm) -- 
        (#3.north west) arc(135:0:2.9mm) -- 
        (#4.east) arc(0:-135:2.9mm) -- 
        (#2.south west);
}

\newcommand{\trianglevertexleft}[5][]{    
    \draw[bigvertex interior, #1] (#2.south west) arc(-135:-225:2.9mm) -- 
        (#3.north west) arc(135:0:2.9mm) -- 
        (#4.east) arc(0:-135:2.9mm) -- 
        (#2.south west); 
	\hiddenvertices{#2}{#3}{#4}{#5}
	\trianglevertexleftborder{#2}{#3}{#4}
}

\newcommand{\trianglevertexrightborder}[4][]{
    \draw[bigvertex border, #1] (#2.south east) arc(-45:45:2.9mm) -- 
        (#3.north east) arc(45:180:2.9mm) -- 
        (#4.west) arc(180:315:2.9mm) -- 
        (#2.south east);
}

\newcommand{\trianglevertexright}[5][]{   
    \draw[bigvertex interior, #1] (#2.south east) arc(-45:45:2.9mm) -- 
        (#3.north east) arc(45:180:2.9mm) -- 
        (#4.west) arc(180:315:2.9mm) -- 
        (#2.south east); 
	\hiddenvertices{#2}{#3}{#4}{#5}
	\trianglevertexrightborder{#2}{#3}{#4}
}

\newcommand{\squarevertexborder}[5][]{
    \draw[bigvertex border] (#2.west) arc(180:90:2.9mm) -- 
        (#3.north) arc(90:0:2.9mm) -- 
        (#4.east) arc(0:-90:2.9mm) -- 
        (#5.south) arc(-90:-180:2.9mm) -- 
        (#2.west);
}

\newcommand{\squarevertex}[5][]{    
    \draw[bigvertex interior, #1] (#2.west) arc(180:90:2.9mm) -- 
        (#3.north) arc(90:0:2.9mm) -- 
        (#4.east) arc(0:-90:2.9mm) -- 
        (#5.south) arc(-90:-180:2.9mm) -- 
        (#2.west); 
	\hiddenvertices{#2}{#3}{#4}{#5}
	\squarevertexborder{#2}{#3}{#4}{#5}
}

\newcommand{\gravisvertexborder}[3][]{
    \draw[bigvertex border, #1] (#2.south west) arc(225:45:2.9mm) -- 
        (#3.north east) arc(45:-135:2.9mm) -- 
        (#2.south west);
}

\newcommand{\gravisvertex}[3][]{    
    \draw[bigvertex interior,#1] (#2.south west) arc(225:45:2.9mm) -- 
        (#3.north east) arc(45:-135:2.9mm) -- 
        (#2.south west); 
	\hiddenvertices{#2}{#3}{}{}
	\gravisvertexborder{#2}{#3}
}

\newcommand{\akutvertexborder}[3][]{
    \draw[bigvertex border, #1] (#2.north west) arc(135:-45:2.9mm) -- 
        (#3.south east) arc(-45:-225:2.9mm) -- 
        (#2.north west);
}

\newcommand{\akutvertex}[3][]{    
    \draw[bigvertex interior,#1] (#2.north west) arc(135:-45:2.9mm) -- 
        (#3.south east) arc(-45:-225:2.9mm) -- 
        (#2.north west); 
	\hiddenvertices{#2}{#3}{}{}
	\akutvertexborder{#2}{#3}
}

\newcommand{\goalvertex}[3][]{
    \node[vertex interior,color=green!40] at (#2) (#3) {};
    \node[vertex border,#1] at (#3) {\small #3};
}

\newcommand{\numberedgoalvertex}[4][]{
    \node[vertex interior,color=green!40] at (#2) (#3) {};
    \node[vertex border,#1] at (#3) {\small #4};
}

\newcommand{\goalvertexnolabel}[3][]{
    \node[vertex interior,color=green!40] at (#2) (#3) {};
    \node[vertex border,#1] at (#3) {};
}

\newcommand{\transition}[5][]{\draw[<->,>=stealth,transition,#1] (#2)--(#3);}

\newcommand{\circletransition}{140:1.25mm}
\newcommand{\selftransition}[2][0:0mm]{
    \draw[->,>=stealth,transition] (#2) +(#1) arc(-45:225:1.5mm);}

\newcommand{\picfulltransitiongraphbase}[2]{%
  \begin{tikzpicture}[scale=1.4]
    #1{0,0}{LRR}
    %
    \draw[transition,->,>=stealth](-0.6,0) -- (LRR);
    %
    #1{1,0}{LLL}
    #1{1,1}{LLR}
    #1{1,-1}{LRL}
    %
    #1{2,1.5}{ALR}
    #1{2,0.5}{ALL}
    #1{2,-0.5}{BLL}
    #1{2,-1.5}{BRL}
    %
    #1{3,1.5}{ARL}
    #1{3,0.5}{ARR}
    #1{3,-0.5}{BRR}
    #1{3,-1.5}{BLR}
    %
    #2{4,0}{RRR}
    #2{4,1}{RRL}
    #2{4,-1}{RLR}
    %
    #2{5,0}{RLL}
    %
    \transition{LLL}{LRL}{move(A,R)}{move(A,L)}
    \transition{LLL}{LLR}{move(B,R)}{move(B,L)}
    \transition{LRL}{LRR}{move(B,R)}{move(B,L)}
    \transition{LLR}{LRR}{move(A,R)}{move(A,L)}
    %
    \transition{LLL}{ALL}{pick(A,L)}{drop(A,L)}
    \transition{LLL}{BLL}{pick(B,L)}{drop(B,L)}
    \transition{LLR}{ALR}{pick(A,L)}{drop(A,L)}
    \transition{LRL}{BRL}{pick(B,L)}{drop(B,L)}
    \transition{ALL}{ALR}{move(A,R)}{move(A,L)}
    \transition{BLL}{BRL}{move(B,R)}{move(B,L)}
    %
    \transition{ALR}{ARR}{move(A,R)}{move(A,L)}
    \transition{ALL}{ARL}{move(A,R)}{move(A,L)}
    \transition{BRL}{BRR}{move(B,R)}{move(B,L)}
    \transition{BLL}{BLR}{move(B,R)}{move(B,L)}
    %
    \transition{RRR}{ARR}{pick(A,R)}{drop(A,R)}
    \transition{RRR}{BRR}{pick(B,R)}{drop(B,R)}
    \transition{RRL}{ARL}{pick(A,R)}{drop(A,R)}
    \transition{RLR}{BLR}{pick(B,R)}{drop(B,R)}
    \transition{ARR}{ARL}{move(A,L)}{move(A,R)}
    \transition{BRR}{BLR}{move(B,L)}{move(B,R)}
    %
    \transition{RRR}{RLR}{move(A,L)}{move(A,R)}
    \transition{RRR}{RRL}{move(B,L)}{move(B,R)}
    \transition{RLR}{RLL}{move(B,L)}{move(B,R)}
    \transition{RRL}{RLL}{move(A,L)}{move(A,R)}
  \end{tikzpicture}%
}

\newcommand{\picfulltransitiongraph}{%
  \picfulltransitiongraphbase{\vertex}{\goalvertex}}

\newcommand{\picfulltransitiongraphnolabels}{%
  \picfulltransitiongraphbase{\vertexnolabel}{\goalvertexnolabel}}

\newcommand{\picperfectabstraction}{%
  \begin{tikzpicture}[scale=1.4]
    \vertex[text=gray]{0,0}{LRR}
    %
    \draw[transition,->,>=stealth](-0.6,0) -- (LRR);
    %
    \vertex{1,1}{LLR}
    \vertex{1,0}{LLL}
    \vertex{1,-1}{LRL}
    \verticalvertex{LLR}{LRL}{LLL}{}
    %
    \vertex{2,1.5}{ALR}
    \vertex{2,0.5}{ALL}
    \vertex{2,-0.5}{BLL}
    \vertex{2,-1.5}{BRL}
    \verticalvertex{ALR}{BRL}{ALL}{BLL}
    %
    \vertex{3,1.5}{ARL}
    \vertex{3,0.5}{ARR}
    \vertex{3,-0.5}{BRR}
    \vertex{3,-1.5}{BLR}
    \verticalvertex{ARL}{BLR}{ARR}{BRR}
    %
    %
    \vertex{4,0}{RRR}
    \vertex{4,1}{RRL}
    \vertex{4,-1}{RLR}
    \vertex{5,0}{RLL}
    \trianglevertexright[fill=green!40]{RLL}{RRL}{RLR}{RRR}
    %
    \transition{LLL}{LRR}{}{}
    %
    \transition{LLL}{LLL-|ALL.west}{}{}
    %
    \transition{LLL-|ALL.east}{RRR-|ARL.west}{}{}
    %
    \transition{RRR}{RRR-|ARL.east}{}{}
    %
    \selftransition[\circletransition]{LLR.north east}
    \selftransition[\circletransition]{ALR.north east}
    \selftransition[\circletransition]{ARL.north east}
    \selftransition[\circletransition]{RRL.north east}
  \end{tikzpicture}%
}

\newcommand{\piconestateabstraction}{%
  \begin{tikzpicture}[scale=1.4]
    \vertex[text=gray]{0,0}{LRR}
    %
    \draw[transition,->,>=stealth](-0.6,0) -- (LRR);
    %
    \vertex{1,1}{LLR}
    \vertex{1,0}{LLL}
    \vertex{1,-1}{LRL}
    \vertex{2,1.5}{ALR}
    \vertex{2,0.5}{ALL}
    \vertex{2,-0.5}{BLL}
    \vertex{2,-1.5}{BRL}
    \vertex{3,1.5}{ARL}
    \vertex{3,0.5}{ARR}
    \vertex{3,-0.5}{BRR}
    \vertex{3,-1.5}{BLR}
    \vertex{4,0}{RRR}
    \vertex{4,1}{RRL}
    \vertex{4,-1}{RLR}
    \vertex{5,0}{RLL}
	
    \draw[bigvertex interior,color=green!40] (LRR.230) arc(-130:-230:2.9mm)
    -- (LLR.140) arc(-220:-240:2.9mm)
    -- (ALR.120) arc(-240:-270:2.9mm) -- (ARL.north) arc(90:60:2.9mm)
    -- (RRL.60) arc (-300:-320:2.9mm) -- (RLL.50) arc(50:-50:2.9mm)
	-- (RLR.320) arc (-40:-60:2.9mm) -- (BLR.300) arc (-60:-90:2.9mm)
	-- (BRL.south) arc (-90:-120:2.9mm) -- (LRL.240) arc (-120:-140:2.9mm)
	-- (LRR.230);
	\hiddenvertices{LRR}{LLR}{LLL}{LRL}
	\hiddenvertices{ALR}{ALL}{BLL}{BRL}
	\hiddenvertices{ARL}{ARR}{BRR}{BLR}
	\hiddenvertices{RRR}{RRL}{RLR}{RLL}
    \draw[bigvertex border] (LRR.230) arc(-130:-230:2.9mm)
    -- (LLR.140) arc(-220:-240:2.9mm)
    -- (ALR.120) arc(-240:-270:2.9mm) -- (ARL.north) arc(90:60:2.9mm)
    -- (RRL.60) arc (-300:-320:2.9mm) -- (RLL.50) arc(50:-50:2.9mm)
	-- (RLR.320) arc (-40:-60:2.9mm) -- (BLR.300) arc (-60:-90:2.9mm)
	-- (BRL.south) arc (-90:-120:2.9mm) -- (LRL.240) arc (-120:-140:2.9mm)
	-- (LRR.230);
	
    \selftransition[\circletransition]{ARL.north east}
  \end{tikzpicture}%
}

\newcommand{\picexampleabstraction}[1][\verticalvertex]{%
  \begin{tikzpicture}[scale=1.4]
    \vertex[text=gray]{0,0}{LRR}
    %
    \draw[transition,->,>=stealth](-0.6,0) -- (LRR);
    %
    \vertex{1,1}{LLR}
    \vertex{1,0}{LLL}
    \vertex{1,-1}{LRL}
    #1{LLR}{LRL}{LLL}{}
    %
    \vertex{2,1.5}{ALR}
    \vertex{3,1.5}{ARL}
    \vertex{2,0.5}{ALL}
    \vertex{3,0.5}{ARR}
    \vertex{2,-0.5}{BLL}
    \vertex{2,-1.5}{BRL}
    \vertex{3,-0.5}{BRR}
    \vertex{3,-1.5}{BLR}
    \squarevertex{ALR}{ARL}{BLR}{BRL}
	\hiddenvertices{ALL}{ARR}{BLL}{BRR}
    %
    \vertex{4,0}{RRR}
    \vertex{4,1}{RRL}
    \vertex{4,-1}{RLR}
    \vertex{5,0}{RLL}
    \trianglevertexright[fill=green!40]{RLL}{RRL}{RLR}{RRR}
    %
    \draw[transition,->,>=stealth](-0.6,0) -- (LRR);
    %
    \transition{LRR}{LLL}{}{}
    \transition{LLL}{LLL-|ALL.west}{}{}
    \transition{RRR}{RRR-|ARL.east}{}{}
    %
    \selftransition[\circletransition]{LLR.north east}
    \selftransition[\circletransition]{RRL.north east}
    \selftransition[0:6mm]{ALR.north}
  \end{tikzpicture}
}

\newcommand{\picprojectionpackage}{%
  \begin{tikzpicture}[scale=1.4]
    \vertex{0,0}{LRR}
    \vertex{1,0}{LLL}
    \vertex{1,1}{LLR}
    \vertex{1,-1}{LRL}
    \trianglevertexleft{LRR}{LLR}{LRL}{LLL}
    %
    \vertex{2,1.5}{ALR}
    \vertex{3,1.5}{ARL}
    \vertex{2,0.5}{ALL}
    \vertex{3,0.5}{ARR}
    \squarevertex{ALR}{ARL}{ARR}{ALL}
    %
    \vertex{2,-0.5}{BLL}
    \vertex{2,-1.5}{BRL}
    \vertex{3,-0.5}{BRR}
    \vertex{3,-1.5}{BLR}
    \squarevertex{BLL}{BRR}{BLR}{BRL}
    %
    \vertex{4,0}{RRR}
    \vertex{4,1}{RRL}
    \vertex{4,-1}{RLR}
    \vertex{5,0}{RLL}
    \trianglevertexright[fill=green!40]{RLL}{RRL}{RLR}{RRR}
    %
    \draw[transition,->,>=stealth](-0.6,0) -- (LRR);
    %
    \transition{LLR}{LLR-|ALR.west}{}{}
    \transition{LRL}{LRL-|BRL.west}{}{}
    \transition{RRL}{RRL-|ARL.east}{}{}
    \transition{RLR}{RLR-|ARL.east}{}{}
    \selftransition[\circletransition]{LLR.north east}
    \selftransition[\circletransition]{RRL.north east}
    \selftransition[0:6mm]{ALR.north}
    \selftransition[0:6mm]{BLL.north}
  \end{tikzpicture}
}

\newcommand{\picprojectionpackagetruck}{
  \begin{tikzpicture}[scale=1.4]
    \vertex{0,0}{LRR}
    \vertex{1,-1}{LRL}
    \gravisvertex{LRR}{LRL}
    %
    \vertex{1,0}{LLL}
    \vertex{1,1}{LLR}
    \verticalvertex{LLR}{LLL}{}{}
    %
    \vertex{2,1.5}{ALR}
    \vertex{2,0.5}{ALL}
    \verticalvertex{ALR}{ALL}{}{}
    %
    \vertex{3,1.5}{ARL}
    \vertex{3,0.5}{ARR}
    \verticalvertex{ARL}{ARR}{}{}
    %
    \only<beamer:1|handout:0>{
      \vertex{3,-1.5}{BLR}
      \vertex{2,-0.5}{BLL}
      \vertex{3,-0.5}{BRR}
      \vertex{2,-1.5}{BRL}
      \gravisvertex{BLL}{BLR}
      \akutvertex{BRR}{BRL}
    }%
    \only<2>{
      \vertex{3,-1.5}{BRR}
      \vertex{2,-0.5}{BLL}
      \vertex{3,-0.5}{BLR}
      \vertex{2,-1.5}{BRL}
      \horizontalvertex{BLL}{BLR}
      \horizontalvertex{BRL}{BRR}
    }%
    %
    \vertex{4,0}{RRR}
    \vertex{4,1}{RRL}
    \verticalvertex[fill=green!40]{RRL}{RRR}{}{}
    %
    \vertex{4,-1}{RLR}
    \vertex{5,0}{RLL}
    \akutvertex[fill=green!40]{RLL}{RLR}
    %
    \draw[transition,->,>=stealth](-0.6,0) -- (LRR);
    \transition{intersection of LRR.north east--LRL.north east and LRR.east--LLL.west}
      {LLL.north-|LLR.west}{}{}
    \transition{ALL.north west-|LLR.east}{LLR.east-|ALL.west}{}{}
    \transition{LLR.east-|ALL.east}{LLR.east-|ARR.west}{}{}
    \transition{LLR.east-|ARL.east}{ARR.north east-|RRL.west}{}{}
    \transition{RRR.north-|RRR.east}
      {intersection of RLL.north west--RLR.north west and RLL.west--RRR.east}{}{}
    %
    \transition{LLL.north-|LLL.east}{BLL}{}{}
    \transition{LRL}{BRL}{}{}
    \only<beamer:1|handout:0>{\transition{BRR}{RRR.west|-RRR.north}{}{}}
    \only<2>{\transition{BRR.north east}{RRR.south west}{}{}}
    \transition{BLR}{RLR}{}{}
    \only<beamer:1|handout:0>{\transition{BRL.south east}{BLR.south west}{}{}}
    \only<2>{\transition{BLL.south-|BLL.east}{BRL.north-|BRL.east}{}{}}
    \selftransition[\circletransition]{LRR.north east}
    \selftransition[\circletransition]{LLR.north east}
    \selftransition[\circletransition]{ALR.north east}
    \selftransition[\circletransition]{ARL.north east}
    \selftransition[\circletransition]{RRL.north east}
    \selftransition[\circletransition]{RLL.north east}
    \only<beamer:1|handout:0>{
      \selftransition[\circletransition]{BLL.north east}
      \selftransition[\circletransition]{BRR.north east}
    }%
    \only<2>{
      \selftransition[1:8mm]{BLL.north}
      \selftransition[1:8mm]{BRL.north}
    }%
  \end{tikzpicture}%
}

\newcommand{\picrefinement}{%
  \begin{tikzpicture}[scale=1.4]
    \vertex{0,0}{LRR}
    \vertex{1,0}{LLL}
    \vertex{1,1}{LLR}
    \vertex{1,-1}{LRL}
    \trianglevertexleft{LRR}{LLR}{LRL}{LLL}
    \gravisvertexborder[subvertex]{LRR}{LRL}
    \verticalvertexborder[subvertex]{LLR}{LLL}
    \trianglevertexleftborder{LRR}{LLR}{LRL}{LLL}
    %
    \vertex{2,1.5}{ALR}
    \vertex{3,1.5}{ARL}
    \vertex{2,0.5}{ALL}
    \vertex{3,0.5}{ARR}
    \squarevertex{ALR}{ARL}{ARR}{ALL}
    \verticalvertexborder[subvertex]{ALR}{ALL}
    \verticalvertexborder[subvertex]{ARL}{ARR}
    \squarevertexborder{ALR}{ARL}{ARR}{ALL}
    %
    \vertex{2,-0.5}{BLL}
    \vertex{2,-1.5}{BRL}
    \vertex{3,-0.5}{BLR}
    \vertex{3,-1.5}{BRR}
    \squarevertex{BLL}{BLR}{BRR}{BRL}
    \horizontalvertexborder[subvertex]{BLL}{BLR}
    \horizontalvertexborder[subvertex]{BRL}{BRR}
    \squarevertexborder{BLL}{BLR}{BRR}{BRL}
    %
    \vertex{4,0}{RRR}
    \vertex{4,1}{RRL}
    \vertex{4,-1}{RLR}
    \vertex{5,0}{RLL}
    \trianglevertexright[fill=green!40]{RLL}{RRL}{RLR}{RRR}
    \verticalvertexborder[subvertex]{RRL}{RRR}
    \akutvertexborder[subvertex]{RLL}{RLR}
    \trianglevertexrightborder{RLL}{RRL}{RLR}{RRR}
	%
    \draw[transition,->,>=stealth](-0.6,0) -- (LRR);
    %
    \transition{LLR}{LLR-|ALR.west}{}{}
    \transition{LRL}{LRL-|BRL.west}{}{}
    \transition{RRL}{RRL-|ARL.east}{}{}
    \transition{RLR}{RLR-|ARL.east}{}{}
    \selftransition[\circletransition]{LLR.north east}
    \selftransition[\circletransition]{RRL.north east}
    \selftransition[0:6mm]{ALR.north}
    \selftransition[0:6mm]{BLL.north}
  \end{tikzpicture}
}


\newcommand{\pictwopackages}{%
  \begin{tikzpicture}[scale=1.28]
    \vertex{0,0}{LLR}
    \vertex{1,0}{LLL}
    \vertex{2,.5}{ILL}
    \vertex{2,-.5}{LIL}
    \vertex{3,0}{IIL}
    \vertex{4,0}{IIR}
    \vertex{5,.5}{RIR}
    \vertex{5,-.5}{IRR}
    \goalvertex{6,0}{RRR}
    \goalvertex{7,0}{RRL}
    \vertex{2.3,1.2}{ILR}
    \vertex{3.1,1.5}{RLR}
    \vertex{3.85,1.5}{RLL}
    \vertex{4.7,1.2}{RIL}
    \vertex{2.3,-1.2}{LIR}
    \vertex{3.1,-1.5}{LRR}
    \vertex{3.85,-1.5}{LRL}
    \vertex{4.7,-1.2}{IRL}

    \draw[transition,->,>=stealth](-0.6,0) -- (LLR);
    \transition{LLR}{LLL}{}{}
    \transition{LLL}{ILL}{}{}
    \transition{LLL}{LIL}{}{}
    \transition{ILL}{IIL}{}{}
    \transition{LIL}{IIL}{}{}
    \transition{IIL}{IIR}{}{}
    \transition{IIR}{RIR}{}{}
    \transition{IIR}{IRR}{}{}
    \transition{RIR}{RRR}{}{}
    \transition{IRR}{RRR}{}{}
    \transition{RRR}{RRL}{}{}
    \transition{ILL}{ILR}{}{}
    \transition{RLR}{ILR}{}{}
    \transition{RLR}{RLL}{}{}
    \transition{RIL}{RLL}{}{}
    \transition{RIL}{RIR}{}{}
    \transition{LIL}{LIR}{}{}
    \transition{LRR}{LIR}{}{}
    \transition{LRR}{LRL}{}{}
    \transition{LRL}{IRL}{}{}
    \transition{IRR}{IRL}{}{}
  \end{tikzpicture}
}

\newlength{\vertexround}
\setlength{\vertexround}{3.14mm}
\newcommand{\pictwopackagesabstractpacktwo}{%
  \begin{tikzpicture}[scale=1.28]
    \vertex{0,0}{LLR}
    \vertex{1,0}{LLL}
    \vertex{2,.5}{ILL}
    \vertex{2,-.5}{LIL}
    \vertex{3,0}{IIL}
    \vertex{4,0}{IIR}
    \vertex{5,.5}{RIR}
    \vertex{5,-.5}{IRR}
    \goalvertex{6,0}{RRR}
    \goalvertex{7,0}{RRL}
    \vertex{2.3,1.2}{ILR}
    \vertex{3.1,1.5}{RLR}
    \vertex{3.85,1.5}{RLL}
    \vertex{4.7,1.2}{RIL}
    \vertex{2.3,-1.2}{LIR}
    \vertex{3.1,-1.5}{LRR}
    \vertex{3.85,-1.5}{LRL}
    \vertex{4.7,-1.2}{IRL}
    
    \draw[bigvertex interior] (LLR.south) arc(-90:-240:\vertexround)
    -- (ILR.120) arc(120:100:\vertexround) 
	-- (RLR.110) arc (110:90:\vertexround)
    -- (RLL.north) arc(90:-60:\vertexround) -- (ILL.-60)
    -- (LLL.300) arc (-60:-90:\vertexround) -- (LLR.south);
	\hiddenvertices{LLR}{LLL}{ILL}{ILR}
	\hiddenvertices{RLR}{RLL}{}{}
    \draw[bigvertex border] (LLR.south) arc(-90:-240:\vertexround)
    -- (ILR.120) arc(120:100:\vertexround) 
	-- (RLR.120) arc (120:90:\vertexround)
    -- (RLL.north) arc(90:-60:\vertexround) -- (ILL.-60)
    -- (LLL.300) arc (-60:-90:\vertexround) -- (LLR.south);
    
    \draw[bigvertex interior] (LIR.310) arc(-50:-165:\vertexround)
    -- (LIL.200) arc(-160:-240:\vertexround) 
    -- (RIL.120) arc(120:10:\vertexround)
    -- (RIR.20) arc(20:-50:\vertexround)
    -- (LIR.310);
	\hiddenvertices{LIR}{LIL}{IIL}{IIR}
	\hiddenvertices{RIR}{RIL}{}{}
    \draw[bigvertex border] (LIR.310) arc(-50:-165:\vertexround)
    -- (LIL.200) arc(-160:-240:\vertexround) 
    -- (RIL.120) arc(120:10:\vertexround)
    -- (RIR.20) arc(20:-50:\vertexround)
    -- (LIR.310);

    \draw[bigvertex interior, color=green!40] (RRL.north) arc(90:-60:\vertexround)
    -- (IRL.-60) arc(-60:-80:\vertexround)
	-- (LRL.-70) arc (-70:-90:\vertexround)
    -- (LRR.south) arc(-90:-240:\vertexround) -- (IRR.120)
    -- (RRR.120) arc (120:90:\vertexround) -- (RRL.north);
	\hiddenvertices{LRR}{LRL}{IRL}{IRR}
	\hiddenvertices{RRR}{RRL}{}{}
    \draw[bigvertex border] (RRL.north) arc(90:-60:\vertexround)
    -- (IRL.-60) arc(-60:-80:\vertexround)
	-- (LRL.-70) arc (-70:-90:\vertexround)
    -- (LRR.south) arc(-90:-240:\vertexround) -- (IRR.120)
    -- (RRR.120) arc (120:90:\vertexround) -- (RRL.north);

    \draw[transition,->,>=stealth](-0.6,0) -- (LLR);
    \transition{ILL.-60}{intersection of LIL.-240 -- RIL.120 and ILR.180 -- LRR.180}{}{}
    \transition{IRR.120}{intersection of RIR.-50 -- LIR.310 and IRL.0 -- RLL.0}{}{}
    \selftransition[-.5mm:-3mm]{RRL.north}
    \selftransition[-.1mm:1mm]{RIL.north}
    \begin{scope}[yscale=-1]
      \selftransition[-.5mm:-3mm]{LLL.south}
    \end{scope}

    \only<all:5>{
      \transition[gray,->]{LLR}{LLL}{}{}
      \transition[gray,->]{LLL}{ILL}{}{}
      \transition[gray,->]{ILL}{IIL}{}{}
      \transition[gray,->]{IIL}{IIR}{}{}
      \transition[gray,->]{IIR}{RIR}{}{}
      \transition[gray,->]{RIR}{RRR}{}{}
    }
  \end{tikzpicture}
}

\newcommand{\pictwopackagesabstractpackone}{%
  \begin{tikzpicture}[scale=1.28]
    \vertex{0,0}{LLR}
    \vertex{1,0}{LLL}
    \vertex{2,.5}{ILL}
    \vertex{2,-.5}{LIL}
    \vertex{3,0}{IIL}
    \vertex{4,0}{IIR}
    \vertex{5,.5}{RIR}
    \vertex{5,-.5}{IRR}
    \goalvertex{6,0}{RRR}
    \goalvertex{7,0}{RRL}
    \vertex{2.3,1.2}{ILR}
    \vertex{3.1,1.5}{RLR}
    \vertex{3.85,1.5}{RLL}
    \vertex{4.7,1.2}{RIL}
    \vertex{2.3,-1.2}{LIR}
    \vertex{3.1,-1.5}{LRR}
    \vertex{3.85,-1.5}{LRL}
    \vertex{4.7,-1.2}{IRL}
    
    \draw[bigvertex interior] (LLR.north) arc(90:240:\vertexround)
    -- (LIR.-120) arc(-120:-100:\vertexround) 
    -- (LRR.-110) arc (-110:-90:\vertexround)
    -- (LRL.south) arc(-90:60:\vertexround) -- (LIL.60)
    -- (LLL.-300) arc (60:90:\vertexround) -- (LLR.north);
    \hiddenvertices{LLR}{LLL}{LIL}{LIR}
    \hiddenvertices{LRR}{LRL}{}{}
    \draw[bigvertex border] (LLR.north) arc(90:240:\vertexround)
    -- (LIR.-120) arc(-120:-100:\vertexround) 
    -- (LRR.-110) arc (-110:-90:\vertexround)
    -- (LRL.south) arc(-90:60:\vertexround) -- (LIL.60)
    -- (LLL.-300) arc (60:90:\vertexround) -- (LLR.north);
    
    \draw[bigvertex interior] (ILR.-310) arc(50:165:\vertexround)
    -- (ILL.-200) arc(160:240:\vertexround) 
    -- (IRL.-120) arc(-120:-10:\vertexround)
    -- (IRR.-20) arc(-20:50:\vertexround)
    -- (ILR.-310);
    \hiddenvertices{ILR}{ILL}{IIL}{IIR}
    \hiddenvertices{IRR}{IRL}{}{}
    \draw[bigvertex border] (ILR.-310) arc(50:165:\vertexround)
    -- (ILL.-200) arc(160:240:\vertexround) 
    -- (IRL.-120) arc(-120:-10:\vertexround)
    -- (IRR.-20) arc(-20:50:\vertexround)
    -- (ILR.-310);

    \draw[bigvertex interior, color=green!40] (RRL.south) arc(-90:60:\vertexround)
    -- (RIL.60) arc(60:80:\vertexround)
    -- (RLL.70) arc (70:90:\vertexround)
    -- (RLR.north) arc(90:240:\vertexround) -- (RIR.-120)
    -- (RRR.-120) arc (-120:-90:\vertexround) -- (RRL.south);
    \hiddenvertices{RLR}{RLL}{RIL}{RIR}
    \hiddenvertices{RRR}{RRL}{}{}
    \draw[bigvertex border] (RRL.south) arc(-90:60:\vertexround)
    -- (RIL.60) arc(60:80:\vertexround)
    -- (RLL.70) arc (70:90:\vertexround)
    -- (RLR.north) arc(90:240:\vertexround) -- (RIR.-120)
    -- (RRR.-120) arc (-120:-90:\vertexround) -- (RRL.south);

    \draw[transition,->,>=stealth](-0.6,0) -- (LLR);
    \transition{LIL.60}{intersection of ILL.240 -- IRL.240 and LIR.180 -- RLR.180}{}{}
    \transition{LIL.60}{intersection of ILL.240 -- IRL.-120 and LIR.-180 -- RLR.-180}{}{}
    \transition{RIR.-120}{intersection of IRR.50 -- ILR.-310 and RIL.0 -- LRL.0}{}{}
    \selftransition[-.5mm:-3mm]{LLL.north}
    \selftransition[-.1mm:1mm]{ILR.north}
    \begin{scope}[yscale=-1]
      \selftransition[-.5mm:-3mm]{RRL.south}
    \end{scope}

    \only<all:3>{
      \transition[gray,->]{LLR}{LLL}{}{}
      \transition[gray,->]{LLL}{ILL}{}{}
      \transition[gray,->]{ILL}{IIL}{}{}
      \transition[gray,->]{IIL}{IIR}{}{}
      \transition[gray,->]{IIR}{RIR}{}{}
      \transition[gray,->]{RIR}{RRR}{}{}
    }
  \end{tikzpicture}
}

\newcommand{\bendtransition}[6][]{\draw[->,>=stealth,transition,#1] (#2) edge [#6] node[above=-1mm,sloped,#5] {\scriptsize #4} (#3);}
\newcommand{\bendtransitionnolabel}[6][]{\draw[->,>=stealth,transition,#1] (#2) edge [#6] node[above=-1mm,sloped,#5] {} (#3);}
\newcommand{\picatomicprojectionpackage}{%
  \begin{tikzpicture}[->,>=stealth,looseness=.5,bend angle=20]
       	\vertex{0,0}{L}
	\vertex{2,1}{A}
	\vertex{2,-1}{B}
	\goalvertex{4,0}{R}

	\draw[transition,->,>=stealth](-0.8,0) -- (L);
	\bendtransition{L}{L}{M{\wildcard}{\wildcard}{\wildcard}}{above left}{loop above}
	\bendtransition{L}{A}{PAL}{}{bend left}
	\bendtransition{A}{L}{DAL}{}{bend left}
	\bendtransition{A}{A}{M{\wildcard}{\wildcard}{\wildcard}}{above=1mm}{loop above}
	\bendtransition{A}{R}{DAR}{}{bend left}
	\bendtransition{R}{A}{PAR}{}{bend left}
	\bendtransition{R}{R}{M{\wildcard}{\wildcard}{\wildcard}}{above right}{loop above}
	\bendtransition{R}{B}{PBR}{below=-2mm}{bend left}
	\bendtransition{B}{R}{DBR}{below=-2mm}{bend left}
	\bendtransition{B}{B}{M{\wildcard}{\wildcard}{\wildcard}}{below=-1mm}{loop below}
	\bendtransition{B}{L}{DBL}{below=-2mm}{bend left}
	\bendtransition{L}{B}{PBL}{below=-2mm}{bend left}
  \end{tikzpicture}
}

\newcommand{\picatomicprojectionpackagenumberedstates}{%
  \begin{tikzpicture}[->,>=stealth,looseness=.5,bend angle=20]
        \numberedvertex{0,0}{L}{0}
	\numberedvertex{2,1}{A}{2}
	\numberedvertex{2,-1}{B}{3}
	\numberedgoalvertex{4,0}{R}{1}

	\draw[transition,->,>=stealth](-0.8,0) -- (L);
	\bendtransition{L}{L}{M{\wildcard}{\wildcard}{\wildcard}}{above left}{loop above}
	\bendtransition{L}{A}{PAL}{}{bend left}
	\bendtransition{A}{L}{DAL}{}{bend left}
	\bendtransition{A}{A}{M{\wildcard}{\wildcard}{\wildcard}}{above=1mm}{loop above}
	\bendtransition{A}{R}{DAR}{}{bend left}
	\bendtransition{R}{A}{PAR}{}{bend left}
	\bendtransition{R}{R}{M{\wildcard}{\wildcard}{\wildcard}}{above right}{loop above}
	\bendtransition{R}{B}{PBR}{below=-2mm}{bend left}
	\bendtransition{B}{R}{DBR}{below=-2mm}{bend left}
	\bendtransition{B}{B}{M{\wildcard}{\wildcard}{\wildcard}}{below=-1mm}{loop below}
	\bendtransition{B}{L}{DBL}{below=-2mm}{bend left}
	\bendtransition{L}{B}{PBL}{below=-2mm}{bend left}
  \end{tikzpicture}
}

\newcommand{\picatomicprojectiontrucka}{%
  \begin{tikzpicture}[->,>=stealth,looseness=.5]
	\goalvertex{0,0}{L}
	\goalvertex{4,0}{R}

	\draw[transition,->,>=stealth](4.8,0) -- (R);
	\bendtransition{L}{L}{\parbox{22mm}{PAL,DAL,MB{\wildcard}{\wildcard},\\PB{\wildcard},DB{\wildcard}}}{above}{loop above}
	\bendtransition{L}{R}{MALR}{}{bend left}
	\bendtransition{R}{L}{MARL}{}{bend left}
	\bendtransition{R}{R}{\parbox{22mm}{PAR,DAR,MB{\wildcard}{\wildcard},\\PB{\wildcard},DB{\wildcard}}}{above}{loop above}
  \end{tikzpicture}
}

\newcommand{\picatomicprojectiontruckb}{%
  \begin{tikzpicture}[->,>=stealth,looseness=.5]
	\goalvertex{0,0}{L}
	\goalvertex{4,0}{R}

	\draw[transition,->,>=stealth](4.8,0) -- (R);
	\bendtransition{L}{L}{\parbox{22mm}{PBL,DBL,MA{\wildcard}{\wildcard},\\PA{\wildcard},DA{\wildcard}}}{above}{loop above}
	\bendtransition{L}{R}{MBLR}{}{bend left}
	\bendtransition{R}{L}{MBRL}{}{bend left}
	\bendtransition{R}{R}{\parbox{22mm}{PBR,DBR,MA{\wildcard}{\wildcard},\\PA{\wildcard},DA{\wildcard}}}{above}{loop above}
  \end{tikzpicture}
}

\newcommand{\picproductpackagetrucka}[1][]{%
  \begin{tikzpicture}[->,>=stealth,looseness=.5,bend angle=20,#1]
  	\vertex{0,0}{LL}
  	\vertex{2,0}{LR}
	\vertex{3,2}{AL}
	\vertex{5,2}{AR}
	\vertex{3,-2}{BL}
	\vertex{5,-2}{BR}
  	\goalvertex{6,0}{RL}
  	\goalvertex{8,0}{RR}
	
	\draw[transition,->,>=stealth](2.6,0.6) -- (LR);
	\bendtransition{LL}{LR}{MALR}{}{bend left}
	\bendtransition{LR}{LL}{MARL}{below=-2mm}{bend left}
	\bendtransition{AL}{AR}{MALR}{}{bend left}
	\bendtransition{AR}{AL}{MARL}{below=-2mm}{bend left}
	\bendtransition{BL}{BR}{MALR}{}{bend left}
	\bendtransition{BR}{BL}{MARL}{below=-2mm}{bend left}
	\bendtransition{RL}{RR}{MALR}{}{bend left}
	\bendtransition{RR}{RL}{MARL}{below=-2mm}{bend left}
	
	\bendtransition{LL}{AL}{PAL}{}{bend left=50}
	\bendtransition{AL}{LL}{DAL}{}{bend right}
	\bendtransition{AR}{RR}{DAR}{}{bend left=50}
	\bendtransition{RR}{AR}{PAR}{}{bend right}
	\bendtransition{RR}{BR}{PBR}{}{bend left=50}
	\bendtransition{BR}{RR}{DBR}{}{bend right}
	\bendtransition{BL}{LL}{DBL}{}{bend left=50}
	\bendtransition{LL}{BL}{PBL}{}{bend right}
	
	\bendtransition{LR}{BR}{PBL}{near start}{bend left=50}
	\bendtransition{BR}{LR}{DBL}{near end,below=-2mm}{bend right}
	\bendtransition{BL}{RL}{DBR}{near end}{bend left=50}
	\bendtransition{RL}{BL}{PBR}{near start,below=-2mm}{bend right}
	
	\bendtransition{LL}{LL}{MB{\wildcard}{\wildcard}}{above=2mm,pos=.9,sloped=false}{loop left}
	\bendtransition{RR}{RR}{MB{\wildcard}{\wildcard}}{above=2mm,pos=.1,sloped=false}{loop right}
	\bendtransition{AL}{AL}{MB{\wildcard}{\wildcard}}{}{loop above}
	\bendtransition{AR}{AR}{MB{\wildcard}{\wildcard}}{}{loop above}
	\bendtransition{LR}{LR}{MB{\wildcard}{\wildcard}}{}{loop above}
	\bendtransition{RL}{RL}{MB{\wildcard}{\wildcard}}{}{loop above}
	\bendtransition{BL}{BL}{MB{\wildcard}{\wildcard}}{below=-1mm}{loop below}
	\bendtransition{BR}{BR}{MB{\wildcard}{\wildcard}}{below=-1mm}{loop below}
  \end{tikzpicture}
}

\newcommand{\picproductpackagetruckanumberedstates}[1][]{%
  \begin{tikzpicture}[->,>=stealth,looseness=.5,bend angle=20,#1]
  	\numberedvertex{0,0}{LL}{0}
  	\numberedvertex{2,0}{LR}{1}
	\numberedvertex{3,2}{AL}{4}
	\numberedvertex{5,2}{AR}{5}
	\numberedvertex{3,-2}{BL}{6}
	\numberedvertex{5,-2}{BR}{7}
  	\numberedgoalvertex{6,0}{RL}{2}
  	\numberedgoalvertex{8,0}{RR}{3}
	
	\draw[transition,->,>=stealth](2.6,0.6) -- (LR);
	\bendtransition{LL}{LR}{MALR}{}{bend left}
	\bendtransition{LR}{LL}{MARL}{below=-2mm}{bend left}
	\bendtransition{AL}{AR}{MALR}{}{bend left}
	\bendtransition{AR}{AL}{MARL}{below=-2mm}{bend left}
	\bendtransition{BL}{BR}{MALR}{}{bend left}
	\bendtransition{BR}{BL}{MARL}{below=-2mm}{bend left}
	\bendtransition{RL}{RR}{MALR}{}{bend left}
	\bendtransition{RR}{RL}{MARL}{below=-2mm}{bend left}
	
	\bendtransition{LL}{AL}{PAL}{}{bend left=50}
	\bendtransition{AL}{LL}{DAL}{}{bend right}
	\bendtransition{AR}{RR}{DAR}{}{bend left=50}
	\bendtransition{RR}{AR}{PAR}{}{bend right}
	\bendtransition{RR}{BR}{PBR}{}{bend left=50}
	\bendtransition{BR}{RR}{DBR}{}{bend right}
	\bendtransition{BL}{LL}{DBL}{}{bend left=50}
	\bendtransition{LL}{BL}{PBL}{}{bend right}
	
	\bendtransition{LR}{BR}{PBL}{near start}{bend left=50}
	\bendtransition{BR}{LR}{DBL}{near end,below=-2mm}{bend right}
	\bendtransition{BL}{RL}{DBR}{near end}{bend left=50}
	\bendtransition{RL}{BL}{PBR}{near start,below=-2mm}{bend right}
	
	\bendtransition{LL}{LL}{MB{\wildcard}{\wildcard}}{above=2mm,pos=.9,sloped=false}{loop left}
	\bendtransition{RR}{RR}{MB{\wildcard}{\wildcard}}{above=2mm,pos=.1,sloped=false}{loop right}
	\bendtransition{AL}{AL}{MB{\wildcard}{\wildcard}}{}{loop above}
	\bendtransition{AR}{AR}{MB{\wildcard}{\wildcard}}{}{loop above}
	\bendtransition{LR}{LR}{MB{\wildcard}{\wildcard}}{}{loop above}
	\bendtransition{RL}{RL}{MB{\wildcard}{\wildcard}}{}{loop above}
	\bendtransition{BL}{BL}{MB{\wildcard}{\wildcard}}{below=-1mm}{loop below}
	\bendtransition{BR}{BR}{MB{\wildcard}{\wildcard}}{below=-1mm}{loop below}
  \end{tikzpicture}
}

\newcommand{\subpicatomicprojectionpackagesmall}{%
  \begin{scope}[->,>=stealth,looseness=.5,bend angle=20,scale=.7,transform shape]
	\vertex{0,0}{L}
	\vertex{2,1}{A}
	\vertex{2,-1}{B}
	\goalvertex{4,0}{R}

	\draw[transition,->,>=stealth](-0.8,0) -- (L);
	\bendtransition{L}{L}{M{\wildcard}{\wildcard}{\wildcard}}{above left}{loop above}
	\bendtransition{L}{A}{PAL}{}{bend left}
	\bendtransition{A}{L}{DAL}{}{bend left}
	\bendtransition{A}{A}{M{\wildcard}{\wildcard}{\wildcard}}{above=1mm}{loop above}
	\bendtransition{A}{R}{DAR}{}{bend left}
	\bendtransition{R}{A}{PAR}{}{bend left}
	\bendtransition{R}{R}{M{\wildcard}{\wildcard}{\wildcard}}{above right}{loop above}
	\bendtransition{R}{B}{PBR}{below=-2mm}{bend left}
	\bendtransition{B}{R}{DBR}{below=-2mm}{bend left}
	\bendtransition{B}{B}{M{\wildcard}{\wildcard}{\wildcard}}{below=-1mm}{loop below}
	\bendtransition{B}{L}{DBL}{below=-2mm}{bend left}
	\bendtransition{L}{B}{PBL}{below=-2mm}{bend left}
	
	\only<all:2>{\vertex[draw=red,thick]{2,1}{A}}
	\only<all:3>{\vertex[draw=red,thick]{0,0}{L}}
	\only<all:3>{\draw[transition,->,>=stealth,red](-0.8,0) -- (L);}
	\only<all:4>{\goalvertex[fill=green]{4,0}{R}}
	\only<all:5>{\bendtransition[red]{L}{A}{PAL}{}{bend left}}
	\only<all:6>{\bendtransition[red]{L}{L}{M{\wildcard}{\wildcard}{\wildcard}}{above left}{loop above}}
	\only<all:7>{\bendtransition[red]{L}{B}{PBL}{below=-2mm}{bend left}}
	\only<all:8>{\bendtransition[red]{A}{A}{M{\wildcard}{\wildcard}{\wildcard}}{above=1mm}{loop above}}
  \end{scope}
}

\newcommand{\subpicatomicprojectiontruckasmall}{%
  \begin{scope}[->,>=stealth,looseness=.5,scale=.7,transform shape]
	\goalvertex{0,0}{L}
	\goalvertex{4,0}{R}

	\draw[transition,->,>=stealth](4.8,0) -- (R);
	\bendtransition{L}{L}{\parbox{22mm}{PAL,DAL,MB{\wildcard}{\wildcard},\\PB{\wildcard},DB{\wildcard}}}{above}{loop above}
	\bendtransition{L}{R}{MALR}{}{bend left}
	\bendtransition{R}{L}{MARL}{}{bend left}
	\bendtransition{R}{R}{\parbox{22mm}{PAR,DAR,MB{\wildcard}{\wildcard},\\PB{\wildcard},DB{\wildcard}}}{above}{loop above}

	\only<all:2>{\goalvertex[draw=red,thick]{0,0}{L}}
	\only<all:3>{\goalvertex[draw=red,thick]{4,0}{R}}
	\only<all:3>{\draw[transition,->,>=stealth,red](4.8,0) -- (R);}
	\only<all:4>{\goalvertex[fill=green]{0,0}{L}}
	\only<all:5>{\bendtransition[red]{L}{L}{\parbox{22mm}{PAL\color{black},DAL,MB{\wildcard}{\wildcard},\\PB{\wildcard},DB{\wildcard}}}{above}{loop above}}
	\only<all:6>{\bendtransition[red]{L}{R}{MALR}{}{bend left}}
	\only<all:7>{\bendtransition[red]{R}{R}{\parbox{22mm}{\color{black}PAR,DAR,MB{\wildcard}{\wildcard},\\{\color{red}PB{\wildcard}},DB{\wildcard}}}{above}{loop above}}
	\only<all:8>{\bendtransition[red]{L}{L}{\parbox{22mm}{\color{black}PAL,DAL,{\color{red}MB{\wildcard}{\wildcard}},\\PB{\wildcard},DB{\wildcard}}}{above}{loop above}}
  \end{scope}
}

\newcommand{\subpicproductpackagetruckasmall}{%
  \begin{scope}[->,>=stealth,looseness=.5,,bend angle=20,scale=.7,transform shape]
  	\vertex{0,0}{LL}
  	\vertex{2,0}{LR}
	\vertex{3,2}{AL}
	\vertex{5,2}{AR}
	\vertex{3,-2}{BL}
	\vertex{5,-2}{BR}
  	\goalvertex{6,0}{RL}
  	\goalvertex{8,0}{RR}
	
	\draw[transition,->,>=stealth](2.6,0.6) -- (LR);
	\bendtransition{LL}{LR}{MALR}{}{bend left}
	\bendtransition{LR}{LL}{MARL}{below=-2mm}{bend left}
	\bendtransition{AL}{AR}{MALR}{}{bend left}
	\bendtransition{AR}{AL}{MARL}{below=-2mm}{bend left}
	\bendtransition{BL}{BR}{MALR}{}{bend left}
	\bendtransition{BR}{BL}{MARL}{below=-2mm}{bend left}
	\bendtransition{RL}{RR}{MALR}{}{bend left}
	\bendtransition{RR}{RL}{MARL}{below=-2mm}{bend left}
	
	\bendtransition{LL}{AL}{PAL}{}{bend left=50}
	\bendtransition{AL}{LL}{DAL}{}{bend right}
	\bendtransition{AR}{RR}{DAR}{}{bend left=50}
	\bendtransition{RR}{AR}{PAR}{}{bend right}
	\bendtransition{RR}{BR}{PBR}{}{bend left=50}
	\bendtransition{BR}{RR}{DBR}{}{bend right}
	\bendtransition{BL}{LL}{DBL}{}{bend left=50}
	\bendtransition{LL}{BL}{PBL}{}{bend right}
	
	\bendtransition{LR}{BR}{PBL}{near start}{bend left=50}
	\bendtransition{BR}{LR}{DBL}{near end,below=-2mm}{bend right}
	\bendtransition{BL}{RL}{DBR}{near end}{bend left=50}
	\bendtransition{RL}{BL}{PBR}{near start,below=-2mm}{bend right}
	
	\bendtransition{LL}{LL}{MB{\wildcard}{\wildcard}}{above=2mm,sloped=false,pos=.9}{loop left}
	\bendtransition{RR}{RR}{MB{\wildcard}{\wildcard}}{above=2mm,sloped=false,pos=.1}{loop right}
	\bendtransition{AL}{AL}{MB{\wildcard}{\wildcard}}{}{loop above}
	\bendtransition{AR}{AR}{MB{\wildcard}{\wildcard}}{}{loop above}
	\bendtransition{LR}{LR}{MB{\wildcard}{\wildcard}}{}{loop above}
	\bendtransition{RL}{RL}{MB{\wildcard}{\wildcard}}{}{loop above}
	\bendtransition{BL}{BL}{MB{\wildcard}{\wildcard}}{below=-1mm}{loop below}
	\bendtransition{BR}{BR}{MB{\wildcard}{\wildcard}}{below=-1mm}{loop below}
	
	\only<all:2>{\vertex[draw=red,thick]{3,2}{AL}}
	\only<all:3>{\vertex[draw=red,thick]{2,0}{LR}}
	\only<all:3>{\draw[transition,->,>=stealth,red](2.6,0.6) -- (LR);}
  	\only<all:4>{\goalvertex[fill=green]{6,0}{RL}}
	\only<all:5>{\bendtransition[red]{LL}{AL}{PAL}{}{bend left=50}}
	\only<all:6>{\bendtransition[red]{LL}{LR}{MALR}{}{bend left}}
	\only<all:7>{\bendtransition[red]{LR}{BR}{PBL}{near start}{bend left=50}}
	\only<all:8>{\bendtransition[red]{AL}{AL}{MB{\wildcard}{\wildcard}}{}{loop above}}
  \end{scope}
}

\newcommand{\picproductpackagetruckaillustration}{%
  \begin{tikzpicture}[->,>=stealth,looseness=.5,,bend angle=20,scale=0.98]
  	\subpicatomicprojectionpackagesmall
	\node (otimes) at (4,0) {\LARGE $\otimes$};	
	\begin{scope}[xshift=5cm]
  	\subpicatomicprojectiontruckasmall
	\end{scope}
	\node (otimes) at (0,-3) {\LARGE $=$};	
	\begin{scope}[xshift=2cm,yshift=-3cm]
  	\subpicproductpackagetruckasmall
	\end{scope}
  \end{tikzpicture}
}


\newcommand{\picshrinkexampleone}[1][]{%
  \begin{tikzpicture}[->,>=stealth,looseness=.5,,bend angle=20,#1]
  	\vertex{0,0}{LL}
  	\vertex{2,0}{LR}
	\only<all:1-4>{\vertex{3,2}{AL}}
	\only<all:1-4>{\vertex{5,2}{AR}}
	\only<all:4>{\vertex[draw=red,thick]{3,2}{AL}}
	\only<all:4>{\vertex[draw=red,thick]{5,2}{AR}}
	\only<all:5-8>{\vertex{4,2}{A}}
	\only<all:8>{\vertex[draw=red,thick]{4,2}{A}}

	\only<all:1-6>{\vertex{3,-2}{BL}}
	\only<all:1-6>{\vertex{5,-2}{BR}}
	\only<all:6>{\vertex[draw=red,thick]{3,-2}{BL}}
	\only<all:6>{\vertex[draw=red,thick]{5,-2}{BR}}
	\only<all:7-8>{\vertex{4,-2}{B}}
	\only<all:8>{\vertex[draw=red,thick]{4,-2}{B}}
	\only<all:9->{\vertex{4,0}{I}}

	\only<all:1>{\goalvertex{6,0}{RL}}
  	\only<all:1>{\goalvertex{8,0}{RR}}
  	\only<all:2>{\goalvertex[draw=red,thick]{6,0}{RL}}
  	\only<all:2>{\goalvertex[draw=red,thick]{8,0}{RR}}
	\only<all:3->{\goalvertex{7,0}{R}}

	
	\draw[transition,->,>=stealth](2.6,0.6) -- (LR);
	\bendtransition{LL}{LR}{MALR}{}{bend left}
	\only<all:1-8>{\bendtransition{LR}{LL}{MARL}{below=-2mm}{bend left}}
	\only<all:9->{\bendtransition{LR}{LL}{MARL}{}{bend left}}
	\only<all:1-4>{\bendtransition{AL}{AR}{MALR}{}{bend left}}
	\only<all:1-4>{\bendtransition{AR}{AL}{MARL}{below=-2mm}{bend left}}
	\only<all:1-6>{\bendtransition{BL}{BR}{MALR}{}{bend left}}
	\only<all:1-6>{\bendtransition{BR}{BL}{MARL}{below=-2mm}{bend left}}
	\only<all:1-2>{\bendtransition{RL}{RR}{MALR}{}{bend left}}
	\only<all:1-2>{\bendtransition{RR}{RL}{MARL}{below=-2mm}{bend left}}
	
	\only<all:1-4>{\bendtransition{LL}{AL}{PAL}{}{bend left=50}}
	\only<all:1-4>{\bendtransition{AL}{LL}{DAL}{}{bend right}}
	\only<all:5-8>{\bendtransition{LL}{A}{PAL}{}{bend left=50}}
	\only<all:5-8>{\bendtransition{A}{LL}{DAL}{}{bend right}}
	\only<all:1-2>{\bendtransition{AR}{RR}{DAR}{}{bend left=50}}
	\only<all:1-2>{\bendtransition{RR}{AR}{PAR}{}{bend right}}
	\only<all:3-4>{\bendtransition{AR}{R}{DAR}{}{bend left=50}}
	\only<all:3-4>{\bendtransition{R}{AR}{PAR}{}{bend right}}
	\only<all:5-8>{\bendtransition{A}{R}{DAR}{}{bend left=50}}
	\only<all:5-8>{\bendtransition{R}{A}{PAR}{}{bend right}}
	\only<all:1-2>{\bendtransition{RR}{BR}{PBR}{}{bend left=50}}
	\only<all:1-2>{\bendtransition{BR}{RR}{DBR}{}{bend right}}
	\only<all:3-6>{\bendtransition{R}{BR}{PBR}{}{bend left=50}}
	\only<all:3-6>{\bendtransition{BR}{R}{DBR}{}{bend right}}
	\only<all:7-8>{\bendtransition{R}{B}{PBR}{}{bend left=50}}
	\only<all:7-8>{\bendtransition{B}{R}{DBR}{}{bend right}}
	\only<all:1-6>{\bendtransition{BL}{LL}{DBL}{}{bend left=50}}
	\only<all:1-6>{\bendtransition{LL}{BL}{PBL}{}{bend right}}
	\only<all:7-8>{\bendtransition{B}{LL}{DBL}{}{bend left=50}}
	\only<all:7-8>{\bendtransition{LL}{B}{PBL}{}{bend right}}
	
	\only<all:1-6>{\bendtransition{LR}{BR}{PBL}{near start}{bend left=50}}
	\only<all:1-6>{\bendtransition{BR}{LR}{DBL}{near end,below=-2mm}{bend right}}
	\only<all:7-8>{\bendtransition{LR}{B}{PBL}{near start}{bend left=50}}
	\only<all:7-8>{\bendtransition{B}{LR}{DBL}{near end,below=-2mm}{bend right}}
	\only<all:1-2>{\bendtransition{BL}{RL}{DBR}{near end}{bend left=50}}%
	\only<all:1-2>{\bendtransition{RL}{BL}{PBR}{near start,below=-2mm}{bend right}}%
	\only<all:3-6>{\bendtransition{BL}{R}{DBR}{near end}{bend left=50}}%
	\only<all:3-6>{\bendtransition{R}{BL}{PBR}{near start,below=-2mm}{bend right}}%
	
	\bendtransition{LL}{LL}{MB{\wildcard}{\wildcard}}{above=2mm,sloped=false,pos=.9}{loop left}
	\only<all:1-2>{\bendtransition{RR}{RR}{MB{\wildcard}{\wildcard}}{above=2mm,sloped=false,pos=.1}{loop right}}
	\only<all:1-4>{\bendtransition{AL}{AL}{MB{\wildcard}{\wildcard}}{}{loop above}}
	\only<all:1-4>{\bendtransition{AR}{AR}{MB{\wildcard}{\wildcard}}{}{loop above}}
	\only<all:5-8>{\bendtransition{A}{A}{M{\wildcard}{\wildcard}{\wildcard}}{}{loop above}}
	\bendtransition{LR}{LR}{MB{\wildcard}{\wildcard}}{}{loop above}
	\only<all:1-2>{\bendtransition{RL}{RL}{MB{\wildcard}{\wildcard}}{}{loop above}}
	\only<all:1-6>{\bendtransition{BL}{BL}{MB{\wildcard}{\wildcard}}{below=-1mm}{loop below}}
	\only<all:7-8>{\bendtransition{B}{B}{M{\wildcard}{\wildcard}{\wildcard}}{below=-1mm}{loop below}}
	\only<all:3->{\bendtransition{R}{R}{M{\wildcard}{\wildcard}{\wildcard}}{above=2mm,sloped=false}{loop right}}
	\only<all:1-6>{\bendtransition{BR}{BR}{MB{\wildcard}{\wildcard}}{below=-1mm}{loop below}}

	\only<all:9->{\bendtransition{I}{R}{D{\wildcard}R}{}{bend left}}
	\only<all:9->{\bendtransition{R}{I}{P{\wildcard}R}{}{bend left}}
	\only<all:9->{\bendtransition{I}{I}{M{\wildcard}{\wildcard}{\wildcard}}{}{loop above}}
	\only<all:9->{\bendtransition{LR}{I}{PBL}{}{bend left}}
	\only<all:9->{\bendtransition{I}{LR}{DBL}{}{bend left}}
	\only<all:9->{\bendtransition{LL}{I}{P{\wildcard}L}{}{bend right=55}}
	\only<all:9->{\bendtransition{I}{LL}{D{\wildcard}L}{below=-2mm}{bend left=80}}
  \end{tikzpicture}
}

\newcommand{\picshrinkexampleonenumberedstates}[1][]{%
  \begin{tikzpicture}[->,>=stealth,looseness=.5,,bend angle=20,#1]
        \useasboundingbox (-1.5,-3.5) rectangle (9.5,3.5);
        %% \draw (-1.5,-3.5) rectangle (9.5,3.5);
  	\numberedvertex{0,0}{LL}{0}
  	\numberedvertex{2,0}{LR}{1}
	\only<all:1-5>{\numberedvertex{3,2}{AL}{4}}
	\only<all:1-5>{\numberedvertex{5,2}{AR}{5}}
	\only<all:5>{\numberedvertex[draw=red,thick]{3,2}{AL}{4}}
	\only<all:5>{\numberedvertex[draw=red,thick]{5,2}{AR}{5}}
	\only<all:6-9>{\numberedvertex{4,2}{A}{4}}
	\only<all:9>{\numberedvertex[draw=red,thick]{4,2}{A}{4}}

	\only<all:1-7>{\numberedvertex{3,-2}{BL}{6}}
	\only<all:1-7>{\numberedvertex{5,-2}{BR}{7}}
	\only<all:7>{\numberedvertex[draw=red,thick]{3,-2}{BL}{6}}
	\only<all:7>{\numberedvertex[draw=red,thick]{5,-2}{BR}{7}}
	\only<all:8-9>{\numberedvertex{4,-2}{B}{6}}
	\only<all:9>{\numberedvertex[draw=red,thick]{4,-2}{B}{6}}
	\only<all:10-11>{\numberedvertex{4,0}{I}{4}}
        \only<all:12>{\numberedvertex{4,0}{I}{\alert{4$\mapsto$3}}}
        \only<all:13->{\numberedvertex{4,0}{I}{\alert<all:13>{3}}}

	\only<all:1-2>{\numberedgoalvertex{6,0}{RL}{2}}
  	\only<all:1-2>{\numberedgoalvertex{8,0}{RR}{3}}
  	\only<all:3>{\numberedgoalvertex[draw=red,thick]{6,0}{RL}{2}}
  	\only<all:3>{\numberedgoalvertex[draw=red,thick]{8,0}{RR}{3}}
	\only<all:4->{\numberedgoalvertex{7,0}{R}{2}}

	
	\draw[transition,->,>=stealth](2.6,0.6) -- (LR);
	\bendtransition{LL}{LR}{MALR}{}{bend left}
	\only<all:1-9>{\bendtransition{LR}{LL}{MARL}{below=-2mm}{bend left}}
	\only<all:10->{\bendtransition{LR}{LL}{MARL}{}{bend left}}
	\only<all:1-5>{\bendtransition{AL}{AR}{MALR}{}{bend left}}
	\only<all:1-5>{\bendtransition{AR}{AL}{MARL}{below=-2mm}{bend left}}
	\only<all:1-7>{\bendtransition{BL}{BR}{MALR}{}{bend left}}
	\only<all:1-7>{\bendtransition{BR}{BL}{MARL}{below=-2mm}{bend left}}
	\only<all:1-3>{\bendtransition{RL}{RR}{MALR}{}{bend left}}
	\only<all:1-3>{\bendtransition{RR}{RL}{MARL}{below=-2mm}{bend left}}
	
	\only<all:1-5>{\bendtransition{LL}{AL}{PAL}{}{bend left=50}}
	\only<all:1-5>{\bendtransition{AL}{LL}{DAL}{}{bend right}}
	\only<all:6-9>{\bendtransition{LL}{A}{PAL}{}{bend left=50}}
	\only<all:6-9>{\bendtransition{A}{LL}{DAL}{}{bend right}}
	\only<all:1-3>{\bendtransition{AR}{RR}{DAR}{}{bend left=50}}
	\only<all:1-3>{\bendtransition{RR}{AR}{PAR}{}{bend right}}
	\only<all:4-5>{\bendtransition{AR}{R}{DAR}{}{bend left=50}}
	\only<all:4-5>{\bendtransition{R}{AR}{PAR}{}{bend right}}
	\only<all:6-9>{\bendtransition{A}{R}{DAR}{}{bend left=50}}
	\only<all:6-9>{\bendtransition{R}{A}{PAR}{}{bend right}}
	\only<all:1-3>{\bendtransition{RR}{BR}{PBR}{}{bend left=50}}
	\only<all:1-3>{\bendtransition{BR}{RR}{DBR}{}{bend right}}
	\only<all:4-7>{\bendtransition{R}{BR}{PBR}{}{bend left=50}}
	\only<all:4-7>{\bendtransition{BR}{R}{DBR}{}{bend right}}
	\only<all:8-9>{\bendtransition{R}{B}{PBR}{}{bend left=50}}
	\only<all:8-9>{\bendtransition{B}{R}{DBR}{}{bend right}}
	\only<all:1-7>{\bendtransition{BL}{LL}{DBL}{}{bend left=50}}
	\only<all:1-7>{\bendtransition{LL}{BL}{PBL}{}{bend right}}
	\only<all:8-9>{\bendtransition{B}{LL}{DBL}{}{bend left=50}}
	\only<all:8-9>{\bendtransition{LL}{B}{PBL}{}{bend right}}
	
	\only<all:1-7>{\bendtransition{LR}{BR}{PBL}{near start}{bend left=50}}
	\only<all:1-7>{\bendtransition{BR}{LR}{DBL}{near end,below=-2mm}{bend right}}
	\only<all:8-9>{\bendtransition{LR}{B}{PBL}{near start}{bend left=50}}
	\only<all:8-9>{\bendtransition{B}{LR}{DBL}{near end,below=-2mm}{bend right}}
	\only<all:1-3>{\bendtransition{BL}{RL}{DBR}{near end}{bend left=50}}%
	\only<all:1-3>{\bendtransition{RL}{BL}{PBR}{near start,below=-2mm}{bend right}}%
	\only<all:4-7>{\bendtransition{BL}{R}{DBR}{near end}{bend left=50}}%
	\only<all:4-7>{\bendtransition{R}{BL}{PBR}{near start,below=-2mm}{bend right}}%
	
	\bendtransition{LL}{LL}{MB{\wildcard}{\wildcard}}{above=2mm,sloped=false,pos=.9}{loop left}
	\only<all:1-3>{\bendtransition{RR}{RR}{MB{\wildcard}{\wildcard}}{above=2mm,sloped=false,pos=.1}{loop right}}
	\only<all:1-5>{\bendtransition{AL}{AL}{MB{\wildcard}{\wildcard}}{}{loop above}}
	\only<all:1-5>{\bendtransition{AR}{AR}{MB{\wildcard}{\wildcard}}{}{loop above}}
	\only<all:6-9>{\bendtransition{A}{A}{M{\wildcard}{\wildcard}{\wildcard}}{}{loop above}}
	\bendtransition{LR}{LR}{MB{\wildcard}{\wildcard}}{}{loop above}
	\only<all:1-3>{\bendtransition{RL}{RL}{MB{\wildcard}{\wildcard}}{}{loop above}}
	\only<all:1-7>{\bendtransition{BL}{BL}{MB{\wildcard}{\wildcard}}{below=-1mm}{loop below}}
	\only<all:8-9>{\bendtransition{B}{B}{M{\wildcard}{\wildcard}{\wildcard}}{below=-1mm}{loop below}}
	\only<all:4->{\bendtransition{R}{R}{M{\wildcard}{\wildcard}{\wildcard}}{above=2mm,sloped=false}{loop right}}
	\only<all:1-7>{\bendtransition{BR}{BR}{MB{\wildcard}{\wildcard}}{below=-1mm}{loop below}}

	\only<all:10->{\bendtransition{I}{R}{D{\wildcard}R}{}{bend left}}
	\only<all:10->{\bendtransition{R}{I}{P{\wildcard}R}{}{bend left}}
	\only<all:10->{\bendtransition{I}{I}{M{\wildcard}{\wildcard}{\wildcard}}{}{loop above}}
	\only<all:10->{\bendtransition{LR}{I}{PBL}{}{bend left}}
	\only<all:10->{\bendtransition{I}{LR}{DBL}{}{bend left}}
	\only<all:10->{\bendtransition{LL}{I}{P{\wildcard}L}{}{bend right=55}}
	\only<all:10->{\bendtransition{I}{LL}{D{\wildcard}L}{below=-2mm}{bend left=80}}
  \end{tikzpicture}
}

\newcommand{\picshrinkexampleoneresult}{%
  \begin{tikzpicture}[->,>=stealth,looseness=.5,,bend angle=20]
  	\vertex{0,0}{LL}
  	\vertex{2,0}{LR}
	\vertex{4,0}{I}
	\goalvertex{7,0}{R}
	
	\draw[transition,->,>=stealth](2.6,0.6) -- (LR);
	\bendtransition{LL}{LR}{MALR}{}{bend left}
	\bendtransition{LR}{LL}{MARL}{}{bend left}
	\bendtransition{LL}{LL}{MB{\wildcard}{\wildcard}}{above=2mm,sloped=false}{loop left}
	\bendtransition{LR}{LR}{MB{\wildcard}{\wildcard}}{}{loop above}
	\bendtransition{R}{R}{M{\wildcard}{\wildcard}{\wildcard}}{above=2mm,sloped=false}{loop right}
	\bendtransition{I}{R}{D{\wildcard}R}{}{bend left}
	\bendtransition{R}{I}{P{\wildcard}R}{}{bend left}
	\bendtransition{I}{I}{M{\wildcard}{\wildcard}{\wildcard}}{}{loop above}
	\bendtransition{LR}{I}{PBL}{}{bend left}
	\bendtransition{I}{LR}{DBL}{}{bend left}
	\bendtransition{LL}{I}{P{\wildcard}L}{}{bend right=55}
	\bendtransition{I}{LL}{D{\wildcard}L}{below=-2mm}{bend left=80}
  \end{tikzpicture}
}

\newcommand{\picshrinkexampleoneresultdifferent}{%
  \begin{tikzpicture}[->,>=stealth,looseness=.5,,bend angle=20]
  	\vertex{0,-1}{LL}
  	\vertex{0,1}{LR}
	\vertex{3,0}{I}
	\goalvertex{6,0}{R}
	
	\draw[transition,->,>=stealth](0.6,1.6) -- (LR);
	\bendtransition{LL}{LR}{MALR}{}{bend left}
	\bendtransition{LR}{LL}{MARL}{}{bend left}
	\bendtransition{LL}{LL}{MB{\wildcard}{\wildcard}}{above=2mm,sloped=false}{loop left}
	\bendtransition{LR}{LR}{MB{\wildcard}{\wildcard}}{above=2mm,sloped=false}{loop left}
	\bendtransition{R}{R}{M{\wildcard}{\wildcard}{\wildcard}}{above=2mm,sloped=false}{loop right}
	\bendtransition{I}{R}{D{\wildcard}R}{}{bend left}
	\bendtransition{R}{I}{P{\wildcard}R}{}{bend left}
	\bendtransition{I}{I}{M{\wildcard}{\wildcard}{\wildcard}}{}{loop above}
	\bendtransition{LR}{I}{PBL}{}{bend left=40}
	\bendtransition{I}{LR}{DBL}{below=-2mm}{bend right}
	\bendtransition{LL}{I}{P{\wildcard}L}{}{bend right}
	\bendtransition{I}{LL}{D{\wildcard}L}{below=-2mm}{bend left=40}
  \end{tikzpicture}
}

\newcommand{\picproductfinal}[1][]{%
  \begin{tikzpicture}[->,>=stealth,looseness=.5,,bend angle=20,#1]
  	\vertex{-.5,2.5}{LRL}
  	\vertex{1,1.5}{LRR}
  	\vertex{-.5,-1.5}{LLL}
  	\vertex{1,-2.5}{LLR}
  	\vertex{3,0.5}{IL}
  	\vertex{4.5,-.5}{IR}
  	\goalvertex{6.5,0.5}{RL}
  	\goalvertex{8,-.5}{RR}

	\draw[transition,->,>=stealth](1.6,1.1) -- (LRR);

	\bendtransition{LRL}{LRR}{MBLR}{below=-2mm}{bend left}
	\bendtransition{LRR}{LRL}{MBRL}{below=-2mm}{bend left}
	\bendtransition{LLL}{LLR}{MBLR}{below=-2mm}{bend left}
	\bendtransition{LLR}{LLL}{MBRL}{below=-2mm}{bend left}
	\bendtransition{IL}{IR}{MBLR}{}{bend left}
	\bendtransition{IR}{IL}{MBRL}{}{bend left}
	\bendtransition{RL}{RR}{MBLR}{}{bend left}
	\bendtransition{RR}{RL}{MBRL}{}{bend left}

	\bendtransition{IL}{RL}{DAR}{}{bend left=40}
	\bendtransition{RL}{IL}{PAR}{below=-2mm}{bend right}
	\bendtransition{IR}{RR}{D{\wildcard}R}{}{bend right}
	\bendtransition{RR}{IR}{P{\wildcard}R}{below=-2mm}{bend left=40}
	\bendtransition{LLL}{IL}{P{\wildcard}L}{near end,below=-2mm}{bend left}
	\bendtransition{IL}{LLL}{D{\wildcard}L}{below=-2mm,near start}{bend left}
	\bendtransition{LLR}{IR}{PAL}{}{bend right}
	\bendtransition{IR}{LLR}{DAL}{below=-2mm}{bend left=40}

	\bendtransition{LLL}{LRL}{MALR}{}{bend left=50}
	\bendtransition{LRL}{LLL}{MARL}{rotate=180,below=0mm}{bend right}
	\bendtransition{LLR}{LRR}{MALR}{near end}{bend left}
	\bendtransition{LRR}{LLR}{MARL}{rotate=180,above=-2mm,near start}{bend left}

	\bendtransition{LRL}{IL}{PBL}{}{bend left=60}
	\bendtransition{IL}{LRL}{DBL}{}{bend right=30}
	\bendtransition{IL}{IL}{MA{\wildcard}{\wildcard}}{below=-1mm}{loop below}
	\bendtransition{IR}{IR}{MA{\wildcard}{\wildcard}}{below=-1mm}{loop below}
	\bendtransition{RR}{RR}{MA{\wildcard}{\wildcard}}{below=-1mm}{loop below}
	\bendtransition{RL}{RL}{MA{\wildcard}{\wildcard}}{}{loop above}
  \end{tikzpicture}
}

\newcommand{\picshrinkexampletworesult}{%
  \begin{tikzpicture}[->,>=stealth,looseness=.5,,bend angle=20]
  	\vertex{0,0}{LRR}
    \node[vertex interior] at (2,0) (LLL) {};
    \node[vertex border] at (2,0) {\parbox{4mm}{\tiny LLL\\LRL\\LLR}};
	\vertex{4,0}{I}
	\goalvertex{6,0}{R}
	
	\draw[transition,->,>=stealth](-0.8,0) -- (LRR);
	\bendtransition{I}{I}{M{\wildcard}{\wildcard}{\wildcard}}{}{loop above}
	\bendtransition{R}{R}{M{\wildcard}{\wildcard}{\wildcard}}{}{loop above}
	\bendtransition{LLL}{LLL}{M{\wildcard}{\wildcard}{\wildcard}}{}{loop above}

	\bendtransition{LRR}{LLL}{M{\wildcard}RL}{}{bend left}
	\bendtransition{LLL}{LRR}{M{\wildcard}LR}{below=-2mm}{bend left}
	\bendtransition{LLL}{I}{P{\wildcard}L}{}{bend left}
	\bendtransition{I}{LLL}{D{\wildcard}L}{below=-2mm}{bend left}
	\bendtransition{I}{R}{D{\wildcard}R}{}{bend left}
	\bendtransition{R}{I}{P{\wildcard}R}{below=-2mm}{bend left}
  \end{tikzpicture}
}

%% \author[M.~Helmert, G.~R\"oger]{Malte Helmert and Gabriele R\"oger}
%% abstraction pictures by Gabi

\begin{document}
\lectureno{11}
\subtitle{Planning as search: pattern database heuristics}
\date{December 13th, 2011}
\maketitles

\section[PDB heuristics]{Pattern databases heuristics}
\subsection[Projections]{Projections and pattern database heuristics}

\begin{frame}{Pattern database heuristics}
  \begin{itemize}
  \item The most commonly used abstraction heuristics in search and
    planning are \alert{pattern database (PDB) heuristics}.
  \item PDB heuristics were originally introduced \\
    for the \alert{15-puzzle} (Culberson \& Schaeffer, 1996) \\
    and for \alert{Rubik's cube} (Korf, 1997).
  \item The first use for \alert{domain-independent planning} \\
    is due to Edelkamp (2001).
  \item Since then, much research has focused on the theoretical
    properties of pattern databases, how to use pattern databases more
    effectively, how to find good patterns, etc.
  \item Pattern databases are a \alert{very active research area} both
    in planning and in (domain-specific) heuristic search.
  \item For many search problems, pattern databases are \\
    the \alert{most effective admissible heuristics} currently known.
  \end{itemize}
\end{frame}

\begin{frame}{Pattern database heuristics informally}
  \begin{block}{Pattern databases: informally}
    A pattern database heuristic for a planning task is an abstraction
    heuristic where
    \begin{itemize}
    \item some aspects of the task are represented in the abstraction
      \alert{with perfect precision}, while
    \item all other aspects of the task are \alert{not represented at
      all}.
    \end{itemize}
  \end{block}

  \begin{example}[15-puzzle]
    \begin{itemize}
    \item Choose a subset $T$ of tiles (the \alert{pattern}).
    \item Faithfully represent the locations of $T$ in the
      abstraction.
    \item Assume that all other tiles and the blank can be anywhere in
      the abstraction.
    \end{itemize}
  \end{example}
\end{frame}

\begin{frame}{Projections}
  Formally, pattern database heuristics are induced abstractions of a
  particular class of homomorphisms called \alert{projections}.

  \begin{definition}[projections]
    Let $\Pi$ be an FDR planning task with variable set $V$ and state
    set $S$. Let $P \subseteq V$, and let $S'$ be the set of states
    over $P$.

    \smallskip

    The \alert{projection} $\pi_P: S \to S'$ is defined as $\pi_P(s)
    := s|_P$\\
    (with $s|_P(v) := s(v)$ for all $v \in P$).

    \smallskip

    We call $P$ the \alert{pattern} of the projection $\pi_P$.
  \end{definition}
  
  In other words, $\pi_P$ maps two states $s_1$ and $s_2$ to the same
  abstract state iff they agree on all variables in $P$.
\end{frame}

\begin{frame}{Pattern database heuristics}
  Abstraction heuristics for projections are called
  \alert{pattern database} (\alert{PDB}) heuristics.

  \begin{definition}[pattern database heuristic]
    The abstraction heuristic induced by $\pi_P$ is called a \\
    \alert{pattern database heuristic} or \alert{PDB heuristic}.

    We write \alert{$h^P$} as a short-hand for $h^{\pi_P}$.
  \end{definition}

  \medskip

  Why are they called \alert{pattern database heuristics}?
  \begin{itemize}
  \item Heuristic values for PDB heuristics are traditionally stored in
    a 1-dimensional table (array) called a \alert{pattern database}
    (PDB). Hence the name ``PDB heuristic''.
  \end{itemize}
\end{frame}

\subsection{Examples}

\begin{frame}{Example: transition system}
  \begin{center}
    \picfulltransitiongraph
  \end{center}
  Logistics problem with one package, two trucks, two locations:
  \begin{itemize}
  \item state variable \textblue{package}: $\{L,R,A,B\}$
  \item state variable \textblue{truck A}: $\{L,R\}$
  \item state variable \textblue{truck B}: $\{L,R\}$
  \end{itemize}
\end{frame}

\begin{frame}{Example: projection}
  Abstraction induced by $\pi_{\{\text{\textblue{package}}\}}$:
  \begin{center}
    \picprojectionpackage
  \end{center}
  \[ h^{\{\text{\textblue{package}}\}}(\text{LRR}) = 2 \]
\end{frame}

\begin{frame}{Example: projection (2)}
  Abstraction induced by $\pi_{\{\text{\textblue{package}},
    \text{\textblue{truck A}}\}}$:
  \begin{center}
    \picprojectionpackagetruck
  \end{center}
  \[ h^{\{\text{\textblue{package}},
        \text{\textblue{truck A}}\}}(\text{LRR}) = 2 \]
\end{frame}
%% TODO: It would be pedagogically preferable if the second projection
%% were actually better than the first. (After all, this example was
%% designed to highlight the *problems* of PDB heuristics.) For
%% example, we could use the 2-package, 1-truck task instead. But then
%% we would need another picture for a 1-package/1-truck abstraction
%% of that problem.

\subsection{Overview}

\begin{frame}{Chapter overview}
  In the rest of this chapter, we will discuss:
  \begin{itemize}
  \item how to \alert{implement} PDB heuristics
  \item how to effectively make use of \alert{multiple} PDB heuristics
  \item how to \alert{find good patterns} for PDB heuristics
  \end{itemize}
\end{frame}

\section[Implementing PDBs]{Implementing pattern database heuristics}
\subsection[Precomputation]{Precomputation step}

\begin{frame}{Pattern database implementation}
  Assume we are given a pattern $P$ for a planning task $\Pi$.

  How do we implement $h^P$?

  \begin{enumerate}
  \item In a \alert{precomputation} step, we compute a graph
    representation for the abstraction $\mathcal T(\Pi)^{\pi_P}$ and
    compute the abstract goal distance for each abstract state.
  \item During search, we use the precomputed abstract goal distances
    in a \alert{lookup} step.
  \end{enumerate}
\end{frame}

\begin{frame}{Precomputation step}
  Let $\Pi$ be a planning task and $P$ a pattern.

  Let $\mathcal T = \mathcal T(\Pi)$ and $\mathcal T' = \mathcal
  T^{\pi_P}$.
  \begin{itemize}
  \item We want to compute a graph representation of $\mathcal T'$.
  \item $\mathcal T'$ is defined through a homomorphism of $\mathcal
    T$.
    \begin{itemize}
    \item For example, each concrete transition induces an abstract
      transition.
    \end{itemize}
  \item However, we cannot \alert{compute} $\mathcal T'$ by iterating
    over all transitions of $\mathcal T$.%
    \begin{itemize}
    \item This would take time $\Omega(\|\mathcal T\|)$.
    \item This is prohibitively large (or else we could solve the
      task using breadth-first search or similar techniques).
    \end{itemize}
  \item Hence, we need a way of computing $\mathcal T'$ in time
    which is \alert{polynomial only in $\|\Pi\|$ and $\|\mathcal
      T'\|$}.
  \end{itemize}
\end{frame}

\begin{frame}{Syntactic projections}
  \begin{definition}[syntactic projection]
    Let $\Pi = \langle V, I, O, \gamma\rangle$ be an FDR planning task, \\
    and let $P \subseteq V$ be a subset of its variables.

    The \alert{syntactic projection $\Pi|_P$} of $\Pi$ to $P$
    is the FDR planning task
    $\langle P, I|_P, \{o|_P \mid o \in O\}, \gamma|_P\rangle$, where
    \begin{itemize}
    \item $\varphi|_P$ for formula $\varphi$ is defined as the
      formula obtained from $\varphi$ by replacing all atoms $(v = d)$
      with $v \notin P$ by $\top$, and
    \item $o|_P$ for operator $o$ is defined by replacing all formulas
      $\varphi$ occurring in the precondition or effect conditions of
      $o$ with $\varphi|_P$ and all atomic effects $(v := d)$ with $v
      \notin P$ with the empty effect $\top$.
    \end{itemize}
  \end{definition}
  Put simply, $\Pi|_P$ throws away all information not pertaining to
  variables in $P$.
\end{frame}

\begin{frame}{Trivially inapplicable operators}
  \begin{definition}[trivially inapplicable operator]
    An operator $\langle \chi, e\rangle$ of a \sasplus\ task is
    called \alert{trivially inapplicable} if
    \begin{itemize}
    \item $\chi$ contains the atoms $(v = d)$ and $(v = d')$ \\
      for some variable $v$ and values $d \neq d'$, or
    \item $e$ contains the effects $(v := d)$ and $(v := d')$ \\
      for some variable $v$ and values $d \neq d'$.
    \end{itemize}
  \end{definition}
  \hilite{Notes:}
  \begin{itemize}
  \item Trivially inapplicable operators are never applicable and \\
    can thus be safely omitted from the task.
  \item Trivially inapplicable operators can be detected in linear time.
  \end{itemize}
\end{frame}

\begin{frame}{Trivially unsolvable {\sasplus} tasks}
  \begin{definition}[trivially unsolvable {\sasplus} tasks]
    A \sasplus\ task $\Pi = \langle V, I, O, \gamma\rangle$ 
    is called \alert{trivially unsolvable} if
    $\gamma$ contains the atoms $(v = d)$ and $(v = d')$
    for some variable $v$ and values $d \neq d'$.
  \end{definition}
  \hilite{Notes:}
  \begin{itemize}
  \item Trivially unsolvable {\sasplus} tasks have no goal states,
    and are hence unsolvable.
  \item Trivially unsolvable {\sasplus} tasks can be detected in
    linear time.
  \end{itemize}
\end{frame}

\begin{frame}{Equivalence theorem for syntactic projections}
  \begin{theorem}[syntactic projections vs.\ projections]
    Let $\Pi$ be a \sasplus\ task that is not trivially unsolvable
    and has no trivially inapplicable operators,
    and let $P$ be a pattern for $\Pi$.

    Then $\mathcal T(\Pi|_P) \graphequiv \mathcal T(\Pi)^{\pi_P}$.
  \end{theorem}
  \begin{proof}
    $\leadsto$ exercises
  \end{proof}
\end{frame}

\begin{frame}{PDB computation}
  Using the equivalence theorem, we can compute pattern databases
  for (not trivially unsolvable) {\sasplus} tasks $\Pi$ and patterns $P$:
  
  \begin{block}{Computing pattern databases}
    \textbf{def} compute-PDB($\Pi$, $P$): \\
    {}\qquad Remove trivially inapplicable operators from $\Pi$. \\
    {}\qquad Compute $\Pi' := \Pi|_P$. \\
    {}\qquad Compute $\mathcal T' := \mathcal T(\Pi')$. \\
    {}\qquad Perform a backward breadth-first search from the goal \\
    {}\qquad\qquad states of $\mathcal T'$ to compute all abstract
    goal distances. \\
    {}\qquad $\textit{PDB} := \text{a table containing all
    goal distances in~} \mathcal T'$ \\
    {}\qquad\textbf{return} \textit{PDB}
  \end{block}
  
  The algorithm runs \alert{in polynomial time and space} in terms of
  $\|\Pi\| + |\textit{PDB}|$.
\end{frame}

\begin{frame}{Generalizations of the equivalence theorem}
  \begin{itemize}
  \item The restrictions to {\sasplus} tasks and to tasks without
    trivially inapplicable operators are necessary.
  \item We can slightly generalize the result if we allow general
    negation-free formulas, but still forbid conditional effects.
    \begin{itemize}
    \item In that case, the unlabeled graph of $\mathcal
      T(\Pi)^{\pi_P}$ is isomorphic to a subgraph of the unlabeled
      graph of $\mathcal T(\Pi|_P)$.
    \item This means that we can use $\mathcal T(\Pi|_P)$ to derive
      an admissible estimate of $h^P$.
    \end{itemize}
  \item With conditional effects, not even this weaker result holds.
  \end{itemize}
\end{frame}

\begin{frame}{Going beyond {\sasplus} tasks}
  \begin{itemize}
  \item Most practical implementations of PDB heuristics are limited
    to {\sasplus} tasks (or modest generalizations).
  \item One way to avoid the issues with general FDR tasks is to
    convert them to equivalent {\sasplus} tasks.
  \item However, most direct conversions can exponentially increase
    the task size in the worst case.
  \end{itemize}
  $\leadsto$ We will only consider {\sasplus} tasks in this chapter.
\end{frame}

\subsection[Lookup]{Lookup step}

\begin{frame}{Lookup step: overview}
  \begin{itemize}
  \item During search, the PDB is the only piece of information
    necessary to represent $h^P$. (It is not necessary to store the
    abstract transition system itself at this point.)
  \item Hence, the space requirements for PDBs during search are
    linear in the number of abstract states $S'$: there is one table
    entry for each abstract state.
  \item During search, $h^P(s)$ is computed by mapping
    $\pi_P(s)$ to a natural number in the range
    $\{0, \dots, |S'| - 1\}$
    using a \alert{perfect hash function}, then looking up the table
    entry for that number.
  \end{itemize}
\end{frame}

\begin{frame}{Lookup step: algorithm}
  Let $P = \{v_1, \dots, v_k\}$ be the pattern.
  \begin{itemize}
  \item We assume that all variable domains are natural numbers
    counted from $0$, \ie, $\mathcal D_v = \{0, 1, \dots, |\mathcal
      D_v| - 1\}$.
  \item For all $i \in \{1, \dots, k\}$, we precompute
    $N_i := \prod_{j=1}^{i-1} |\mathcal D(v_j)|$.
  \end{itemize}

  \smallskip

  Then we can look up heuristic values as follows:
  \begin{block}{Computing pattern database heuristics}
    \textbf{def} PDB-heuristic($s$): \\
    {}\qquad \textit{index} := $\sum_{i=1}^k N_i s(v_i)$ \\
    {}\qquad \textbf{return} \textit{PDB}[\textit{index}]
  \end{block}

  \begin{itemize}
  \item This is a \alert{very fast} operation: it can be performed in
    $O(k)$.
  \item For comparison, most relaxation heuristics need time
    $O(\|\Pi\|)$ per state.
  \end{itemize}
\end{frame}

\begin{frame}<all:2>{Lookup step: example}
  Abstraction induced by $\pi_{\{\text{\textblue{package}},
    \text{\textblue{truck A}}\}}$:
  \begin{center}
    \picprojectionpackagetruck
  \end{center}
\end{frame}

\begin{frame}{Lookup step: example (ctd.)}
  \begin{itemize}
  \item $P = \{v_1, v_2\}$ with $v_1 = \text{\textblue{package}}$,
    $v_2 = \text{\textblue{truck A}}$.
  \item $\mathcal D_{v_1} = \{\val{L}, \val{R}, \val{A}, \val{B}\}
    \approx \{0, 1, 2, 3\}$
  \item $\mathcal D_{v_2} = \{\val{L}, \val{R}\}
    \approx \{0, 1\}$
  \end{itemize}
  
  \smallskip

  \begin{itemize}
  \item[$\leadsto$] $N_1 = \prod_{j=1}^0 |\mathcal D_{v_j}| = 1$,
    $N_2 = \prod_{j=1}^1 |\mathcal D_{v_j}| = 4$
  \item[$\leadsto$] $\textit{index}(s) =
    1 \cdot s(\text{\textblue{package}})
    + 4 \cdot s(\text{\textblue{truck A}})$
  \end{itemize}

  Pattern database:
  \smallskip

  \begin{tabular}{@{}c|rrrrrrrr@{}}
    abstract state & LL & RL & AL & BL & LR & RR & AR & BR \\
    index & 0 & 1 & 2 & 3 & 4 & 5 & 6 & 7 \\
    value & 2 & 0 & 2 & 1 & 2 & 0 & 1 & 1
  \end{tabular}
\end{frame}

\section[Additivity]{Additive patterns for planning tasks}
\subsection[Canonical heuristic function]{The additivity criterion \&
  the canonical heuristic function}

\begin{frame}{Pattern collections}
  \begin{itemize}
  \item The space requirements for a pattern database grow
    \alert{exponentially} with the \alert{number of state variables}
    in the pattern.
  \item This places severe limits on the usefulness of single PDB
    heuristics $h^P$ for larger planning task.
  \item To overcome this limitation, planners using pattern databases
    work with \alert{collections of multiple patterns}.
  \item When using two patterns $P_1$ and $P_2$, it is always possible
    to use the \alert{maximum} of $h^{P_1}$ and $h^{P_2}$ as an
    admissible and consistent heuristic estimate.
  \item However, when possible, it is much preferable to use
    the \alert{sum} of $h^{P_1}$ and $h^{P_2}$ as a heuristic
    estimate, since $h^{P_1} + h^{P_2} \ge \max \{h^{P_1}, h^{P_2}\}$.
  \end{itemize}
\end{frame}

\begin{frame}{Criterion for additive patterns}
  \begin{theorem}[additive pattern sets]
    Let $P_1, \dots, P_k$ be patterns for an FDR planning task $\Pi$.

    \smallskip

    If there exists no operator that has an effect on a variable $v_i
    \in P_i$ and on a variable $v_j \in P_j$ for some $i \neq j$, then
    $\sum_{i=1}^k h^{P_i}$ is an admissible and consistent heuristic
    for $\Pi$.
  \end{theorem}
  \pause
  \begin{proof}
    If there exists no such operator, then no label of $\mathcal
    T(\Pi)$ affects both $\mathcal T(\Pi)^{\pi_{P_i}}$ and $\mathcal
    T(\Pi)^{\pi_{P_j}}$ for $i \neq j$. By the theorem on affecting
    transition labels, this means that any two projections $\pi_{P_i}$
    and $\pi_{P_j}$ are orthogonal. The claim follows with the theorem
    on additivity for orthogonal abstraction mappings.
  \end{proof}

  A pattern set $\{P_1, \dots, P_k\}$ which satisfies the criterion of
  the theorem is called an \alert{additive pattern set} or
  \alert{additive set}.
\end{frame}

\begin{frame}{Finding additive pattern sets}
  The theorem on additive pattern sets gives us a simple criterion to
  decide which pattern heuristics can be admissibly added.

  \smallskip

  Given a \alert{pattern collection $\mathcal C$} (\ie, a set of
  patterns), \\ we can use this information as follows:
  \begin{enumerate}
  \item Build the \alert{compatibility graph} for $\mathcal C$.
    \begin{itemize}
    \item Vertices correspond to patterns $P \in \mathcal C$.
    \item There is an edge between two vertices iff no operator
      affects both incident patterns.
    \end{itemize}
  \item Compute \alert{all maximal cliques} of the graph. \\
    These correspond to maximal additive subsets of $\mathcal C$.
    \begin{itemize}
    \item Computing large cliques is an NP-hard problem, and a
      graph can have exponentially many maximal cliques.
    \item However, there are \alert{output-polynomial} algorithms for
      finding all maximal cliques (Tomita, Tanaka \& Takahashi, 2004)
      which have led to good results in practice.
    \end{itemize}
  \end{enumerate}
\end{frame}

\begin{frame}{The canonical heuristic function}
  \begin{definition}[canonical heuristic function]
    Let $\Pi$ be an FDR planning task, and
    let $\mathcal C$ be a pattern collection for $\Pi$.

    \smallskip

    The \alert{canonical heuristic $h^{\mathcal C}$}
    for pattern collection $\mathcal C$ is defined as
    \[
    h^{\mathcal C}(s) =
    \max_{\mathcal D \in \cliques(\mathcal C)}
    \sum_{P \in \mathcal D} h^P(s),\]
    where $\cliques(\mathcal C)$ is the set of all maximal cliques \\
    in the compatibility graph for $\mathcal C$.
  \end{definition}
  
  For all choices of $\mathcal C$, heuristic $h^{\mathcal C}$ is
  admissible and consistent.
\end{frame}

\begin{frame}{How good is the canonical heuristic function?}
  \begin{itemize}
  \item The canonical heuristic function is the \alert{best possible}
    admissible heuristic we can derive from $\mathcal C$ using
    \alert{our additivity criterion}.
  \item In theory, even better heuristic estimates can be obtained from
    projection heuristics using a \alert{more general additivity
      criterion} based on an idea called \alert{cost partitioning}.
    \begin{itemize}
    \item Optimal polynomial cost partitioning algorithms exist
      (Katz \& Domshlak, 2008a).
    \item However, polynomial is often not fast enough here.
    \item Coming up with good practical ways of cost partitioning for
      PDB heuristics remains an open research issue.
    \end{itemize}
  \end{itemize}
\end{frame}

\begin{frame}{Canonical heuristic function: example}
  \begin{example}
    Consider a planning task with state variables $V = \{v_1, v_2,
    v_3\}$ \\ and the pattern collection $\mathcal C = \{P_1, \dots,
    P_4\}$ with $P_1 = \{v_1, v_2\}$, $P_2 = \{v_1\}$, $P_3 =
    \{v_2\}$ and $P_4 = \{v_3\}$.

    \smallskip

    There are operators affecting each individual variable, and the
    only operators affecting several variables affect $v_1$ and $v_3$.

    \smallskip

    What are the maximal cliques in the compatibility graph for
    $\mathcal C$?

    \pause
    \medskip

    \hilite{Answer:} $\{P_1\}$, $\{P_2, P_3\}$, $\{P_3, P_4\}$

    \pause
    \medskip

    What is the canonical heuristic function $h^{\mathcal C}$?

    \pause
    \medskip

    \hilite{Answer:}
    $\begin{array}[t]{r@{}l}
      h^{\mathcal C} & {} = \max\,\{
        h^{P_1},
        h^{P_2} + h^{P_3},
        h^{P_3} + h^{P_4}
      \} \\
      & {} = \max\,\{
        h^{\{v_1, v_2\}},
        h^{\{v_1\}} + h^{\{v_2\}},
        h^{\{v_2\}} + h^{\{v_3\}}
        \}
    \end{array}$
  \end{example}
\end{frame}

\subsection[Simplification]{Algebraic simplification \& dominance pruning}

\begin{frame}{Computing $h^{\mathcal C}$ efficiently: motivation}
  Consider $h^{\mathcal C} = \max\,\{
    h^{\{v_1, v_2\}},
    h^{\{v_1\}} + h^{\{v_2\}},
    h^{\{v_2\}} + h^{\{v_3\}}
    \}$.
  \begin{itemize}
  \item We need to evaluate this expression for \alert{every search
    node}.
  \item It is thus worth to spend some effort in precomputations to
    make the evaluation \alert{more efficient}.
  \end{itemize}

  \medskip

  A naive implementation requires \alert{8 atomic operations}:
  \begin{itemize}
  \item \alert{4} heuristic lookups (for
    $h^{\{v_1, v_2\}}$, 
    $h^{\{v_1\}}$, 
    $h^{\{v_2\}}$ and
    $h^{\{v_3\}}$),
  \item \alert{2} binary summations and
  \item \alert{2} binary maximizations
  \end{itemize}

  \medskip

  \alert{Can we do better than that?}
\end{frame}

\begin{frame}{Algebraic simplifications}
  One possible simplification is to use \alert{algebraic
    identities} \\
  to reduce the number of operations:
  \begin{align*}
    &
    \max\,\{
      h^{\{v_1, v_2\}},
      h^{\{v_1\}} + h^{\{v_2\}},
      h^{\{v_2\}} + h^{\{v_3\}}
      \} \\
    {} = {} &
    \max\,\{
      h^{\{v_1, v_2\}},
      h^{\{v_2\}} + \max\,\{h^{\{v_1\}}, h^{\{v_3\}}\}
      \}
  \end{align*}
  $\leadsto$ reduces number of operations from \alert{8} to \alert{7}

  \medskip

  \alert{Is there anything else we can do?}
\end{frame}

\begin{frame}{Dominated sum theorem}
  \begin{theorem}[dominated sum]
    Let $\{P_1, \dots, P_k\}$ be an additive pattern set for an FDR
    planning task, and let $P$ be a pattern with $P_i \subseteq P$ for
    all $i \in \{1, \dots, k\}$.
    
    \smallskip

    Then $\sum_{i=1}^k h^{P_i} \le h^P$.
  \end{theorem}

  \pause

  \begin{proofstart}
    Let $\mathcal T'$ be the transition system induced by $\pi_P$,
    and for all $i \in \{1, \dots, k\}$, let $\mathcal T'_i$ be
    the transition system induced by $\pi_{P_i}$.

    Because $P_i \subseteq P$, we can write each projection
    $\pi_{P_i}$ as a projection onto $P$ followed by a projection onto
    $P_i$: $\pi_{P_i} = \pi'_{P_i} \circ \pi_P$. Hence, each $\mathcal
    T'_i$ is a coarsening of $\mathcal T'$, and we can consider the
    $h^{P_i}$ as \alert{abstraction heuristics on $\mathcal T'$},
    where they are also additive.
    %% TODO: This last part "where they are also additive" could do
    %% with a proof, and the "can consider as heuristics on T'" could
    %% do with a changed notation to emphasize the fact that the
    %% domain of the function is different. (For example, use
    %% \tilde{h} instead of h.)
  \end{proofstart}
\end{frame}

\begin{frame}{Dominated sum theorem (ctd.)}
  \begin{proofend}
    We get:
    \begin{align*}
      \sum_{i=1}^k h^{P_i}(s)
      & {} =
      \sum_{i=1}^k h_{\mathcal T'_i}^*(\pi_{P_i}(s))
      = \sum_{i=1}^k h_{\mathcal T'_i}^*(\pi'_{P_i}(\pi_P(s)))
      \\ & {} \stackrel{(1)}{=}
      \sum_{i=1}^k h_{\mathcal T'_i}^*(\pi'_{P_i}(s'))
      \stackrel{(2)}{=} \sum_{i=1}^k h^{P_i}(s')
      \\ & {} \stackrel{(3)}{\le}
      h_{\mathcal T'}^*(s')
      = h_{\mathcal T'}^*(\pi_P(s))
      = h^P(s)
    \end{align*}
    where (1) holds for $s' := \pi_P(s)$, (2) holds because we can
    consider the $h^{P_i}$ as abstraction heuristics on $\mathcal T'$,
    and (3) holds because of the additivity criterion.
  \end{proofend}
\end{frame}

\begin{frame}{Dominated sum corollary}
  \begin{corollary}[dominated sum]
    Let $\{P_1, \dots, P_n\}$ and $\{Q_1, \dots, Q_m\}$ be additive
    pattern sets of an FDR planning task such that each pattern $P_i$
    is a subset of some pattern $Q_j$ (not necessarily proper).

    \smallskip

    Then $\sum_{i=1}^n h^{P_i} \le \sum_{j=1}^m h^{Q_j}$.
  \end{corollary}
  \pause
  \begin{proof}
    \[ \sum_{i=1}^n h^{P_i}
      \stackrel{(1)}{\le} \sum_{j=1}^m \sum_{P_i \subseteq Q_j} h^{P_i}
      \stackrel{(2)}{\le} \sum_{j=1}^m h^{Q_j},
      \]
    where (1) holds because each $P_i$ is contained in some $Q_j$ \\
    and (2) follows from the dominated sum theorem.
  \end{proof}
\end{frame}

\begin{frame}{Dominance pruning}
  \begin{itemize}
  \item We can use the dominated sum corollary to simplify
    the representation of $h^{\mathcal C}$:
    sums that are dominated by other sums can be pruned.
  \item The dominance test can be performed in polynomial time.
  \end{itemize}

  \begin{example}
    \begin{align*}
      &
      \max\,\{
      h^{\{v_1, v_2\}},
      h^{\{v_1\}} + h^{\{v_2\}},
      h^{\{v_2\}} + h^{\{v_3\}}
      \} \\
      {} = {} &
      \max\,\{
      h^{\{v_1, v_2\}},
      h^{\{v_2\}} + h^{\{v_3\}}
      \}
    \end{align*}
    $\leadsto$ reduces number of operations from \alert{8} to
    \alert{5}
  \end{example}
\end{frame}

\section{Pattern selection}
\subsection[Local search]{Pattern selection as local search}

\begin{frame}{Pattern selection as an optimization problem}
  Only one question remains to be answered now \\
  in order to apply PDBs to planning tasks in practice:

  \smallskip

  \alert{How do we automatically find a good pattern collection?}

  \medskip

  \begin{block}{The idea}
    Pattern selection can be cast as an \alert{optimization problem}:
    \begin{itemize}
    \item \hilite{Given:} a set of \alert{candidate solutions} \\
      (= pattern collections which fit into a given memory limit)
    \item \hilite{Find:} a \alert{best possible} solution, or an
      approximation \\
      (= pattern collection with high heuristic quality)
    \end{itemize}
  \end{block}
\end{frame}

\begin{frame}{Pattern selection as local search}
  How to solve this optimization problem?
  \begin{itemize}
  \item For problems of interesting size, we cannot hope to
    find (and prove) a \alert{globally optimal} pattern collection.
    \begin{itemize}
    \item \hilite{Question:} How many candidates are there?
    \end{itemize}
  \item Instead, we try to find \alert{good} solutions by \alert{local
    search}.
  \end{itemize}

  \bigskip

  Two approaches from the literature:
  \begin{itemize}
  \item Edelkamp (2007): using \alert{evolutionary algorithm}
  \item Haslum et al.\ (2007): using \alert{hill-climbing}
  \end{itemize}
  $\leadsto$ we present the main ideas of the second approach here
\end{frame}

\begin{frame}{Pattern selection as hill-climbing}
  \begin{block}{Reminder: Hill-climbing}
    $\sigma := \text{make-root-node}(\alert{\text{init}}())$ \\
    \textbf{forever}: \\
    {}\qquad \textbf{if}
    $\alert{\text{is-goal}}(\text{state}(\sigma))$: \\
    {}\qquad\qquad \textbf{return} $\text{extract-solution}(\sigma)$
    \\
    {}\qquad $\Sigma' := \{\,\text{make-node}(\sigma, o, s) \mid
      \langle o, s\rangle \in
      \alert{\text{succ}}(\text{state}(\sigma))\,\}$ \\
    {}\qquad $\sigma := \text{an element of}~\Sigma'
    \text{~minimizing~} \alert{h}$ (random tie breaking)
  \end{block}

  Four questions to answer to use this for pattern selection:
  \begin{enumerate}
  \item \alert{init:} What is the initial pattern collection?
  \item \alert{is-goal:} When do we terminate?
  \item \alert{succ:} Which collections
    are neighbours of the current collection?
  \item \alert{$h$:} How do we rank the quality of pattern collections?
  \end{enumerate}
\end{frame}

\subsection{Search space}

\begin{frame}{Search space}
  We first discuss the \alert{search space} (init, is-goal, succ).

  \medskip

  The basic idea is that we
  \begin{itemize}
  \item start from \alert{small patterns} of only a single
    variable each,
  \item grow them by \alert{adding slightly larger patterns}, and
  \item stop when \alert{heuristic quality no longer improves}.
  \end{itemize}

  \medskip
  
  To motivate the precise definition of our search space, we need a
  little more theory.
\end{frame}

\begin{frame}{Initial pattern collection}
  \begin{theorem}[non-goal patterns are trivial]
    Let $\Pi$ be a {\sasplus} planning task that is not trivially
    unsolvable, and let $P$ be a pattern for $\Pi$ such that no
    variable in $P$ is mentioned in the goal formula of $\Pi$.

    Then $h^P(s) = 0$ for all states $s$.
  \end{theorem}
  %% NOTE: Not true for general FDR tasks because these can have
  %% unsatisfiable goal formulas, in which case *all* semantic
  %% projections have h value infinity (no goal states => no abstract
  %% goal states).

  \begin{proof}
    All states in the abstraction are goal states.
  \end{proof}

  \smallskip
  
  This motivates our first answer:
  \begin{block}{1. What is the initial pattern collection?}
    The initial pattern collection is
    $\{\{v\}~|~v \text{~is a state variable mentioned in the goal formula}\}$.
  \end{block}
\end{frame}

\begin{frame}{Termination criterion}
  Our second question has a very simple answer:
  \begin{block}{2. When do we terminate?}
    We terminate as soon as the current pattern collection has no
    successors of better quality.
  \end{block}

  Note that this also covers the case where there are no successors
  at all because further growth of the current pattern collection
  would exceed a memory limit.
\end{frame}

\begin{frame}{Search neighborhood: idea}
  Our search neighbourhood is defined through \alert{incremental
    growth} of the current pattern collection.

  \smallskip

  A successor is obtained by
  \begin{itemize}
  \item starting from the \alert{current pattern collection} $\mathcal
    C$,
  \item choosing \alert{one of its patterns} $P \in \mathcal C$
    (without removing it from $\mathcal C$!),
  \item generating a new pattern by extending $P$ with a single
    variable ($P' = P \cup \{v\}$), and
  \item adding $P'$ to $\mathcal C$ to form the new pattern
    collection $\mathcal C'$
  \end{itemize}

  However, not all such collections $\mathcal C'$ are useful.
\end{frame}

\begin{frame}{Causal graphs}
  \begin{definition}[causal graph]
    Let $\Pi = \langle V, I, O, \gamma\rangle$ be an FDR planning task.

    The \alert{causal graph} of $\Pi$, \alert{$\cg(\Pi)$}, is the
    directed graph with vertex set $V$ and an arc from $u \in V$ to
    $v \in V$ iff $u \neq v$ and there exists an operator $o \in O$
    such that:
    \begin{itemize}
    \item $u$ appears anywhere in $o$ (in precondition, effect
      conditions or atomic effects), and
    \item $v$ is modified by an effect of $o$.
    \end{itemize}
  \end{definition}
  \hilite{Idea:} an arc $\langle u, v\rangle$ in the causal graph
  indicates that variable $u$ is in some way relevant for modifying
  the value of $v$
\end{frame}

\begin{frame}{Causally relevant variables}
  \begin{definition}[causally relevant variables]
    Let $\Pi = \langle V, I, O, \gamma\rangle$ be an FDR planning task
    and let $P \subseteq V$ be a pattern for $\Pi$.

    We say that $v \in P$ is \alert{causally relevant for $P$} if
    $\cg(\Pi)$ contains a directed path from $v$ to a variable $v' \in
    P$ that is mentioned in the goal formula $\gamma$.
  \end{definition}
  \hilite{Note:} The definition implies that variables in $P$
  mentioned in the goal are always causally relevant for $P$.
\end{frame}
%% NOTE from Patrik: It would probably make more sense to require that
%% the path stays within P, i.e., exists in the subgraph of the causal
%% graph induced by P. We kind of get the right criterion in the end
%% anyway because we also require causal connectedness, but that's a
%% bit too accidental for my tastes.

\begin{frame}{Causally irrelevant variables are useless}
  \begin{theorem}[causally irrelevant variables are useless]
    Let $P \subseteq V$ be a pattern for an FDR planning task $\Pi$,
    and let $P' \subseteq P$ consist of all variables that are
    causally relevant for $P$.

    Then $h^{P'}(s) = h^P(s)$ for all states $s$.
  \end{theorem}

  \pause

  \begin{proofstart}
    \hilite{$(\le)$:} follows from the dominated sum theorem with $P'
    \subseteq P$

  \pause
    \medskip

    \hilite{$(\ge)$:} Obvious if $h^{P'}(s) = \infty$; else, induction
    over $n = h^{P'}(s)$.

    \begin{itemize}
    \item \hilite{Base case $n = 0$:}

      If $h^{P'}(s) = 0$, then there exists a concrete goal state
      $\tilde{s}$ that agrees with $s$ on all variables in $P'$. If we
      change $\tilde{s}$ so that it agrees with $s$ on all variables in
      $P$, it is still a goal state because we only change variables
      that are not mentioned in the goal.
    \end{itemize}
  \end{proofstart}
\end{frame}

\begin{frame}{Causally irrelevant variables are useless (ctd.)}
  \begin{proofmid}
    \begin{itemize}
    \item \hilite{Inductive case $n \rightarrow n + 1$:}

      If $h^{P'}(s) = n+1$, then there exist concrete states $\tilde{s}$
      and $\tilde{t}$ and an operator $o$ of $\Pi$ such that:
      \begin{itemize}
      \item $s$ and $\tilde{s}$ agree on all variables in $P'$,
      \item $\applyop{o}{\tilde{s}} = \tilde{t}$, and
      \item $h^{P'}(\tilde{t}) = n$.
      \end{itemize}

      \pause

      If we change $\tilde{s}$ and $\tilde{t}$ so that they agree
      with $s$ on all variables of $P$, then still
      $\applyop{o}{\tilde{s}} = \tilde{t}$, because
      \begin{itemize}
      \item $o$ modifies a variable in $P'$ (otherwise
        $\pi_{P'}(\tilde{s}) = \pi_{P'}(\tilde{t})$ contradicting
        $h^{P'}(\tilde{s}) = n + 1 \neq n = h^{P'}(\tilde{t})$), and
        hence
      \item all variables mentioned in $o$ are causally relevant for
        $P'$, which implies that
      \item $o$ mentions no variable in $P \setminus P'$.
      \end{itemize}
      \dots
    \end{itemize}
  \end{proofmid}
\end{frame}

\begin{frame}{Causally irrelevant variables are useless (ctd.)}
  \begin{proofend}
    \begin{itemize}
    \item \dots

      We get:
      \begin{itemize}
      \item $s$ and $\tilde{s}$ agree on all variables in $P$,
      \item $\applyop{o}{\tilde{s}} = \tilde{t}$, and
      \item $h^{P'}(\tilde{t}) = n = h^P(\tilde{t})$ (by the induction
        hypothesis).
      \end{itemize}
      This implies $h^P(s) \le n + 1$, concluding the proof.
    \end{itemize}
  \end{proofend}

  \hilite{Corollary:} There is no point in growing a pattern by adding
  a variable that is causally irrelevant in the resulting pattern.
\end{frame}

\begin{frame}{Causally connected patterns}
  \begin{definition}[causally connected patterns]
    Let $\Pi = \langle V, I, O, \gamma\rangle$ be an FDR planning task
    and let $P \subseteq V$ be a pattern for $\Pi$.

    We say that $P$ is \alert{causally connected} if the subgraph of
    $\cg(\Pi)$ induced by $P$ is weakly connected (\ie, contains a
    path from every vertex to every other vertex, ignoring arc
    directions).
  \end{definition}
\end{frame}

\begin{frame}{Disconnected patterns are decomposable}
  \begin{theorem}[causally disconnected patterns are decomposable]
    Let $P \subseteq V$ be a pattern for a {\sasplus} planning task
    $\Pi$ that is not causally connected, and let $P_1$, $P_2$ be a
    partition of $P$ into non-empty subsets such that $\cg(\Pi)$
    contains no arc between the two sets.

    Then $h^{P_1}(s) + h^{P_2}(s) = h^P(s)$ for all states $s$.
  \end{theorem}
  %% NOTE: Not a theorem for FDR tasks with disjunctive goals.
  %% Consider \gamma = (a \lor b), where a and b are atoms from different
  %% subpatterns. Then each subpattern has h value zero because the
  %% goal can be satisfied by achieving the disjunctive subgoal from
  %% the other subpattern. However, the combined pattern can of course
  %% have non-zero heuristic estimates. The cause of the breakdown is
  %% that our causal graph definition -- or our notion of causal
  %% connectedness -- is in some ways inappropriate for disjunctive
  %% goals.

  \pause

  \begin{proofstart}
    \hilite{$(\le)$:} There is no arc between $P_1$ and $P_2$ in the
    causal graph, and thus there is no operator that affects both
    patterns. Therefore, they are additive, and $h^{P_1} + h^{P_2} \le
    h^P$ follows from the dominated sum theorem.
  \end{proofstart}
\end{frame}

\begin{frame}{Disconnected patterns are decomposable (ctd.)}
  \begin{proofmid}
    \hilite{$(\ge)$:} If $h^{P_1}(s) + h^{P_2}(s) = \infty$, we are
    done.

    Otherwise, proof by induction over $n = h^{P_1}(s) + h^{P_2}(s)$.
    \begin{itemize}
    \item \hilite{Base case $n = 0$:}

      If $h^{P_1}(s) + h^{P_2}(s) = 0$,
      then $h^{P_1}(s) = 0$ and hence $s$ satisfies all goal atoms for
      variables in $P_1$; similarly for $P_2$. Hence, it satisfies all
      goal atoms for variables in $P = P_1 \cup P_2$, and thus
      $h^P(s) = 0$.
    \end{itemize}
  \end{proofmid}
\end{frame}

\begin{frame}{Disconnected patterns are decomposable (ctd.)}
  \begin{proofmid}
    \begin{itemize}
    \item \hilite{Inductive case $n \rightarrow n + 1$:}

      If $h^{P_1}(s) + h^{P_2}(s) = n+1$, then there exist concrete
      states $\tilde{s}$ and $\tilde{t}$, an operator $o$ of $\Pi$ and
      $i \in \{1, 2\}$ such that:
      \begin{enumerate}
      \item $s$ and $\tilde{s}$ agree on all variables in $P_i$,
      \item $\applyop{o}{\tilde{s}} = \tilde{t}$, and
      \item $h^{P_i}(\tilde{t}) = h^{P_i}(\tilde{s}) - 1$.
      \end{enumerate}

      \pause

      Let $j \in \{1, 2\}$ with $j \neq i$. Since $P_1$ and $P_2$ are
      causally disconnected, the operator $o$ does not mention any
      variables in $P_j$. Therefore, we can change $\tilde{s}$ and
      $\tilde{t}$ so that they agree with $s$ on all variables of
      $P$ and still have (2) and (3).

      \dots
    \end{itemize}
  \end{proofmid}
\end{frame}

\begin{frame}{Disconnected patterns are decomposable (ctd.)}
  \begin{proofend}
    \begin{itemize}
    \item \dots

      We get:
      \begin{itemize}
      \item $h^{P_j}(\tilde{t}) = h^{P_j}(\tilde{s})$ (because
        $o$ does not affect variables in $P_j$)
      \item[$\leadsto$] $h^{P_1}(\tilde{t}) + h^{P_2}(\tilde{t}) = n$
      \item[$\leadsto$] $h^P(\tilde{t}) = n$ (by the induction
        hypothesis)
      \item[$\leadsto$] $h^P(\tilde{s}) \le n + 1$ (because
        $\applyop{o}{\tilde{s}} = \tilde{t}$)
      \item[$\leadsto$] $h^P(s) \le n + 1$ (because
        $s$ and $\tilde{s}$ agree on $P$ and hence $h^P(s) =
        h^P(\tilde{s})$).
      \end{itemize}
      This concludes the proof.
    \end{itemize}
  \end{proofend}

  \hilite{Corollary:} There is no point in including a causally
  disconnected pattern in the collection. (Using its connected
  components instead requires less space and gives identical results.)
\end{frame}

\begin{frame}{Search neighbourhood}
  We can now put the pieces together to define our search
  neighbourhood, obtaining the third answer:
  
  \begin{block}{3. Which collections are neighbours of the
      current collection?}
    
    The neighbours of $\mathcal C$ are all pattern collections
    $\mathcal C \cup \{P'\}$ where
    \begin{itemize}
    \item $P' = P \cup \{v\}$ for some $P \in \mathcal C$,
    \item $P' \notin \mathcal C$,
    \item all variables of $P'$ are causally relevant in $P'$,
    \item $P'$ is causally connected, and
    \item all pattern databases in $\mathcal C \cup \{P'\}$ can be
      represented within some prespecified space limit
    \end{itemize}
  \end{block}
\end{frame}

\begin{frame}{Search neighborhood (ctd.)}
  \hilite{Remark:} For causal relevance and connectivity, there is
  a sufficient and necessary criterion which is easy to check:
  \begin{itemize}
  \item $v$ is a predecessor of some $u \in P$ in the causal graph,
    \alert{or}
  \item $v$ is a successor of some $u \in P$ in the causal graph and
    is mentioned in the goal formula.
  \end{itemize}
  %% NOTE: The second possibility appears to be missing in the AAAI
  %% 2007 PDB paper. (That is, the "optimization" there appears to be
  %% too strong.)
\end{frame}

\subsection{Estimating heuristic quality}

\begin{frame}{What is a good pattern collection?}
  \begin{itemize}
  \item The last question we need to answer is \alert{how to rank} \\
    the quality of pattern collections.
  \item This is perhaps the most critical point: without a good
    ranking criterion, pattern collections are chosen blindly.
  \end{itemize}
  \medskip

  The first search-based approach to pattern selection (Edelkamp,
  2007) used the following strategy:
  \begin{itemize}
  \item only additive sets are used as pattern collections
    \begin{itemize}
    \item no need for something like the canonical heuristic function
    \end{itemize}
  \item the quality of a \alert{single pattern} is estimated by \\
    its \alert{mean heuristic value} (the higher, the better)
  \item the quality of a \alert{pattern collection} is estimated by \\
    the \alert{sum} of the individual pattern qualities
  \end{itemize}
\end{frame}

\begin{frame}{Discussion of the mean value approach}
  Pros of the approach:
  \begin{itemize}
  \item mean heuristic values are clearly correlated with search
    performance $\leadsto$ the \alert{quality measure makes sense}
  \item mean heuristic values are quite \alert{easy to calculate}
  \end{itemize}

  \medskip

  Cons of the approach:
  \begin{itemize}
  \item cannot reasonably deal with \alert{infinite} heuristic
    estimates
  \item difficult to generalize to pattern collections that are \\
    \alert{not fully additive}
  \item there are \alert{better predictors} for search performance \\
    than mean heuristic values
  \end{itemize}
\end{frame}

\begin{frame}{So what is a good pattern collection, again?}
  How can be come up with a better quality measure?
  \begin{itemize}
  \item We are chiefly interested in \alert{minimizing the number of
    node expansions} for the \alert{canonical heuristic function}
    during the \alert{actual search phase} of the planner.
  \item There is theoretical work on \alert{predicting node
    expansions} of heuristic search algorithms based on
    parameters of the heuristic (Korf, Reid \& Edelkamp, 2001).
  \item[$\leadsto$] Try to estimate these parameters, then use their
    analysis.
  \end{itemize}
\end{frame}

\begin{frame}{The Korf, Reid \& Edelkamp formula}
  \begin{block}{Korf, Reid \& Edelkamp (2001)}
    In the limit of large $c$, the expected number of node expansions
    for a failed iteration of {\idastar} with depth threshold $c$ is
    \[ E(N, c, P) = \sum_{i=0}^c N_i P(c - i) \]
    where
    \begin{itemize}
    \item $N = \langle N_0, N_1, \dots, N_c\rangle$ is the
      \alert{brute-force tree shape}
      \begin{itemize}
      \item[$\leadsto$] $N_i$: number of search nodes in layer $i$ of
        the brute-force search tree
      \end{itemize}
    \item $P$ is the \alert{equilibrium distribution} of the heuristic
      function
      \begin{itemize}
      \item[$\leadsto$] $P(k)$: probability that node chosen
        uniformly from layer $i$ of the brute-force search tree has
        heuristic value at most $k$, in the limit of large $i$
      \end{itemize}
    \end{itemize}
  \end{block}
\end{frame}

\begin{frame}{Using the formula for quality estimation}
  Some problems when using the formula for quality estimation:
  \begin{itemize}
  \item only holds in the limit
  \item applies to {\idastar}, but most planners use {\astar}
  \item we do not know $N$ or $P$
  \end{itemize}

  \bigskip

  However:
  \begin{itemize}
  \item we do not need \alert{absolute} node estimates, we only need
    to know which heuristic among a set of candidates is \alert{best}
  \item we would expect heuristics good for {\idastar} to be good for
    {\astar} most of the time
  \item we can use \alert{random walks} and \alert{sampling} to get
    approximate estimates without knowing $N$ and $P$
  \end{itemize}
\end{frame}

\begin{frame}{Estimating heuristic quality in practice}
  With some additional assumptions and simplifications, we reduce the
  problem of ranking the quality of pattern collections to the
  following criterion:

  \begin{block}{Measuring degree of improvement}
    \begin{itemize}
    \item Generate $M$ states $s_1, \dots, s_M$ through random walks
      in the search space from the initial state (according to certain
      parameters not discussed in detail).
    \item The \alert{degree of improvement} of a pattern collection
      $\mathcal C'$ which is generated as a successor of collection
      $\mathcal C$ is the \alert{number of sample states} $s_i$ for
      which \alert{$h^{\mathcal C'}(s_i) > h^{\mathcal C}(s_i)$}.
    \end{itemize}
  \end{block}
\end{frame}

\begin{frame}{Computing $h^{\mathcal C'}(s)$}
  \begin{itemize}
  \item So we need to compute $h^{\mathcal C'}(s)$ for some states $s$
    and each candidate successor collection $\mathcal C'$.
  \item We have PDBs for all patterns in $\mathcal C$, but not for the
    new pattern $P' \in \mathcal C'$ (of the form $P \cup \{v\}$ for
    some $P \in \mathcal C$).
  \item If possible, we want to avoid computing the complete pattern
    database except for the best successor (where we will need it
    later anyway).
  \end{itemize}

  \hilite{Idea:}
  \begin{itemize}
  \item For {\sasplus} tasks $\Pi$, $h^{P'}(s)$ is identical to the \\
    \alert{optimal solution length for the syntactic projection
      $\Pi|_{P'}$}.
  \item We can use \alert{any optimal planning algorithm} for this.
  \item In particular, we can use \alert{\astar} search using
    \alert{$h^P$} as a heuristic.
  \end{itemize}
\end{frame}

\section{Literature}
\subsection{References}

\begin{frame}{References\only<all:2->{ (ctd.)}}
  References on planning with pattern databases:
  \begin{thebibliography}{1}
    \only<all:1>{
    \bibitem{} Stefan Edelkamp.
      \newblock Planning with Pattern Databases.
      \newblock \emph{Proc.~ECP 2001}, pp.~13--24, 2001.
      \newblock \alert{First paper} on planning with pattern databases.
    \bibitem{} Stefan Edelkamp.
      \newblock Symbolic Pattern Databases in Heuristic Search Planning.
      \newblock \emph{Proc.~AIPS 2002}, pp.~274--283, 2002.
      \newblock \alert{Uses BDDs} to store pattern databases more
      compactly.
    }
    \only<all:2>{
    \bibitem<2>{} Patrik Haslum, Blai Bonet and H{\'e}ctor Geffner.
      \newblock New Admissible Heuristics for Domain-Independent
      Planning.
      \newblock \emph{Proc.~AAAI 2005}, pp.~1164--1168, 2005.
      \newblock Introduces \alert{constrained PDBs}. \\
      First pattern \alert{selection methods} based on \alert{heuristic
        quality}.
    }
    \only<all:3>{
    \bibitem{} Stefan Edelkamp.
      \newblock Automated Creation of Pattern Database Search
      Heuristics.
      \newblock \emph{Proc.~MoChArt 2006}, pp.~121--135, 2007.
      \newblock First \alert{search-based} pattern selection method.
    \bibitem{} Patrik Haslum, Adi Botea, Malte Helmert, Blai Bonet and
      Sven Koenig.
      \newblock Domain-Independent Construction of Pattern Database
      Heuristics for Cost-Optimal Planning.
      \newblock \emph{Proc.~AAAI 2007}, pp.~1007--1012, 2007.
      \newblock Introduces \alert{canonical heuristic} for
      pattern collections. \\
      Search-based pattern selection based on \alert{Korf, Reid \&
        Edelkamp's theory}.
    }
  \end{thebibliography}
\end{frame}

\section*{Summary}

\begin{frame}{Summary}
  \begin{itemize}
  \item \alert{Pattern database (PDB) heuristics} are abstraction
    heuristics based on \alert{projection} to a subset of variables.
  \item For \sasplus\ tasks, they can easily be implemented via
    \alert{syntactic projections} on the task representation.
  \item PDBs are \alert{lookup tables} that store heuristic values, \\
    indexed by \alert{perfect hash values} for projected states.
  \item PDB values can be looked up \alert{very fast}, \\
    in time $O(k)$ for a projection to $k$ variables.
  \end{itemize}
\end{frame}

\begin{frame}{Summary (ctd.)}
  \begin{itemize}
  \item When faced with multiple PDB heuristics (a \alert{pattern
    collection}), we want to \alert{admissibly add} their values where
    possible, and \alert{maximize} where addition is inadmissible.
  \item The \alert{canonical heuristic function} is the \alert{best
    possible} additive/maximizing combination for a given pattern
    collection given our additivity criterion.
  \item One way to \alert{automatically find a good pattern
    collection} is by performing \alert{search in the space of
    pattern collections}.
  \item One such approach uses hill-climbing search guided by the
    \alert{Korf, Reid and Edelkamp formula}, which tries to estimate
    the quality of a heuristic function.
  \end{itemize}
\end{frame}

\end{document}
