\documentclass{gkibeamer}

\usepackage{tikz}[2008/02/13]

\title[AI Planning]{Principles of AI Planning}
\author[B.~Nebel,\\R.~Mattm\"uller]{Bernhard Nebel and Robert Mattm\"uller}
\date[]{Winter Semester 2011/2012}

%% Use \alert instead of \textred where it makes sense.
%% (\textred makes sense when you *really* want the color to be red,
%% independent of the beamer style settings.)
\newcommand{\hilite}[1]{{\color{structure} #1}}
\newcommand{\textred}[1]{{\color{red} #1}}
\newcommand{\textblue}[1]{{\color{blue} #1}}
\newcommand{\textgreen}[1]{{\color{green} #1}}
\newcommand{\textbrown}[1]{{\color{brown} #1}}
\newcommand{\textviolet}[1]{{\color{violet} #1}}
\newcommand{\textbg}[1]{{\color{bg} #1}}

\newcommand{\eg}{e.\,g.}
\newcommand{\ie}{i.\,e.}

%% State variables and values for state variables
\newcommand{\var}[1]{\textit{#1}}
\newcommand{\val}[1]{\text{#1}}

%% Various stuff
\newcommand{\astar}{A${}^*$}
\newcommand{\idastar}{IDA${}^*$}

%% Blocksworld stuff
\newcommand{\AONB}{\var{A-on-B}}
\newcommand{\AONC}{\var{A-on-C}}
\newcommand{\AOND}{\var{A-on-D}}
\newcommand{\BONA}{\var{B-on-A}}
\newcommand{\BONC}{\var{B-on-C}}
\newcommand{\BOND}{\var{B-on-D}}
\newcommand{\CONA}{\var{C-on-A}}
\newcommand{\CONB}{\var{C-on-B}}
\newcommand{\COND}{\var{C-on-D}}
\newcommand{\DONA}{\var{D-on-A}}
\newcommand{\DONB}{\var{D-on-B}}
\newcommand{\DONC}{\var{D-on-C}}
\newcommand{\AONTABLE}{\var{A-on-T}}
\newcommand{\BONTABLE}{\var{B-on-T}}
\newcommand{\CONTABLE}{\var{C-on-T}}
\newcommand{\CLEARA}{\var{A-clear}}
\newcommand{\CLEARB}{\var{B-clear}}
\newcommand{\CLEARC}{\var{C-clear}}

\newcommand{\applyop}[2]{\textit{app}_{#1}(#2)}
\newcommand{\applyops}[2]{\textit{app}_{#1}(#2)}
\newcommand{\applyplan}[2]{\textit{app}_{#1}(#2)}
\newcommand{\regr}[2]{\textit{regr}_{#1}(#2)}
\newcommand{\regrstrips}[2]{\textit{sregr}_{#1}(#2)}
\newcommand{\eprecon}[2]{\textit{EPC}_{#1}(#2)}

\newcommand{\compl}[1]{\overline{#1}}

\newcommand{\changes}[2]{\lbrack #1\rbrack_{#2}}
\newcommand{\CEF}{\vartriangleright}

\newcommand{\relaxation}[1]{{#1}^+}
\newcommand{\onset}[1]{\textit{on}(#1)}
\newcommand{\hplus}{\ensuremath{h^+}}
\newcommand{\hmax}{\ensuremath{h_{\text{max}}}}
\newcommand{\hadd}{\ensuremath{h_{\text{add}}}}
\newcommand{\hlmcut}{\ensuremath{h_{\text{LM-cut}}}}
\newcommand{\hff}{\ensuremath{h_{\text{FF}}}}
\newcommand{\hcs}{\ensuremath{h_{\text{cs}}}}
\newcommand{\hsa}{\ensuremath{h_{\text{sa}}}}
\newcommand{\hhhh}{\ensuremath{h_{\text{HHH}}}}

\newcommand{\sasplus}{SAS${}^+$}

\newcommand{\graphequiv}{\stackrel{\textup{G}}{\sim}}

\newcommand{\cg}{\ensuremath{\textit{CG}}}

%% Turing machines and complexity theory
\newcommand{\accept}{{\textsf{Y}}}

\newcommand{\easier}{\ensuremath{\le_{\text{p}}}}

\newcommand{\decisionclass}[1]{\ensuremath{\textsf{\textup{#1}}}}
\newcommand{\dtime}{\decisionclass{DTIME}}
\newcommand{\ntime}{\decisionclass{NTIME}}
\newcommand{\dspace}{\decisionclass{DSPACE}}
\newcommand{\nspace}{\decisionclass{NSPACE}}

\newcommand{\ptime}{\decisionclass{P}}
\newcommand{\np}{\decisionclass{NP}}
\newcommand{\pspace}{\decisionclass{PSPACE}}
\newcommand{\npspace}{\decisionclass{NPSPACE}}
\newcommand{\exptime}{\decisionclass{EXP}}
\newcommand{\nexptime}{\decisionclass{NEXP}}
\newcommand{\expspace}{\decisionclass{EXPSPACE}}
\newcommand{\dblexptime}{\decisionclass{2EXP}}

\newcommand{\planex}{\textsc{PlanEx}}
\newcommand{\planlen}{\textsc{PlanLen}}

\newcommand{\ms}[1]{\mathchoice{\mbox{\normalsize\it #1}}{\mbox{\normalsize\it #1}}{\mbox{\scriptsize\it #1}}{\mbox{\tiny\it #1}}}
\newcommand{\image}[2]{\ms{img}_{#1}(#2)}
\newcommand{\spreimage}[2]{\ms{spreimg}_{#1}(#2)}
\newcommand{\wpreimage}[2]{\ms{wpreimg}_{#1}(#2)}
\newcommand{\disbwd}[2]{\delta^{\ms{bwd}}_{#1}(#2)}
\newcommand{\ats}[1]{\ms{scope}(#1)}
\newcommand{\ndopcpc}[2]{\tau^{\ms{nd}}_{#1}(#2)}
\newcommand{\choice}{\mathop{|}}
\newcommand{\opcpc}[2]{\tau_{#1}(#2)}

%% Don't use \begin{proof}/\end{proof} for multi-part proofs because
%% this places a QED symbol.
\newenvironment{proofstart}{\begin{block}{Proof.}}{\end{block}}
\newenvironment{proofmid}{\begin{block}{Proof (ctd.)}}{\end{block}}
\newenvironment{proofend}{\begin{proof}[Proof (ctd.)]}{\end{proof}}

\newtheorem{remark}[theorem]{Remark}

%% Macros to align the width of things.
\newlength{\mywidth}
\newcommand{\setmywidth}[1]{\settowidth{\mywidth}{#1}}
\newcommand{\usemywidth}[1]{\makebox[\mywidth][l]{#1}}
\newcommand{\usemywidthmath}[1]{\usemywidth{\ensuremath{#1}}}

%% A tighter "align" environment.
\newenvironment{tightalign}[1][c]{\par\(\begin{array}[#1]{@{}r@{}l}}
  {\end{array}\)\par}
\newenvironment{wrappedmath}[1][t]{\begin{array}[#1]{@{}l}}{\end{array}}

%% Nondeterministic transition systems with connector arcs
\newcommand{\arc}[4]{
  \begin{pgfscope}
    \pgfsetlinewidth{1pt}
    \pgfpathmoveto{\pgfpointanchor{#3}{center}}
    \pgfpathlineto{\pgfpointanchor{#1}{#2}}
    \pgfpathlineto{\pgfpointanchor{#4}{center}}
    \pgfusepath{clip}
    \pgfpathcircle{\pgfpointanchor{#1}{#2}}{1.8mm}
    \pgfusepath{draw}
  \end{pgfscope}
}

\newcommand{\drawblock}[4]{% Parameters: color, stack, level,  label
  \draw[draw=black,fill=#1!55] (#2,0.8*#3) ++(0,0.8) -- ++(0.8,0) -- ++(0.4,0.4) -- ++(-0.8,0) -- ++(-0.4,-0.4);
  \draw[draw=black,fill=#1!75] (#2,0.8*#3) ++(0.8,0) -- ++(0.4,0.4) -- ++(0,0.8) -- ++(-0.4,-0.4) -- ++(0,-0.8);
  \draw[draw=black,fill=#1]    (#2,0.8*#3) rectangle +(0.8,0.8);
  \path (#2,0.8*#3) +(0.4,0.4) node {#4};
}

\newcommand{\drawblackblock}[3]{% Parameters: stack, level,  label
  \draw[draw=black!90!white,fill=blue!10!white] (#1,0.8*#2) ++(0,0.8) -- ++(0.8,0) -- ++(0.4,0.4) -- ++(-0.8,0) -- ++(-0.4,-0.4);
  \draw[draw=black!90!white,fill=blue!10!white] (#1,0.8*#2) ++(0.8,0) -- ++(0.4,0.4) -- ++(0,0.8) -- ++(-0.4,-0.4) -- ++(0,-0.8);
  \draw[draw=black!90!white,fill=blue!10!white] (#1,0.8*#2) rectangle +(0.8,0.8);
  \path (#1,0.8*#2) +(0.4,0.4) node {#3};
}

\newcommand{\drawtable}[1]{ % Parameter: number of stacks
  \draw[draw=black!80!white,fill=black!50!white] (-0.2,-0.2) -- ++(#1,0) -- ++(0.2,0)
  -- ++(0.8,0.8) -- ++(-0.2,0) -- ++(-#1,0) -- ++(-0.2,0) -- ++(-0.8,-0.8) -- ++(0.2,0);
} 

\newcommand{\drawleftfinger}[2]{ % Parameters: stack, level
  \draw[draw=black!80!white,fill=black!50!white] (#1,0.8*#2) ++(0.1,0.5) -- ++(0.2,0.2)
  -- ++(0,0.6) -- ++(-0.2,-0.2) -- ++(0,-0.6);
  \draw[draw=black!80!white,fill=black!50!white] (#1,0.8*#2) ++(-0.1,0.5) rectangle +(0.2,0.6);
}

\newcommand{\drawrightfinger}[2]{ % Parameters: stack, level
  \draw[draw=black!80!white,fill=black!50!white] (#1,0.8*#2) ++(1.1,0.5) -- ++(0.2,0.2)
  -- ++(0,0.6) -- ++(-0.2,-0.2) -- ++(0,-0.6);
  \draw[draw=black!80!white,fill=black!50!white] (#1,0.8*#2) ++(0.9,0.5) rectangle +(0.2,0.6);
}

\newcommand{\drawhandfront}[2]{ % Parameters: stack, level
  \draw[draw=black!80!white,fill=black!50!white] (#1,0.8*#2) ++(-0.1,0.9) rectangle ++(1.2,0.2);
  \draw[draw=black!50!white,thick] (#1,0.8*#2) ++(-0.095,0.9) -- ++(0.17,0);
  \draw[draw=black!50!white,thick] (#1,0.8*#2) ++(0.91,0.9) -- ++(0.185,0);
}

\newcommand{\drawhandtop}[2]{ % Parameters: stack, level
  \draw[draw=black!80!white,fill=black!50!white] (#1,0.8*#2) ++(-0.1,1.1) -- ++(1.2,0)
  -- ++(0.2,0.2) -- ++(-1.2,0) -- ++(-0.2,-0.2);
}

\newcommand{\drawhandle}[2]{ % Parameters: stack, level
  \draw[draw=black!80!white,fill=black!50!white] (#1,0.8*#2) ++(0.7,1.1) -- ++(0.2,0.2)
  -- ++(0,0.5) -- ++(-0.2,-0.2) --++(0,-0.5);
  \draw[draw=black!80!white,fill=black!50!white] (#1,0.8*#2) ++(0.3,1.6) -- ++(0.4,0)
  -- ++(0.2,0.2) -- ++(-0.4,0) -- ++(-0.2,-0.2);
  \draw[draw=black!80!white,fill=black!50!white] (#1,0.8*#2) ++(0.3,1.1) rectangle ++(0.4,0.5);
  \draw[draw=black!50!white,thick] (#1,0.8*#2) ++(0.305,1.1) -- ++(0.39,0);
}

\newcommand{\drawblackleftfinger}[2]{ % Parameters: stack, level
  \draw[draw=black!90!white,fill=blue!10!white] (#1,0.8*#2) ++(0.1,0.5) -- ++(0.2,0.2)
  -- ++(0,0.6) -- ++(-0.2,-0.2) -- ++(0,-0.6);
  \draw[draw=black!90!white,fill=blue!10!white] (#1,0.8*#2) ++(-0.1,0.5) rectangle +(0.2,0.6);
}

\newcommand{\drawblackrightfinger}[2]{ % Parameters: stack, level
  \draw[draw=black!90!white,fill=blue!10!white] (#1,0.8*#2) ++(1.1,0.5) -- ++(0.2,0.2)
  -- ++(0,0.6) -- ++(-0.2,-0.2) -- ++(0,-0.6);
  \draw[draw=black!90!white,fill=blue!10!white] (#1,0.8*#2) ++(0.9,0.5) rectangle +(0.2,0.6);
}

\newcommand{\drawblackhandfront}[2]{ % Parameters: stack, level
  \draw[draw=black!90!white,fill=blue!10!white] (#1,0.8*#2) ++(-0.1,0.9) rectangle ++(1.2,0.2);
  \draw[draw=blue!10!white, thick] (#1,0.8*#2) ++(-0.095,0.9) -- ++(0.17,0);
  \draw[draw=blue!10!white, thick] (#1,0.8*#2) ++(0.91,0.9) -- ++(0.185,0);
}

\newcommand{\drawblackhandtop}[2]{ % Parameters: stack, level
  \draw[draw=black!90!white,fill=blue!10!white] (#1,0.8*#2) ++(-0.1,1.1) -- ++(1.2,0)
  -- ++(0.2,0.2) -- ++(-1.2,0) -- ++(-0.2,-0.2);
}

\newcommand{\drawblackhandle}[2]{ % Parameters: stack, level
  \draw[draw=black!90!white,fill=blue!10!white] (#1,0.8*#2) ++(0.7,1.1) -- ++(0.2,0.2)
  -- ++(0,0.5) -- ++(-0.2,-0.2) --++(0,-0.5);
  \draw[draw=black!90!white,fill=blue!10!white] (#1,0.8*#2) ++(0.3,1.6) -- ++(0.4,0)
  -- ++(0.2,0.2) -- ++(-0.4,0) -- ++(-0.2,-0.2);
  \draw[draw=black!90!white,fill=blue!10!white] (#1,0.8*#2) ++(0.3,1.1) rectangle ++(0.4,0.5);
  \draw[draw=blue!10!white,thick] (#1,0.8*#2) ++(0.305,1.1) -- ++(0.39,0);
}

\definecolor{Navy}{rgb}{0.3,0.3,1}

\newcommand{\bwstateA}{
  \begin{tikzpicture}[scale=0.45]
    \drawtable{3}
    \drawblock{red}{0}{0}{\textbf{B}}
    \drawblock{Navy}{0}{1}{\textbf{A}}
    \drawblock{green}{1}{0}{\textbf{C}}
    \drawblock{yellow}{2}{0}{\textbf{D}}
    \drawleftfinger{1.5}{1}
    \drawrightfinger{1.5}{1}
    \drawhandfront{1.5}{1}
    \drawhandtop{1.5}{1}
    \drawhandle{1.5}{1}
  \end{tikzpicture}
}

\newcommand{\bwstateB}{
  \begin{tikzpicture}[scale=0.45]
    \drawtable{3}
    \drawblock{red}{0}{0}{\textbf{B}}
    \drawblock{green}{1}{0}{\textbf{C}}
    \drawblock{yellow}{2}{0}{\textbf{D}}
    \drawleftfinger{0.5}{1.75}
    \drawblock{Navy}{0.5}{1.75}{\textbf{A}}
    \drawrightfinger{0.5}{1.75}
    \drawhandfront{0.5}{1.75}
    \drawhandtop{0.5}{1.75}
    \drawhandle{0.5}{1.75}
  \end{tikzpicture}
}

\newcommand{\bwstateC}{
  \begin{tikzpicture}[scale=0.45]
    \drawtable{4}
    \drawblock{red}{0}{0}{\textbf{B}}
    \drawblock{green}{1}{0}{\textbf{C}}
    \drawblock{yellow}{2}{0}{\textbf{D}}
    \drawblock{Navy}{3}{0}{\textbf{A}}
    \drawleftfinger{3}{1}
    \drawrightfinger{3}{1}
    \drawhandfront{3}{1}
    \drawhandtop{3}{1}
    \drawhandle{3}{1}
  \end{tikzpicture}
}

\newcommand{\bwstateD}{
  \begin{tikzpicture}[scale=0.45]
    \drawtable{3}
    \drawblock{red}{0}{0}{\textbf{B}}
    \drawblock{green}{1}{0}{\textbf{C}}
    \drawblock{yellow}{2}{0}{\textbf{D}}
    \drawblock{Navy}{1}{1}{\textbf{A}}
    \drawleftfinger{1}{2}
    \drawrightfinger{1}{2}
    \drawhandfront{1}{2}
    \drawhandtop{1}{2}
    \drawhandle{1}{2}
  \end{tikzpicture}
}

\newcommand{\bwstateAabs}{
  \begin{tikzpicture}[scale=0.45]
    \drawtable{3}
    \drawblock{red}{0}{0}{\textbf{B}}
    \drawblock{Navy}{0}{1}{\textbf{A}}
    \drawblock{green}{1}{0}{\textbf{C}}
    \drawblackblock{2}{0}{\textbf{$\alert{\times}$}}
    \drawleftfinger{1.5}{1}
    \drawrightfinger{1.5}{1}
    \drawhandfront{1.5}{1}
    \drawhandtop{1.5}{1}
    \drawhandle{1.5}{1}
  \end{tikzpicture}
}

\newcommand{\bwstateBabs}{
  \begin{tikzpicture}[scale=0.45]
    \drawtable{3}
    \drawblock{red}{0}{0}{\textbf{B}}
    \drawblock{green}{1}{0}{\textbf{C}}
    \drawblackblock{2}{0}{\textbf{$\alert{\times}$}}
    \drawleftfinger{0.5}{1.75}
    \drawblock{Navy}{0.5}{1.75}{\textbf{A}}
    \drawrightfinger{0.5}{1.75}
    \drawhandfront{0.5}{1.75}
    \drawhandtop{0.5}{1.75}
    \drawhandle{0.5}{1.75}
  \end{tikzpicture}
}

\newcommand{\bwstateCabs}{
  \begin{tikzpicture}[scale=0.45]
    \drawtable{4}
    \drawblock{red}{0}{0}{\textbf{B}}
    \drawblock{green}{1}{0}{\textbf{C}}
    \drawblackblock{2}{0}{\textbf{$\alert{\times}$}}
    \drawblock{Navy}{3}{0}{\textbf{A}}
    \drawleftfinger{3}{1}
    \drawrightfinger{3}{1}
    \drawhandfront{3}{1}
    \drawhandtop{3}{1}
    \drawhandle{3}{1}
  \end{tikzpicture}
}

\newcommand{\bwstateDabs}{
  \begin{tikzpicture}[scale=0.45]
    \drawtable{3}
    \drawblock{red}{0}{0}{\textbf{B}}
    \drawblock{green}{1}{0}{\textbf{C}}
    \drawblackblock{2}{0}{\textbf{$\alert{\times}$}}
    \drawblock{Navy}{1}{1}{\textbf{A}}
    \drawleftfinger{1}{2}
    \drawrightfinger{1}{2}
    \drawhandfront{1}{2}
    \drawhandtop{1}{2}
    \drawhandle{1}{2}
  \end{tikzpicture}
}

\newcommand{\blockA}{
  \begin{tikzpicture}[scale=0.45,baseline=2mm]
    \drawblock{Navy}{0}{0}{\textbf{A}}
  \end{tikzpicture}
}

\newcommand{\blockB}{
  \begin{tikzpicture}[scale=0.45,baseline=2mm]
    \drawblock{red}{0}{0}{\textbf{B}}
  \end{tikzpicture}
}

\newcommand{\blockC}{
  \begin{tikzpicture}[scale=0.45,baseline=2mm]
    \drawblock{green}{0}{0}{\textbf{C}}
  \end{tikzpicture}
}

\newcommand{\blockD}{
  \begin{tikzpicture}[scale=0.45,baseline=2mm]
    \drawblock{yellow}{0}{0}{\textbf{D}}
  \end{tikzpicture}
}

























\newcommand{\bwstateXA}{
  \begin{tikzpicture}[scale=0.45]
    \drawtable{2}
    \drawblock{Navy}{0}{0}{\textbf{A}}
    \drawblock{red}{1}{0}{\textbf{B}}
    \drawleftfinger{0.5}{1}
    \drawrightfinger{0.5}{1}
    \drawhandfront{0.5}{1}
    \drawhandtop{0.5}{1}
    \drawhandle{0.5}{1}
  \end{tikzpicture}
}

\newcommand{\bwstateXB}{
  \begin{tikzpicture}[scale=0.45]
    \drawtable{2}
    \drawleftfinger{0}{1}
    \drawblock{Navy}{0}{1}{\textbf{A}}
    \drawrightfinger{0}{1}
    \drawhandfront{0}{1}
    \drawhandtop{0}{1}
    \drawhandle{0}{1}
    \drawblock{red}{1}{0}{\textbf{B}}
  \end{tikzpicture}
}

\newcommand{\bwstateXC}{
  \begin{tikzpicture}[scale=0.45]
    \drawtable{2}
    \drawblock{Navy}{0}{0}{\textbf{A}}
    \drawleftfinger{1}{1}
    \drawblock{red}{1}{1}{\textbf{B}}
    \drawrightfinger{1}{1}
    \drawhandfront{1}{1}
    \drawhandtop{1}{1}
    \drawhandle{1}{1}
  \end{tikzpicture}
}

\newcommand{\bwstateXD}{
  \begin{tikzpicture}[scale=0.45]
    \drawtable{2}
    \drawblock{red}{0}{0}{\textbf{B}}
    \drawblock{Navy}{0}{1}{\textbf{A}}
    \drawleftfinger{1}{1}
    \drawrightfinger{1}{1}
    \drawhandfront{1}{1}
    \drawhandtop{1}{1}
    \drawhandle{1}{1}
  \end{tikzpicture}
}

\newcommand{\bwstateXE}{
  \begin{tikzpicture}[scale=0.45]
    \drawtable{2}
    \drawblock{Navy}{0}{0}{\textbf{A}}
    \drawblock{red}{0}{1}{\textbf{B}}
    \drawleftfinger{1}{1}
    \drawrightfinger{1}{1}
    \drawhandfront{1}{1}
    \drawhandtop{1}{1}
    \drawhandle{1}{1}
  \end{tikzpicture}
}

\newcommand{\bwstateXAabs}{
  \begin{tikzpicture}[scale=0.45]
    \drawtable{2}
    \drawblock{Navy}{0}{0}{\textbf{A}}
    \drawblackblock{1}{0}{\textbf{$\alert{\times}$}}
    \drawblackleftfinger{0.5}{1}
    \drawblackrightfinger{0.5}{1}
    \drawblackhandfront{0.5}{1}
    \drawblackhandtop{0.5}{1}
    \drawblackhandle{0.5}{1}
  \end{tikzpicture}
}

\newcommand{\bwstateXBabs}{
  \begin{tikzpicture}[scale=0.45]
    \drawtable{2}
    \drawblackleftfinger{0}{1}
    \drawblock{Navy}{0}{1}{\textbf{A}}
    \drawblackrightfinger{0}{1}
    \drawblackhandfront{0}{1}
    \drawblackhandtop{0}{1}
    \drawblackhandle{0}{1}
    \drawblackblock{1}{0}{\textbf{$\alert{\times}$}}
  \end{tikzpicture}
}

\newcommand{\bwstateXCabs}{
  \begin{tikzpicture}[scale=0.45]
    \drawtable{2}
    \drawblock{Navy}{0}{0}{\textbf{A}}
    \drawblackleftfinger{1}{1}
    \drawblackblock{1}{1}{\textbf{$\alert{\times}$}}
    \drawblackrightfinger{1}{1}
    \drawblackhandfront{1}{1}
    \drawblackhandtop{1}{1}
    \drawblackhandle{1}{1}
  \end{tikzpicture}
}

\newcommand{\bwstateXDabs}{
  \begin{tikzpicture}[scale=0.45]
    \drawtable{2}
    \drawblackblock{0}{0}{\textbf{$\alert{\times}$}}
    \drawblock{Navy}{0}{1}{\textbf{A}}
    \drawblackleftfinger{1}{1}
    \drawblackrightfinger{1}{1}
    \drawblackhandfront{1}{1}
    \drawblackhandtop{1}{1}
    \drawblackhandle{1}{1}
  \end{tikzpicture}
}

\newcommand{\bwstateXEabs}{
  \begin{tikzpicture}[scale=0.45]
    \drawtable{2}
    \drawblock{Navy}{0}{0}{\textbf{A}}
    \drawblackblock{0}{1}{\textbf{$\alert{\times}$}}
    \drawblackleftfinger{1}{1}
    \drawblackrightfinger{1}{1}
    \drawblackhandfront{1}{1}
    \drawblackhandtop{1}{1}
    \drawblackhandle{1}{1}
  \end{tikzpicture}
}



\newcommand{\bwstateYAabs}{
  \begin{tikzpicture}[scale=0.45]
    \drawtable{1}
    \drawblock{Navy}{0}{0}{\textbf{A}}
  \end{tikzpicture}
}

\newcommand{\bwstateYBabs}{
  \begin{tikzpicture}[scale=0.45]
    \drawblackleftfinger{0}{0}
    \drawblock{Navy}{0}{0}{\textbf{A}}
    \drawblackrightfinger{0}{0}
    \drawblackhandfront{0}{0}
    \drawblackhandtop{0}{0}
    \drawblackhandle{0}{0}
  \end{tikzpicture}
}

\newcommand{\bwstateYDabs}{
  \begin{tikzpicture}[scale=0.45]
    \drawblackblock{0}{0}{{\color{gray}\textbf{B}}}
    \drawblock{Navy}{0}{1}{\textbf{A}}
  \end{tikzpicture}
}


\usetikzlibrary{positioning}

\begin{document}
\subtitle{14.~Strong nondeterministic planning}
\date{January 24th, 2012}
\maketitles

\begin{frame}
  \frametitle{Strong planning}

  We will consider one of the simplest cases of nondeterministic
  planning by restricting attention to \alert{strong plans}.
\end{frame}

\section{Concepts}
\subsection{Strong plans}

\begin{frame}{Strong plans}
  Recall the definition of strong plans:
  
  \medskip
  
  \begin{definition}[strong plan]
    Let $S$ be the set of states of a planning task $\mathcal T$. Then a
    \alert{strong plan} for $\Pi$ is a function $\pi: S_\pi \to O$ for
    some subset $S_\pi \subseteq S$ such that
    \begin{itemize}
    \item $\pi(s)$ is applicable in $s$ for all $s \in S_\pi$,
    \item $S_\pi(s_0) \subseteq S_\pi$ ($\pi$ is closed),
    \item $S_\pi(s') \cap S_\star \neq \emptyset$ for all $s' \in S_\pi(s_0)$ ($\pi$ is proper), and
    \item there is no state $s' \in S_\pi(s_0)$ such that $s'$ is
      reachable from $s'$ following $\pi$ in a strictly positive number
      of steps ($\pi$ is acyclic).
    \end{itemize}
  \end{definition}
\end{frame}

\begin{frame}{Strong plans}
  \begin{block}{Execution of a strong plan}
    \begin{enumerate}
    \item Determine the current state $s$.
    \item If $s$ is a goal state or $\pi(s)$ is not defined then
      terminate.
    \item Execute action $\pi(s)$.
    \item Repeat from first step.
    \end{enumerate}
  \end{block}
\end{frame}

\begin{frame}<handout:1>[label=strong-plan-blocksworld]
  \frametitle{Strong plans}
  \begin{center}
    \begin{tikzpicture}[>=latex,xscale=6,yscale=2.75]
      \tikzstyle{hypernode}=[draw,shape=rectangle,draw=black,fill=blue!10!white]
      \tikzstyle{hyperedge}=[thick,rounded corners=20pt,->]
      %% step 1
      \visible<1->{
        \node[hypernode] at (0.5,2) (nA) {\bwstateA};
        \node[left=of nA] (dummyI) {};
        \draw[hyperedge] (dummyI) -- (nA);
      }
      %% step 2
      \visible<2->{
        \node[hypernode] at (0,1)   (nB) {\bwstateB};
        \node[hypernode] at (1,1)   (nC) {\bwstateC};
        \path (nB.south) ++(0.5,-0.5) node (dummyB) {};
        \path (nC.west)  ++(-0.5,0.2) node (dummyC) {};
        \draw[hyperedge] (nA.south) -- (nB);
        \draw[hyperedge] (nA.south) -- (nC);
        \arc{nA}{south}{nB}{nC}
        \node (labelA) at (0.8,1.5) {\footnotesize{(pick-up A B)}};
      }
      %% step 3
      \visible<3->{
        \node[hypernode,double] at (0,0)   (nD) {\bwstateD};
        %% \draw[hyperedge] (nB.south) -- ++(0.5,-0.5) -- (nC);
        \draw[hyperedge] (nB.south) -- (nD);
        %% \arc{nB}{south}{nD}{dummyB}
        \node[anchor=east] (labelB) at (0.0,0.5) {\footnotesize{(put-on A C)}};
      }
      %% step 4
      \visible<4>{
        \draw[hyperedge] (nC.west) -- (nB);
        %% \draw[hyperedge] (nC.west) -- ++(-0.5,0.2) -- ++(0.25,0.25) -- (nC);
        \node[anchor=north] (labelC) at (0.5,1) {\footnotesize{(pick-up-from-table A)}};
        %% \arc{nC}{west}{nB}{dummyC}
      }
    \end{tikzpicture}
  \end{center}
\end{frame}

\only<handout>{
  \againframe<handout:4>{strong-plan-blocksworld}
}

\subsection{Images}

\begin{frame}
  \frametitle{Images}

  \begin{block}{Image}
    The \alert{image} of a set $T$ of states with respect to an
    operator $o$ is the set of those states that can be reached by
    executing $o$ in a state in $T$.
  \end{block}

  \begin{center}
    \begin{pgfpicture}{0cm}{1cm}{6.0cm}{5.5cm}
      \pgfsetlinewidth{0.7pt}

      \pgfputat{\pgfxy(1.3,4.4)}{\pgfbox[left,left]{$T$}}
      %% DASHED LINES
      \pgfsetdash{{3pt}{3pt}}{0pt}
      \pgfellipse[stroke]{\pgfxy(1,3)}{\pgfxy(0.5,0)}{\pgfxy(0,1.5)}
      \pgfputat{\pgfxy(4.3,4.4)}{\pgfbox[left,left]{$\image{o}{T}$}}
      %% DASHED LINES
      \pgfsetdash{{3pt}{3pt}}{0pt}
      \pgfellipse[stroke]{\pgfxy(4,3)}{\pgfxy(0.5,0)}{\pgfxy(0,1.5)}
      %% NORMAL LINES
      \pgfsetdash{{3pt}{0pt}}{3pt}
      \pgfnodecircle{N01}[fill]{\pgfxy(1,2)}{0.1cm}
      \pgfnodecircle{N02}[fill]{\pgfxy(1,3)}{0.1cm}
      \pgfnodecircle{N03}[fill]{\pgfxy(1,4)}{0.1cm}
      \pgfnodecircle{N11}[fill]{\pgfxy(4,2)}{0.1cm}
      \pgfnodecircle{N12}[fill]{\pgfxy(4,3)}{0.1cm}
      \pgfnodecircle{N13}[fill]{\pgfxy(4,4)}{0.1cm}
      \pgfnodecircle{N14}[fill]{\pgfxy(4,5)}{0.1cm}
      \pgfsetendarrow{\pgfarrowtriangle{4pt}}
      \pgfnodeconnline{N01}{N11}
      \pgfnodeconnline{N01}{N12}
      \pgfnodeconnline{N02}{N12}
      \pgfnodeconnline{N02}{N13}
      \pgfnodeconnline{N03}{N13}
    \end{pgfpicture}
  \end{center}
\end{frame}

\begin{frame}
  \frametitle{Images}

  \begin{definition}[image of a state]
    $\image{o}{s} = \{ s'\in S | s \xrightarrow{o} s' \} = \applyop{o}{s}$
  \end{definition}
  
  \begin{definition}[image of a set of states]
    $\image{o}{T} = \bigcup_{s\in T}\image{o}{s}$
  \end{definition}
\end{frame}

\subsection{Weak preimages}

\begin{frame}
  \frametitle{Weak preimages}

  \begin{block}{Weak preimage}
    The \alert{weak preimage} of a set $T$ of states with respect to
    an operator $o$ is the set of those states from which a state in
    $T$ can be reached by executing $o$.
  \end{block}
  
  \begin{center}
    \begin{pgfpicture}{0cm}{1cm}{5cm}{5.5cm}
      \pgfsetlinewidth{0.7pt}
      
      \pgfputat{\pgfxy(1.3,4.4)}{\pgfbox[left,left]{$\wpreimage{o}{T}$}}
      %% DASHED LINES
      \pgfsetdash{{3pt}{3pt}}{0pt}
      \pgfellipse[stroke]{\pgfxy(1,3)}{\pgfxy(0.5,0)}{\pgfxy(0,1.5)}
      \pgfputat{\pgfxy(4.3,4.4)}{\pgfbox[left,left]{$T$}}
      %% DASHED LINES
      \pgfsetdash{{3pt}{3pt}}{0pt}
      \pgfellipse[stroke]{\pgfxy(4,3.5)}{\pgfxy(0.5,0)}{\pgfxy(0,1)}
      %% NORMAL LINES
      \pgfsetdash{{3pt}{0pt}}{3pt}
      \pgfnodecircle{N01}[fill]{\pgfxy(1,2)}{0.1cm}
      \pgfnodecircle{N02}[fill]{\pgfxy(1,3)}{0.1cm}
      \pgfnodecircle{N03}[fill]{\pgfxy(1,4)}{0.1cm}
      \pgfnodecircle{N04}[fill]{\pgfxy(1,5)}{0.1cm}
      \pgfnodecircle{N11}[fill]{\pgfxy(4,2)}{0.1cm}
      \pgfnodecircle{N12}[fill]{\pgfxy(4,3)}{0.1cm}
      \pgfnodecircle{N13}[fill]{\pgfxy(4,4)}{0.1cm}
      \pgfsetendarrow{\pgfarrowtriangle{4pt}}
      \pgfnodeconnline{N01}{N11}
      \pgfnodeconnline{N01}{N12}
      \pgfnodeconnline{N02}{N12}
      \pgfnodeconnline{N02}{N13}
      \pgfnodeconnline{N03}{N13}
    \end{pgfpicture}
  \end{center}
\end{frame}

\begin{frame}
  \frametitle{Weak preimages}

  \begin{definition}[weak preimage of a state]
    $\wpreimage{o}{s'} = \{ s\in S | s \xrightarrow{o} s' \}$
  \end{definition}

  \begin{definition}[weak preimage of a set of states]
    $\wpreimage{o}{T} = \bigcup_{s\in T}\wpreimage{o}{s}$.
  \end{definition}
\end{frame}

\subsection{Strong preimages}
\begin{frame}
  \frametitle{Strong preimages}

  \begin{block}{Strong preimage}
    The \alert{strong preimage} of a set $T$ of states with
    respect to an operator $o$ is the set of
    those states from which a state in $T$ is always reached when
    executing $o$.
  \end{block}

  \begin{center}
    \begin{pgfpicture}{0cm}{1cm}{5cm}{5.5cm}
      \pgfsetlinewidth{0.7pt}

      \pgfputat{\pgfxy(1.3,4.4)}{\pgfbox[left,left]{$\spreimage{o}{T}$}}
      %% DASHED LINES
      \pgfsetdash{{3pt}{3pt}}{0pt}
      \pgfellipse[stroke]{\pgfxy(1,3.5)}{\pgfxy(0.5,0)}{\pgfxy(0,1)}
      \pgfputat{\pgfxy(4.3,4.4)}{\pgfbox[left,left]{$T$}}
      %% DASHED LINES
      \pgfsetdash{{3pt}{3pt}}{0pt}
      \pgfellipse[stroke]{\pgfxy(4,3.5)}{\pgfxy(0.5,0)}{\pgfxy(0,1)}
      %% NORMAL LINES
      \pgfsetdash{{3pt}{0pt}}{3pt}
      \pgfnodecircle{N01}[fill]{\pgfxy(1,2)}{0.1cm}
      \pgfnodecircle{N02}[fill]{\pgfxy(1,3)}{0.1cm}
      \pgfnodecircle{N03}[fill]{\pgfxy(1,4)}{0.1cm}
      \pgfnodecircle{N04}[fill]{\pgfxy(1,5)}{0.1cm}
      \pgfnodecircle{N11}[fill]{\pgfxy(4,2)}{0.1cm}
      \pgfnodecircle{N12}[fill]{\pgfxy(4,3)}{0.1cm}
      \pgfnodecircle{N13}[fill]{\pgfxy(4,4)}{0.1cm}
      \pgfsetendarrow{\pgfarrowtriangle{4pt}}
      \pgfnodeconnline{N01}{N11}
      \pgfnodeconnline{N01}{N12}
      \pgfnodeconnline{N02}{N12}
      \pgfnodeconnline{N02}{N13}
      \pgfnodeconnline{N03}{N13}
    \end{pgfpicture}
  \end{center}
\end{frame}

\begin{frame}
  \frametitle{Strong preimages}

  \begin{definition}[strong preimage of a set of states]
    $\spreimage{o}{T} = \{ s\in S~|~\exists s'\in T: s \xrightarrow{o} s' \land
    \image{o}{s} \subseteq T \}$
  \end{definition}
\end{frame}

\section{Basic algorithms}

\begin{frame}
  \frametitle{Algorithms for strong planning}

  \begin{enumerate}
  \item \alert{Heuristic search} (forward)

    Strong planning can be viewed as AND/OR graph search.
    \begin{description}[AND nodes]
    \item[OR nodes:] Choice between operators
    \item[AND nodes:] Nondeterministically reached state
    \end{description}
    Heuristic AND-OR search algorithms: \\
    AO*, Proof Number Search, \dots
  \item \alert{Dynamic programming} (backward)
    
    Compute operator/distance/value for a state based on the
    operators/distances/values of its all successor states.
    \begin{enumerate}
    \item Zero actions needed for goal states.
    \item If states with $i$ actions to goals are known,
      states with $\leq i+1$ actions to goals can be easily identified.
    \end{enumerate}
    Automatic reuse of already found plan suffixes.
  \end{enumerate}
\end{frame}

\subsection{AND/OR graph search}

\begin{frame}<handout:1>[label=and-or-search]
  \frametitle{AND/OR search}
  \begin{center}
    \begin{pgfpicture}{1.15cm}{0.6cm}{11.35cm}{7.6cm}
      \pgfsetxvec{\pgfpoint{1.31cm}{0cm}}
      \pgfsetyvec{\pgfpoint{0cm}{1cm}}

      \pgfnodecircle{N0}[fill]{\pgfxy(4.75,7.5)}{1.1mm}
      \only<all:2>{\pgfnodecircle{N00}[fill]{\pgfxy(2.75,6)}{1.1mm}}
      \only<all:2>{\pgfnodecircle{N01}[fill]{\pgfxy(6.75,6)}{1.1mm}}
      \pgfnodecircle{N000}[fill]{\pgfxy(1.75,4.5)}{1.1mm}
      \pgfnodecircle{N001}[fill]{\pgfxy(3.75,4.5)}{1.1mm}
      \pgfnodecircle{N010}[fill]{\pgfxy(5.75,4.5)}{1.1mm}
      \pgfnodecircle{N011}[fill]{\pgfxy(7.75,4.5)}{1.1mm}

      \only<all:2>{\pgfnodecircle{N0000}[fill]{\pgfxy(1.25,2.5)}{1.1mm}}
      \only<all:2>{\pgfnodecircle{N0001}[fill]{\pgfxy(2.25,2.5)}{1.1mm}}
      \only<all:2>{\pgfnodecircle{N0010}[fill]{\pgfxy(3.25,2.5)}{1.1mm}}
      \only<all:2>{\pgfnodecircle{N0011}[fill]{\pgfxy(4.25,2.5)}{1.1mm}}
      \only<all:2>{\pgfnodecircle{N0100}[fill]{\pgfxy(5.25,2.5)}{1.1mm}}
      \only<all:2>{\pgfnodecircle{N0101}[fill]{\pgfxy(6.25,2.5)}{1.1mm}}
      \only<all:2>{\pgfnodecircle{N0110}[fill]{\pgfxy(7.25,2.5)}{1.1mm}}
      \only<all:2>{\pgfnodecircle{N0111}[fill]{\pgfxy(8.25,2.5)}{1.1mm}}
      
      \pgfnodecircle{N00000}[fill]{\pgfxy(1,1)}{1.0mm}
      \pgfnodecircle{N00001}[fill]{\pgfxy(1.5,1)}{1.0mm}
      \pgfnodecircle{N00010}[fill]{\pgfxy(2,1)}{1.0mm}
      \pgfnodecircle{N00011}[fill]{\pgfxy(2.5,1)}{1.0mm}
      \pgfnodecircle{N00100}[fill]{\pgfxy(3,1)}{1.0mm}
      \pgfnodecircle{N00101}[fill]{\pgfxy(3.5,1)}{1.0mm}
      \pgfnodecircle{N00110}[fill]{\pgfxy(4,1)}{1.0mm}
      \pgfnodecircle{N00111}[fill]{\pgfxy(4.5,1)}{1.0mm}
      \pgfnodecircle{N01000}[fill]{\pgfxy(5,1)}{1.0mm}
      \pgfnodecircle{N01001}[fill]{\pgfxy(5.5,1)}{1.0mm}
      \pgfnodecircle{N01010}[fill]{\pgfxy(6,1)}{1.0mm}
      \pgfnodecircle{N01011}[fill]{\pgfxy(6.5,1)}{1.0mm}
      \pgfnodecircle{N01100}[fill]{\pgfxy(7,1)}{1.0mm}
      \pgfnodecircle{N01101}[fill]{\pgfxy(7.5,1)}{1.0mm}
      \pgfnodecircle{N01110}[fill]{\pgfxy(8,1)}{1.0mm}
      \pgfnodecircle{N01111}[fill]{\pgfxy(8.5,1)}{1.0mm}
      
      \only<all:2>{\pgfputat{\pgfxy(4.75,7)}{\pgfbox[center,center]{\Large OR}}}
      \only<all:2>{\pgfputat{\pgfxy(1.75,3.0)}{\pgfbox[center,center]{OR}}}
      \only<all:2>{\pgfputat{\pgfxy(3.75,3.0)}{\pgfbox[center,center]{OR}}}
      \only<all:2>{\pgfputat{\pgfxy(5.75,3.0)}{\pgfbox[center,center]{OR}}}
      \only<all:2>{\pgfputat{\pgfxy(7.75,3.0)}{\pgfbox[center,center]{OR}}}
      
      \pgfputat{\pgfxy(1.35,4.5)}{\pgfbox[left,center]{$s_1$}}
      \pgfputat{\pgfxy(3.35,4.5)}{\pgfbox[left,center]{$s_2$}}
      \pgfputat{\pgfxy(5.35,4.5)}{\pgfbox[left,center]{$s_3$}}
      \pgfputat{\pgfxy(7.35,4.5)}{\pgfbox[left,center]{$s_4$}}

      \pgfputat{\pgfxy(1,0.7)}{\pgfbox[center,center]{\small $s_5$}}
      \pgfputat{\pgfxy(1.5,0.7)}{\pgfbox[center,center]{\small $s_6$}}
      \pgfputat{\pgfxy(2,0.7)}{\pgfbox[center,center]{\small $s_7$}}
      \pgfputat{\pgfxy(2.5,0.7)}{\pgfbox[center,center]{\small $s_8$}}
      \pgfputat{\pgfxy(3,0.7)}{\pgfbox[center,center]{\small $s_9$}}
      \pgfputat{\pgfxy(3.5,0.7)}{\pgfbox[center,center]{\small $s_{10}$}}
      \pgfputat{\pgfxy(4,0.7)}{\pgfbox[center,center]{\small $s_{11}$}}
      \pgfputat{\pgfxy(4.5,0.7)}{\pgfbox[center,center]{\small $s_{12}$}}
      \pgfputat{\pgfxy(5,0.7)}{\pgfbox[center,center]{\small $s_{13}$}}
      \pgfputat{\pgfxy(5.5,0.7)}{\pgfbox[center,center]{\small $s_{14}$}}
      \pgfputat{\pgfxy(6,0.7)}{\pgfbox[center,center]{\small $s_{15}$}}
      \pgfputat{\pgfxy(6.5,0.7)}{\pgfbox[center,center]{\small $s_{16}$}}
      \pgfputat{\pgfxy(7,0.7)}{\pgfbox[center,center]{\small $s_{17}$}}
      \pgfputat{\pgfxy(7.5,0.7)}{\pgfbox[center,center]{\small $s_{18}$}}
      \pgfputat{\pgfxy(8,0.7)}{\pgfbox[center,center]{\small $s_{19}$}}
      \pgfputat{\pgfxy(8.5,0.7)}{\pgfbox[center,center]{\small $s_{20}$}}

      \only<all:2>{
        \pgfsetlinewidth{1pt}
        \pgfsetendarrow{\pgfarrowtriangle{6pt}}

        \color{red}
        \pgfnodeconnline{N0}{N00}
        \color{blue}
        \pgfnodeconnline{N0}{N01}
        \color{black}
        \pgfnodeconnline{N00}{N000}
        \pgfnodeconnline{N00}{N001}
        \pgfnodeconnline{N01}{N010}
        \pgfnodeconnline{N01}{N011}
        \color{red}
        \pgfnodeconnline{N000}{N0000}
        \color{blue}
        \pgfnodeconnline{N000}{N0001}
        \color{red}
        \pgfnodeconnline{N010}{N0100}
        \color{blue}
        \pgfnodeconnline{N010}{N0101}
        \color{red}
        \pgfnodeconnline{N001}{N0010}
        \color{blue}
        \pgfnodeconnline{N001}{N0011}
        \color{red}
        \pgfnodeconnline{N011}{N0110}
        \color{blue}
        \pgfnodeconnline{N011}{N0111}

        \pgfsetlinewidth{1pt}
        \pgfsetendarrow{\pgfarrowtriangle{6pt}}
        
        \color{black}
        \pgfnodeconnline{N0000}{N00000}
        \pgfnodeconnline{N0000}{N00001}
        \pgfnodeconnline{N0001}{N00010}
        \pgfnodeconnline{N0001}{N00011}
        \pgfnodeconnline{N0100}{N01000}
        \pgfnodeconnline{N0100}{N01001}
        \pgfnodeconnline{N0101}{N01010}
        \pgfnodeconnline{N0101}{N01011}
        \pgfnodeconnline{N0010}{N00100}
        \pgfnodeconnline{N0010}{N00101}
        \pgfnodeconnline{N0011}{N00110}
        \pgfnodeconnline{N0011}{N00111}
        \pgfnodeconnline{N0110}{N01100}
        \pgfnodeconnline{N0110}{N01101}
        \pgfnodeconnline{N0111}{N01110}
        \pgfnodeconnline{N0111}{N01111}
      }

      \only<all:1>{
        \pgfsetlinewidth{1pt}
        \pgfsetendarrow{\pgfarrowtriangle{6pt}}

        \color{red}
        \pgfnodeconnline{N0}{N000}
        \pgfnodeconnline{N0}{N001}
        \color{blue}
        \pgfnodeconnline{N0}{N010}
        \pgfnodeconnline{N0}{N011}
        \color{red}
        \pgfnodeconnline{N000}{N00000}
        \pgfnodeconnline{N000}{N00001}
        \color{blue}
        \pgfnodeconnline{N000}{N00010}
        \pgfnodeconnline{N000}{N00011}
        \color{red}
        \pgfnodeconnline{N010}{N01000}
        \pgfnodeconnline{N010}{N01001}
        \color{blue}
        \pgfnodeconnline{N010}{N01010}
        \pgfnodeconnline{N010}{N01011}
        \color{red}
        \pgfnodeconnline{N001}{N00100}
        \pgfnodeconnline{N001}{N00101}
        \color{blue}
        \pgfnodeconnline{N001}{N00110}
        \pgfnodeconnline{N001}{N00111}
        \color{red}
        \pgfnodeconnline{N011}{N01100}
        \pgfnodeconnline{N011}{N01101}
        \color{blue}
        \pgfnodeconnline{N011}{N01110}
        \pgfnodeconnline{N011}{N01111}
        
        \pgfsetlinewidth{1pt}
        \pgfsetendarrow{\pgfarrowtriangle{6pt}}
      }
    \end{pgfpicture}
  \end{center}
\end{frame}

\only<handout>{
  \againframe<handout:2>{and-or-search}
}

\subsection{Dynamic programming}

\begin{frame}
  \frametitle{Dynamic programming}
  
  \begin{block}{Planning by dynamic programming}
    If for all successors of state $s$ with respect to operator $o$ a
    plan exists, assign operator $o$ to $s$.

    \begin{description}[Inductive case $i \ge 1$:]
    \item[Base case $i=0$:]
      In goal states there is nothing to do.

    \item[Inductive case $i\ge 1$:]
      If there is $o\in O$ such that for all $s' \in \image{o}{s}$,
      the state $s'$ is a goal state or $\pi(s')$ was assigned in
      an earlier iteration, then assign $\pi(s)=o$.
    \end{description}
  \end{block}

  \begin{block}{Backward distances}
    If $s$ is assigned a value on iteration $i\ge 1$, then
    the \alert{backward distance} of $s$ is $i$.

    The dynamic programming algorithm essentially computes
    the \alert{backward distances} of states.
  \end{block}
\end{frame}

\begin{frame}
  \frametitle{Backward distances}

  \begin{example}
  \begin{center}
    \begin{pgfpicture}{-9.45cm}{0.8cm}{0.75cm}{6.7cm}
      \pgfsetxvec{\pgfpoint{2cm}{0cm}}
      \pgfsetlinewidth{0.7pt}
      \pgfputat{\pgfxy(-2.5,6.5)}{\pgfbox[center,center]{\large distance to $G$}}
      \pgfputat{\pgfxy(-4.5,6)}{\pgfbox[center,center]{$\infty$}}
      \pgfputat{\pgfxy(-3,6)}{\pgfbox[center,center]{3}}
      \pgfputat{\pgfxy(-2,6)}{\pgfbox[center,center]{2}}
      \pgfputat{\pgfxy(-1,6)}{\pgfbox[center,center]{1}}
      \pgfputat{\pgfxy(0,6)}{\pgfbox[center,center]{0}}
      \pgfputat{\pgfxy(0.3,5)}{\pgfbox[center,center]{$G$}}
      %% DASHED LINES
      \pgfsetdash{{3pt}{3pt}}{0pt}
      \pgfellipse[stroke]{\pgfxy(0,3)}{\pgfxy(0.3,0)}{\pgfxy(0,2)}
      %% NORMAL LINES
      \pgfsetdash{{3pt}{0pt}}{3pt}
      \pgfnodecircle{N01}[fill]{\pgfxy(0,2)}{0.1cm}
      \pgfnodecircle{N02}[fill]{\pgfxy(0,4)}{0.1cm}
      \pgfnodecircle{N11}[fill]{\pgfxy(-1,2)}{0.1cm}
      \pgfnodecircle{N12}[fill]{\pgfxy(-1,4)}{0.1cm}
      \pgfnodecircle{N21}[fill]{\pgfxy(-2,1)}{0.1cm}
      \pgfnodecircle{N22}[fill]{\pgfxy(-2,3)}{0.1cm}
      \pgfnodecircle{N23}[fill]{\pgfxy(-2,5)}{0.1cm}
      \pgfnodecircle{N31}[fill]{\pgfxy(-3,2)}{0.1cm}
      \pgfnodecircle{N32}[fill]{\pgfxy(-3,4)}{0.1cm}
      \pgfnodecircle{N41}[fill]{\pgfxy(-4.5,2)}{0.1cm}
      \pgfnodecircle{N42}[fill]{\pgfxy(-4.5,4)}{0.1cm}
      \pgfsetendarrow{\pgfarrowtriangle{6pt}}
      \color{blue}
      \pgfnodeconnline{N31}{N41}
      \pgfnodeconnline{N32}{N23}
      \pgfnodeconnline{N32}{N41}
      \pgfnodeconncurve{N42}{N42}{50}{130}{20mm}{20mm}
      \pgfnodeconnline{N41}{N42}
      \pgfnodeconnline{N42}{N32}
      \pgfnodeconnline{N23}{N22}
      \pgfnodeconnline{N32}{N31}
      \color{red}
      \pgfnodeconnline{N11}{N01}
      \pgfnodeconnline{N12}{N01}
      \pgfnodeconnline{N12}{N02}
      \pgfnodeconnline{N22}{N01}
      \pgfnodeconnline{N22}{N11}
      \pgfnodeconnline{N22}{N12}
      \pgfnodeconnline{N23}{N12}
      \pgfnodeconnline{N21}{N11}
      \pgfnodeconnline{N32}{N22}
      \pgfnodeconnline{N31}{N21}
      \pgfnodeconnline{N31}{N11}
    \end{pgfpicture}
  \end{center}
  \end{example}
\end{frame}

\subsection{Backward distances}

\begin{frame}
  \frametitle{Backward distances}

  \begin{definition}[backward distance sets]
    Let $G$ be a set of states and $O$ a set of operators.
    
    The \alert{backward distance sets} $D^{\ms{bwd}}_i$ for $G$ and
    $O$ consist of those states for which there is a guarantee of
    reaching a state in $G$ with at most $i$ operator applications
    using operators in $O$:
    \begin{align*}
      D^{\ms{bwd}}_0 & := G \\
      D^{\ms{bwd}}_i & := D^{\ms{bwd}}_{i-1} \cup
      \bigcup_{o \in O} \spreimage{o}{D^{\ms{bwd}}_{i - 1}}
      \text{~for all~} i \ge 1
    \end{align*}
  \end{definition}
\end{frame}

\begin{frame}
  \frametitle{Backward distances}

  \begin{definition}[backward distance]
    Let $G$ be a set of states and $O$ a set of operators, and
    let $D^{\ms{bwd}}_0, D^{\ms{bwd}}_1, \dots$ be the backward
    distance sets for $G$ and $O$. Then the \alert{backward distance}
    of a state $s$ for $G$ and $O$ is
    \[
    \disbwd{G}{s} = \min \{ i \in \mathbb N \,|\, s \in D^{\ms{bwd}}_i \}
    \]
    (where $\min\emptyset = \infty$).
  \end{definition}
\end{frame}

\begin{frame}
  \frametitle{Strong plans based on distances}

  Let $\Pi = \langle V, I, O, \gamma\rangle$ be a nondeterministic
  planning task with state set $S$ and goal states $S_\star$.

  \begin{block}{Extraction of a strong plan from distance sets}
    \begin{enumerate}
    \item Let $S' \subseteq S$ be those states having a finite backward
      distance for $\gamma$ and $O$.
    \item Let $s \in S'$ be a state with distance $i = \disbwd{\gamma}{s} \ge 1$.
    \item Assign to $\pi(s)$ any operator $o \in O$ such that
      $\image{o}{s} \subseteq D^{\ms{bwd}}_{i-1}$. 
      Hence $o$ decreases the backward distance by at least one.
    \end{enumerate}
    Then $\pi$ is a strong plan for $\mathcal T$ iff
    $I \in S'$.
  \end{block}

  \textblue{Question:} What is the \alert{worst-case} runtime of the
  algorithm?

  \textblue{Question:} What is the \alert{best-case} runtime of the
  algorithm \\
  \textbg{Question:} if most states have a finite backward distance?
\end{frame}

\section{Efficient algorithm}

\subsection{Main algorithm}

\begin{frame}
  \frametitle{Making the algorithm a logic-based algorithm}

  \begin{itemize}
  \item An algorithm that represents the states \alert{explicitly}
    stops being feasible at about $10^8$ or $10^9$ states.
  \item For planning with bigger transition systems \alert{structural
      properties} of the transition system have to be taken advantage
    of.
  \item As before, representing state sets as \alert{propositional
      formulae} (or \alert{BDDs}) often allows taking advantage of the
    structural properties: a formula (or BDD) that represents a set of
    states or a transition relation that has certain regularities may
    be very small in comparison to the set or relation.
  \item In the following, we will present an algorithm using a
    boolean-formula representation (without going into the details of
    how to implement it using BDDs).
  \end{itemize}
\end{frame}

\begin{frame}
  \frametitle{Making the algorithm a logic-based algorithm}

  \structure{Remark:} The following algorithm assumes a propositional
  representation of the state space as opposed to a finite-domain
  representation. We have already seen how to translate an FDR
  encoding into a propositional encoding in Chapter~9 (cf.\ definition
  of the ``induced propositional planning task'').

  \smallskip

  Therefore, for the rest of the present chapter, we will assume
  without loss of generality that all $v \in V$ are propositional
  variables with domain $\mathcal D_v = \{ 0, 1 \}$.
\end{frame}

\begin{frame}[t]
  \frametitle{Breadth-first search with progression and
    state sets (deterministic case)}

  \begin{block}{Progression breadth-first search}
    \textbf{def} bfs-progression($V$, $I$, $O$, $\gamma$): \\
    {}\qquad$\textit{goal} := \textit{formula-to-set}(\gamma)$ \\
    {}\qquad$\textit{reached} := \{I\}$ \\
    {}\qquad\textbf{loop}: \\
    {}\qquad\qquad\textbf{if} $\textit{reached} \cap \textit{goal}
    \neq \emptyset$: \\
    {}\qquad\qquad\qquad\textbf{return} solution found \\
    {}\qquad\qquad$\textit{new-reached} := \textit{reached} \cup
    \bigcup_{o \in O} \image{o}{\textit{reached}}$ \\
    {}\qquad\qquad\textbf{if} $\textit{new-reached} = 
    \textit{reached}$: \\
    {}\qquad\qquad\qquad\textbf{return} no solution exists \\
    {}\qquad\qquad$\textit{reached} := \textit{new-reached}$
  \end{block}
  
  $\leadsto$ This can easily be transformed into a \alert{regression}
  algorithm.
\end{frame}

\begin{frame}[t]
  \frametitle{Breadth-first search with regression and
    state sets (deterministic case)}

  \begin{block}{Regression breadth-first search}
    \textbf{def} bfs-regression($V$, $I$, $O$, $\gamma$): \\
    {}\qquad$\textit{init} := I$ \\
    {}\qquad$\textit{reached} := \textit{formula-to-set}(\gamma)$ \\
    {}\qquad\textbf{loop}: \\
    {}\qquad\qquad\textbf{if} $\textit{init} \in
    \textit{reached}$: \\
    {}\qquad\qquad\qquad\textbf{return} solution found \\
    {}\qquad\qquad$\textit{new-reached} := \textit{reached} \cup
    \bigcup_{o \in O} \wpreimage{o}{\textit{reached}}$ \\
    {}\qquad\qquad\textbf{if} $\textit{new-reached} = 
    \textit{reached}$: \\
    {}\qquad\qquad\qquad\textbf{return} no solution exists \\
    {}\qquad\qquad$\textit{reached} := \textit{new-reached}$
  \end{block}
  
  \begin{itemize}
  \item This algorithm is very similar to the dynamic programming
    algorithm for the nondeterministic case!
  \end{itemize}
\end{frame}

\begin{frame}[t]
  \frametitle{Breadth-first search with regression and
    state sets (nondeterministic case)}

  \begin{block}{Regression breadth-first search}
    \textbf{def} bfs-regression($V$, $I$, $O$, $\gamma$): \\
    {}\qquad$\textit{init} := I$ \\
    {}\qquad$\textit{reached} := \textit{formula-to-set}(\gamma)$ \\
    {}\qquad\textbf{loop}: \\
    {}\qquad\qquad\textbf{if} $\textit{init} \in \textit{reached}$: \\
    {}\qquad\qquad\qquad\textbf{return} solution found \\
    {}\qquad\qquad$\textit{new-reached} := \textit{reached} \cup
    \alert{\bigcup_{o \in O} \spreimage{o}{\textit{reached}}}$ \\
    {}\qquad\qquad\textbf{if} $\textit{new-reached} = 
    \textit{reached}$: \\
    {}\qquad\qquad\qquad\textbf{return} no solution exists \\
    {}\qquad\qquad$\textit{reached} := \textit{new-reached}$
  \end{block}

  \begin{itemize}
  \item How do we define \alert{\textit{spreimg}} with set-theoretic
    (BDD) operations?
  \end{itemize}
\end{frame}





































\begin{frame}
  \frametitle{Computing strong preimages}

  \begin{block}{Strong preimages}
    \begin{align*}
      \spreimage{o}{T}
      & = \{s \in S~|~\exists s' \in T: s \xrightarrow{o} s' \land \image{o}{s} \subseteq T\} \\
      & = \{s \in S~|~(\exists s' \in S:
      s \xrightarrow{o} s' \land s' \in T) \land {} \\
      & \phantom{= \{s' \in S~|~}
      \{s' \in S~|~s \xrightarrow{o} s'\} \subseteq T\} \\
      & = \{s \in S~|~(\exists s' \in S:
      s \xrightarrow{o} s' \land s' \in T) \land {} \\
      & \phantom{= \{s' \in S~|~}
      (\forall s' \in S: s \xrightarrow{o} s' \rightarrow (s' \in T)) \} \\
      & = \{s \in S~|~(\exists s' \in S:
      s \xrightarrow{o} s' \land s' \in T) \land {} \\
      & \phantom{= \{s' \in S~|~}
      (\neg \exists s' \in S: s \xrightarrow{o} s' \land s' \notin T) \}
    \end{align*}
  \end{block}
\end{frame}

\begin{frame}
  \frametitle{Computing strong preimages with boolean function operations}

  \begin{align*}
    \spreimage{o}{T} & = \{s \in S~|~(\exists s' \in S:
    s \xrightarrow{o} s' \land s' \in T) \land {} \\
    & \phantom{= \{s' \in S~|~}
    (\neg \exists s' \in S: s \xrightarrow{o} s' \land s' \notin T) \}
  \end{align*}

  \begin{block}{Strong preimages with boolean functions}
  DEFINE SUBSTITUTION

  DEFINE EXISTENTIAL ABSTRACTION (ONE VARIABLE)

  DEFINE EXISTENTIAL ABSTRACTION (SET OF VARIABLES)

  DEFINE REGRESSION/SPREIMG FORMULA
  \end{block}



  Are we done?
  
  No, because we have not yet shown how to compute $\tau_A(o)$
  for \alert{nondeterministic} operators.
\end{frame}


\subsection[Transitions]{Representing operator transitions}

\begin{frame}
  \frametitle{Transition formula for nondeterministic operators}

  The formula $\opcpc{A}{o}$ must express
  \begin{itemize}
  \item the conditions for applicability of $o$,
  \item how $o$ \alert{changes} state variables, and
  \item which state variables $o$ \alert{does not change}.
  \end{itemize}
  
  A significant difficulty lies in the third requirement because
  \alert{different variables} may be affected depending on
  nondeterministic choices.
\end{frame}


\begin{frame}
  \frametitle{Transition formula for nondeterministic operators}

  \begin{block}{$\opcpc{A}{o}$ for deterministic operators
      $o = \langle c, e\rangle$}
    \begin{align*}
      \opcpc{A}{o} = c \land{}
      & \bigwedge_{a \in A} ((\eprecon{a}{e} \lor
      (a \land \neg \eprecon{\neg a}{e})) \leftrightarrow a') \\
      \land & \bigwedge_{a \in A} \neg
      (\eprecon{a}{e} \land \eprecon {\neg a}{e})
    \end{align*}
  \end{block}

  For $o = \langle c, e_1 \choice \dots \choice e_n\rangle$
  where each $e_i$ is deterministic:
  \begin{align*}
    \opcpc{A}{o} = c \land{}
    & \bigvee_{i=1}^n \bigwedge_{a \in A} ((\eprecon{a}{e_i} \lor
    (a \land \neg \eprecon{\neg a}{e_i})) \leftrightarrow a') \\
    \land & \bigwedge_{i=1}^n \bigwedge_{a \in A} \neg
    (\eprecon{a}{e_i} \land \eprecon {\neg a}{e_i})
  \end{align*}
\end{frame}


\section{Summary}

\begin{frame}
  \frametitle{Summary}

  \begin{itemize}
  \item We have considered the special case of nondeterministic planning
    where
    \begin{itemize}
    \item planning tasks are \alert{fully observable} and
    \item we are interested in \alert{strong plans}.
    \end{itemize}
  \item We have introduced important concepts also relevant
    to other variants of nondeterministic planning such as
    \begin{itemize}
    \item \alert{images} and
    \item \alert{weak and strong preimages}.
    \end{itemize}
  \item We have discussed some basic classes of algorithms:
    \begin{itemize}
    \item \alert{forward search} in AND/OR graphs, and
    \item \alert{backward induction} by dynamic programming.
    \end{itemize}
  \item Finally, we have shown how to make a dynamic programming
    algorithm more efficient by exploiting logic- or set-based
    representations such as BDDs.
  \end{itemize}
\end{frame}

\end{document}
