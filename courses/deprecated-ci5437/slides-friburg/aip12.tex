\documentclass{gkibeamer}

\usepackage{tikz}[2008/02/13]

\title[AI Planning]{Principles of AI Planning}
\author[B.~Nebel,\\R.~Mattm\"uller]{Bernhard Nebel and Robert Mattm\"uller}
\date[]{Winter Semester 2011/2012}

%% Use \alert instead of \textred where it makes sense.
%% (\textred makes sense when you *really* want the color to be red,
%% independent of the beamer style settings.)
\newcommand{\hilite}[1]{{\color{structure} #1}}
\newcommand{\textred}[1]{{\color{red} #1}}
\newcommand{\textblue}[1]{{\color{blue} #1}}
\newcommand{\textgreen}[1]{{\color{green} #1}}
\newcommand{\textbrown}[1]{{\color{brown} #1}}
\newcommand{\textviolet}[1]{{\color{violet} #1}}
\newcommand{\textbg}[1]{{\color{bg} #1}}

\newcommand{\eg}{e.\,g.}
\newcommand{\ie}{i.\,e.}

%% State variables and values for state variables
\newcommand{\var}[1]{\textit{#1}}
\newcommand{\val}[1]{\text{#1}}

%% Various stuff
\newcommand{\astar}{A${}^*$}
\newcommand{\idastar}{IDA${}^*$}

%% Blocksworld stuff
\newcommand{\AONB}{\var{A-on-B}}
\newcommand{\AONC}{\var{A-on-C}}
\newcommand{\AOND}{\var{A-on-D}}
\newcommand{\BONA}{\var{B-on-A}}
\newcommand{\BONC}{\var{B-on-C}}
\newcommand{\BOND}{\var{B-on-D}}
\newcommand{\CONA}{\var{C-on-A}}
\newcommand{\CONB}{\var{C-on-B}}
\newcommand{\COND}{\var{C-on-D}}
\newcommand{\DONA}{\var{D-on-A}}
\newcommand{\DONB}{\var{D-on-B}}
\newcommand{\DONC}{\var{D-on-C}}
\newcommand{\AONTABLE}{\var{A-on-T}}
\newcommand{\BONTABLE}{\var{B-on-T}}
\newcommand{\CONTABLE}{\var{C-on-T}}
\newcommand{\CLEARA}{\var{A-clear}}
\newcommand{\CLEARB}{\var{B-clear}}
\newcommand{\CLEARC}{\var{C-clear}}

\newcommand{\applyop}[2]{\textit{app}_{#1}(#2)}
\newcommand{\applyops}[2]{\textit{app}_{#1}(#2)}
\newcommand{\applyplan}[2]{\textit{app}_{#1}(#2)}
\newcommand{\regr}[2]{\textit{regr}_{#1}(#2)}
\newcommand{\regrstrips}[2]{\textit{sregr}_{#1}(#2)}
\newcommand{\eprecon}[2]{\textit{EPC}_{#1}(#2)}

\newcommand{\compl}[1]{\overline{#1}}

\newcommand{\changes}[2]{\lbrack #1\rbrack_{#2}}
\newcommand{\CEF}{\vartriangleright}

\newcommand{\relaxation}[1]{{#1}^+}
\newcommand{\onset}[1]{\textit{on}(#1)}
\newcommand{\hplus}{\ensuremath{h^+}}
\newcommand{\hmax}{\ensuremath{h_{\text{max}}}}
\newcommand{\hadd}{\ensuremath{h_{\text{add}}}}
\newcommand{\hlmcut}{\ensuremath{h_{\text{LM-cut}}}}
\newcommand{\hff}{\ensuremath{h_{\text{FF}}}}
\newcommand{\hcs}{\ensuremath{h_{\text{cs}}}}
\newcommand{\hsa}{\ensuremath{h_{\text{sa}}}}
\newcommand{\hhhh}{\ensuremath{h_{\text{HHH}}}}

\newcommand{\sasplus}{SAS${}^+$}

\newcommand{\graphequiv}{\stackrel{\textup{G}}{\sim}}

\newcommand{\cg}{\ensuremath{\textit{CG}}}

%% Turing machines and complexity theory
\newcommand{\accept}{{\textsf{Y}}}

\newcommand{\easier}{\ensuremath{\le_{\text{p}}}}

\newcommand{\decisionclass}[1]{\ensuremath{\textsf{\textup{#1}}}}
\newcommand{\dtime}{\decisionclass{DTIME}}
\newcommand{\ntime}{\decisionclass{NTIME}}
\newcommand{\dspace}{\decisionclass{DSPACE}}
\newcommand{\nspace}{\decisionclass{NSPACE}}

\newcommand{\ptime}{\decisionclass{P}}
\newcommand{\np}{\decisionclass{NP}}
\newcommand{\pspace}{\decisionclass{PSPACE}}
\newcommand{\npspace}{\decisionclass{NPSPACE}}
\newcommand{\exptime}{\decisionclass{EXP}}
\newcommand{\nexptime}{\decisionclass{NEXP}}
\newcommand{\expspace}{\decisionclass{EXPSPACE}}
\newcommand{\dblexptime}{\decisionclass{2EXP}}

\newcommand{\planex}{\textsc{PlanEx}}
\newcommand{\planlen}{\textsc{PlanLen}}

\newcommand{\ms}[1]{\mathchoice{\mbox{\normalsize\it #1}}{\mbox{\normalsize\it #1}}{\mbox{\scriptsize\it #1}}{\mbox{\tiny\it #1}}}
\newcommand{\image}[2]{\ms{img}_{#1}(#2)}
\newcommand{\spreimage}[2]{\ms{spreimg}_{#1}(#2)}
\newcommand{\wpreimage}[2]{\ms{wpreimg}_{#1}(#2)}
\newcommand{\disbwd}[2]{\delta^{\ms{bwd}}_{#1}(#2)}
\newcommand{\ats}[1]{\ms{scope}(#1)}
\newcommand{\ndopcpc}[2]{\tau^{\ms{nd}}_{#1}(#2)}
\newcommand{\choice}{\mathop{|}}
\newcommand{\opcpc}[2]{\tau_{#1}(#2)}

%% Don't use \begin{proof}/\end{proof} for multi-part proofs because
%% this places a QED symbol.
\newenvironment{proofstart}{\begin{block}{Proof.}}{\end{block}}
\newenvironment{proofmid}{\begin{block}{Proof (ctd.)}}{\end{block}}
\newenvironment{proofend}{\begin{proof}[Proof (ctd.)]}{\end{proof}}

\newtheorem{remark}[theorem]{Remark}

%% Macros to align the width of things.
\newlength{\mywidth}
\newcommand{\setmywidth}[1]{\settowidth{\mywidth}{#1}}
\newcommand{\usemywidth}[1]{\makebox[\mywidth][l]{#1}}
\newcommand{\usemywidthmath}[1]{\usemywidth{\ensuremath{#1}}}

%% A tighter "align" environment.
\newenvironment{tightalign}[1][c]{\par\(\begin{array}[#1]{@{}r@{}l}}
  {\end{array}\)\par}
\newenvironment{wrappedmath}[1][t]{\begin{array}[#1]{@{}l}}{\end{array}}

%% Nondeterministic transition systems with connector arcs
\newcommand{\arc}[4]{
  \begin{pgfscope}
    \pgfsetlinewidth{1pt}
    \pgfpathmoveto{\pgfpointanchor{#3}{center}}
    \pgfpathlineto{\pgfpointanchor{#1}{#2}}
    \pgfpathlineto{\pgfpointanchor{#4}{center}}
    \pgfusepath{clip}
    \pgfpathcircle{\pgfpointanchor{#1}{#2}}{1.8mm}
    \pgfusepath{draw}
  \end{pgfscope}
}


\tikzstyle{vertex interior}=[fill,minimum size=0.80cm,circle,
    inner sep=0pt,outer sep=0.2pt,color=red!20]
\tikzstyle{vertex border}=[draw,thin,minimum size=0.80cm,circle,
    inner sep=0pt,outer sep=0.2pt,color=black]
\tikzstyle{vertex hidden}=[draw,minimum size=0.80cm,circle,
    inner sep=0pt,outer sep=0.2pt, color=gray, text=gray]
\tikzstyle{subvertex}=[color=gray]

\tikzstyle{bigvertex interior}=[fill=red!20]
\tikzstyle{bigvertex border}=[]

\tikzstyle{transition}=[semithick]

\newcommand{\vertex}[3][]{
    \node[vertex interior] at (#2) (#3) {};
    \node[vertex border,#1] at (#3) {\small #3};
}

\newcommand{\numberedvertex}[4][]{
    \node[vertex interior] at (#2) (#3) {};
    \node[vertex border,#1] at (#3) {\small #4};
}

\newcommand{\vertexnolabel}[3][]{
    \node[vertex interior] at (#2) (#3) {};
    \node[vertex border,#1] at (#3) {};
}

\newcommand{\wildcard}{\ensuremath{\star}}

% Usage: \if\blank{#1}...\else...\fi
\catcode`\@=11 % as in plain.tex
\long\def\blank#1{\bl@nk#1@@..\bl@nk}%
\long\def\bl@nk#1#2@#3#4\bl@nk{#3#4}
\catcode`\@=12

\long\def\test#1{\begingroup \toks0{[#1]}%
    \newlinechar`\/\message{/\the\toks0:
        \if\blank{#1}EMPTY\else NOT empty\fi%
    }\endgroup}

\newcommand{\hiddenvertex}[1]{
    \if \blank{#1}
    \else
        \node[vertex hidden] at (#1) {#1};
    \fi
}

\newcommand{\hiddenvertices}[4]{
    \hiddenvertex{#1}
    \hiddenvertex{#2}
    \hiddenvertex{#3}
    \hiddenvertex{#4}
}

\newcommand{\verticalvertexborder}[3][]{
    \draw[bigvertex border, #1] (#2.west) arc(180:0:2.9mm) -- 
        (#3.east) arc(0:-180:2.9mm) -- (#2.west);
}

\newcommand{\verticalvertex}[5][]{
    \draw[bigvertex interior, #1] (#2.west) arc(180:0:2.9mm) -- 
        (#3.east) arc(0:-180:2.9mm) -- (#2.west);
	\hiddenvertices{#2}{#3}{#4}{#5}
	\verticalvertexborder{#2}{#3}
}

\newcommand{\verticalgoalvertex}{\verticalvertex[color=green!40]}

\newcommand{\horizontalvertexborder}[3][]{
    \draw[bigvertex border, #1] (#2.south) arc(-90:-270:2.9mm) -- 
        (#3.north) arc(90:-90:2.9mm) -- (#2.south);
}

\newcommand{\horizontalvertex}[3][]{
    \draw[bigvertex interior, #1] (#2.south) arc(-90:-270:2.9mm) -- 
        (#3.north) arc(90:-90:2.9mm) -- (#2.south);
	\hiddenvertices{#2}{#3}{}{}
	\horizontalvertexborder{#2}{#3}
}

\newcommand{\trianglevertexleftborder}[4][]{
    \draw[bigvertex border,#1] (#2.south west) arc(-135:-225:2.9mm) -- 
        (#3.north west) arc(135:0:2.9mm) -- 
        (#4.east) arc(0:-135:2.9mm) -- 
        (#2.south west);
}

\newcommand{\trianglevertexleft}[5][]{    
    \draw[bigvertex interior, #1] (#2.south west) arc(-135:-225:2.9mm) -- 
        (#3.north west) arc(135:0:2.9mm) -- 
        (#4.east) arc(0:-135:2.9mm) -- 
        (#2.south west); 
	\hiddenvertices{#2}{#3}{#4}{#5}
	\trianglevertexleftborder{#2}{#3}{#4}
}

\newcommand{\trianglevertexrightborder}[4][]{
    \draw[bigvertex border, #1] (#2.south east) arc(-45:45:2.9mm) -- 
        (#3.north east) arc(45:180:2.9mm) -- 
        (#4.west) arc(180:315:2.9mm) -- 
        (#2.south east);
}

\newcommand{\trianglevertexright}[5][]{   
    \draw[bigvertex interior, #1] (#2.south east) arc(-45:45:2.9mm) -- 
        (#3.north east) arc(45:180:2.9mm) -- 
        (#4.west) arc(180:315:2.9mm) -- 
        (#2.south east); 
	\hiddenvertices{#2}{#3}{#4}{#5}
	\trianglevertexrightborder{#2}{#3}{#4}
}

\newcommand{\squarevertexborder}[5][]{
    \draw[bigvertex border] (#2.west) arc(180:90:2.9mm) -- 
        (#3.north) arc(90:0:2.9mm) -- 
        (#4.east) arc(0:-90:2.9mm) -- 
        (#5.south) arc(-90:-180:2.9mm) -- 
        (#2.west);
}

\newcommand{\squarevertex}[5][]{    
    \draw[bigvertex interior, #1] (#2.west) arc(180:90:2.9mm) -- 
        (#3.north) arc(90:0:2.9mm) -- 
        (#4.east) arc(0:-90:2.9mm) -- 
        (#5.south) arc(-90:-180:2.9mm) -- 
        (#2.west); 
	\hiddenvertices{#2}{#3}{#4}{#5}
	\squarevertexborder{#2}{#3}{#4}{#5}
}

\newcommand{\gravisvertexborder}[3][]{
    \draw[bigvertex border, #1] (#2.south west) arc(225:45:2.9mm) -- 
        (#3.north east) arc(45:-135:2.9mm) -- 
        (#2.south west);
}

\newcommand{\gravisvertex}[3][]{    
    \draw[bigvertex interior,#1] (#2.south west) arc(225:45:2.9mm) -- 
        (#3.north east) arc(45:-135:2.9mm) -- 
        (#2.south west); 
	\hiddenvertices{#2}{#3}{}{}
	\gravisvertexborder{#2}{#3}
}

\newcommand{\akutvertexborder}[3][]{
    \draw[bigvertex border, #1] (#2.north west) arc(135:-45:2.9mm) -- 
        (#3.south east) arc(-45:-225:2.9mm) -- 
        (#2.north west);
}

\newcommand{\akutvertex}[3][]{    
    \draw[bigvertex interior,#1] (#2.north west) arc(135:-45:2.9mm) -- 
        (#3.south east) arc(-45:-225:2.9mm) -- 
        (#2.north west); 
	\hiddenvertices{#2}{#3}{}{}
	\akutvertexborder{#2}{#3}
}

\newcommand{\goalvertex}[3][]{
    \node[vertex interior,color=green!40] at (#2) (#3) {};
    \node[vertex border,#1] at (#3) {\small #3};
}

\newcommand{\numberedgoalvertex}[4][]{
    \node[vertex interior,color=green!40] at (#2) (#3) {};
    \node[vertex border,#1] at (#3) {\small #4};
}

\newcommand{\goalvertexnolabel}[3][]{
    \node[vertex interior,color=green!40] at (#2) (#3) {};
    \node[vertex border,#1] at (#3) {};
}

\newcommand{\transition}[5][]{\draw[<->,>=stealth,transition,#1] (#2)--(#3);}

\newcommand{\circletransition}{140:1.25mm}
\newcommand{\selftransition}[2][0:0mm]{
    \draw[->,>=stealth,transition] (#2) +(#1) arc(-45:225:1.5mm);}

\newcommand{\picfulltransitiongraphbase}[2]{%
  \begin{tikzpicture}[scale=1.4]
    #1{0,0}{LRR}
    %
    \draw[transition,->,>=stealth](-0.6,0) -- (LRR);
    %
    #1{1,0}{LLL}
    #1{1,1}{LLR}
    #1{1,-1}{LRL}
    %
    #1{2,1.5}{ALR}
    #1{2,0.5}{ALL}
    #1{2,-0.5}{BLL}
    #1{2,-1.5}{BRL}
    %
    #1{3,1.5}{ARL}
    #1{3,0.5}{ARR}
    #1{3,-0.5}{BRR}
    #1{3,-1.5}{BLR}
    %
    #2{4,0}{RRR}
    #2{4,1}{RRL}
    #2{4,-1}{RLR}
    %
    #2{5,0}{RLL}
    %
    \transition{LLL}{LRL}{move(A,R)}{move(A,L)}
    \transition{LLL}{LLR}{move(B,R)}{move(B,L)}
    \transition{LRL}{LRR}{move(B,R)}{move(B,L)}
    \transition{LLR}{LRR}{move(A,R)}{move(A,L)}
    %
    \transition{LLL}{ALL}{pick(A,L)}{drop(A,L)}
    \transition{LLL}{BLL}{pick(B,L)}{drop(B,L)}
    \transition{LLR}{ALR}{pick(A,L)}{drop(A,L)}
    \transition{LRL}{BRL}{pick(B,L)}{drop(B,L)}
    \transition{ALL}{ALR}{move(A,R)}{move(A,L)}
    \transition{BLL}{BRL}{move(B,R)}{move(B,L)}
    %
    \transition{ALR}{ARR}{move(A,R)}{move(A,L)}
    \transition{ALL}{ARL}{move(A,R)}{move(A,L)}
    \transition{BRL}{BRR}{move(B,R)}{move(B,L)}
    \transition{BLL}{BLR}{move(B,R)}{move(B,L)}
    %
    \transition{RRR}{ARR}{pick(A,R)}{drop(A,R)}
    \transition{RRR}{BRR}{pick(B,R)}{drop(B,R)}
    \transition{RRL}{ARL}{pick(A,R)}{drop(A,R)}
    \transition{RLR}{BLR}{pick(B,R)}{drop(B,R)}
    \transition{ARR}{ARL}{move(A,L)}{move(A,R)}
    \transition{BRR}{BLR}{move(B,L)}{move(B,R)}
    %
    \transition{RRR}{RLR}{move(A,L)}{move(A,R)}
    \transition{RRR}{RRL}{move(B,L)}{move(B,R)}
    \transition{RLR}{RLL}{move(B,L)}{move(B,R)}
    \transition{RRL}{RLL}{move(A,L)}{move(A,R)}
  \end{tikzpicture}%
}

\newcommand{\picfulltransitiongraph}{%
  \picfulltransitiongraphbase{\vertex}{\goalvertex}}

\newcommand{\picfulltransitiongraphnolabels}{%
  \picfulltransitiongraphbase{\vertexnolabel}{\goalvertexnolabel}}

\newcommand{\picperfectabstraction}{%
  \begin{tikzpicture}[scale=1.4]
    \vertex[text=gray]{0,0}{LRR}
    %
    \draw[transition,->,>=stealth](-0.6,0) -- (LRR);
    %
    \vertex{1,1}{LLR}
    \vertex{1,0}{LLL}
    \vertex{1,-1}{LRL}
    \verticalvertex{LLR}{LRL}{LLL}{}
    %
    \vertex{2,1.5}{ALR}
    \vertex{2,0.5}{ALL}
    \vertex{2,-0.5}{BLL}
    \vertex{2,-1.5}{BRL}
    \verticalvertex{ALR}{BRL}{ALL}{BLL}
    %
    \vertex{3,1.5}{ARL}
    \vertex{3,0.5}{ARR}
    \vertex{3,-0.5}{BRR}
    \vertex{3,-1.5}{BLR}
    \verticalvertex{ARL}{BLR}{ARR}{BRR}
    %
    %
    \vertex{4,0}{RRR}
    \vertex{4,1}{RRL}
    \vertex{4,-1}{RLR}
    \vertex{5,0}{RLL}
    \trianglevertexright[fill=green!40]{RLL}{RRL}{RLR}{RRR}
    %
    \transition{LLL}{LRR}{}{}
    %
    \transition{LLL}{LLL-|ALL.west}{}{}
    %
    \transition{LLL-|ALL.east}{RRR-|ARL.west}{}{}
    %
    \transition{RRR}{RRR-|ARL.east}{}{}
    %
    \selftransition[\circletransition]{LLR.north east}
    \selftransition[\circletransition]{ALR.north east}
    \selftransition[\circletransition]{ARL.north east}
    \selftransition[\circletransition]{RRL.north east}
  \end{tikzpicture}%
}

\newcommand{\piconestateabstraction}{%
  \begin{tikzpicture}[scale=1.4]
    \vertex[text=gray]{0,0}{LRR}
    %
    \draw[transition,->,>=stealth](-0.6,0) -- (LRR);
    %
    \vertex{1,1}{LLR}
    \vertex{1,0}{LLL}
    \vertex{1,-1}{LRL}
    \vertex{2,1.5}{ALR}
    \vertex{2,0.5}{ALL}
    \vertex{2,-0.5}{BLL}
    \vertex{2,-1.5}{BRL}
    \vertex{3,1.5}{ARL}
    \vertex{3,0.5}{ARR}
    \vertex{3,-0.5}{BRR}
    \vertex{3,-1.5}{BLR}
    \vertex{4,0}{RRR}
    \vertex{4,1}{RRL}
    \vertex{4,-1}{RLR}
    \vertex{5,0}{RLL}
	
    \draw[bigvertex interior,color=green!40] (LRR.230) arc(-130:-230:2.9mm)
    -- (LLR.140) arc(-220:-240:2.9mm)
    -- (ALR.120) arc(-240:-270:2.9mm) -- (ARL.north) arc(90:60:2.9mm)
    -- (RRL.60) arc (-300:-320:2.9mm) -- (RLL.50) arc(50:-50:2.9mm)
	-- (RLR.320) arc (-40:-60:2.9mm) -- (BLR.300) arc (-60:-90:2.9mm)
	-- (BRL.south) arc (-90:-120:2.9mm) -- (LRL.240) arc (-120:-140:2.9mm)
	-- (LRR.230);
	\hiddenvertices{LRR}{LLR}{LLL}{LRL}
	\hiddenvertices{ALR}{ALL}{BLL}{BRL}
	\hiddenvertices{ARL}{ARR}{BRR}{BLR}
	\hiddenvertices{RRR}{RRL}{RLR}{RLL}
    \draw[bigvertex border] (LRR.230) arc(-130:-230:2.9mm)
    -- (LLR.140) arc(-220:-240:2.9mm)
    -- (ALR.120) arc(-240:-270:2.9mm) -- (ARL.north) arc(90:60:2.9mm)
    -- (RRL.60) arc (-300:-320:2.9mm) -- (RLL.50) arc(50:-50:2.9mm)
	-- (RLR.320) arc (-40:-60:2.9mm) -- (BLR.300) arc (-60:-90:2.9mm)
	-- (BRL.south) arc (-90:-120:2.9mm) -- (LRL.240) arc (-120:-140:2.9mm)
	-- (LRR.230);
	
    \selftransition[\circletransition]{ARL.north east}
  \end{tikzpicture}%
}

\newcommand{\picexampleabstraction}[1][\verticalvertex]{%
  \begin{tikzpicture}[scale=1.4]
    \vertex[text=gray]{0,0}{LRR}
    %
    \draw[transition,->,>=stealth](-0.6,0) -- (LRR);
    %
    \vertex{1,1}{LLR}
    \vertex{1,0}{LLL}
    \vertex{1,-1}{LRL}
    #1{LLR}{LRL}{LLL}{}
    %
    \vertex{2,1.5}{ALR}
    \vertex{3,1.5}{ARL}
    \vertex{2,0.5}{ALL}
    \vertex{3,0.5}{ARR}
    \vertex{2,-0.5}{BLL}
    \vertex{2,-1.5}{BRL}
    \vertex{3,-0.5}{BRR}
    \vertex{3,-1.5}{BLR}
    \squarevertex{ALR}{ARL}{BLR}{BRL}
	\hiddenvertices{ALL}{ARR}{BLL}{BRR}
    %
    \vertex{4,0}{RRR}
    \vertex{4,1}{RRL}
    \vertex{4,-1}{RLR}
    \vertex{5,0}{RLL}
    \trianglevertexright[fill=green!40]{RLL}{RRL}{RLR}{RRR}
    %
    \draw[transition,->,>=stealth](-0.6,0) -- (LRR);
    %
    \transition{LRR}{LLL}{}{}
    \transition{LLL}{LLL-|ALL.west}{}{}
    \transition{RRR}{RRR-|ARL.east}{}{}
    %
    \selftransition[\circletransition]{LLR.north east}
    \selftransition[\circletransition]{RRL.north east}
    \selftransition[0:6mm]{ALR.north}
  \end{tikzpicture}
}

\newcommand{\picprojectionpackage}{%
  \begin{tikzpicture}[scale=1.4]
    \vertex{0,0}{LRR}
    \vertex{1,0}{LLL}
    \vertex{1,1}{LLR}
    \vertex{1,-1}{LRL}
    \trianglevertexleft{LRR}{LLR}{LRL}{LLL}
    %
    \vertex{2,1.5}{ALR}
    \vertex{3,1.5}{ARL}
    \vertex{2,0.5}{ALL}
    \vertex{3,0.5}{ARR}
    \squarevertex{ALR}{ARL}{ARR}{ALL}
    %
    \vertex{2,-0.5}{BLL}
    \vertex{2,-1.5}{BRL}
    \vertex{3,-0.5}{BRR}
    \vertex{3,-1.5}{BLR}
    \squarevertex{BLL}{BRR}{BLR}{BRL}
    %
    \vertex{4,0}{RRR}
    \vertex{4,1}{RRL}
    \vertex{4,-1}{RLR}
    \vertex{5,0}{RLL}
    \trianglevertexright[fill=green!40]{RLL}{RRL}{RLR}{RRR}
    %
    \draw[transition,->,>=stealth](-0.6,0) -- (LRR);
    %
    \transition{LLR}{LLR-|ALR.west}{}{}
    \transition{LRL}{LRL-|BRL.west}{}{}
    \transition{RRL}{RRL-|ARL.east}{}{}
    \transition{RLR}{RLR-|ARL.east}{}{}
    \selftransition[\circletransition]{LLR.north east}
    \selftransition[\circletransition]{RRL.north east}
    \selftransition[0:6mm]{ALR.north}
    \selftransition[0:6mm]{BLL.north}
  \end{tikzpicture}
}

\newcommand{\picprojectionpackagetruck}{
  \begin{tikzpicture}[scale=1.4]
    \vertex{0,0}{LRR}
    \vertex{1,-1}{LRL}
    \gravisvertex{LRR}{LRL}
    %
    \vertex{1,0}{LLL}
    \vertex{1,1}{LLR}
    \verticalvertex{LLR}{LLL}{}{}
    %
    \vertex{2,1.5}{ALR}
    \vertex{2,0.5}{ALL}
    \verticalvertex{ALR}{ALL}{}{}
    %
    \vertex{3,1.5}{ARL}
    \vertex{3,0.5}{ARR}
    \verticalvertex{ARL}{ARR}{}{}
    %
    \only<beamer:1|handout:0>{
      \vertex{3,-1.5}{BLR}
      \vertex{2,-0.5}{BLL}
      \vertex{3,-0.5}{BRR}
      \vertex{2,-1.5}{BRL}
      \gravisvertex{BLL}{BLR}
      \akutvertex{BRR}{BRL}
    }%
    \only<2>{
      \vertex{3,-1.5}{BRR}
      \vertex{2,-0.5}{BLL}
      \vertex{3,-0.5}{BLR}
      \vertex{2,-1.5}{BRL}
      \horizontalvertex{BLL}{BLR}
      \horizontalvertex{BRL}{BRR}
    }%
    %
    \vertex{4,0}{RRR}
    \vertex{4,1}{RRL}
    \verticalvertex[fill=green!40]{RRL}{RRR}{}{}
    %
    \vertex{4,-1}{RLR}
    \vertex{5,0}{RLL}
    \akutvertex[fill=green!40]{RLL}{RLR}
    %
    \draw[transition,->,>=stealth](-0.6,0) -- (LRR);
    \transition{intersection of LRR.north east--LRL.north east and LRR.east--LLL.west}
      {LLL.north-|LLR.west}{}{}
    \transition{ALL.north west-|LLR.east}{LLR.east-|ALL.west}{}{}
    \transition{LLR.east-|ALL.east}{LLR.east-|ARR.west}{}{}
    \transition{LLR.east-|ARL.east}{ARR.north east-|RRL.west}{}{}
    \transition{RRR.north-|RRR.east}
      {intersection of RLL.north west--RLR.north west and RLL.west--RRR.east}{}{}
    %
    \transition{LLL.north-|LLL.east}{BLL}{}{}
    \transition{LRL}{BRL}{}{}
    \only<beamer:1|handout:0>{\transition{BRR}{RRR.west|-RRR.north}{}{}}
    \only<2>{\transition{BRR.north east}{RRR.south west}{}{}}
    \transition{BLR}{RLR}{}{}
    \only<beamer:1|handout:0>{\transition{BRL.south east}{BLR.south west}{}{}}
    \only<2>{\transition{BLL.south-|BLL.east}{BRL.north-|BRL.east}{}{}}
    \selftransition[\circletransition]{LRR.north east}
    \selftransition[\circletransition]{LLR.north east}
    \selftransition[\circletransition]{ALR.north east}
    \selftransition[\circletransition]{ARL.north east}
    \selftransition[\circletransition]{RRL.north east}
    \selftransition[\circletransition]{RLL.north east}
    \only<beamer:1|handout:0>{
      \selftransition[\circletransition]{BLL.north east}
      \selftransition[\circletransition]{BRR.north east}
    }%
    \only<2>{
      \selftransition[1:8mm]{BLL.north}
      \selftransition[1:8mm]{BRL.north}
    }%
  \end{tikzpicture}%
}

\newcommand{\picrefinement}{%
  \begin{tikzpicture}[scale=1.4]
    \vertex{0,0}{LRR}
    \vertex{1,0}{LLL}
    \vertex{1,1}{LLR}
    \vertex{1,-1}{LRL}
    \trianglevertexleft{LRR}{LLR}{LRL}{LLL}
    \gravisvertexborder[subvertex]{LRR}{LRL}
    \verticalvertexborder[subvertex]{LLR}{LLL}
    \trianglevertexleftborder{LRR}{LLR}{LRL}{LLL}
    %
    \vertex{2,1.5}{ALR}
    \vertex{3,1.5}{ARL}
    \vertex{2,0.5}{ALL}
    \vertex{3,0.5}{ARR}
    \squarevertex{ALR}{ARL}{ARR}{ALL}
    \verticalvertexborder[subvertex]{ALR}{ALL}
    \verticalvertexborder[subvertex]{ARL}{ARR}
    \squarevertexborder{ALR}{ARL}{ARR}{ALL}
    %
    \vertex{2,-0.5}{BLL}
    \vertex{2,-1.5}{BRL}
    \vertex{3,-0.5}{BLR}
    \vertex{3,-1.5}{BRR}
    \squarevertex{BLL}{BLR}{BRR}{BRL}
    \horizontalvertexborder[subvertex]{BLL}{BLR}
    \horizontalvertexborder[subvertex]{BRL}{BRR}
    \squarevertexborder{BLL}{BLR}{BRR}{BRL}
    %
    \vertex{4,0}{RRR}
    \vertex{4,1}{RRL}
    \vertex{4,-1}{RLR}
    \vertex{5,0}{RLL}
    \trianglevertexright[fill=green!40]{RLL}{RRL}{RLR}{RRR}
    \verticalvertexborder[subvertex]{RRL}{RRR}
    \akutvertexborder[subvertex]{RLL}{RLR}
    \trianglevertexrightborder{RLL}{RRL}{RLR}{RRR}
	%
    \draw[transition,->,>=stealth](-0.6,0) -- (LRR);
    %
    \transition{LLR}{LLR-|ALR.west}{}{}
    \transition{LRL}{LRL-|BRL.west}{}{}
    \transition{RRL}{RRL-|ARL.east}{}{}
    \transition{RLR}{RLR-|ARL.east}{}{}
    \selftransition[\circletransition]{LLR.north east}
    \selftransition[\circletransition]{RRL.north east}
    \selftransition[0:6mm]{ALR.north}
    \selftransition[0:6mm]{BLL.north}
  \end{tikzpicture}
}


\newcommand{\pictwopackages}{%
  \begin{tikzpicture}[scale=1.28]
    \vertex{0,0}{LLR}
    \vertex{1,0}{LLL}
    \vertex{2,.5}{ILL}
    \vertex{2,-.5}{LIL}
    \vertex{3,0}{IIL}
    \vertex{4,0}{IIR}
    \vertex{5,.5}{RIR}
    \vertex{5,-.5}{IRR}
    \goalvertex{6,0}{RRR}
    \goalvertex{7,0}{RRL}
    \vertex{2.3,1.2}{ILR}
    \vertex{3.1,1.5}{RLR}
    \vertex{3.85,1.5}{RLL}
    \vertex{4.7,1.2}{RIL}
    \vertex{2.3,-1.2}{LIR}
    \vertex{3.1,-1.5}{LRR}
    \vertex{3.85,-1.5}{LRL}
    \vertex{4.7,-1.2}{IRL}

    \draw[transition,->,>=stealth](-0.6,0) -- (LLR);
    \transition{LLR}{LLL}{}{}
    \transition{LLL}{ILL}{}{}
    \transition{LLL}{LIL}{}{}
    \transition{ILL}{IIL}{}{}
    \transition{LIL}{IIL}{}{}
    \transition{IIL}{IIR}{}{}
    \transition{IIR}{RIR}{}{}
    \transition{IIR}{IRR}{}{}
    \transition{RIR}{RRR}{}{}
    \transition{IRR}{RRR}{}{}
    \transition{RRR}{RRL}{}{}
    \transition{ILL}{ILR}{}{}
    \transition{RLR}{ILR}{}{}
    \transition{RLR}{RLL}{}{}
    \transition{RIL}{RLL}{}{}
    \transition{RIL}{RIR}{}{}
    \transition{LIL}{LIR}{}{}
    \transition{LRR}{LIR}{}{}
    \transition{LRR}{LRL}{}{}
    \transition{LRL}{IRL}{}{}
    \transition{IRR}{IRL}{}{}
  \end{tikzpicture}
}

\newlength{\vertexround}
\setlength{\vertexround}{3.14mm}
\newcommand{\pictwopackagesabstractpacktwo}{%
  \begin{tikzpicture}[scale=1.28]
    \vertex{0,0}{LLR}
    \vertex{1,0}{LLL}
    \vertex{2,.5}{ILL}
    \vertex{2,-.5}{LIL}
    \vertex{3,0}{IIL}
    \vertex{4,0}{IIR}
    \vertex{5,.5}{RIR}
    \vertex{5,-.5}{IRR}
    \goalvertex{6,0}{RRR}
    \goalvertex{7,0}{RRL}
    \vertex{2.3,1.2}{ILR}
    \vertex{3.1,1.5}{RLR}
    \vertex{3.85,1.5}{RLL}
    \vertex{4.7,1.2}{RIL}
    \vertex{2.3,-1.2}{LIR}
    \vertex{3.1,-1.5}{LRR}
    \vertex{3.85,-1.5}{LRL}
    \vertex{4.7,-1.2}{IRL}
    
    \draw[bigvertex interior] (LLR.south) arc(-90:-240:\vertexround)
    -- (ILR.120) arc(120:100:\vertexround) 
	-- (RLR.110) arc (110:90:\vertexround)
    -- (RLL.north) arc(90:-60:\vertexround) -- (ILL.-60)
    -- (LLL.300) arc (-60:-90:\vertexround) -- (LLR.south);
	\hiddenvertices{LLR}{LLL}{ILL}{ILR}
	\hiddenvertices{RLR}{RLL}{}{}
    \draw[bigvertex border] (LLR.south) arc(-90:-240:\vertexround)
    -- (ILR.120) arc(120:100:\vertexround) 
	-- (RLR.120) arc (120:90:\vertexround)
    -- (RLL.north) arc(90:-60:\vertexround) -- (ILL.-60)
    -- (LLL.300) arc (-60:-90:\vertexround) -- (LLR.south);
    
    \draw[bigvertex interior] (LIR.310) arc(-50:-165:\vertexround)
    -- (LIL.200) arc(-160:-240:\vertexround) 
    -- (RIL.120) arc(120:10:\vertexround)
    -- (RIR.20) arc(20:-50:\vertexround)
    -- (LIR.310);
	\hiddenvertices{LIR}{LIL}{IIL}{IIR}
	\hiddenvertices{RIR}{RIL}{}{}
    \draw[bigvertex border] (LIR.310) arc(-50:-165:\vertexround)
    -- (LIL.200) arc(-160:-240:\vertexround) 
    -- (RIL.120) arc(120:10:\vertexround)
    -- (RIR.20) arc(20:-50:\vertexround)
    -- (LIR.310);

    \draw[bigvertex interior, color=green!40] (RRL.north) arc(90:-60:\vertexround)
    -- (IRL.-60) arc(-60:-80:\vertexround)
	-- (LRL.-70) arc (-70:-90:\vertexround)
    -- (LRR.south) arc(-90:-240:\vertexround) -- (IRR.120)
    -- (RRR.120) arc (120:90:\vertexround) -- (RRL.north);
	\hiddenvertices{LRR}{LRL}{IRL}{IRR}
	\hiddenvertices{RRR}{RRL}{}{}
    \draw[bigvertex border] (RRL.north) arc(90:-60:\vertexround)
    -- (IRL.-60) arc(-60:-80:\vertexround)
	-- (LRL.-70) arc (-70:-90:\vertexround)
    -- (LRR.south) arc(-90:-240:\vertexround) -- (IRR.120)
    -- (RRR.120) arc (120:90:\vertexround) -- (RRL.north);

    \draw[transition,->,>=stealth](-0.6,0) -- (LLR);
    \transition{ILL.-60}{intersection of LIL.-240 -- RIL.120 and ILR.180 -- LRR.180}{}{}
    \transition{IRR.120}{intersection of RIR.-50 -- LIR.310 and IRL.0 -- RLL.0}{}{}
    \selftransition[-.5mm:-3mm]{RRL.north}
    \selftransition[-.1mm:1mm]{RIL.north}
    \begin{scope}[yscale=-1]
      \selftransition[-.5mm:-3mm]{LLL.south}
    \end{scope}

    \only<all:5>{
      \transition[gray,->]{LLR}{LLL}{}{}
      \transition[gray,->]{LLL}{ILL}{}{}
      \transition[gray,->]{ILL}{IIL}{}{}
      \transition[gray,->]{IIL}{IIR}{}{}
      \transition[gray,->]{IIR}{RIR}{}{}
      \transition[gray,->]{RIR}{RRR}{}{}
    }
  \end{tikzpicture}
}

\newcommand{\pictwopackagesabstractpackone}{%
  \begin{tikzpicture}[scale=1.28]
    \vertex{0,0}{LLR}
    \vertex{1,0}{LLL}
    \vertex{2,.5}{ILL}
    \vertex{2,-.5}{LIL}
    \vertex{3,0}{IIL}
    \vertex{4,0}{IIR}
    \vertex{5,.5}{RIR}
    \vertex{5,-.5}{IRR}
    \goalvertex{6,0}{RRR}
    \goalvertex{7,0}{RRL}
    \vertex{2.3,1.2}{ILR}
    \vertex{3.1,1.5}{RLR}
    \vertex{3.85,1.5}{RLL}
    \vertex{4.7,1.2}{RIL}
    \vertex{2.3,-1.2}{LIR}
    \vertex{3.1,-1.5}{LRR}
    \vertex{3.85,-1.5}{LRL}
    \vertex{4.7,-1.2}{IRL}
    
    \draw[bigvertex interior] (LLR.north) arc(90:240:\vertexround)
    -- (LIR.-120) arc(-120:-100:\vertexround) 
    -- (LRR.-110) arc (-110:-90:\vertexround)
    -- (LRL.south) arc(-90:60:\vertexround) -- (LIL.60)
    -- (LLL.-300) arc (60:90:\vertexround) -- (LLR.north);
    \hiddenvertices{LLR}{LLL}{LIL}{LIR}
    \hiddenvertices{LRR}{LRL}{}{}
    \draw[bigvertex border] (LLR.north) arc(90:240:\vertexround)
    -- (LIR.-120) arc(-120:-100:\vertexround) 
    -- (LRR.-110) arc (-110:-90:\vertexround)
    -- (LRL.south) arc(-90:60:\vertexround) -- (LIL.60)
    -- (LLL.-300) arc (60:90:\vertexround) -- (LLR.north);
    
    \draw[bigvertex interior] (ILR.-310) arc(50:165:\vertexround)
    -- (ILL.-200) arc(160:240:\vertexround) 
    -- (IRL.-120) arc(-120:-10:\vertexround)
    -- (IRR.-20) arc(-20:50:\vertexround)
    -- (ILR.-310);
    \hiddenvertices{ILR}{ILL}{IIL}{IIR}
    \hiddenvertices{IRR}{IRL}{}{}
    \draw[bigvertex border] (ILR.-310) arc(50:165:\vertexround)
    -- (ILL.-200) arc(160:240:\vertexround) 
    -- (IRL.-120) arc(-120:-10:\vertexround)
    -- (IRR.-20) arc(-20:50:\vertexround)
    -- (ILR.-310);

    \draw[bigvertex interior, color=green!40] (RRL.south) arc(-90:60:\vertexround)
    -- (RIL.60) arc(60:80:\vertexround)
    -- (RLL.70) arc (70:90:\vertexround)
    -- (RLR.north) arc(90:240:\vertexround) -- (RIR.-120)
    -- (RRR.-120) arc (-120:-90:\vertexround) -- (RRL.south);
    \hiddenvertices{RLR}{RLL}{RIL}{RIR}
    \hiddenvertices{RRR}{RRL}{}{}
    \draw[bigvertex border] (RRL.south) arc(-90:60:\vertexround)
    -- (RIL.60) arc(60:80:\vertexround)
    -- (RLL.70) arc (70:90:\vertexround)
    -- (RLR.north) arc(90:240:\vertexround) -- (RIR.-120)
    -- (RRR.-120) arc (-120:-90:\vertexround) -- (RRL.south);

    \draw[transition,->,>=stealth](-0.6,0) -- (LLR);
    \transition{LIL.60}{intersection of ILL.240 -- IRL.240 and LIR.180 -- RLR.180}{}{}
    \transition{LIL.60}{intersection of ILL.240 -- IRL.-120 and LIR.-180 -- RLR.-180}{}{}
    \transition{RIR.-120}{intersection of IRR.50 -- ILR.-310 and RIL.0 -- LRL.0}{}{}
    \selftransition[-.5mm:-3mm]{LLL.north}
    \selftransition[-.1mm:1mm]{ILR.north}
    \begin{scope}[yscale=-1]
      \selftransition[-.5mm:-3mm]{RRL.south}
    \end{scope}

    \only<all:3>{
      \transition[gray,->]{LLR}{LLL}{}{}
      \transition[gray,->]{LLL}{ILL}{}{}
      \transition[gray,->]{ILL}{IIL}{}{}
      \transition[gray,->]{IIL}{IIR}{}{}
      \transition[gray,->]{IIR}{RIR}{}{}
      \transition[gray,->]{RIR}{RRR}{}{}
    }
  \end{tikzpicture}
}

\newcommand{\bendtransition}[6][]{\draw[->,>=stealth,transition,#1] (#2) edge [#6] node[above=-1mm,sloped,#5] {\scriptsize #4} (#3);}
\newcommand{\bendtransitionnolabel}[6][]{\draw[->,>=stealth,transition,#1] (#2) edge [#6] node[above=-1mm,sloped,#5] {} (#3);}
\newcommand{\picatomicprojectionpackage}{%
  \begin{tikzpicture}[->,>=stealth,looseness=.5,bend angle=20]
       	\vertex{0,0}{L}
	\vertex{2,1}{A}
	\vertex{2,-1}{B}
	\goalvertex{4,0}{R}

	\draw[transition,->,>=stealth](-0.8,0) -- (L);
	\bendtransition{L}{L}{M{\wildcard}{\wildcard}{\wildcard}}{above left}{loop above}
	\bendtransition{L}{A}{PAL}{}{bend left}
	\bendtransition{A}{L}{DAL}{}{bend left}
	\bendtransition{A}{A}{M{\wildcard}{\wildcard}{\wildcard}}{above=1mm}{loop above}
	\bendtransition{A}{R}{DAR}{}{bend left}
	\bendtransition{R}{A}{PAR}{}{bend left}
	\bendtransition{R}{R}{M{\wildcard}{\wildcard}{\wildcard}}{above right}{loop above}
	\bendtransition{R}{B}{PBR}{below=-2mm}{bend left}
	\bendtransition{B}{R}{DBR}{below=-2mm}{bend left}
	\bendtransition{B}{B}{M{\wildcard}{\wildcard}{\wildcard}}{below=-1mm}{loop below}
	\bendtransition{B}{L}{DBL}{below=-2mm}{bend left}
	\bendtransition{L}{B}{PBL}{below=-2mm}{bend left}
  \end{tikzpicture}
}

\newcommand{\picatomicprojectionpackagenumberedstates}{%
  \begin{tikzpicture}[->,>=stealth,looseness=.5,bend angle=20]
        \numberedvertex{0,0}{L}{0}
	\numberedvertex{2,1}{A}{2}
	\numberedvertex{2,-1}{B}{3}
	\numberedgoalvertex{4,0}{R}{1}

	\draw[transition,->,>=stealth](-0.8,0) -- (L);
	\bendtransition{L}{L}{M{\wildcard}{\wildcard}{\wildcard}}{above left}{loop above}
	\bendtransition{L}{A}{PAL}{}{bend left}
	\bendtransition{A}{L}{DAL}{}{bend left}
	\bendtransition{A}{A}{M{\wildcard}{\wildcard}{\wildcard}}{above=1mm}{loop above}
	\bendtransition{A}{R}{DAR}{}{bend left}
	\bendtransition{R}{A}{PAR}{}{bend left}
	\bendtransition{R}{R}{M{\wildcard}{\wildcard}{\wildcard}}{above right}{loop above}
	\bendtransition{R}{B}{PBR}{below=-2mm}{bend left}
	\bendtransition{B}{R}{DBR}{below=-2mm}{bend left}
	\bendtransition{B}{B}{M{\wildcard}{\wildcard}{\wildcard}}{below=-1mm}{loop below}
	\bendtransition{B}{L}{DBL}{below=-2mm}{bend left}
	\bendtransition{L}{B}{PBL}{below=-2mm}{bend left}
  \end{tikzpicture}
}

\newcommand{\picatomicprojectiontrucka}{%
  \begin{tikzpicture}[->,>=stealth,looseness=.5]
	\goalvertex{0,0}{L}
	\goalvertex{4,0}{R}

	\draw[transition,->,>=stealth](4.8,0) -- (R);
	\bendtransition{L}{L}{\parbox{22mm}{PAL,DAL,MB{\wildcard}{\wildcard},\\PB{\wildcard},DB{\wildcard}}}{above}{loop above}
	\bendtransition{L}{R}{MALR}{}{bend left}
	\bendtransition{R}{L}{MARL}{}{bend left}
	\bendtransition{R}{R}{\parbox{22mm}{PAR,DAR,MB{\wildcard}{\wildcard},\\PB{\wildcard},DB{\wildcard}}}{above}{loop above}
  \end{tikzpicture}
}

\newcommand{\picatomicprojectiontruckb}{%
  \begin{tikzpicture}[->,>=stealth,looseness=.5]
	\goalvertex{0,0}{L}
	\goalvertex{4,0}{R}

	\draw[transition,->,>=stealth](4.8,0) -- (R);
	\bendtransition{L}{L}{\parbox{22mm}{PBL,DBL,MA{\wildcard}{\wildcard},\\PA{\wildcard},DA{\wildcard}}}{above}{loop above}
	\bendtransition{L}{R}{MBLR}{}{bend left}
	\bendtransition{R}{L}{MBRL}{}{bend left}
	\bendtransition{R}{R}{\parbox{22mm}{PBR,DBR,MA{\wildcard}{\wildcard},\\PA{\wildcard},DA{\wildcard}}}{above}{loop above}
  \end{tikzpicture}
}

\newcommand{\picproductpackagetrucka}[1][]{%
  \begin{tikzpicture}[->,>=stealth,looseness=.5,bend angle=20,#1]
  	\vertex{0,0}{LL}
  	\vertex{2,0}{LR}
	\vertex{3,2}{AL}
	\vertex{5,2}{AR}
	\vertex{3,-2}{BL}
	\vertex{5,-2}{BR}
  	\goalvertex{6,0}{RL}
  	\goalvertex{8,0}{RR}
	
	\draw[transition,->,>=stealth](2.6,0.6) -- (LR);
	\bendtransition{LL}{LR}{MALR}{}{bend left}
	\bendtransition{LR}{LL}{MARL}{below=-2mm}{bend left}
	\bendtransition{AL}{AR}{MALR}{}{bend left}
	\bendtransition{AR}{AL}{MARL}{below=-2mm}{bend left}
	\bendtransition{BL}{BR}{MALR}{}{bend left}
	\bendtransition{BR}{BL}{MARL}{below=-2mm}{bend left}
	\bendtransition{RL}{RR}{MALR}{}{bend left}
	\bendtransition{RR}{RL}{MARL}{below=-2mm}{bend left}
	
	\bendtransition{LL}{AL}{PAL}{}{bend left=50}
	\bendtransition{AL}{LL}{DAL}{}{bend right}
	\bendtransition{AR}{RR}{DAR}{}{bend left=50}
	\bendtransition{RR}{AR}{PAR}{}{bend right}
	\bendtransition{RR}{BR}{PBR}{}{bend left=50}
	\bendtransition{BR}{RR}{DBR}{}{bend right}
	\bendtransition{BL}{LL}{DBL}{}{bend left=50}
	\bendtransition{LL}{BL}{PBL}{}{bend right}
	
	\bendtransition{LR}{BR}{PBL}{near start}{bend left=50}
	\bendtransition{BR}{LR}{DBL}{near end,below=-2mm}{bend right}
	\bendtransition{BL}{RL}{DBR}{near end}{bend left=50}
	\bendtransition{RL}{BL}{PBR}{near start,below=-2mm}{bend right}
	
	\bendtransition{LL}{LL}{MB{\wildcard}{\wildcard}}{above=2mm,pos=.9,sloped=false}{loop left}
	\bendtransition{RR}{RR}{MB{\wildcard}{\wildcard}}{above=2mm,pos=.1,sloped=false}{loop right}
	\bendtransition{AL}{AL}{MB{\wildcard}{\wildcard}}{}{loop above}
	\bendtransition{AR}{AR}{MB{\wildcard}{\wildcard}}{}{loop above}
	\bendtransition{LR}{LR}{MB{\wildcard}{\wildcard}}{}{loop above}
	\bendtransition{RL}{RL}{MB{\wildcard}{\wildcard}}{}{loop above}
	\bendtransition{BL}{BL}{MB{\wildcard}{\wildcard}}{below=-1mm}{loop below}
	\bendtransition{BR}{BR}{MB{\wildcard}{\wildcard}}{below=-1mm}{loop below}
  \end{tikzpicture}
}

\newcommand{\picproductpackagetruckanumberedstates}[1][]{%
  \begin{tikzpicture}[->,>=stealth,looseness=.5,bend angle=20,#1]
  	\numberedvertex{0,0}{LL}{0}
  	\numberedvertex{2,0}{LR}{1}
	\numberedvertex{3,2}{AL}{4}
	\numberedvertex{5,2}{AR}{5}
	\numberedvertex{3,-2}{BL}{6}
	\numberedvertex{5,-2}{BR}{7}
  	\numberedgoalvertex{6,0}{RL}{2}
  	\numberedgoalvertex{8,0}{RR}{3}
	
	\draw[transition,->,>=stealth](2.6,0.6) -- (LR);
	\bendtransition{LL}{LR}{MALR}{}{bend left}
	\bendtransition{LR}{LL}{MARL}{below=-2mm}{bend left}
	\bendtransition{AL}{AR}{MALR}{}{bend left}
	\bendtransition{AR}{AL}{MARL}{below=-2mm}{bend left}
	\bendtransition{BL}{BR}{MALR}{}{bend left}
	\bendtransition{BR}{BL}{MARL}{below=-2mm}{bend left}
	\bendtransition{RL}{RR}{MALR}{}{bend left}
	\bendtransition{RR}{RL}{MARL}{below=-2mm}{bend left}
	
	\bendtransition{LL}{AL}{PAL}{}{bend left=50}
	\bendtransition{AL}{LL}{DAL}{}{bend right}
	\bendtransition{AR}{RR}{DAR}{}{bend left=50}
	\bendtransition{RR}{AR}{PAR}{}{bend right}
	\bendtransition{RR}{BR}{PBR}{}{bend left=50}
	\bendtransition{BR}{RR}{DBR}{}{bend right}
	\bendtransition{BL}{LL}{DBL}{}{bend left=50}
	\bendtransition{LL}{BL}{PBL}{}{bend right}
	
	\bendtransition{LR}{BR}{PBL}{near start}{bend left=50}
	\bendtransition{BR}{LR}{DBL}{near end,below=-2mm}{bend right}
	\bendtransition{BL}{RL}{DBR}{near end}{bend left=50}
	\bendtransition{RL}{BL}{PBR}{near start,below=-2mm}{bend right}
	
	\bendtransition{LL}{LL}{MB{\wildcard}{\wildcard}}{above=2mm,pos=.9,sloped=false}{loop left}
	\bendtransition{RR}{RR}{MB{\wildcard}{\wildcard}}{above=2mm,pos=.1,sloped=false}{loop right}
	\bendtransition{AL}{AL}{MB{\wildcard}{\wildcard}}{}{loop above}
	\bendtransition{AR}{AR}{MB{\wildcard}{\wildcard}}{}{loop above}
	\bendtransition{LR}{LR}{MB{\wildcard}{\wildcard}}{}{loop above}
	\bendtransition{RL}{RL}{MB{\wildcard}{\wildcard}}{}{loop above}
	\bendtransition{BL}{BL}{MB{\wildcard}{\wildcard}}{below=-1mm}{loop below}
	\bendtransition{BR}{BR}{MB{\wildcard}{\wildcard}}{below=-1mm}{loop below}
  \end{tikzpicture}
}

\newcommand{\subpicatomicprojectionpackagesmall}{%
  \begin{scope}[->,>=stealth,looseness=.5,bend angle=20,scale=.7,transform shape]
	\vertex{0,0}{L}
	\vertex{2,1}{A}
	\vertex{2,-1}{B}
	\goalvertex{4,0}{R}

	\draw[transition,->,>=stealth](-0.8,0) -- (L);
	\bendtransition{L}{L}{M{\wildcard}{\wildcard}{\wildcard}}{above left}{loop above}
	\bendtransition{L}{A}{PAL}{}{bend left}
	\bendtransition{A}{L}{DAL}{}{bend left}
	\bendtransition{A}{A}{M{\wildcard}{\wildcard}{\wildcard}}{above=1mm}{loop above}
	\bendtransition{A}{R}{DAR}{}{bend left}
	\bendtransition{R}{A}{PAR}{}{bend left}
	\bendtransition{R}{R}{M{\wildcard}{\wildcard}{\wildcard}}{above right}{loop above}
	\bendtransition{R}{B}{PBR}{below=-2mm}{bend left}
	\bendtransition{B}{R}{DBR}{below=-2mm}{bend left}
	\bendtransition{B}{B}{M{\wildcard}{\wildcard}{\wildcard}}{below=-1mm}{loop below}
	\bendtransition{B}{L}{DBL}{below=-2mm}{bend left}
	\bendtransition{L}{B}{PBL}{below=-2mm}{bend left}
	
	\only<all:2>{\vertex[draw=red,thick]{2,1}{A}}
	\only<all:3>{\vertex[draw=red,thick]{0,0}{L}}
	\only<all:3>{\draw[transition,->,>=stealth,red](-0.8,0) -- (L);}
	\only<all:4>{\goalvertex[fill=green]{4,0}{R}}
	\only<all:5>{\bendtransition[red]{L}{A}{PAL}{}{bend left}}
	\only<all:6>{\bendtransition[red]{L}{L}{M{\wildcard}{\wildcard}{\wildcard}}{above left}{loop above}}
	\only<all:7>{\bendtransition[red]{L}{B}{PBL}{below=-2mm}{bend left}}
	\only<all:8>{\bendtransition[red]{A}{A}{M{\wildcard}{\wildcard}{\wildcard}}{above=1mm}{loop above}}
  \end{scope}
}

\newcommand{\subpicatomicprojectiontruckasmall}{%
  \begin{scope}[->,>=stealth,looseness=.5,scale=.7,transform shape]
	\goalvertex{0,0}{L}
	\goalvertex{4,0}{R}

	\draw[transition,->,>=stealth](4.8,0) -- (R);
	\bendtransition{L}{L}{\parbox{22mm}{PAL,DAL,MB{\wildcard}{\wildcard},\\PB{\wildcard},DB{\wildcard}}}{above}{loop above}
	\bendtransition{L}{R}{MALR}{}{bend left}
	\bendtransition{R}{L}{MARL}{}{bend left}
	\bendtransition{R}{R}{\parbox{22mm}{PAR,DAR,MB{\wildcard}{\wildcard},\\PB{\wildcard},DB{\wildcard}}}{above}{loop above}

	\only<all:2>{\goalvertex[draw=red,thick]{0,0}{L}}
	\only<all:3>{\goalvertex[draw=red,thick]{4,0}{R}}
	\only<all:3>{\draw[transition,->,>=stealth,red](4.8,0) -- (R);}
	\only<all:4>{\goalvertex[fill=green]{0,0}{L}}
	\only<all:5>{\bendtransition[red]{L}{L}{\parbox{22mm}{PAL\color{black},DAL,MB{\wildcard}{\wildcard},\\PB{\wildcard},DB{\wildcard}}}{above}{loop above}}
	\only<all:6>{\bendtransition[red]{L}{R}{MALR}{}{bend left}}
	\only<all:7>{\bendtransition[red]{R}{R}{\parbox{22mm}{\color{black}PAR,DAR,MB{\wildcard}{\wildcard},\\{\color{red}PB{\wildcard}},DB{\wildcard}}}{above}{loop above}}
	\only<all:8>{\bendtransition[red]{L}{L}{\parbox{22mm}{\color{black}PAL,DAL,{\color{red}MB{\wildcard}{\wildcard}},\\PB{\wildcard},DB{\wildcard}}}{above}{loop above}}
  \end{scope}
}

\newcommand{\subpicproductpackagetruckasmall}{%
  \begin{scope}[->,>=stealth,looseness=.5,,bend angle=20,scale=.7,transform shape]
  	\vertex{0,0}{LL}
  	\vertex{2,0}{LR}
	\vertex{3,2}{AL}
	\vertex{5,2}{AR}
	\vertex{3,-2}{BL}
	\vertex{5,-2}{BR}
  	\goalvertex{6,0}{RL}
  	\goalvertex{8,0}{RR}
	
	\draw[transition,->,>=stealth](2.6,0.6) -- (LR);
	\bendtransition{LL}{LR}{MALR}{}{bend left}
	\bendtransition{LR}{LL}{MARL}{below=-2mm}{bend left}
	\bendtransition{AL}{AR}{MALR}{}{bend left}
	\bendtransition{AR}{AL}{MARL}{below=-2mm}{bend left}
	\bendtransition{BL}{BR}{MALR}{}{bend left}
	\bendtransition{BR}{BL}{MARL}{below=-2mm}{bend left}
	\bendtransition{RL}{RR}{MALR}{}{bend left}
	\bendtransition{RR}{RL}{MARL}{below=-2mm}{bend left}
	
	\bendtransition{LL}{AL}{PAL}{}{bend left=50}
	\bendtransition{AL}{LL}{DAL}{}{bend right}
	\bendtransition{AR}{RR}{DAR}{}{bend left=50}
	\bendtransition{RR}{AR}{PAR}{}{bend right}
	\bendtransition{RR}{BR}{PBR}{}{bend left=50}
	\bendtransition{BR}{RR}{DBR}{}{bend right}
	\bendtransition{BL}{LL}{DBL}{}{bend left=50}
	\bendtransition{LL}{BL}{PBL}{}{bend right}
	
	\bendtransition{LR}{BR}{PBL}{near start}{bend left=50}
	\bendtransition{BR}{LR}{DBL}{near end,below=-2mm}{bend right}
	\bendtransition{BL}{RL}{DBR}{near end}{bend left=50}
	\bendtransition{RL}{BL}{PBR}{near start,below=-2mm}{bend right}
	
	\bendtransition{LL}{LL}{MB{\wildcard}{\wildcard}}{above=2mm,sloped=false,pos=.9}{loop left}
	\bendtransition{RR}{RR}{MB{\wildcard}{\wildcard}}{above=2mm,sloped=false,pos=.1}{loop right}
	\bendtransition{AL}{AL}{MB{\wildcard}{\wildcard}}{}{loop above}
	\bendtransition{AR}{AR}{MB{\wildcard}{\wildcard}}{}{loop above}
	\bendtransition{LR}{LR}{MB{\wildcard}{\wildcard}}{}{loop above}
	\bendtransition{RL}{RL}{MB{\wildcard}{\wildcard}}{}{loop above}
	\bendtransition{BL}{BL}{MB{\wildcard}{\wildcard}}{below=-1mm}{loop below}
	\bendtransition{BR}{BR}{MB{\wildcard}{\wildcard}}{below=-1mm}{loop below}
	
	\only<all:2>{\vertex[draw=red,thick]{3,2}{AL}}
	\only<all:3>{\vertex[draw=red,thick]{2,0}{LR}}
	\only<all:3>{\draw[transition,->,>=stealth,red](2.6,0.6) -- (LR);}
  	\only<all:4>{\goalvertex[fill=green]{6,0}{RL}}
	\only<all:5>{\bendtransition[red]{LL}{AL}{PAL}{}{bend left=50}}
	\only<all:6>{\bendtransition[red]{LL}{LR}{MALR}{}{bend left}}
	\only<all:7>{\bendtransition[red]{LR}{BR}{PBL}{near start}{bend left=50}}
	\only<all:8>{\bendtransition[red]{AL}{AL}{MB{\wildcard}{\wildcard}}{}{loop above}}
  \end{scope}
}

\newcommand{\picproductpackagetruckaillustration}{%
  \begin{tikzpicture}[->,>=stealth,looseness=.5,,bend angle=20,scale=0.98]
  	\subpicatomicprojectionpackagesmall
	\node (otimes) at (4,0) {\LARGE $\otimes$};	
	\begin{scope}[xshift=5cm]
  	\subpicatomicprojectiontruckasmall
	\end{scope}
	\node (otimes) at (0,-3) {\LARGE $=$};	
	\begin{scope}[xshift=2cm,yshift=-3cm]
  	\subpicproductpackagetruckasmall
	\end{scope}
  \end{tikzpicture}
}


\newcommand{\picshrinkexampleone}[1][]{%
  \begin{tikzpicture}[->,>=stealth,looseness=.5,,bend angle=20,#1]
  	\vertex{0,0}{LL}
  	\vertex{2,0}{LR}
	\only<all:1-4>{\vertex{3,2}{AL}}
	\only<all:1-4>{\vertex{5,2}{AR}}
	\only<all:4>{\vertex[draw=red,thick]{3,2}{AL}}
	\only<all:4>{\vertex[draw=red,thick]{5,2}{AR}}
	\only<all:5-8>{\vertex{4,2}{A}}
	\only<all:8>{\vertex[draw=red,thick]{4,2}{A}}

	\only<all:1-6>{\vertex{3,-2}{BL}}
	\only<all:1-6>{\vertex{5,-2}{BR}}
	\only<all:6>{\vertex[draw=red,thick]{3,-2}{BL}}
	\only<all:6>{\vertex[draw=red,thick]{5,-2}{BR}}
	\only<all:7-8>{\vertex{4,-2}{B}}
	\only<all:8>{\vertex[draw=red,thick]{4,-2}{B}}
	\only<all:9->{\vertex{4,0}{I}}

	\only<all:1>{\goalvertex{6,0}{RL}}
  	\only<all:1>{\goalvertex{8,0}{RR}}
  	\only<all:2>{\goalvertex[draw=red,thick]{6,0}{RL}}
  	\only<all:2>{\goalvertex[draw=red,thick]{8,0}{RR}}
	\only<all:3->{\goalvertex{7,0}{R}}

	
	\draw[transition,->,>=stealth](2.6,0.6) -- (LR);
	\bendtransition{LL}{LR}{MALR}{}{bend left}
	\only<all:1-8>{\bendtransition{LR}{LL}{MARL}{below=-2mm}{bend left}}
	\only<all:9->{\bendtransition{LR}{LL}{MARL}{}{bend left}}
	\only<all:1-4>{\bendtransition{AL}{AR}{MALR}{}{bend left}}
	\only<all:1-4>{\bendtransition{AR}{AL}{MARL}{below=-2mm}{bend left}}
	\only<all:1-6>{\bendtransition{BL}{BR}{MALR}{}{bend left}}
	\only<all:1-6>{\bendtransition{BR}{BL}{MARL}{below=-2mm}{bend left}}
	\only<all:1-2>{\bendtransition{RL}{RR}{MALR}{}{bend left}}
	\only<all:1-2>{\bendtransition{RR}{RL}{MARL}{below=-2mm}{bend left}}
	
	\only<all:1-4>{\bendtransition{LL}{AL}{PAL}{}{bend left=50}}
	\only<all:1-4>{\bendtransition{AL}{LL}{DAL}{}{bend right}}
	\only<all:5-8>{\bendtransition{LL}{A}{PAL}{}{bend left=50}}
	\only<all:5-8>{\bendtransition{A}{LL}{DAL}{}{bend right}}
	\only<all:1-2>{\bendtransition{AR}{RR}{DAR}{}{bend left=50}}
	\only<all:1-2>{\bendtransition{RR}{AR}{PAR}{}{bend right}}
	\only<all:3-4>{\bendtransition{AR}{R}{DAR}{}{bend left=50}}
	\only<all:3-4>{\bendtransition{R}{AR}{PAR}{}{bend right}}
	\only<all:5-8>{\bendtransition{A}{R}{DAR}{}{bend left=50}}
	\only<all:5-8>{\bendtransition{R}{A}{PAR}{}{bend right}}
	\only<all:1-2>{\bendtransition{RR}{BR}{PBR}{}{bend left=50}}
	\only<all:1-2>{\bendtransition{BR}{RR}{DBR}{}{bend right}}
	\only<all:3-6>{\bendtransition{R}{BR}{PBR}{}{bend left=50}}
	\only<all:3-6>{\bendtransition{BR}{R}{DBR}{}{bend right}}
	\only<all:7-8>{\bendtransition{R}{B}{PBR}{}{bend left=50}}
	\only<all:7-8>{\bendtransition{B}{R}{DBR}{}{bend right}}
	\only<all:1-6>{\bendtransition{BL}{LL}{DBL}{}{bend left=50}}
	\only<all:1-6>{\bendtransition{LL}{BL}{PBL}{}{bend right}}
	\only<all:7-8>{\bendtransition{B}{LL}{DBL}{}{bend left=50}}
	\only<all:7-8>{\bendtransition{LL}{B}{PBL}{}{bend right}}
	
	\only<all:1-6>{\bendtransition{LR}{BR}{PBL}{near start}{bend left=50}}
	\only<all:1-6>{\bendtransition{BR}{LR}{DBL}{near end,below=-2mm}{bend right}}
	\only<all:7-8>{\bendtransition{LR}{B}{PBL}{near start}{bend left=50}}
	\only<all:7-8>{\bendtransition{B}{LR}{DBL}{near end,below=-2mm}{bend right}}
	\only<all:1-2>{\bendtransition{BL}{RL}{DBR}{near end}{bend left=50}}%
	\only<all:1-2>{\bendtransition{RL}{BL}{PBR}{near start,below=-2mm}{bend right}}%
	\only<all:3-6>{\bendtransition{BL}{R}{DBR}{near end}{bend left=50}}%
	\only<all:3-6>{\bendtransition{R}{BL}{PBR}{near start,below=-2mm}{bend right}}%
	
	\bendtransition{LL}{LL}{MB{\wildcard}{\wildcard}}{above=2mm,sloped=false,pos=.9}{loop left}
	\only<all:1-2>{\bendtransition{RR}{RR}{MB{\wildcard}{\wildcard}}{above=2mm,sloped=false,pos=.1}{loop right}}
	\only<all:1-4>{\bendtransition{AL}{AL}{MB{\wildcard}{\wildcard}}{}{loop above}}
	\only<all:1-4>{\bendtransition{AR}{AR}{MB{\wildcard}{\wildcard}}{}{loop above}}
	\only<all:5-8>{\bendtransition{A}{A}{M{\wildcard}{\wildcard}{\wildcard}}{}{loop above}}
	\bendtransition{LR}{LR}{MB{\wildcard}{\wildcard}}{}{loop above}
	\only<all:1-2>{\bendtransition{RL}{RL}{MB{\wildcard}{\wildcard}}{}{loop above}}
	\only<all:1-6>{\bendtransition{BL}{BL}{MB{\wildcard}{\wildcard}}{below=-1mm}{loop below}}
	\only<all:7-8>{\bendtransition{B}{B}{M{\wildcard}{\wildcard}{\wildcard}}{below=-1mm}{loop below}}
	\only<all:3->{\bendtransition{R}{R}{M{\wildcard}{\wildcard}{\wildcard}}{above=2mm,sloped=false}{loop right}}
	\only<all:1-6>{\bendtransition{BR}{BR}{MB{\wildcard}{\wildcard}}{below=-1mm}{loop below}}

	\only<all:9->{\bendtransition{I}{R}{D{\wildcard}R}{}{bend left}}
	\only<all:9->{\bendtransition{R}{I}{P{\wildcard}R}{}{bend left}}
	\only<all:9->{\bendtransition{I}{I}{M{\wildcard}{\wildcard}{\wildcard}}{}{loop above}}
	\only<all:9->{\bendtransition{LR}{I}{PBL}{}{bend left}}
	\only<all:9->{\bendtransition{I}{LR}{DBL}{}{bend left}}
	\only<all:9->{\bendtransition{LL}{I}{P{\wildcard}L}{}{bend right=55}}
	\only<all:9->{\bendtransition{I}{LL}{D{\wildcard}L}{below=-2mm}{bend left=80}}
  \end{tikzpicture}
}

\newcommand{\picshrinkexampleonenumberedstates}[1][]{%
  \begin{tikzpicture}[->,>=stealth,looseness=.5,,bend angle=20,#1]
        \useasboundingbox (-1.5,-3.5) rectangle (9.5,3.5);
        %% \draw (-1.5,-3.5) rectangle (9.5,3.5);
  	\numberedvertex{0,0}{LL}{0}
  	\numberedvertex{2,0}{LR}{1}
	\only<all:1-5>{\numberedvertex{3,2}{AL}{4}}
	\only<all:1-5>{\numberedvertex{5,2}{AR}{5}}
	\only<all:5>{\numberedvertex[draw=red,thick]{3,2}{AL}{4}}
	\only<all:5>{\numberedvertex[draw=red,thick]{5,2}{AR}{5}}
	\only<all:6-9>{\numberedvertex{4,2}{A}{4}}
	\only<all:9>{\numberedvertex[draw=red,thick]{4,2}{A}{4}}

	\only<all:1-7>{\numberedvertex{3,-2}{BL}{6}}
	\only<all:1-7>{\numberedvertex{5,-2}{BR}{7}}
	\only<all:7>{\numberedvertex[draw=red,thick]{3,-2}{BL}{6}}
	\only<all:7>{\numberedvertex[draw=red,thick]{5,-2}{BR}{7}}
	\only<all:8-9>{\numberedvertex{4,-2}{B}{6}}
	\only<all:9>{\numberedvertex[draw=red,thick]{4,-2}{B}{6}}
	\only<all:10-11>{\numberedvertex{4,0}{I}{4}}
        \only<all:12>{\numberedvertex{4,0}{I}{\alert{4$\mapsto$3}}}
        \only<all:13->{\numberedvertex{4,0}{I}{\alert<all:13>{3}}}

	\only<all:1-2>{\numberedgoalvertex{6,0}{RL}{2}}
  	\only<all:1-2>{\numberedgoalvertex{8,0}{RR}{3}}
  	\only<all:3>{\numberedgoalvertex[draw=red,thick]{6,0}{RL}{2}}
  	\only<all:3>{\numberedgoalvertex[draw=red,thick]{8,0}{RR}{3}}
	\only<all:4->{\numberedgoalvertex{7,0}{R}{2}}

	
	\draw[transition,->,>=stealth](2.6,0.6) -- (LR);
	\bendtransition{LL}{LR}{MALR}{}{bend left}
	\only<all:1-9>{\bendtransition{LR}{LL}{MARL}{below=-2mm}{bend left}}
	\only<all:10->{\bendtransition{LR}{LL}{MARL}{}{bend left}}
	\only<all:1-5>{\bendtransition{AL}{AR}{MALR}{}{bend left}}
	\only<all:1-5>{\bendtransition{AR}{AL}{MARL}{below=-2mm}{bend left}}
	\only<all:1-7>{\bendtransition{BL}{BR}{MALR}{}{bend left}}
	\only<all:1-7>{\bendtransition{BR}{BL}{MARL}{below=-2mm}{bend left}}
	\only<all:1-3>{\bendtransition{RL}{RR}{MALR}{}{bend left}}
	\only<all:1-3>{\bendtransition{RR}{RL}{MARL}{below=-2mm}{bend left}}
	
	\only<all:1-5>{\bendtransition{LL}{AL}{PAL}{}{bend left=50}}
	\only<all:1-5>{\bendtransition{AL}{LL}{DAL}{}{bend right}}
	\only<all:6-9>{\bendtransition{LL}{A}{PAL}{}{bend left=50}}
	\only<all:6-9>{\bendtransition{A}{LL}{DAL}{}{bend right}}
	\only<all:1-3>{\bendtransition{AR}{RR}{DAR}{}{bend left=50}}
	\only<all:1-3>{\bendtransition{RR}{AR}{PAR}{}{bend right}}
	\only<all:4-5>{\bendtransition{AR}{R}{DAR}{}{bend left=50}}
	\only<all:4-5>{\bendtransition{R}{AR}{PAR}{}{bend right}}
	\only<all:6-9>{\bendtransition{A}{R}{DAR}{}{bend left=50}}
	\only<all:6-9>{\bendtransition{R}{A}{PAR}{}{bend right}}
	\only<all:1-3>{\bendtransition{RR}{BR}{PBR}{}{bend left=50}}
	\only<all:1-3>{\bendtransition{BR}{RR}{DBR}{}{bend right}}
	\only<all:4-7>{\bendtransition{R}{BR}{PBR}{}{bend left=50}}
	\only<all:4-7>{\bendtransition{BR}{R}{DBR}{}{bend right}}
	\only<all:8-9>{\bendtransition{R}{B}{PBR}{}{bend left=50}}
	\only<all:8-9>{\bendtransition{B}{R}{DBR}{}{bend right}}
	\only<all:1-7>{\bendtransition{BL}{LL}{DBL}{}{bend left=50}}
	\only<all:1-7>{\bendtransition{LL}{BL}{PBL}{}{bend right}}
	\only<all:8-9>{\bendtransition{B}{LL}{DBL}{}{bend left=50}}
	\only<all:8-9>{\bendtransition{LL}{B}{PBL}{}{bend right}}
	
	\only<all:1-7>{\bendtransition{LR}{BR}{PBL}{near start}{bend left=50}}
	\only<all:1-7>{\bendtransition{BR}{LR}{DBL}{near end,below=-2mm}{bend right}}
	\only<all:8-9>{\bendtransition{LR}{B}{PBL}{near start}{bend left=50}}
	\only<all:8-9>{\bendtransition{B}{LR}{DBL}{near end,below=-2mm}{bend right}}
	\only<all:1-3>{\bendtransition{BL}{RL}{DBR}{near end}{bend left=50}}%
	\only<all:1-3>{\bendtransition{RL}{BL}{PBR}{near start,below=-2mm}{bend right}}%
	\only<all:4-7>{\bendtransition{BL}{R}{DBR}{near end}{bend left=50}}%
	\only<all:4-7>{\bendtransition{R}{BL}{PBR}{near start,below=-2mm}{bend right}}%
	
	\bendtransition{LL}{LL}{MB{\wildcard}{\wildcard}}{above=2mm,sloped=false,pos=.9}{loop left}
	\only<all:1-3>{\bendtransition{RR}{RR}{MB{\wildcard}{\wildcard}}{above=2mm,sloped=false,pos=.1}{loop right}}
	\only<all:1-5>{\bendtransition{AL}{AL}{MB{\wildcard}{\wildcard}}{}{loop above}}
	\only<all:1-5>{\bendtransition{AR}{AR}{MB{\wildcard}{\wildcard}}{}{loop above}}
	\only<all:6-9>{\bendtransition{A}{A}{M{\wildcard}{\wildcard}{\wildcard}}{}{loop above}}
	\bendtransition{LR}{LR}{MB{\wildcard}{\wildcard}}{}{loop above}
	\only<all:1-3>{\bendtransition{RL}{RL}{MB{\wildcard}{\wildcard}}{}{loop above}}
	\only<all:1-7>{\bendtransition{BL}{BL}{MB{\wildcard}{\wildcard}}{below=-1mm}{loop below}}
	\only<all:8-9>{\bendtransition{B}{B}{M{\wildcard}{\wildcard}{\wildcard}}{below=-1mm}{loop below}}
	\only<all:4->{\bendtransition{R}{R}{M{\wildcard}{\wildcard}{\wildcard}}{above=2mm,sloped=false}{loop right}}
	\only<all:1-7>{\bendtransition{BR}{BR}{MB{\wildcard}{\wildcard}}{below=-1mm}{loop below}}

	\only<all:10->{\bendtransition{I}{R}{D{\wildcard}R}{}{bend left}}
	\only<all:10->{\bendtransition{R}{I}{P{\wildcard}R}{}{bend left}}
	\only<all:10->{\bendtransition{I}{I}{M{\wildcard}{\wildcard}{\wildcard}}{}{loop above}}
	\only<all:10->{\bendtransition{LR}{I}{PBL}{}{bend left}}
	\only<all:10->{\bendtransition{I}{LR}{DBL}{}{bend left}}
	\only<all:10->{\bendtransition{LL}{I}{P{\wildcard}L}{}{bend right=55}}
	\only<all:10->{\bendtransition{I}{LL}{D{\wildcard}L}{below=-2mm}{bend left=80}}
  \end{tikzpicture}
}

\newcommand{\picshrinkexampleoneresult}{%
  \begin{tikzpicture}[->,>=stealth,looseness=.5,,bend angle=20]
  	\vertex{0,0}{LL}
  	\vertex{2,0}{LR}
	\vertex{4,0}{I}
	\goalvertex{7,0}{R}
	
	\draw[transition,->,>=stealth](2.6,0.6) -- (LR);
	\bendtransition{LL}{LR}{MALR}{}{bend left}
	\bendtransition{LR}{LL}{MARL}{}{bend left}
	\bendtransition{LL}{LL}{MB{\wildcard}{\wildcard}}{above=2mm,sloped=false}{loop left}
	\bendtransition{LR}{LR}{MB{\wildcard}{\wildcard}}{}{loop above}
	\bendtransition{R}{R}{M{\wildcard}{\wildcard}{\wildcard}}{above=2mm,sloped=false}{loop right}
	\bendtransition{I}{R}{D{\wildcard}R}{}{bend left}
	\bendtransition{R}{I}{P{\wildcard}R}{}{bend left}
	\bendtransition{I}{I}{M{\wildcard}{\wildcard}{\wildcard}}{}{loop above}
	\bendtransition{LR}{I}{PBL}{}{bend left}
	\bendtransition{I}{LR}{DBL}{}{bend left}
	\bendtransition{LL}{I}{P{\wildcard}L}{}{bend right=55}
	\bendtransition{I}{LL}{D{\wildcard}L}{below=-2mm}{bend left=80}
  \end{tikzpicture}
}

\newcommand{\picshrinkexampleoneresultdifferent}{%
  \begin{tikzpicture}[->,>=stealth,looseness=.5,,bend angle=20]
  	\vertex{0,-1}{LL}
  	\vertex{0,1}{LR}
	\vertex{3,0}{I}
	\goalvertex{6,0}{R}
	
	\draw[transition,->,>=stealth](0.6,1.6) -- (LR);
	\bendtransition{LL}{LR}{MALR}{}{bend left}
	\bendtransition{LR}{LL}{MARL}{}{bend left}
	\bendtransition{LL}{LL}{MB{\wildcard}{\wildcard}}{above=2mm,sloped=false}{loop left}
	\bendtransition{LR}{LR}{MB{\wildcard}{\wildcard}}{above=2mm,sloped=false}{loop left}
	\bendtransition{R}{R}{M{\wildcard}{\wildcard}{\wildcard}}{above=2mm,sloped=false}{loop right}
	\bendtransition{I}{R}{D{\wildcard}R}{}{bend left}
	\bendtransition{R}{I}{P{\wildcard}R}{}{bend left}
	\bendtransition{I}{I}{M{\wildcard}{\wildcard}{\wildcard}}{}{loop above}
	\bendtransition{LR}{I}{PBL}{}{bend left=40}
	\bendtransition{I}{LR}{DBL}{below=-2mm}{bend right}
	\bendtransition{LL}{I}{P{\wildcard}L}{}{bend right}
	\bendtransition{I}{LL}{D{\wildcard}L}{below=-2mm}{bend left=40}
  \end{tikzpicture}
}

\newcommand{\picproductfinal}[1][]{%
  \begin{tikzpicture}[->,>=stealth,looseness=.5,,bend angle=20,#1]
  	\vertex{-.5,2.5}{LRL}
  	\vertex{1,1.5}{LRR}
  	\vertex{-.5,-1.5}{LLL}
  	\vertex{1,-2.5}{LLR}
  	\vertex{3,0.5}{IL}
  	\vertex{4.5,-.5}{IR}
  	\goalvertex{6.5,0.5}{RL}
  	\goalvertex{8,-.5}{RR}

	\draw[transition,->,>=stealth](1.6,1.1) -- (LRR);

	\bendtransition{LRL}{LRR}{MBLR}{below=-2mm}{bend left}
	\bendtransition{LRR}{LRL}{MBRL}{below=-2mm}{bend left}
	\bendtransition{LLL}{LLR}{MBLR}{below=-2mm}{bend left}
	\bendtransition{LLR}{LLL}{MBRL}{below=-2mm}{bend left}
	\bendtransition{IL}{IR}{MBLR}{}{bend left}
	\bendtransition{IR}{IL}{MBRL}{}{bend left}
	\bendtransition{RL}{RR}{MBLR}{}{bend left}
	\bendtransition{RR}{RL}{MBRL}{}{bend left}

	\bendtransition{IL}{RL}{DAR}{}{bend left=40}
	\bendtransition{RL}{IL}{PAR}{below=-2mm}{bend right}
	\bendtransition{IR}{RR}{D{\wildcard}R}{}{bend right}
	\bendtransition{RR}{IR}{P{\wildcard}R}{below=-2mm}{bend left=40}
	\bendtransition{LLL}{IL}{P{\wildcard}L}{near end,below=-2mm}{bend left}
	\bendtransition{IL}{LLL}{D{\wildcard}L}{below=-2mm,near start}{bend left}
	\bendtransition{LLR}{IR}{PAL}{}{bend right}
	\bendtransition{IR}{LLR}{DAL}{below=-2mm}{bend left=40}

	\bendtransition{LLL}{LRL}{MALR}{}{bend left=50}
	\bendtransition{LRL}{LLL}{MARL}{rotate=180,below=0mm}{bend right}
	\bendtransition{LLR}{LRR}{MALR}{near end}{bend left}
	\bendtransition{LRR}{LLR}{MARL}{rotate=180,above=-2mm,near start}{bend left}

	\bendtransition{LRL}{IL}{PBL}{}{bend left=60}
	\bendtransition{IL}{LRL}{DBL}{}{bend right=30}
	\bendtransition{IL}{IL}{MA{\wildcard}{\wildcard}}{below=-1mm}{loop below}
	\bendtransition{IR}{IR}{MA{\wildcard}{\wildcard}}{below=-1mm}{loop below}
	\bendtransition{RR}{RR}{MA{\wildcard}{\wildcard}}{below=-1mm}{loop below}
	\bendtransition{RL}{RL}{MA{\wildcard}{\wildcard}}{}{loop above}
  \end{tikzpicture}
}

\newcommand{\picshrinkexampletworesult}{%
  \begin{tikzpicture}[->,>=stealth,looseness=.5,,bend angle=20]
  	\vertex{0,0}{LRR}
    \node[vertex interior] at (2,0) (LLL) {};
    \node[vertex border] at (2,0) {\parbox{4mm}{\tiny LLL\\LRL\\LLR}};
	\vertex{4,0}{I}
	\goalvertex{6,0}{R}
	
	\draw[transition,->,>=stealth](-0.8,0) -- (LRR);
	\bendtransition{I}{I}{M{\wildcard}{\wildcard}{\wildcard}}{}{loop above}
	\bendtransition{R}{R}{M{\wildcard}{\wildcard}{\wildcard}}{}{loop above}
	\bendtransition{LLL}{LLL}{M{\wildcard}{\wildcard}{\wildcard}}{}{loop above}

	\bendtransition{LRR}{LLL}{M{\wildcard}RL}{}{bend left}
	\bendtransition{LLL}{LRR}{M{\wildcard}LR}{below=-2mm}{bend left}
	\bendtransition{LLL}{I}{P{\wildcard}L}{}{bend left}
	\bendtransition{I}{LLL}{D{\wildcard}L}{below=-2mm}{bend left}
	\bendtransition{I}{R}{D{\wildcard}R}{}{bend left}
	\bendtransition{R}{I}{P{\wildcard}R}{below=-2mm}{bend left}
  \end{tikzpicture}
}

%% \author[M.~Helmert, G.~R\"oger]{Malte Helmert and Gabriele R\"oger}
%% abstraction pictures by Gabi

\begin{document}
\subtitle{12.~Planning as search: merge-and-shrink abstractions}
\date{January 13th, 2012}
\maketitles

\section{Motivation}
\subsection[PDB limitations]{Limitations of pattern databases}

\begin{frame}{Beyond pattern databases}
  \begin{itemize}
  \item Despite their popularity, pattern databases have some
    \alert{fundamental limitations} ($\leadsto$ example on next
    slides).
  \item In this chapter, we study a class of
    abstractions called \alert{merge-and-shrink abstractions}.
  \item Merge-and-shrink abstractions can be seen as a \\
    \alert{proper generalization} of pattern databases.
    \begin{itemize}
    \item They can do everything that pattern databases can do
      (modulo polynomial extra effort).
    \item They can do some things that pattern databases cannot.
    \end{itemize}
  \item Initial experiments with merge-and-shrink abstractions have
    shown very promising results.
  \item They have provably greater \alert{representational power} than
    pattern databases for many common planning domains.
  \end{itemize}
\end{frame}

\begin{frame}{Back to the running example}
  \begin{center}
    \picfulltransitiongraph
  \end{center}
  Logistics problem with one package, two trucks, two locations:
  \begin{itemize}
  \item state variable \textblue{package}: $\{L,R,A,B\}$
  \item state variable \textblue{truck A}: $\{L,R\}$
  \item state variable \textblue{truck B}: $\{L,R\}$
  \end{itemize}
\end{frame}

\begin{frame}{Example: projection}
  Project to $\{\text{\textblue{package}}\}$:
  \begin{center}
    \picprojectionpackage
  \end{center}
\end{frame}

\begin{frame}{Example: projection (2)}
   Project to $\{\text{\textblue{package}},
   \text{\textblue{truck A}}\}$:
   \begin{center}
     \picprojectionpackagetruck
   \end{center}
\end{frame}

\begin{frame}{Limitations of projections}
  How accurate is the PDB heuristic?
  \begin{itemize}
  \item consider \alert{generalization of the example}: \\
    $N$ trucks, $M$ locations (fully connected), still one package
  \item consider \alert{any} pattern that is proper subset of
    variable set $V$
  \item $h(s_0) \le 2$ $\leadsto$ \alert{no better} than atomic
    projection to $\text{\textblue{package}}$
  \end{itemize}

  \bigskip

  These values cannot be improved by maximizing over several patterns
  or using additive patterns.

  \bigskip

  \alert{Merge-and-shrink abstractions} can represent heuristics with
  $h(s_0) \ge 3$ for tasks of this kind of any size.

  Time and space requirements are \alert{polynomial in $N$ and $M$}.
\end{frame}

\subsection{Main ideas}

\begin{frame}{Merge-and-shrink abstractions: main idea}
  \begin{block}{Main idea of merge-and-shrink abstractions}
    (due to Dr\"ager, Finkbeiner \& Podelski, 2006): \medskip

    Instead of \alert{perfectly} reflecting \alert{a few} state
    variables, \\
    reflect \alert{all} state variables, but in a \alert{potentially
      lossy} way.
  \end{block}
\end{frame}

\begin{frame}{The need for succinct abstraction mappings}
  \begin{itemize}
  \item One major difficulty for non-PDB abstractions is to
    \alert{succinctly represent the abstraction mapping}.
  \item For pattern databases, this is easy because the abstraction
    mappings -- projections -- are very \alert{structured}.
  \item For less rigidly structured abstraction mappings, we need
    another idea.
  \end{itemize}
\end{frame}

\begin{frame}{Merge-and-shrink abstractions: idea}
  \begin{itemize}
  \item The main idea underlying merge-and-shrink abstractions is
    that given two abstractions $\mathcal A$ and $\mathcal A'$, we can
    \alert{merge} them into a new \alert{product abstraction}.
  \item The product abstraction \alert{captures all information} of
    both abstractions and can be \alert{better informed than either}.
  \item It can even be better informed than their \alert{sum}.
  \item By merging a set of very simple abstractions, we can in theory
    represent \alert{arbitrary} abstractions of an {\sasplus} task.
  \item In practice, due to memory limitations, such abstractions can
    become too large. In that case, we can \alert{shrink} them by
    abstracting them further using \alert{any abstraction} on an
    intermediate result, then \alert{continue the merging process}.
  \end{itemize}
\end{frame}

\subsection{Running example}

\begin{frame}{Running example: explanations}
  \begin{itemize}
  \item \alert{Atomic projections} -- projections to a single state
    variable -- play an important role in this chapter.
  \item Unlike previous chapters, \alert{transition labels} are
    critically important in this chapter.
  \item Hence we now look at the transition systems for atomic
    projections of our example task, including transition labels.
  \item We abbreviate operator names as in these examples:
    \begin{itemize}
    \item \hilite{MALR:} \alert{m}ove truck \alert{A} from
      \alert{l}eft to \alert{r}ight
    \item \hilite{DAR:} \alert{d}rop package from truck \alert{A} at
      \alert{r}ight location
    \item \hilite{PBL:} \alert{p}ick up package with truck \alert{B} at
      \alert{l}eft location
    \end{itemize}
  \item We abbreviate parallel arcs with \alert{commas} and
    \alert{wildcards ($\star$)} in the labels as in these examples:
    \begin{itemize}
    \item \hilite{PAL, DAL:} two parallel arcs labeled \hilite{PAL}
      and \hilite{DAL}
    \item \hilite{MA$\star\star$:} two parallel arcs labeled
      \hilite{MALR} and \hilite{MARL}
    \end{itemize}
  \end{itemize}
\end{frame}

\begin{frame}{Running example: atomic projection for package}
  $\mathcal T^{\pi_{\{\hilite{\text{package}}\}}}$:
  \begin{center}
    \picatomicprojectionpackage
  \end{center}
\end{frame}

\begin{frame}{Running example: atomic projection for truck A}
  $\mathcal T^{\pi_{\{\hilite{\text{truck A}}\}}}$:
  \begin{center}
    \picatomicprojectiontrucka
  \end{center}
\end{frame}

\begin{frame}{Running example: atomic projection for truck B}
  $\mathcal T^{\pi_{\{\hilite{\text{truck B}}\}}}$:
  \begin{center}
    \picatomicprojectiontruckb
  \end{center}
\end{frame}

\section{Synchronized products}
\subsection{Definition}

\begin{frame}{Synchronized product of transition systems}
  \begin{definition}[synchronized product of transition systems]
    For $i \in \{1, 2\}$, let
    $\mathcal T_i = \langle S_i, L, T_i, {s_0}_i, {S_\star}_i\rangle $ be
    transition systems with identical label set.

    \smallskip

    The \alert{synchronized product} of $\mathcal T_1$ and $\mathcal
    T_2$, in symbols \alert{$\mathcal T_1 \otimes \mathcal T_2$}, is
    the transition system $\mathcal T_{\otimes} = \langle S_{\otimes},
    L, T_{\otimes}, {s_0}_{\otimes}, {S_\star}_{\otimes}\rangle$ with
    \begin{itemize}
    \item $S_{\otimes} := S_1 \times S_2$
    \item $T_{\otimes} := \{
      \langle \langle s_1, s_2\rangle, l, \langle t_1,
      t_2\rangle\rangle \mid
      \only<beamer>{\begin{array}[t]{@{}l@{}}}
      \langle s_1, l, t_1\rangle \in T_1 \text{~and~}
      \only<beamer>{\\}
      \langle s_2, l, t_2\rangle \in T_2
      \}
      \only<beamer>{\end{array}}$
    \item ${s_0}_{\otimes} := \langle{s_0}_1,{s_0}_2\rangle$
    \item ${S_\star}_{\otimes} := {S_\star}_1 \times {S_\star}_2$
    \end{itemize}
  \end{definition}
\end{frame}

\begin{frame}{Synchronized product of functions}
  \begin{definition}[synchronized product of functions]
    Let $\alpha_1: S \to S_1$ and $\alpha_2: S \to S_2$ be
    functions with identical domain.

    \smallskip

    The \alert{synchronized product} of $\alpha_1$ and $\alpha_2$,
    in symbols \alert{$\alpha_1 \otimes \alpha_2$},
    is the function $\alpha_{\otimes}: S \to S_1 \times S_2$
    defined as $\alpha_{\otimes}(s) = \langle \alpha_1(s),
    \alpha_2(s)\rangle$.
  \end{definition}
\end{frame}

\subsection{Example}

\begin{frame}{Example: synchronized product}
  $\mathcal T^{\pi_{\{\hilite{\text{package}}\}}} \otimes
  \mathcal T^{\pi_{\{\hilite{\text{truck A}}\}}}$:
  \begin{center}
    \picproductpackagetrucka
  \end{center}
\end{frame}

\begin{frame}{Example: computation of synchronized product}
  $\mathcal T^{\pi_{\{\hilite{\text{package}}\}}} \otimes
  \mathcal T^{\pi_{\{\hilite{\text{truck A}}\}}}$:
  \alert{\only<all:2>{$S_{\otimes} = S_1 \times S_2$}%
    \only<all:3>{${s_0}_{\otimes} = \langle{s_0}_1,{s_0}_2\rangle$}%
    \only<all:4>{${S_\star}_{\otimes} = {S_\star}_1 \times {S_\star}_2$}%
    \only<all:5->{$T_{\otimes} := \{\langle \langle s_1, s_2\rangle,
        l, \langle t_1, t_2\rangle\rangle \mid \dots\}$}}
  \begin{center}
    \picproductpackagetruckaillustration
  \end{center}
\end{frame}

\subsection{Properties}

\begin{frame}{Synchronized products are abstractions}
  \begin{theorem}[synchronized products are abstractions]
    For $i \in \{1, 2\}$, let $\mathcal T_i$ be an abstraction of
    transition system $\mathcal T$ with abstraction mapping
    $\alpha_i$ such that $\alpha_1 \otimes \alpha_2$ is surjective.

    \smallskip

    Then $\mathcal T_{\otimes} := \mathcal T_1 \otimes \mathcal T_2$
    is an abstraction of $\mathcal T$ with abstraction mapping
    $\alpha_{\otimes} := \alpha_1 \otimes \alpha_2$, and $\langle
    \mathcal T_{\otimes}, \alpha_{\otimes}\rangle$ is a refinement of
    $\langle \mathcal T_1, \alpha_1\rangle$
    and of $\langle \mathcal T_2, \alpha_2\rangle$.
  \end{theorem}
  \hilite{Remark:} If $\alpha_1 \otimes \alpha_2$ is not surjective,
  then the proof also holds if we restrict $\mathcal T_{\otimes}$ to
  the states in the image of $\alpha_1 \otimes \alpha_2$.
\end{frame}

\begin{frame}{Synchronized products are abstractions (ctd.)}
  \begin{proofstart}
    We prove the first part. The refinement property is then easy to
    see: the mapping $\langle s_1, s_2\rangle \mapsto s_i$ is a strict
    homomorphism from $\mathcal T_{\otimes}$ to $\mathcal T_i$ for $i
    \in \{1, 2\}$.

    \pause
    \medskip

    To show that $\mathcal T_{\otimes}$ is an abstraction of $\mathcal
    T$ with mapping $\alpha_{\otimes}$, we need to show that
    $\alpha_{\otimes}$ is surjective and preserves initial states,
    goal states and transitions.

  \end{proofstart}
\end{frame}

\begin{frame}{Synchronized products are abstractions (ctd.)}
  \begin{proofmid}
    Let $\mathcal T = \langle S, L, T, s_0, S_\star\rangle$, and let
    $\mathcal T_i = \langle S_i, L, T_i, {s_0}_i, {S_\star}_i\rangle$ for
    $i \in \{1, 2, \otimes\}$.

    \begin{itemize}
    \item \hilite{$\alpha_{\otimes}$ surjective:} This is given in the
      premise.
      \pause

    \item \hilite{$\alpha_{\otimes}$ preserves the initial state:}
      
      $\alpha_{\otimes}(s_0) = \langle \alpha_1(s_0), \alpha_2(s_0)\rangle$
      \quad (definition of $\alpha_{\otimes}$)

      $\phantom{\alpha_{\otimes}(s_0)} = \langle {s_0}_1, {s_0}_2 \rangle$
      \quad (abstraction property for $\mathcal T_1$, $\mathcal T_2$)

      $\phantom{\alpha_{\otimes}(s_0)} = {s_0}_{\otimes}$
      \quad (definition of ${s_0}_{\otimes}$)
    \end{itemize}
  \end{proofmid}
\end{frame}

\begin{frame}{Synchronized products are abstractions (ctd.)}
  \begin{proofend}
    \begin{itemize}
    \item \hilite{$\alpha_{\otimes}$ preserves goal states:}
      
      Let $s \in S_\star$. Then

      $\alpha_{\otimes}(s) = \langle \alpha_1(s), \alpha_2(s)\rangle$
      \quad (definition of $\alpha_{\otimes}$)

      $\phantom{\alpha_{\otimes}(s)} \in {S_\star}_1 \times {S_\star}_2$
      \quad (abstraction property for $\mathcal T_1$, $\mathcal T_2$)

      $\phantom{\alpha_{\otimes}(s)} = {S_\star}_{\otimes}$
      \quad (definition of ${S_\star}_{\otimes}$)
      \pause

    \item \hilite{$\alpha_{\otimes}$ preserves transitions:}

      Let $\langle s, l, t\rangle \in T$.

      $\leadsto$ $\langle \alpha_1(s), l, \alpha_1(t)\rangle \in T_1$,
      $\langle \alpha_2(s), l, \alpha_2(t)\rangle \in T_2$

      $\leadsto$ $\langle \langle \alpha_1(s), \alpha_2(s)\rangle, l,
      \langle \alpha_1(t), \alpha_2(t)\rangle\rangle \in T_{\otimes}$

      $\leadsto$ $\langle \alpha_{\otimes}(s), l,
      \alpha_{\otimes}(t)\rangle \in T_{\otimes}$
    \end{itemize}
  \end{proofend}
\end{frame}

\begin{frame}{Preserving strict homomorphisms}
  \begin{itemize}
  \item It would be very nice if we could also prove that if $\mathcal
    T_1$ and $\mathcal T_2$ are \alert{strictly homomorphic} abstractions,
    then so is $\mathcal T_1 \otimes \mathcal T_2$.
  \item However, this is \alert{not true} in general.
  \item It is \alert{not even} true for {\sasplus} tasks.
  \item However, there is an important \alert{sufficient condition}
    for preserving the strict homomorphism property.
  \end{itemize}
\end{frame}

\begin{frame}{Synchronized products and strict homomorphisms}
  \begin{theorem}[synchronized products and strict homomorphisms]
    Let $\Pi$ be an {\sasplus} planning task with variable set $V$,
    and let $V_1$ and $V_2$ be disjoint subsets of $V$.

    \smallskip

    For $i \in \{1, 2\}$, let $\mathcal T_i$ be a strictly homomorphic
    abstraction of $\mathcal T(\Pi)$ with mapping $\alpha_i$ such
    that $\langle \mathcal T_i, \alpha_i\rangle$ is a coarsening of
    $\langle \mathcal T^{\pi_{V_i}}, \pi_{V_i}\rangle$.

    \medskip
    
    Then $\alpha_{\otimes} := \alpha_1 \otimes \alpha_2$ is surjective
    and $\mathcal T_{\otimes} := \mathcal T_1 \otimes \mathcal T_2$ is
    a strictly homomorphic abstraction of $\mathcal T(\Pi)$ with mapping
    $\alpha_{\otimes}$.
  \end{theorem}

  \medskip

  \begin{corollary}[Synchronized products of projections]
    Let $\Pi$ be an {\sasplus} planning task with variable set $V$,
    and let $V_1$ and $V_2$ be disjoint subsets of $V$.

    Then $\mathcal T^{\pi_{V_1}} \otimes \mathcal T^{\pi_{V_2}} \sim
    \mathcal T^{\pi_{V_1 \cup V_2}}$.
  \end{corollary}

  (Proofs omitted.)

  %% \medskip

  %% \hilite{Note:} In this special case, we do not need to
  %% \alert{require} that $\alpha_{\otimes}$ is surjective but can
  %% \alert{conclude} it from the other premises.
\end{frame}

%% \begin{frame}{Synchronized products of projections}
%%   \begin{corollary}[Synchronized products of projections]
%%     Let $\Pi$ be an {\sasplus} planning task with variable set $V$,
%%     and let $V_1$ and $V_2$ be disjoint subsets of $V$.

%%     Then $\mathcal T^{\pi_{V_1}} \otimes \mathcal T^{\pi_{V_2}} \sim
%%     \mathcal T^{\pi_{V_1 \cup V_2}}$.
%%   \end{corollary}
%%   \begin{proof}
%%     \begin{itemize}
%%     \item By the theorem, $\mathcal T_{\otimes} := \mathcal T^{\pi_{V_1}}
%%       \otimes \mathcal T^{\pi_{V_2}}$ is a strictly homomorphic abstraction of
%%       $\mathcal T(\Pi)$ with mapping $\pi_{V_1} \otimes \pi_{V_2}$.
%%     \item $\mathcal T^{\pi_{V_1 \cup V_2}}$ is a strictly homomorphic
%%       abstraction of $\mathcal T(\Pi)$ with mapping $\pi_{V_1 \cup
%%         V_2}$.
%%     \end{itemize}
%%     $\pi_{V_1} \otimes \pi_{V_2}$ and $\pi_{V_1 \cup V_2}$ are
%%     identical functions up to renaming of abstract states, and
%%     strictly homomorphic abstractions are uniquely determined by the
%%     abstraction function, so the abstractions must be isomorphic.
%%   \end{proof}
%% \end{frame}

\begin{frame}{Example: product for disjoint projections}
  $\mathcal T^{\pi_{\{\hilite{\text{package}}\}}} \otimes
  \mathcal T^{\pi_{\{\hilite{\text{truck A}}\}}}
  \sim
  \mathcal T^{\pi_{\{\hilite{\text{package}}, \hilite{\text{truck A}}\}}}$:
  \begin{center}
    \picproductpackagetrucka
  \end{center}
\end{frame}

\begin{frame}{Recovering $\mathcal T(\Pi)$ from the atomic projections}
  \begin{itemize}
  \item By repeated application of the corollary, we can recover
    \alert{all pattern database abstractions} of an {\sasplus} planning
    task from the abstractions for atomic projections.
  \item In particular, by computing the product of \alert{all} atomic
    projections, we can recover the abstraction for the
    \alert{identity abstraction} $\text{id} = \pi_V$.
  \end{itemize}
  \begin{corollary}[Recovering $\mathcal T(\Pi)$ from the atomic
      projections]
    Let $\Pi$ be an {\sasplus} planning task with variable set $V$.

    Then $\mathcal T(\Pi) \sim \bigotimes_{v \in V} \mathcal T^{\pi_{\{v\}}}$.
  \end{corollary}
  \begin{itemize}
  \item This is an important result because it shows that the
    abstractions for atomic projections \alert{contain all
      information} of an {\sasplus} task.
  \end{itemize}
\end{frame}

\section{Algorithm}
\subsection{Merge steps and shrink steps}

\begin{frame}{Generic merge-and-shrink abstractions: outline}
  Using the results from the previous section, we can develop the
  ideas of a \alert{generic abstraction computation procedure} that
  \alert{takes all state variables into account}:
  \begin{itemize}
  \item \alert{Initialization step:} Compute all abstract transition
    systems for atomic projections to form the initial abstraction
    collection.
  \item \alert{Merge steps:} Combine two abstractions
    in the collection by replacing them with their synchronized
    product. (Stop once only one abstraction is left.)
  \item \alert{Shrink steps:} If the abstractions in the collection
    are too large to compute their synchronized product, make them
    smaller by abstracting them further (applying an arbitrary
    strict homomorphism to them).
  \end{itemize}
  We explain these steps with our running example.
\end{frame}

\begin{frame}{Initialization step: atomic projection for package}
  $\mathcal T^{\pi_{\{\hilite{\text{package}}\}}}$:
  \begin{center}
    \picatomicprojectionpackage
  \end{center}
\end{frame}

\begin{frame}{Initialization step: atomic projection for truck A}
  $\mathcal T^{\pi_{\{\hilite{\text{truck A}}\}}}$:
  \begin{center}
    \picatomicprojectiontrucka
  \end{center}
\end{frame}

\begin{frame}{Initialization step: atomic projection for truck B}
  $\mathcal T^{\pi_{\{\hilite{\text{truck B}}\}}}$:
  \begin{center}
    \picatomicprojectiontruckb
  \end{center}
  current collection: $\{
    \mathcal T^{\pi_{\{\hilite{\text{package}}\}}},
    \mathcal T^{\pi_{\{\hilite{\text{truck A}}\}}},
    \mathcal T^{\pi_{\{\hilite{\text{truck B}}\}}}\}$
\end{frame}

\begin{frame}{First merge step}
  $\mathcal T_1 := \mathcal T^{\pi_{\{\hilite{\text{package}}\}}}
  \otimes \mathcal T^{\pi_{\{\hilite{\text{truck A}}\}}}$:
  \begin{center}
    \picproductpackagetrucka[scale=0.94]
  \end{center}
  current collection: $\{
    \mathcal T_1,
    \mathcal T^{\pi_{\{\hilite{\text{truck B}}\}}}\}$  
\end{frame}

\begin{frame}{Need to simplify?}
  \begin{itemize}
  \item If we have sufficient memory available, we can now compute
    $\mathcal T_1 \otimes \mathcal T^{\pi_{\{\hilite{\text{truck
            B}}\}}}$, which would recover the complete transition
    system of the task.
  \item However, to illustrate the general idea, let us assume that we
    do not have sufficient memory for this product.
  \item More specifically, we will assume that after each product
    operation we need to reduce the result abstraction to \alert{four
      states} to obey memory constraints.
  \item So we need to reduce $\mathcal T_1$ to four states. We have a
    lot of leeway in deciding \alert{how exactly} to abstract
    $\mathcal T_1$.
  \item In this example, we simply use an abstraction that leads to a
    good result in the end.
  \end{itemize}
\end{frame}

\begin{frame}{First shrink step}
  $\mathcal T_2 := \text{some abstraction of~} \mathcal T_1$
  \begin{center}
    \picshrinkexampleone[scale=0.94]
  \end{center}
  \only<all:10>{current collection: $\{\mathcal T_2,
      \mathcal T^{\pi_{\{\hilite{\text{truck B}}\}}}\}$}
\end{frame}

\begin{frame}{Second merge step}
  $\mathcal T_3 := \mathcal T_2
  \otimes \mathcal T^{\pi_{\{\hilite{\text{truck B}}\}}}$:
  \begin{center}
    \picproductfinal[scale=0.94]
  \end{center}
  current collection: $\{\mathcal T_3\}$
\end{frame}

\begin{frame}{Another shrink step?}
  \begin{itemize}
  \item Normally we could stop now and use the distances in the final
    abstraction as our heuristic function.
  \item However, if there were further state variables to integrate,
    we would simplify further, \eg\ leading to the following
    abstraction (again with four states):
  \end{itemize}
  \begin{center}
    \picshrinkexampletworesult
  \end{center}
  \begin{itemize}
  \item We get a heuristic value of 3 for the initial state,
    \alert{better than any PDB heuristic} that is a proper
    abstraction.
  \item The example generalizes to more locations and trucks, even if
    we stick to the size limit of $4$ (after merging).
  \end{itemize}
\end{frame}

\subsection{Abstraction mapping}

\begin{frame}{How to represent the abstraction mapping?}
  \hilite{Idea:} the computation of the abstraction mapping
  follows the sequence of product computations
  \begin{itemize}
  \item For the \alert{atomic abstractions} for $\pi_{\{v\}}$, we
    generate a \alert{one-dimensional table} that denotes which value
    in $\mathcal D_v$ corresponds to which abstract state.
  \item During the \alert{merge} (product) step $\mathcal A :=
    \mathcal A_1 \otimes \mathcal A_2$, we generate a
    \alert{two-dimensional table} that denotes which pair of states of
    $\mathcal A_1$ and $\mathcal A_2$ corresponds to which state of
    $\mathcal A$.
  \item During the \alert{shrink} (abstraction) steps, we make sure to
    keep the table in sync with the abstraction choices.
  \end{itemize}
\end{frame}

\begin{frame}{How to represent the abstraction mapping? (ctd.)}
  \hilite{Idea:} the computation of the abstraction mapping
  follows the sequence of product computations
  \begin{itemize}
  \item Once we have computed the final abstraction, we compute all
    \alert{abstract goal distances} and store them in a
    \alert{one-dimensional table}.
  \item At this point, we can \alert{throw away} all the abstractions
    \\
    -- we just need to keep the tables.
  \item During \alert{search}, we do a sequence of table lookups to
    navigate from the atomic abstraction states to the final
    abstraction state and heuristic value

    $\leadsto$ $2|V|$ lookups, $O(|V|)$ time
    %% |V| lookups for the atomic projections, |V| - 1 for the merges,
    %% 1 for the heuristic value
  \end{itemize}
  Again, we illustrate the process with our running example.
\end{frame}

\begin{frame}{Abstraction mapping example: atomic abstractions}
  Computing abstraction mappings for the atomic abstractions is
  simple. Just number the states (domain values) consecutively
  and generate a table of references to the states:

  \begin{center}
    \only<all:1>{\picatomicprojectionpackage}%
    \only<all:2>{\picatomicprojectionpackagenumberedstates}%
  \end{center}

  \begin{overprint}
    \onslide<all:2>
    \begin{center}
      \begin{tabular}{cccc}
        $L$ & $R$ & $A$ & $B$ \\ \hline
        0   & 1   & 2   & 3
      \end{tabular}
    \end{center}
  \end{overprint}
\end{frame}

\begin{frame}{Abstraction mapping example: merge step}
  For product abstractions $\mathcal A_1 \otimes \mathcal A_2$, we
  again number the product states consecutively and generate a table
  that links state pairs of $\mathcal A_1$ and $\mathcal A_2$ to
  states of $\mathcal A$:

  \begin{columns}
    \begin{column}{7.2cm}
      \renewcommand{\bendtransition}{\bendtransitionnolabel}%
      \only<all:1>{\picproductpackagetrucka[scale=0.63]}%
      \only<all:2>{\picproductpackagetruckanumberedstates[scale=0.63]}%
    \end{column}
    \begin{column}{4.0cm}
      \only<all:2>{\footnotesize
      \begin{tabular}{c|cc}
        & $s_2 = 0$ & $s_2 = 1$ \\ \hline
        $s_1 = 0$ & 0 & 1 \\
        $s_1 = 1$ & 2 & 3 \\
        $s_1 = 2$ & 4 & 5 \\
        $s_1 = 3$ & 6 & 7 \\
      \end{tabular}}
    \end{column}
  \end{columns}
\end{frame}

\begin{frame}{Maintaining the mapping when shrinking}
  \begin{itemize}
  \item The hard part in representing the abstraction mapping is to
    keep it consistent when shrinking.
  \item In theory, this is easy to do:
    \begin{itemize}
    \item When combining states $i$ and $j$, arbitrarily use one of
      them (say $i$) as the number of the new state.
    \item Find all table entries in the table for this abstraction
      which map to the other state $j$ and change them to $i$.
    \end{itemize}
  \item However, doing a table scan each time two states are combined
    is very inefficient.
  \item Fortunately, there also is an efficient implementation which
    takes constant time per combination.
  \end{itemize}
\end{frame}

\begin{frame}{Maintaining the mapping efficiently}
  \begin{itemize}
  \item Associate each abstract state with a linked list, representing
    \alert{all table entries that map to this state}.
  \item Before starting the shrink operation, initialize the
    lists by scanning through the table, then \alert{discard the table}.
  \item While shrinking, when combining $i$ and $j$, \alert{splice the
    list elements of $j$ into the list elements of $i$}.
    \begin{itemize}
    \item For linked lists, this is a \alert{constant-time operation}.
    \end{itemize}
  \item Once shrinking is completed, renumber all abstract states so
    that there are no gaps in the numbering.
  \item Finally, regenerate the mapping table from the linked list
    information.
  \end{itemize}
\end{frame}

\begin{frame}{Abstraction mapping example: shrink step}
  \begin{overprint}
    \onslide<all:1>
    \alert{Representation before shrinking:}
    \onslide<all:2>
    \alert{1. Convert table to linked lists and discard it.}
    \onslide<all:3-11>
    \alert{2. When combining $i$ and $j$, splice $\textit{list}_j$
      into $\textit{list}_i$.}
    \onslide<all:12-13>
    \alert{3. Renumber abstract states consecutively.}
    \onslide<all:14->
    \alert{4. Regenerate the mapping table from the linked lists.}
  \end{overprint}

  \bigskip

  \begin{columns}[c]
    \begin{column}{7.2cm}%
      \begin{center}
        \renewcommand{\bendtransition}{\bendtransitionnolabel}%
        \picshrinkexampleonenumberedstates[scale=0.55]
      \end{center}
    \end{column}
    \begin{column}{4.0cm}%
      \only<all:2->{\footnotesize
        $\textit{list}_0 = \{(0, 0)\}$ \\
        $\textit{list}_1 = \{(0, 1)\}$ \\
        \alert<all:3-4>{$\textit{list}_2 =
          \only<all:-3>{\{(1, 0)\}}
          \only<all:4->{\{(1, 0), (1, 1)\}}
          $} \\
        \alert<all:3-4,12-13>{$\textit{list}_3 =
          \only<all:-3>{\{(1, 1)\}}
          \only<all:4-12>{\emptyset}
          \only<all:13->{\{\begin{array}[t]{@{}l@{}}
              (2, 0), (2, 1), \\ (3, 0), (3, 1)\}\end{array}}
          $} \\
        \alert<all:5-6,9-10,12-13>{$\textit{list}_4 =
          \only<all:-5>{\{(2, 0)\}}
          \only<all:6-9>{\{(2, 0), (2, 1)\}}
          \only<all:10-12>{\{\begin{array}[t]{@{}l@{}}
              (2, 0), (2, 1), \\ (3, 0), (3, 1)\}\end{array}}
          \only<all:13->{\emptyset}
          $} \\
        \alert<all:5-6>{$\textit{list}_5 =
          \only<all:-5>{\{(2, 1)\}}
          \only<all:6->{\emptyset}
          $} \\
        \alert<all:9-10>{%
          \alert<all:7-8>{$\textit{list}_6 =
            \only<all:-7>{\{(3, 0)\}}
            \only<all:8-9>{\{(3, 0), (3, 1)\}}
            \only<all:10->{\emptyset}
            $}} \\
        \alert<all:7-8>{$\textit{list}_7 =
          \only<all:-7>{\{(3, 1)\}}
          \only<all:8->{\emptyset}
          $}
      }
    \end{column}
  \end{columns}
  \begin{overprint}
    \onslide<all:1-2,15>
    \begin{columns}
      \begin{column}{7.2cm}
      \end{column}
      \begin{column}{4.0cm}
        \footnotesize
        \only<all:2>{\color{gray}}%
        \begin{tabular}{c|cc}
          & $s_2 = 0$ & $s_2 = 1$ \\ \hline
          $s_1 = 0$
          & \only<all:1-2>{0}\only<all:15>{0}
          & \only<all:1-2>{1}\only<all:15>{1}
          \\
          $s_1 = 1$
          & \only<all:1-2>{2}\only<all:15>{2}
          & \only<all:1-2>{3}\only<all:15>{2}
          \\
          $s_1 = 2$
          & \only<all:1-2>{4}\only<all:15>{3}
          & \only<all:1-2>{5}\only<all:15>{3}
          \\
          $s_1 = 3$
          & \only<all:1-2>{6}\only<all:15>{3}
          & \only<all:1-2>{7}\only<all:15>{3}
        \end{tabular}
      \end{column}
    \end{columns}
  \end{overprint}
\end{frame}

\begin{frame}{The final heuristic representation}
  \setlength{\tabcolsep}{5pt}
  At the end, our heuristic is represented by six tables:
  \begin{itemize}
  \item three one-dimensional tables for the atomic abstractions:
    
    \medskip

    \begin{footnotesize}
      \begin{tabular}{c|cccc}
        $T_{\text{\hilite{package}}}$ & $L$ & $R$ & $A$ & $B$ \\ \hline
        & 0   & 1   & 2   & 3
      \end{tabular}
      \quad
      \begin{tabular}{c|cc}
        $T_{\text{\hilite{truck A}}}$ & $L$ & $R$ \\ \hline
        & 0   & 1
      \end{tabular}
      \quad
      \begin{tabular}{c|cc}
        $T_{\text{\hilite{truck B}}}$ & $L$ & $R$ \\ \hline
        & 0   & 1
      \end{tabular}
    \end{footnotesize}
  \item two tables for the two merge and subsequent shrink steps:

    \medskip

    \begin{footnotesize}
      \begin{tabular}{c|cc}
        $T^1_{\text{\hilite{m\&s}}}$ & $s_2 = 0$ & $s_2 = 1$ \\ \hline
        $s_1 = 0$ & 0 & 1 \\
        $s_1 = 1$ & 2 & 2 \\
        $s_1 = 2$ & 3 & 3 \\
        $s_1 = 3$ & 3 & 3
      \end{tabular}
      \quad
      \begin{tabular}{c|cc}
        $T^2_{\text{\hilite{m\&s}}}$ & $s_2 = 0$ & $s_2 = 1$ \\ \hline
        $s_1 = 0$ & 1 & 1 \\
        $s_1 = 1$ & 1 & 0 \\
        $s_1 = 2$ & 2 & 2 \\
        $s_1 = 3$ & 3 & 3
      \end{tabular}
    \end{footnotesize}
  \item one table with goal distances for the final abstraction:

    \medskip

    \begin{footnotesize}
      \begin{tabular}{c|cccc}
        $T_{\hilite{h}}$  & $s = 0$ & $s = 1$ & $s = 2$ & $s = 3$ \\ \hline
        $h(s)$ & 3       & 2       & 1       & 0
      \end{tabular}
    \end{footnotesize}
  \end{itemize}

  \medskip

  Given a state $s = \{
    \text{\hilite{package}} \mapsto p,
    \text{\hilite{truck A}} \mapsto a,
    \text{\hilite{truck B}} \mapsto b\}$, \\
  its heuristic value is then looked up as:
  \begin{itemize}
  \item $h(s) =
    T_{\hilite{h}}[
      T^2_{\text{\hilite{m\&s}}}[
        T^1_{\text{\hilite{m\&s}}}[
          T_{\text{\hilite{package}}}[p],
          T_{\text{\hilite{truck A}}}[a]],
        T_{\text{\hilite{truck B}}}[b]]]$
  \end{itemize}
\end{frame}

\subsection{Concrete algorithm}

\begin{frame}{Towards a concrete algorithm}
  \begin{itemize}
  \item We have now described how merge-and-shrink abstractions work
    \alert{in general}.
  \item However, we have not said how exactly to decide
    \begin{itemize}
    \item \alert{which abstractions to combine} in a merge step and
    \item \alert{when and how to further abstract} in a shrink step.
    \end{itemize}
  \item There are \alert{many possibilities here} (just like there are
    many possible PDB heuristics).
  \item Only \alert{one} concrete method, called
    \alert{\hhhh}, has been explored so far in planning,
    which we will now discuss briefly.
  \end{itemize}
\end{frame}
    
\begin{frame}{Generic algorithm template}
  \begin{block}{Generic abstraction computation algorithm}
    \begin{tabular}{l}
      \textit{abs} := $\{\mathcal T^{\pi_{\{v\}}} \mid v \in V\}$ \\
      \textbf{while} \textit{abs} contains more than one
      abstraction: \\
      \hspace*{1cm}
      \textblue{select} $\mathcal A_1$, $\mathcal A_2$
      from \textit{abs} \\
      \hspace*{1cm}
      \textblue{shrink} $\mathcal A_1$ and/or $\mathcal A_2$
      until $\textit{size}(\mathcal A_1) \cdot
      \textit{size}(\mathcal A_2) \le N$ \\
      \hspace*{1cm} \textit{abs} := $\text{\textit{abs}}
      \setminus \{\mathcal A_1, \mathcal A_2\}
      \cup \{\mathcal A_1 \otimes \mathcal A_2\}$ \\
      \textbf{return} the remaining abstraction in \textit{abs}
    \end{tabular}
  \end{block}
  $N$: parameter bounding number of abstract states

  \bigskip

  \textblue{Questions for practical implementation:}
  \begin{itemize}
  \item Which abstractions to select?
    $\leadsto$ \alert{merging strategy}
  \item How to shrink an abstraction?
    $\leadsto$ \alert{shrinking strategy}
  \item How to choose $N$?
    $\leadsto$ usually: as high as memory allows
  \end{itemize}
\end{frame}

\begin{frame}{Merging strategy}
  \textblue{Which abstractions to select?}

  \begin{block}{\hhhh: Linear merging strategy}
    In each iteration after the first, choose the abstraction
    computed in the previous iteration as $\mathcal A_1$.

    $\leadsto$ fully defined by an ordering of atomic projections
  \end{block}

  \textblue{Rationale:} only maintains one ``complex'' abstraction
  at a time

  \medskip

  \begin{block}{\hhhh: Ordering of atomic projections}
    \begin{itemize}
    \item Start with a goal variable.
    \item Add variables that appear in preconditions of operators
      affecting previous variables.
    \item If that is not possible, add a goal variable.
    \end{itemize}
  \end{block}

  \textblue{Rationale:} increases $h$ quickly
  (cf.\ causal graph criteria for growing patterns)
\end{frame}

\begin{frame}{Shrinking strategy}
  \textblue{Which abstractions to shrink?}
  
  \begin{block}{\hhhh: only shrink the product}
    If at all possible, don't shrink atomic abstractions,
    but only product abstractions.
  \end{block}

  \textblue{Rationale:} Product abstractions are more likely to
  contain significant redundancies and symmetries.
\end{frame}

\begin{frame}{Shrinking strategy (ctd.)}
  \textblue{How to shrink an abstraction?}
  
  \begin{block}{\hhhh: $f$-preserving shrinking strategy}
    Repeatedly combine abstract states with \\
    \alert{identical} abstract goal distances (\alert{$h$ values}) and \\
    \alert{identical} abstract initial state distances (\alert{$g$
      values}).
  \end{block}
  
  \textblue{Rationale:}
  preserves heuristic value and overall graph shape

  \begin{block}{\hhhh: Tie-breaking criterion}
    Prefer combining states where \alert{$g+h$ is high}.

    In case of ties, combine states where \alert{$h$ is high}.
  \end{block}
  \textblue{Rationale:} states with high $g+h$ values are less likely to
  be explored by \astar, so inaccuracies there matter less
\end{frame}

\section{Conclusion}
\subsection[Properties]{Theoretical properties and practical performance}

\begin{frame}{Properties of merge-and-shrink abstractions}
  \begin{itemize}
  \item We conclude by briefly mentioning a number of
    \alert{theoretical} properties of merge-and-shrink abstractions
    (without proof).
  \item While these theoretical results are interesting, heuristics in
    planning usually need to be justified by good \alert{empirical
      performance}.
  \item Regarding empirical performance, initial results for {\hhhh}
    are \alert{very encouraging}, outperforming pattern databases (and
    all other admissible heuristics) on a number of benchmark domains.
  \item However, merge-and-shrink abstractions are \alert{not nearly
    as well studied} (and understood) as pattern databases, so the
    \alert{jury is still out}.
  \end{itemize}
\end{frame}

\begin{frame}{Theoretical properties: as good as PDBs}
  \begin{block}{As powerful as PDBs}
    \alert{Pattern database heuristics} are a \alert{special case} of
    our abstraction heuristics, and arise naturally as a side product.
  \end{block}

  \begin{itemize}
  \item More precisely, PDB heuristics are merge-and-shrink
    abstractions without shrink steps (terminating heuristic
    computation as soon as space runs out).
  \item However, specialized PDB algorithms are faster than the
    generic merge-and-shrink algorithm.
  \item This performance difference is only \alert{polynomial}, but
    this does not mean that it does not matter in practice!
  \item Still, this shows that \alert{representational power} is
    \alert{at least as large} as that of PDB heuristics.
  \end{itemize}
\end{frame}

\begin{frame}{Theoretical properties: better than PDBs}
  \begin{block}{Greater representational power}
    In \alert{some planning domains} where polynomial-sized pattern
    database heuristics have unbounded error (Gripper, Schedule, two
    Promela variants), merge-and-shrink abstractions can obtain
    \alert{perfect heuristics} in polynomial time with suitable
    merging/shrinking strategies.
  \end{block}

  \begin{itemize}
  \item This shows that \alert{representational power} is
    \alert{strictly greater} than that of PDB heuristics.
  \item However, it does \alert{not} mean that we know good general
    (domain-independent) merging/shrinking strategies that will
    generate these perfect heuristics in practice.
  \end{itemize}
\end{frame}

\begin{frame}{Theoretical properties: additivity}
  \begin{block}{Get additivity for free}
    If $P_1$ and $P_2$ are \alert{additive patterns} of a {\sasplus}
    task, then for \alert{all} $h$-preserving merge-and-shrink
    abstractions $\mathcal A_1$ of $\mathcal T^{\pi_{P_1}}$, $\mathcal
    A_2$ of $\mathcal T^{\pi_{P_2}}$ and $\mathcal A$ of $\mathcal A_1
    \otimes \mathcal A_2$, the abstraction heuristic for
    \alert{$\mathcal A$ dominates $h^{P_1} + h^{P_2}$}. (An
    abstraction is \alert{$h$-preserving} if $\alpha(s) = \alpha(s')$
    only for $s$, $s'$ with same abstract goal distance.)
  \end{block}
  
  \begin{itemize}
  \item One can derive a similar \alert{theory of additivity} for
    merge-and-shrink abstraction as for pattern databases.
  \item However, this result shows that this is \alert{not as
    necessary} as for pattern databases: additivity is
    \alert{exploited automatically} by a single merge-and-shrink
    abstraction to some extent.
  \item Still, experimental results show that there is sometimes a
    benefit of using multiple merge-and-shrink abstractions. (However,
    so far only maximization has been explored.)
  \end{itemize}
\end{frame}

\subsection{Literature}

\begin{frame}{Literature}
  References on merge-and-shrink abstractions:
  \begin{thebibliography}{1}
  \bibitem{} Klaus Dr\"ager, Bernd Finkbeiner and Andreas Podelski.
    \newblock Directed Model Checking with Distance-Preserving
    Abstractions.
    \newblock \emph{Proc.~SPIN 2006}, pp.~19--34, 2006.
    \newblock \alert{Introduces} merge-and-shrink abstractions (for
    model-checking).
  \bibitem{} Malte Helmert, Patrik Haslum and J\"org Hoffmann.
    \newblock Flexible Abstraction Heuristics for Optimal Sequential
    Planning.
    \newblock \emph{Proc.~ICAPS 2007}, pp.~176--183, 2007.
    \newblock Introduces merge-and-shrink abstractions \alert{for
      planning}. \\ Most ideas of this chapter come from this paper.
  \end{thebibliography}
\end{frame}

\section*{Summary}
\begin{frame}{Summary}
  \begin{itemize}
  \item \alert{Projections} \alert{perfectly} reflect \alert{a few} state
    variables. \alert{Merge-and-shrink abstractions} are a
    \alert{generalization} that can reflect \alert{all} state variables, but in
    a \alert{potentially lossy} way.
  \item The \alert{merge steps} combine two abstractions by replacing them
    with their \alert{synchronized product}.
  \item The \alert{shrink steps} make an abstraction smaller by abstracting it
    further.
  \item The resulting \alert{abstraction mapping} is represented by a set of
    reference tables. 
  \end{itemize}
\end{frame}

\end{document}
